% LaTeX source for ``การเรียนรู้ของเครื่องสำหรับเคมีควอนตัม (Machine Learning for Quantum Chemistry)''
% Copyright (c) 2022 รังสิมันต์ เกษแก้ว (Rangsiman Ketkaew).

% License: Creative Commons Attribution-NonCommercial-NoDerivatives 4.0 International (CC BY-NC-ND 4.0)
% https://creativecommons.org/licenses/by-nc-nd/4.0/

\chapter{ชุดข้อมูลทางเคมี}
\label{ch:dataset}

%--------------------------
\section{ปริภูมิเคมี}
%--------------------------

ปริภูมิเคมี (Chemical space) 

%--------------------------
\section{ชุดข้อมูลมาตรฐาน}
%--------------------------

QM9 เป็นหนึ่งใน Dataset ที่ได้รับความนิยมมากในสายงานวิจัยเคมีควอนตัม โดยเฉพาะงานวิจัยทางด้าน ML และ DL ซึ่งถูกใช้อย่างแพร่หลายตั้งแต่ปี
ค.ศ. 2014 เป็นต้นมา\cite{ramakrishnan2014} โดยบทความวิจัยชิ้นแรกที่ได้รับการตีพิมพ์นั้นได้รายงานค่าความแม่นยำและความผิดพลาดว่าไม่เกิน 
10 kcal/mol ซึ่งถือว่าเยอะมาก ๆ และต่อมาได้มีการพัฒนาระเบียบวิธีวิจัยรวมไปถึงโมเดล ML และ Descriptor ใหม่ ๆ จนทำให้ในปัจจุบันนั้น%
นักวิจัยสามารถที่จะทำนายหรือพยากรณ์ค่าพลังงานของโมเลกุลทางเคมีอินทรีย์ขนาดเล็กได้แม่นยำมากโดยมีค่าความคลาดเคลื่อนประมาณ 1 kcal/mol 
หรือต่ำกว่านั้น ซึ่งเป็นค่าที่เรียกว่า Chemical Accuracy หรือเทียบเท่ากับค่าความคลาดเคลื่อนของเครื่องมือทดลองทางเคมี

QM9 ประกอบไปด้วยข้อมูลคุณสมบัติอิเล็กทรอนิกส์ของโมเลกุลมากถุง 134,000 โมเลกุล โดยโมเลกุลทั้งหมดนั้นจะเป็นโมเลกุลที่มีธาตุพื้นฐานเป็น%
องค์ประกอบ ประกอบไปด้วย คาร์บอน ไนโตรเจน ออกซิเจน ไฮโดรเจน และฟลูออรีน โดย Feature หลักของ QM9 ก็จะมีพิกัดคาร์ทีเชียนของอะตอม%
ในโมเลกุลซึ่งได้มาจากการคำนวณการปรับโครงสร้างด้วยระเบียบวิธี B3LYP/6-31G(2df,p) และนอกจากนี้ยังมีค่า Label หรือค่าที่ไว้ใช้ในการเปรียบ%
เทียบการพยากรณ์ดังแสดงในตารางที่ \ref{tab:qm9_feature}
\footnote{โมเดล ML ที่เหมาะสมสำหรับการฝึกสอนด้วย QM9 นั้นจะต้องไม่ขึ้นกับ Translation, Rotation และ Permutation}

\begin{table}[H]
    \centering
    \caption{ข้อมูล Feature ของชุดข้อมูล QM9}
    \label{tab:qm9_feature}
    \small
    \begin{tabular}{llll}\toprule
    \textbf{ดัชนี} &\textbf{ชื่อ} &\textbf{หน่วย} &\textbf{คำอธิบาย} \\\midrule
    0 &index &- &Consecutive, 1-based integer identifier of molecule \\
    1 &A &GHz &Rotational constant A \\
    2 &B &GHz &Rotational constant B \\
    3 &C &GHz &Rotational constant C \\
    4 &mu &Debye &Dipole moment \\
    5 &alpha &Bohr$^3$ &Isotropic polarizability \\
    6 &homo &Hartree &พลังงานของ Highest occupied molecular orbital (HOMO) \\
    7 &lumo &Hartree &พลังงานของ Lowest unoccupied molecular orbital (LUMO) \\
    8 &gap &Hartree &Gap (พลังงานระหว่าง LUMO and HOMO) \\
    9 &r2 &Bohr$^2$ &Electronic spatial extent \\
    10 &zpve &Hartree &Zero point vibrational energy \\
    11 &U0 &Hartree &Internal energy at 0 K \\
    12 &U &Hartree &Internal energy at 298.15 K \\
    13 &H &Hartree &Enthalpy at 298.15 K \\
    14 &G &Hartree &Free energy at 298.15 K \\
    15 &Cv &cal/(mol K) &Heat capacity at 298.15 K \\
    \bottomrule
    \end{tabular}
\end{table}

นอกจาก QM9 แล้วยังมีชุดข้อมูลอื่น ๆ ที่นักวิจัยมักจะนำมาใช้ในการฝึกสอนโมเดลและทำวิจัย เช่น QM7 และ QM8 ซึ่งก็จะมี Label สำหรับวัตถุประสงค์%
ในการฝึกสอนโมเดลในการเพิ่มความสามารถการพยากรณ์คุณสมบัติเคมีของโมเลกุลที่ต่างกันออกไป โดยชุดข้อมูลที่กล่าวมาทั้งหมดนั้นสามารถดาวน์โหลด%
ได้ฟรีจากเว็บไซต์ \url{http://quantum-machine.org/datasets}

%--------------------------
\section{การสร้างชุดข้อมูล}
%--------------------------
