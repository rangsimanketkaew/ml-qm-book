% LaTeX source for ``การเรียนรู้ของเครื่องสำหรับเคมีควอนตัม (Machine Learning for Quantum Chemistry)''
% Copyright (c) 2022 รังสิมันต์ เกษแก้ว (Rangsiman Ketkaew).

% License: Creative Commons Attribution-NonCommercial-NoDerivatives 4.0 International (CC BY-NC-ND 4.0)
% https://creativecommons.org/licenses/by-nc-nd/4.0/

\chapter{การเรียนรู้แบบไม่มีผู้สอน}
\label{ch:unsup_ml}

การเรียนรู้แบบไม่มีผู้สอนหรือ Unsupervised Learning เป็นเทคนิคที่อาจจะเรียกว่าได้ตรงข้ามกับ supervised learning ก็ได้
เพราะว่าเทคนิคประเภทนี้จะเป็นการเทรนโมเดลแบบไม่มีการบอกคำตอบหรือ Output ให้โมเดลได้รับรู้ ดังนั้นสิ่งที่โมเดลจะต้องพยายามทำออกมา
ให้ได้คือเป็นการเรียนรู้หาความสัมพันธ์ (Relation) หรือ สหสัมพันธ์ (Correlation) ระหว่างข้อมูลแต่ละตัวภายในชุดข้อมูล (Input) 
ที่เราสนใจได้ได้ป้อนเข้าไป

%--------------------------
\section{Weighted Pair Group Method with Arithmetic Mean (WPGMA)}
%--------------------------

เป็นเทคนิคการจัดกลุ่ม (Clustering) แบบ Hierarchical โดยถูกพัฒนาขึ้นมาโดยใช้ Pairwise Similarity Matrix.\cite{sokal1958} 

%--------------------------
\section{Principal Component Analysis (PCA)}
%--------------------------

การวิเคราะห์องค์ประกอบหลัก เป็นวิธีทางสถิติที่ถูกเอามาใช้เพื่อรับมือกับข้อมูลที่มี
จำวนหลายมิติหรือมีหลายตัวแปร โดย PCA สามารถหาความสัมพันธ์ของตัวแปรเหล่านั้นโดยทำการลดขนาดของมิติโดยสร้างชุดข้อมูลใหม่ที่อาศัย
แกนอ้างอิงจากชุดข้อมูลเดิม ซึ่งจำนวนมิติที่ถูกลดลงนั้นก็มีจำนวนมิติเพียง 2 หรือ 3 มิติเท่านั้น ซึ่งทำให้ง่ายต่อการตีความและวิเคราะห์ข้อมูล เช่น
การจัดกลุ่มชุดข้อมูลโดยจำแนกตาม Feature ซึ่ง Feature แต่ละคู่จะมีคุณสมบัติ Orthogonality หรือตั้งฉากกันนั่นเอง ทำให้เราสามารถแสดง
ผลลัพธ์ของ PCA ออกมาได้ในปริภูมิทั่วไป

%--------------------------
\section{Autoencoder (AE)}
%--------------------------

ตัวเข้ารหัสแบบอัตโนมัติ เป็นอัลกอริทึม Unsupervised Learning แบบหนึ่งที่สร้างโมเดล ANN โดย
มีรูปแบบของ network ที่เฉพาะตัวนั่นก็คือจะทำการลดหรือบีบอัดข้อมูล (Encoder) และทำการถอดรหัส (Decoder) 
ออกมาเป็นข้อมูลเดิม\cite{ballard1987} ตัวโมเดล AE มีความพิเศษคือจะมีลักษณะของความสมมาตร นอกจากนี้ยังมีความแตกต่างจาก PCA 
นั่นก็คือสามารถบีบอัดหรือลดจำนวนมิติของข้อมูลแบบไม่เป็นเส้นตรงได้ (Nonlinear) ได้ด้วยการใช้ Nonlinear Activation Function 

%--------------------------
\section{K-means Clustering Algorithm}
%--------------------------

