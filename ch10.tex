% LaTeX source for ``การเรียนรู้ของเครื่องสำหรับเคมีควอนตัม (Machine Learning for Quantum Chemistry)''
% Copyright (c) 2022 รังสิมันต์ เกษแก้ว (Rangsiman Ketkaew).

% License: Creative Commons Attribution-NonCommercial-NoDerivatives 4.0 International (CC BY-NC-ND 4.0)
% https://creativecommons.org/licenses/by-nc-nd/4.0/

\chapter{การทำนายคุณสมบัติเชิงโมเลกุล}
\label{ch:predict_molprop}

%--------------------------
\section{Best Practice}
%--------------------------

Best Practice หรือแนวทางปฏิบัติหรือลำดับขั้นตอนสำหรับการนำ ML มาประยุกต์กับเคมีควอนตัมสามารถแบ่งออกเป็น 5 ขั้นตอนง่าย ๆ ได้ดังนี้

\begin{enumerate}
    \item การเลือก dataset
    \item การทำความสะอาดข้อมูลดิบ (raw data)
    \item คำนวณหา Representation ต่อไปได้ ซึ่งขั้นตอนนี้ก็สำคัญมาก ๆ เพราะการเลือก Representation 
    ที่จะใช้ในการเทรนโมเดลก็ต้องสอดคล้องกับ output ที่เราต้องการจะทำนาย -> และหลังจากนั้นก็เป็นขั้นตอนของ
    \item การเลือก ML algorithm เพื่อให้เหมาะสมกับโจทย์ของเรา 
    \item การเทรนโมเดลและการประเมินโมเดลโดยการทำ Validation
\end{enumerate}

%--------------------------
\section{การเลือกโมเดลที่เหมาะสม}
%--------------------------

%--------------------------
\section{การทำนายพลังงาน Potential Energy Surface}
%--------------------------

การทำนายพื้นผิวพลังงานศักย์ (Potential Energy Surface หรือ PES)

Machine Learning Potentials (MLP) แบ่งออกได้เป็นสองประเภทคือ Kernel-based Potential กับ Neural Network-based Potential

- Gaussian Approximation Potentials (GAP)\cite{bartok2010}
- Moment Tensor Potentials (MTP)\cite{shapeev2016}
- Spectral Neighbor Analysis Potentials (SNAP)\cite{thompson2015}


- High-dimensional Neural Network Potentials (HDNNP)\cite{behler2007}
- ANAKIN-ME หรือเรียกสั้น ๆ ว่า ANI (ชื่อเต็มคือ (Accurate NeurAl networK engINe for Molecular Energies)
    - ANI-1x\cite{smith2017}
    - ANI-1ccx\cite{smith2018}
    - ANI-2x\cite{smith2019}

%--------------------------
\section{การจำลอง Force Field}
%--------------------------

%--------------------------
\section{การทำนายพลังงานกระตุ้นของปฏิกิริยาเคมี}
%--------------------------

Activation energy หรือพลังงานกระตุ้น เป็นค่าพลังงานที่บ่งบอกถึงความยากง่ายในการทำให้ปฏิริยาเคมีสามารถดำเนินไปได้ ณ สภาวะหนึ่ง ๆ

%--------------------------
\section{การทำนาย Dipole Moment}
%--------------------------

%--------------------------
\section{การทำนาย Electronic Coupling}
%--------------------------

Electronic Coupling คือค่าความเกี่ยวเนื่องเชิงอิเล็กทรอนิกส์ระหว่าง 2 สถานะใด ๆ เช่นสถานะเริ่มต้นและสถานะสิ้นสุดในกระบวนการทางควอนตัม

- Nonadiabatic Coupling
- Electron transfer Coupling

%--------------------------
\section{การทำนายสเปคตรัม}
%--------------------------


%--------------------------
\section{บทความวิชาการเพิ่มเติม}
%--------------------------

นอกเหนือจากการนำ ML ไปใช้สำหรับการทำนาย Parameter ต่าง ๆ แล้ว ถ้าหากผู้อ่านสนใจการประยุกต์ใช้ ML กับงานทางด้านอื่น ๆ ของเคมีควอนตัม 
สามารถอ่านบทความวิชาการเพิ่มเติมได้จากวารสารวิชาการชั้นแนวหน้า เช่น Journal of Chemical Theory and Computation (JCTC), 
Journal of Chemical Physics (JCP), และ Journal of Physical Chemistry A (JPCA)

โดยผู้เขียนได้เลือกงานวิจัยที่มีความโดดเด่นและเหมาะสำหรับผู้เริ่มต้นศึกษาเป็นอย่างยิ่ง ซึ่งน่าจะช่วยให้ผู้อ่านเห็นภาพรวมของโจทย์งานวิจัยในปัจจุบันที่กำลังมาแรง
บทความที่คัดเลือกมาประกอบไปด้วย Review ที่ใช้ ML ในการเรียนรู้ Force Field สำหรับงานทางด้าน QM/MD หรือนำมาใช้ในการทำนาย Free Energy Landscape 
ไปจนถึงการพัฒนา Model ML เพื่อทำนาย Molecular Properties เช่น Dipole Moment และ Polarizability 

\begin{enumerate}
    \item \enquote{PhysNet: A Neural Network for Predicting Energies, Forces, Dipole Moments, and Partial Charges}\cite{unke2019}\\
    ตีพิมพ์เมื่อวันที่ 01 พฤษภาคม ค.ศ. 2019
    
    \item \enquote{Comparison of the Performance of Machine Learning Models in Representing High-Dimensional Free Energy Surfaces and Generating Observables}\cite{cendagorta2020}\\
    ตีพิมพ์เมื่อวันที่ 10 เมษายน ค.ศ. 2020
    
    \item \enquote{Kernel-Based Machine Learning for Efficient Simulations of Molecular Liquids}\cite{scherer2020}\\
    ตีพิมพ์เมื่อวันที่ 13 เมษายน ค.ศ. 2020

    \item \enquote{Machine Learning Force Fields}\cite{unke2021}\\
    ตีพิมพ์เมื่อวันที่ 11 มีนาคม ค.ศ. 2021\\

    \item \enquote{The Rise of Neural Networks for Materials and Chemical Dynamics}\cite{kulichenko2021}\\
    ตีพิมพ์เมื่อวันที่ 1 กรกฎาคม ค.ศ. 2021\\

\end{enumerate}
