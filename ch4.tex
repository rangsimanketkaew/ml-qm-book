% LaTeX source for ``การเรียนรู้ของเครื่องสำหรับเคมีควอนตัม (Machine Learning for Quantum Chemistry)''
% Copyright (c) 2022 รังสิมันต์ เกษแก้ว (Rangsiman Ketkaew).

% License: Creative Commons Attribution-NonCommercial-NoDerivatives 4.0 International (CC BY-NC-ND 4.0)
% https://creativecommons.org/licenses/by-nc-nd/4.0/

\chapter{การเรียนรู้เชิงลึก}
\label{ch:dl}

%--------------------------
\section{โครงข่ายประสาทเทียม}
%--------------------------

โครงข่ายประสาทเทียม (Neural Network)

การเรียนรู้เชิงลึกรูปแบบที่มาตรฐานที่สุดคือการเรียนรู้แบบมีผู้สอนด้วยโมเดลแบบไม่เป็นเชิงเส้น (Supervised learning with nonlinear model)

%--------------------------
\subsection{Forwardpropagation}
%--------------------------

%--------------------------
\subsection{Backpropagation}
%--------------------------

%--------------------------
\section{Activation Function}
%--------------------------

ฟังก์ชันกระตุ้น

%--------------------------
\section{Learning Layers}
%--------------------------

ชั้นการเรียนรู้

%--------------------------
\section{Loss Function}
%--------------------------

Loss Function หรือ Cost Function หรือ Error Function คือฟังก์ชันความคลาดเคลื่อน

%--------------------------
\section{Optimizer}
%--------------------------

ตัวปรับประสิทธิภาพการเรียนรู้

%--------------------------
\section{Architectures}
%--------------------------

สถาปัตยกรรมของโครงข่ายประสาทเทียม

