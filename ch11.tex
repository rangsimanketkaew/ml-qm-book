% LaTeX source for ``การเรียนรู้ของเครื่องสำหรับเคมีควอนตัม (Machine Learning for Quantum Chemistry)''
% Copyright (c) 2022 รังสิมันต์ เกษแก้ว (Rangsiman Ketkaew).

% License: Creative Commons Attribution-NonCommercial-NoDerivatives 4.0 International (CC BY-NC-ND 4.0)
% https://creativecommons.org/licenses/by-nc-nd/4.0/

\chapter{โมเดลการเรียนรู้ของเครื่องสำหรับเคมีควอนตัม}
\label{ch:chem_ml}

%--------------------------
\section{SchNet และ SchNetOrb}
%--------------------------

%--------------------------
\section{GDML และ sGDML}
%--------------------------

%--------------------------
\section{$\Delta$ML}
%--------------------------

Delta-ML ($\Delta$ML) เป็นเทคนิคที่ใช้ค่าความแตกต่างระหว่างค่าอ้างอิง (Reference หรือจะเรียก Label ก็ได้) จากวิธีการคำนวณที่มีความ%
แม่นยำต่ำกับความแม่นยำสูงมาใช้ในการเทรนโมเดล (จึงเป็นที่มาว่าทำไมถึงเรียกว่า Detla) ซึ่งการทำแบบนี้จะช่วยให้โมเดลสามารถเรียนรู้การเชื่อมโยง 
(Transferability) ไปยังค่าที่ต้องการทำนายได้อย่างถูกต้องและแม่นยำมากขึ้น โดยจะมีความถูกต้องเทียบเคียงกับการใช้วิธีแบบดั้งเดิมที่มีความแม่นยำสูง 
(เช่น Post-HF) ตัวอย่างของการใช้ $\Delta$ML คือการใช้ค่าความแตกต่างของพลังงานที่ได้จากการคำนวณด้วยวิธี DFT และ CCSD(T) 
($y_{DFT} - y_{CCSD(T)}$) มาฝึกสอนโมเดล

จริง ๆ แล้ว $\Delta$ML ก็เป็นเทคนิคอันนึงที่มีแนวคิดมาจากความพยายามที่ต้องการจะทำให้โมเดลสามารถเรียนรู้ได้จากค่าความผิดพลาด (Error) 
โดยเริ่มมีการเอามาใช้กันมากขึ้นในช่วงปีที่ผ่านมา (ในช่วงแรกถูกใช้เยอะแค่ในเฉพาะกลุ่มวิจัยในโซนยุโรป สำหรับการเอามาทำนายพลังงานและ%
เกรเดียนต์ของพลังงาน (Energy Gradient) ซึ่งก็สอดคล้องกับแรงของแต่ละอะตอมในโมเลกุลโมเลกุลนั่นเอง

%--------------------------
\section{Graph Neural Network}
%--------------------------

Graph Neural Network หรือโครงข่ายประสาทแบบกราฟ เป็น NN รูปแบบหนึ่งซึ่งใช้การมองโครงสร้างข้อมูลให้อยู่ในรูปแบบของกราฟ
โดยไอเดียนี้ได้ถูกเสนอตั้งแต่ปี ค.ศ. 2008\cite{scarselli2009,zhou2020}

%--------------------------
\subsection{Message Passing Neural Network}
%--------------------------

Message Passing Neural Network หรือ MPNN ถูกนำเสนอครั้งแรกเมื่อปี ค.ศ. 2018\cite{gilmer2017}

%--------------------------
\section{Molecule Attention Transformer}
%--------------------------
