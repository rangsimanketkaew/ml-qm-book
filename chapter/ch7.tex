% LaTeX source for ``การเรียนรู้ของเครื่องสำหรับเคมีควอนตัม (Machine Learning for Quantum Chemistry)''
% Copyright (c) 2022 รังสิมันต์ เกษแก้ว (Rangsiman Ketkaew).

% License: Creative Commons Attribution-NonCommercial-NoDerivatives 4.0 International (CC BY-NC-ND 4.0)
% https://creativecommons.org/licenses/by-nc-nd/4.0/

\chapter{วิธีคำนวณทางโครงสร้างเชิงอิเล็กทรอนิกส์}
\label{ch:el_strct}

\begin{figure}[htbp]
    \centering
    \includegraphics[width=0.9\linewidth]{fig/electron_density.png}
    \caption{ความหนาแน่นของอิเล็กตรอนของโมเลกุล 2,3-(S,S)-dimethyloxirane ซึ่งคำนวณด้วยวิธี Real-Time Density 
    Functional Theory}
    \label{fig:elec_density}
\end{figure}

ส่วนที่สองของหนังสือเล่มนี้จะเกี่ยวข้องกับเคมีควอนตัมเป็นหลัก เคมีควอนตัมเป็นพื้นฐานสำคัญของการพัฒนาเทคนิคสำหรับการวิเคราะห์คุณสมบัติ%
ของโมเลกุลโดยนักเคมีนั้นส่วนใหญ่แล้วจะใช้เทคนิคทางสเปกโทรสโกปี เช่น Infrared (IR) Spectroscopy, Nuclear Magnetic Resonance 
(NMR) Spectroscopy, และ Scanning Probe Microscopy ซึ่งเทคนิคเหล่านี้ล้วนเกี่ยวข้องกับการคำนวณหาพลังงานในระดับโมเลกุล 
นอกจากนี้แล้วเคมีควอนตัมยังเกี่ยวข้องกับการศึกษาสถานะพื้น (Ground State) และสถานะกระตุ้น (Excited State) ของอะตอมแต่ละตัว 
รวมไปถึงการศึกษากลไกการเกิดปฏิริยาเคมีและสถานะทรานซิชั่น (Transition State) ที่เป็นสถานะที่เกิดขึ้นในการเปลี่ยนแปลงโครงสร้างของ%
โมเลกุล การที่เราเข้าใจองค์ความรู้ขั้นพื้นฐานในระดับอะตอมและโมเลกุลนั้นทำให้เราประยุกต์ใช้และนำไปสู่การศึกษาคุณสมบัติของโมเลกุลในระดับที่%
ใหญ่ขึ้นได้ เช่น เทอร์โมไดนามิกส์ (Thermodynamics) และจลนศาสตร์เชิงเคมี (Chemical Kinetics) ซึ่งนำไปสู่การพัฒนาแบบจำลองทาง%
คณิตศาสตร์ต่าง ๆ ซึ่งสามารถใช้คอมพิวเตอร์ในการศึกษาระบบที่เราสนใจได้ก่อนที่จะไปศึกษาจริงในห้องทดลอง

บทนี้ซึ่งเป็นบทแรกของส่วนที่สองนั้นผู้อ่านจะได้ศึกษาโครงสร้างเชิงอิเล็กทรอนิกส์ (Electronic Structure) ของโมเลกุล ซึ่งเป็นการศึกษาว่า%
อิเล็กตรอนที่อยู่ภายในอะตอมและโมเลกุลนั้นมีพฤติกรรมอย่างไรทั้งในสถาวะพื้นและสถาวะกระตุ้น โดยเราจะมาดูรายละเอียดของทฤษฎีควอนตัมใน%
มุมมองของนักเคมีทฤษฎีรวมไปถึงการพัฒนาระเบียบวิธีการคำนวณเพื่อใช้ในการอธิบายอันตรกิริยาระหว่างอิเล็กตรอนและศึกษาคุณสมบัติของอะตอม%
และโมเลกุลต่อไป
\idxboth{โครงสร้างเชิงอิเล็กทรอนิกส์}{Electronic Structure}

%--------------------------
\section{ฟังก์ชันคลื่น}
\label{sec:wavefunction}
\idxboth{ฟังก์ชันคลื่น}{Wavefunction}
%--------------------------

%--------------------------
\subsection{ประวัติศาสตร์และความสำคัญของสมการชโรดิงเงอร์}
\label{ssec:schrodinger_eq}
\idxboth{ฟังก์ชันคลื่น!สมการชโรดิงเงอร์}{Wavefunction!Schr\"{o}dinger Equation}
%--------------------------

โมเลกุลเป็นหน่วยพื้นฐานของสิ่งต่าง ๆ รอบตัวเรา โมเลกุลก็คือกลุ่มของอะตอมหลาย ๆ อะตอมมารวมกัน และในอะตอมนั้นเราสนใจพฤติกรรมของ%
อิเล็กตรอนเป็นพิเศษ ในวิชากลศาสตร์ควอนตัมนั้นเราจะอธิบายพฤติกรรมของโมเลกุลโดยมุ่งเน้นไปที่อิเล็กตรอนซึ่งสามารถที่จะถูกอธิบายได้ด้วย%
ฟังก์ชันทางคณิตศาสตร์ที่เรียกว่า \enquote{\textit{ฟังก์ชันคลื่น (Wavefunction)}} ซึ่งถูกพัฒนาขึ้นมาเพื่อเป็นแนวคิดสำหรับการอธิบาย%
อิเล็กตรอนและระบบที่ประกอบไปด้วยอิเล็กตรอนหลายตัว โดยหนึ่งในสมการที่โด่งดังที่สุดสมการหนึ่งของวงการวิทยาศาสตร์นั่นคือสมการชโรดิงเงอร์ 
(Schr\"{o}dinger Equation)\autocite{schleich2013} ซึ่งนำเสนอโดยศาสตราจารย์ Erwin R. J. A. Schr\"{o}dinger 
(นักฟิสิกส์เชื้อสายออสเตรีย-ไอริช ซึ่งในขณะนั้นดำรงตำแหน่งอยู่ที่ University of Zurich) สำหรับการอธิบาย Wavefunction 
โดย Schr\"{o}dinger ได้ตีพิมพ์บทความงานวิจัยในวารสาร Annalen der Physik\footnote{ปัจจุบันนี้วารสาร Annalen der Physik 
ยังตีพิมพ์บทความวิชาการอย่างต่อเนื่อง} ในปี ค.ศ. 1926 ที่ต่อเนื่องกันเป็นจำนวน 4 บทความในซีรีย์ที่ชื่อว่า \textit{Quantisierung als 
Eigenwertproblem} โดยบทความฉบับแรกนั้นเป็นการนำเสนอสมการ \textbf{Time-independent Schr\"{o}dinger Equation}%
\autocite{schrodinger1926} และในเวลาต่อมา Schr\"{o}dinger ก็สามารถพิสูจน์หารูปแบบของสมการ \textbf{Time-dependent 
Schr\"{o}dinger Equation} และตีพิมพ์ในบทความฉบับที่ 4 ได้สำเร็จ\autocite{schrodinger1926a} โดยสมการชโรดิงเงอร์ถูกนำมาใช้%
ในการศึกษาระบบทางกลศาสตร์ควอนตัมซึ่งการแก้สมการชโรดิงเงอร์ได้นั้นจะทำให้ได้มาซึ่งผลเฉลยของสมการคณิตศาสตร์ที่อธิบาย Wavefunction 
ได้นั่นเอง

จากผลงานดังกล่าวทำให้ศาสตราจารย์ Erwin Schr\"{o}dinger ได้รับรางวัลโนเบลสาขาฟิสิกส์ ค.ศ. 1933 ร่วมกับศาสตราจารย์ Paul A. M. 
Dirac (ศาสตราจารย์ที่ University of Cambridge) ซึ่งเป็นหนึ่งในผู้บุกเบิกกลศาสตร์ควอนตัมและการพัฒนาสมการดิแรก (Dirac Equation) 
ซึ่งถูกนำมาใช้อธิบายพฤติกรรมของแฟร์มิออน (Fermions)%
\footnote{อ้างอิง \url{https://www.nobelprize.org/prizes/physics/1933/summary}}

\begin{figure}[htbp]
    \centering
    \includegraphics[width=0.8\linewidth]{fig/time-inde-schrodinger-eq.png}
    \caption{ส่วนหนึ่งของบทความฉบับแรกที่ตีพิมพ์โดย Erwin Schr\"{o}dinger ในเดือนมกราคมของปี ค.ศ. 1926 โดยเสนอสมการ 
    Time-independent Schr\"{o}dinger Equation (สมการที่ (1") และ (5))}
    \label{fig:schrodinger_paper_1}
\end{figure}

\begin{figure}[htbp]
    \centering
    \includegraphics[width=0.8\linewidth]{fig/time-dep-schrodinger-eq.png}
    \caption{ส่วนหนึ่งของบทความฉบับที่ 4 ที่ตีพิมพ์โดย Erwin Schr\"{o}dinger โดยเสนอสมการ Time-dependent Schr\"{o}dinger 
    Equation}
    \label{fig:schrodinger_paper_4}
\end{figure}

สมการชโรดิงเงอร์สามารถแบ่งออกได้เป็นสองแบบคือแบบที่ไม่ขึ้นกับเวลาและแบบที่ขึ้นกับเวลา ดังนี้

\noindent $\bullet$ \textbf{1. Time-independent Schr\"{o}dinger Equation}

\fbox{%
\begin{minipage}{0.9\linewidth}
    \begin{equation}\label{eq:tise}
        \hat{H} \Psi = E \Psi
    \end{equation}
\end{minipage}}

\noindent $\bullet$ \textbf{2. Time-dependent Schr\"{o}dinger Equation}

\fbox{%
\begin{minipage}{0.9\linewidth}
    \begin{equation}\label{eq:tdse}
        i \hbar \frac{d}{d t} \Psi(t) = \hat{H} \Psi(t)
    \end{equation}
\end{minipage}}

โดย Wavefunction $\Psi(t)$ ที่เป็นฟังก์ชันไอเกน (Eigenfunction) นั้นจะบรรจุข้อมูลเชิงอิเล็กทรอนิกส์ทุกอย่างเกี่ยวกับระบบของเราเอาไว้%
\autocite{szabo1996,cramer2004,jensen2017} ซึ่งระบบในที่นี้ก็คือโมเลกุล โดยสมการข้างต้นเป็นการคำนวณหาพลังงานของระบบโดยใช้ 
Hamiltonian Operator $(\hat{H})$ ซึ่งเป็น Operator ที่สอดคล้องกับพลังงาน ซึ่งจริง ๆ แล้วค่าไอเกน (Eigenvalue) ของสมการข้างต้น 
(สมการที่ \ref{eq:tdse} และ \ref{eq:tise}) จะเป็นคุณสมบัติของโมเลกุลอะไรก็ได้ ตราบใดที่เราใช้ Operator ที่สอดคล้องกับคุณสมบัตินั้น ๆ 
\idxboth{ฟังก์ชันไอเกน}{Eigenfunction}
\idxboth{ค่าไอเกน}{Eigenvalue}

%--------------------------
\subsection{คุณสมบัติของฟังก์ชันคลื่น}
\label{ssec:wavefunc_prop}
\idxboth{ฟังก์ชันคลื่น!คุณสมบัติ}{Wavefunction!Properties}
%--------------------------

ฟังก์ชันคลื่นเชิงอิเล็กทรอนิกส์ที่ได้มาจากผลเฉลยที่ถูกต้องหรือได้มาจากการประมาณค่านั้นจะต้องมีคุณสมบัติต่อไปนี้ 

\begin{itemize}[topsep=0pt]
    \item ฟังก์ชันมีค่าที่แน่นอนและมีขอบเขต (Be Finite)
    
    \item ฟังก์ชันมีความต่อเนื่องและหาค่าได้ตลอดทั้งโดเมน (Be Continuous)
    
    \item มีผลเฉลยเพียงแค่ค่าเดียวเท่านั้นสำหรับโดเมนหนึ่งค่า (Single Values) กล่าวคือโดเมนหรืออินพุต $x$ จะต้องให้เรนจ์หรือเอาต์พุต 
    $y$ แค่หนึ่งค่าเท่านั้น
    
    \item เป็นฟังก์ชันที่มีคุณสมบัติในการมองอิเล็กตรอนทุก ๆ ตัวเหมือนกัน (Indistinguishability of Electron)
    
    \item ค่ายกกำลังสองของฟังก์ชันเป็นการกระจายตัวของความน่าจะเป็น
    
    \item ต้องมีความปฏิสมมาตร (Antisymmetry) กล่าวคืออิเล็กตรอนนั้นคือเฟอร์มิออน (Fermion)\autocite{atkins2010} 
    ดังนั้นฟังก์ชันคลื่นจะต้องเปลี่ยนมีการเปลี่ยนเครื่องหมายเมื่ออิเล็กตรอนสองตัวใด ๆ มีการแลกเปลี่ยนพิกัดเชิงพื้นที่หรือพิกัดเชิงสปินกัน
\end{itemize}

ถ้าฟังก์ชันนั้นไม่มีคุณสมบัติข้างต้นนี้จะถือว่าไม่มีความเหมาะสมในการนำมาใช้งานและจะให้ผลการคำนวณที่ผิดพลาด

%--------------------------
\section{แฮมิลโทเนียน}
\label{sec:hamiltonian}
%--------------------------

Hamiltonian เป็นสิ่งที่สำคัญมากในเคมีควอนตัมเพราะเปรียบเสมือนเป็นกุญแจที่สามารถไขรหัสหาคำตอบหรือความลับจาก Wavefunction ได้
โดย Hamiltonian Operator ที่เรานำมาใช้งานนั้นจริง ๆ แล้วก็คือ Operator สำหรับการหาพลังงานรวมนั่นเอง โดยเป็นผลรวมของ Operator 
พลังงานจลน์และพลังงานศักย์
\idxen{Operator}
\idxboth{แฮมิลโทเนียน}{Hamiltonian}

\begin{equation}\label{eq:hamil}
    \hat{H} = \hat{T} + \hat{V}
\end{equation}

\noindent โดยที่พลังงานจลน์นั้นสามารถเขียนให้อยู่ในรูปของ Momentum Operator ได้โดยพิสูจน์จากพลังงานจลน์ในกรณีแบบดั้งเดิม ดังนี้
\idxboth{โมเมนตัม}{Momentum}
\idxboth{พลังงานจลน์}{Kinetic Energy}

\begin{align}
    T &= \frac{1}{2}mv^{2}_{x} \\
      &= \frac{(mv_{x})^{2}}{2m}
\end{align}

\noindent ทำการจัดรูปใหม่แล้วทำการแทนเทอม $mv_{x}$ ด้วย Momentum Opeator ในทางกลศาสตร์ควอนตัม $(-ih\frac{d}{dx})$ 
จะได้ Operator ใหม่ดังนี้

\begin{equation}\label{eq:kin_ener_oper}
    \hat{T} = -\frac{\hbar^{2}}{2m}\frac{d^{2}}{dx^{2}}
\end{equation}

สำหรับพลังงานศักย์นั้นตรงไปตรงมา นั่นคือเราสามารถเขียนพลังงานศักย์ในทางควอนตัมได้แบบเดียวกับกรณีกลศาสตร์ดั้งเดิมได้เลย ดังนี้
\idxboth{พลังงานศักย์}{Potential Energy}

\begin{equation}\label{eq:pot_ener_oper}
    \hat{V} = V(x)
\end{equation}

เมื่อเรานำ Operator ของทั้งสองพลังงาน (สมการที่ \ref{eq:kin_ener_oper} และสมการที่ \ref{eq:pot_ener_oper}) มารวมกันเรา%
จะได้ Hamiltonian Operator ดังนี้

\begin{equation}
    \hat{H} = -\frac{\hbar^{2}}{2m}\frac{d^{2}}{dx^{2}} + V(x)
\end{equation}

ลำดับต่อมาคือเราจะมาทำการพิจารณาพลังงานศักย์กันก่อนเพราะว่าไม่ซับซ้อนเหมือนกับกรณีของพลังงานจลน์ โดยพลังงานศักย์ที่เราจะพิจารณาก็คือ%
พลังงานงานศักย์คูลอมบ์ (Coulomb Potential Energy หรือ Operator นั่นเอง) โดยมีสมการดังต่อไปนี้
\idxboth{พลังงานงานศักย์!พลังงานคูลอมบ์}{Potential Energy!Coulomb Energy}

\begin{equation}
    E_{q_{1}q_{2}} = q_{1}\frac{q_{2}}{4\pi\epsilon_{0}|\bm{R}|}
\end{equation}

โดยเมื่อเราพิจารณาระบบง่าย ๆ เช่น อะตอมไฮโดรเจนซึ่งมี 1 อิเล็กตรอนและ 1 นิวเคลียส แล้วกำหนดจุดกำเนิด (Origin Point) ซึ่งมีระยะห่าง%
จากอิเล็กตรอนเท่ากับ $แผ่{r}$ หน่วยและมีระยะห่างจากนิวเคลียสเท่ากับ $\bm{R}$ หน่วย จะได้ว่าระยะห่างระหว่างอิเล็กตรอนและนิวเคลียสคือ 
$\bm{r}-\bm{R}$ หน่วย ดังนั้นเราสามารถเขียน Hamiltonian Operator ได้ดังนี้

\begin{multline}\label{eq:hamil_hydrogen}
    \hat{H} = -\underbrace{\frac{\hbar^{2}}{2M} \left( \pdv[2]{X} + \pdv[2]{Y} + \pdv[2]{Z} \right)}_{%
                \text{Nuclear Kinetic Energy}} 
              \\
              -\underbrace{\frac{\hbar^{2}}{2m} \left( \pdv[2]{x} + \pdv[2]{y} + \pdv[2]{z} \right)}_{%
                \text{Electronic Kinetic Energy}}
              \\
              -\underbrace{\frac{1}{4\pi\epsilon_{0}}\frac{e^{2}}{|\bm{r}-\bm{R}|}}_{%
                \text{Electron-Nucleus Attraction}}
\end{multline}

\noindent โดยเราสามารถใช้สัญลักษณ์ $\nabla^{2}$ หรือ Laplace Operator ($\nabla$ อ่านว่า Nabla) ซึ่งเป็นอนุพันธ์อันดับที่สอง%
ของพลังงานจลน์ของนิวเคลียส (เทอมแรก) และของพลังงานจลน์ของอิเล็กตรอน (เทอมที่สอง) ของสมการที่ \ref{eq:hamil_hydrogen} 
โดยสามารถเขียนสมการใหม่ได้ดังนี้

\begin{equation}\label{eq:hamil_reduced}
    \hat{H} = -\frac{\hbar^{2}}{2M} \nabla^{2}_{\bm{R}} - \frac{\hbar^{2}}{2m} \nabla^{2}_{\bm{r}}
              -\frac{1}{4\pi\epsilon_{0}}\frac{e^{2}}{|\bm{r}-\bm{R}|}
\end{equation}

ถึงแม้ว่าสมการที่ \ref{eq:hamil_reduced} มีความเรียบง่ายแล้วแต่ว่าในเคมีควอนตัมนั้นเราจะไม่ได้ใช้สมการของ Operator ที่อยู่ในหน่วย
SI (SI Units) โดยนักเคมีทฤษฎีนั้นจะใช้หน่วยอะตอม (Atomic Units หรือย่อได้เป็น a.u. หรือบางครั้งก็เขียนแค่ au)%
\footnote{อ่านรายละเอียดเกี่ยวกับ Atomic Units ได้ที่ \url{https://en.wikipedia.org/wiki/Hartree_atomic_units}} 
ซึ่งเมื่อเราเขียนสมการในรูปของ Atomic Units แล้วจะได้สมการที่เรียบง่ายกว่าเดิม ดังนี้

\begin{equation}\label{eq:hamil_au}
    \hat{H} = -\frac{1}{2M} \nabla^{2}_{\bm{R}} 
              -\frac{1}{2} \nabla^{2}_{\bm{r}}
              -\frac{1}{|\bm{r}-\bm{R}|}
\end{equation}

\noindent โดยจะสังเกตได้ว่าตัวแปรที่เกี่ยวข้องกับอิเล็กตรอนนั้นจะถูกลดรูปไป ปริมาณที่กำหนดให้มีหน่วยเป็น Atomic Units ได้มีดังนี้
\idxen{Atomic Units}
\idxen{SI Units}

\begin{table}[H]
    \centering
    \caption{เปรียบเทียบปริมาณทางเคมีควอนตัมในหน่วย Atomic Units และ SI Units}
    \label{tab:atomic_units}
    \begin{tabular}{lll}\toprule
    \textbf{ปริมาณ} &\textbf{Atomic Unit} &\textbf{ค่าในหน่วย SI} \\\midrule
    พลังงาน & $\hbar^{2}/m_{e}a_{0}$ (Hartree) & $4.36 \times 10^{-18} J$ \\
    ประจุ & $e$ & $1.60 \times 10^{-19} C$ \\
    ความยาว & $a_{0}$ & $5.29 \times 10^{-11} m$ \\
    มวล & $m_{e}$ & $9.11 \times 10^{-31} kg$ \\
    \bottomrule
    \end{tabular}
\end{table}

สำหรับกรณีของระบบที่มีอิเล็กตรอนมากกว่าหนึ่งตัว เช่น อะตอมฮีเลียมที่มี 2 อิเล็กตรอน เราสามารถกระจายเทอมของ Hamiltonian ได้ดังนี้

\begin{multline}\label{eq:hamil_he_au}
    \hat{H} = -\frac{1}{2M} \nabla^{2}_{\bm{R}} 
              -\frac{1}{2} \nabla^{2}_{\bm{r_{1}}}
              -\frac{1}{2} \nabla^{2}_{\bm{r_{2}}}
              -\frac{2}{|\bm{r_{1}}-\bm{R}|}
              \\
              -\frac{2}{|\bm{r_{2}}-\bm{R}|}
              +\frac{1}{|\bm{r_{1}}-\bm{r_{1}}|}
\end{multline}

\noindent โดยทั้ง 6 เทอมคือพลังงานจลน์ของนิวเคลียส, พลังงานจลน์ของอิเล็กตรอนตัวที่ 1, พลังงานจลน์ของอิเล็กตรอนตัวที่ 2, 
แรงดึงดูดระหว่างอิเล็กตรอนตัวที่ 1 และนิวเคลียส, แรงดึงดูดระหว่างอิเล็กตรอนตัวที่ 2 และนิวเคลียส, และแรงผลักระหว่างอิเล็กตรอน ตามลำดับ 

นอกจากเราสามารถใช้การประมาณของบอร์น-ออปเพนไฮเมอร์ (Born-Oppenheimer (BO) Approximation) ซึ่งเป็นเทคนิคที่นำมาใช้เพื่อการ%
ประมาณว่า Wavefunction ของโมเลกุลนั้นขึ้นอยู่กับตำแหน่งของอิเล็กตรอนเพียงอย่างเดียวและไม่ขึ้นกับตำแหน่งของนิวเคลียสเนื่องจากว่ามวลของ%
นิวเคลียวนั้นเยอะกว่ามวลของอิเล็กตรอนมาก ซึ่งถ้าหากใช้ BO Approximation กับ Hamiltonian ของอะตอมฮีเลียมนั้น เทอมแรกของสมการที่ 
\ref{eq:hamil_he_au} จะไม่ถูกนำมาพิจารณาในการคำนวณพลังงานของระบบ

%--------------------------
\section{การแก้สมการฟังก์ชันคลื่นเพื่อคำนวณพลังงาน}
\label{sec:wavefunc_ener}
%--------------------------

\begin{figure}[htbp]
    \centering
    \includegraphics[width=0.9\linewidth]{fig/hydrogen_density_plots.png}
    \caption{แบบจำลองของออร์บิทัลเชิงอะตอม (Atomic Orbitals) ของอิเล็กตรอนของอะตอมไฮโดรเจนที่ระดับพลังงานที่แตกต่างกัน
    โดยความเข้มของสีที่ไฮไลท์บ่งบอกถึงโอกาสที่จะพบอิเล็กตรอน ณ ตำแหน่งนั้น 
    (เครดิตภาพ: \url{https://en.wikipedia.org/wiki/Atomic_orbital})}
    \label{fig:hydrogen_density}
\end{figure}

หนึ่งในเป้าหมายสำคัญของกลศาสตร์ควอนตัมเชิงโมเลกุล (Molecular Quantum Mechanics) ก็คือการหาวิธีแก้สมการ Time-independent 
Schr\"{o}dinger Equation เพื่อให้ได้มาซึ่งคำตอบหรือผลเฉลยที่แม่นยำมากที่สุด ซึ่งจะช่วยให้นักเคมีคำนวณสามารถคำนวณคุณสมบัติโครงสร้าง%
เชิงอิเล็กทรอนิกส์ (Electronic Structure) ของโมเลกุล โดยหัวข้อแรกของบทนี้ที่เราจะมาดูกันแบบละเอียดก็คือการใช้เทคนิคควอนตัมเชิงคำนวณ%
และอาศัยการประมาณค่าในการแก้สมการดังกล่าว โดยทั่วไปนั้นจะมีวิธีการหลัก ๆ 2 วิธีที่สามารถช่วยให้เราหาคำตอบของสมการชโรดิงเงอร์ ได้นั่นคือ 
\textbf{\textit{ab initio} method} ซึ่งเป็นวิธีที่ความแม่นยำของผลลัพธ์ที่ได้จากการแก้สมการนั้นจะขึ้นอยู่กับโมเดลที่เรานำมาใช้ในการอธิบาย 
Wavefunction ของระบบของเรา (โมเลกุลจะถูกมองเป็น Many-body System) วิธี \textit{ab initio} ที่ได้มีการพัฒนากันมาตั้งแต่อดีตจน%
ถึงปัจจุบันนั้นมีหลายวิธีมาก\autocite{friesner2005,helgaker2014,jensen2017} วิธีทีได้รับความนิยมมีดังไปนี้
\idxboth{โครงสร้างเชิงอิเล็กทรอนิกส์}{Electronic Structure}

\noindent \textbf{วิธี Hartree-Fock}
\begin{itemize}[topsep=0pt,noitemsep]
    \item Hartree-Fock (HF)
    \item Restricted open-shell Hartree-Fock (ROHF)
    \item Unrestricted Hartree-Fock (UHF)
\end{itemize}

\noindent \textbf{วิธี Post-Hartree-Fock}
\begin{itemize}[topsep=0pt,noitemsep]
    \item Møller-Plesset Perturbation Theory (MPn)
    \item Configuration Interaction (CI)
    \item Coupled Cluster (CC)
    \item Quadratic Configuration Interaction (QCI)
    \item Quantum Chemistry Composite Methods
\end{itemize}

\noindent \textbf{วิธี Multi-reference}
\begin{itemize}[topsep=0pt,noitemsep]
    \item Multi-configurational Self-consistent Field (MCSCF) รวมถึงวิธี CASSCF and RASSCF
    \item Multi-reference Configuration Interaction (MRCI)
    \item n-electron Valence State Perturbation Theory (NEVPT)
    \item Complete Active Space Perturbation Theory (CASPTn)
    \item State Universal Multi-reference Coupled-cluster Theory (SUMR-CC)
\end{itemize}

นอกจากนี้เป็นที่ทราบกันดีว่าสำหรับโมเลกุลที่มีขนาดใหญ่นั้นการคำนวณด้วยวิธี \textit{ab initio} มีความสิ้นเปลืองสูงมาก (Computationally 
Expensive)\autocite{grabowski2011} ดังนั้นจึงเป็นที่มาของการพัฒนาวิธีการที่สองนั่นคือ \textbf{Semiempirical method}%
\autocite{thiel2014,christensen2016,kriz2020} ซึ่งจะใช้แนวคิดในการตีความ Hamiltonian ในรูปแบบที่ง่ายกว่าซึ่งอ้างอิงด้วยออร์บิทัล%
เชิงโมเลกุล (Molecular Orbital หรือ MO) และอาศัยค่าพารามิเตอร์ที่ได้จากการทดลองเพื่อเพิ่มความแม่นยำ อย่างไรก็ตาม วิธี Density 
Functional Theory (DFT) ก็ถูกพัฒนาขึ้นมาเพื่อแก้ปัญหาที่เราจะต้องมาแก้หรือประมาณค่า Wavefunction ตรง ๆ ซึ่งทำได้ยากโดยเฉพาะกรณี%
ที่ระบบมีหลายอิเล็กตรอน ดังนั้นในปัจจุบันการคำนวณเชิงควอนตัมส่วนใหญ่จึงเป็นการใช้ DFT เพราะว่ามีความสิ้นเปลืองของการคำนวณที่ต่ำมากเมื่อ%
เทียบกับสองวิธีข้างต้นทีได้กล่าวไปนั่นเอง
\idxboth{วิธีแบบกึ่งการทดลอง}{Semiempirical Method}
\idxboth{ออร์บิทัลเชิงโมเลกุล}{Molecular Orbital}
\idxboth{ทฤษฎีฟังก์ชันนอลความหนาแน่น}{Density Functional Theory}

ตัวอย่างของความสำเร็จในการแก้สมการ Wavefunction ก็คือผลลัพธ์ที่แน่นอนของคุณสมบัติของระบบ กรณีที่เราสามารถหาผลเฉลยได้แน่นอนก็คือ%
ระบบที่มีอิเล็กตรอน 1 ตัว ตามแสดงในภาพที่ \ref{fig:hydrogen_density} ซึ่งเป็นแบบจำลองของออร์บิทัลเชิงอะตอม (Atomic Orbital 
หรือ AO) ของอิเล็กตรอนของอะตอมไฮโดรเจน ผู้อ่านสามารถศึกษาการเขียนโค้ดสำหรับพล็อตออร์บิทัลของอะตอทไฮโดรเจนได้ที่ภาพผนวกหัวข้อที่ 
\ref{ch:hydro_orbitals}
\idxboth{ออร์บิทัลเชิงอะตอม}{Atomic Orbital}

%--------------------------
\subsection{วิธี Self-consistent Field}
\label{ssec:scf}
\idxen{Self-consistent Field}
%--------------------------

ในหัวข้อนี้เราจะมาพูดถึงการแก้สมการชโรดิงเงอร์โดยใช้วิธีที่ชื่อว่า Self-consistent Field (SCF) ซึ่งเป็นการประมาณค่า Hamiltonian 
แบบวนซ้ำ (เป็นที่มาของคำว่า \textit{Self-consistent} ซึ่งมีความหมายประมาณว่าเป็นดำเนินการเปรียบเทียบพารามิเตอร์ใหม่กับพารามิเตอร์%
เดิมโดยที่ยังคงใช้โมเดลอันเดียวกัน) เริ่มต้นเราจะต้องมาดูกันก่อนว่าการมอง Wavefunction ของระบบหลายอิเล็กตรอนสำหรับวิธี SCF นั้นจะมีการ%
ตัดสิ่งที่ซับซ้อนออกไปนั่นก็คืออันตรกิริยาแรงผลักระหว่างอิเล็กตรอน (Electron-electron Repulsion) โดย Wavefunction สามารถถูกอธิบาย%
ได้ด้วยสมการชโรดิงเงอร์ที่ไม่ขึ้นกับเวลา ดังต่อไปนี้\autocite{cramer2004}

\begin{equation}\label{eq:tise_elec}
    H^{\circ} \Psi^{\circ} = E^{\circ} \Psi^{\circ}
\end{equation}

โดยกำหนดให้ $H^{\circ} = \sum^{N}_{i=1} h_{i}$ เมื่อ $h$ คือ Hamiltonian สำหรับอิเล็กตรอนตัวที่ $i$ ในระบบที่มีอิเล็กตรอน 
$N$ ตัว นั่นคือสมการสำหรับระบบที่มีอิเล็กตรอน $N$ ตัวนั้น จะสามารถถูกแยกออกมาได้เป็นสมการของระบบหนึ่งอิเล็กตรอนได้ $N$ สมการและ 
Wavefunction ของอิเล็กตรอนหนึ่งตัวนั้นจริง ๆ แล้วก็คือออร์บิทัล (Orbital) เราจึงสามารถเขียนสมการของอิเล็กตรอนหนึ่งตัวโดยอ้างอิงจาก%
สมการที่ \ref{eq:tise_elec} ได้เป็นสมการที่จำเพาะเจาะจงมากขึ้น ดังนี้
\idxboth{ออร์บิทัล}{Orbital}

\begin{equation}\label{eq:tise_elec_i}
    h_{i} \Psi^{\circ}(i) = E^{\circ}_{m} \Psi^{\circ}(i)
\end{equation}

\noindent โดยที่ $E^{\circ}_{m}$ คือพลังงานของอิเล็กตรอนหนึ่งตัวใน MO ซึ่งเขียนแทนด้วย $m$ นั่นเอง สำหรับระบบที่อิเล็กตรอนไม่ขึ้นต่อกัน
\idxboth{ออร์บิทัลเชิงโมเลกุล}{Molecular Orbital}

ด้วยเหตุนี้ Wavefunction รวมของระบบ $(\Psi^{\circ})$ จึงสามารถเขียนให้อยู่ในรูปของ Wavefunction ของอิเล็กตรอนหนึ่งตัวได้ดังนี้

\begin{equation}
    \Psi^{\circ} = \psi^{\circ}_{a}(1) \psi^{\circ}_{b}(1) \dots \psi^{\circ}_{z}(N)
\end{equation}

\noindent ซึ่ง Wavefunction ด้านบนนี้จะขึ้นอยู่กับพิกัดของอิเล็กตรอนทุกตัวและขึ้นกับตำแหน่งของนิวเคลียสหรืออะตอมด้วย%
\footnote{ตอนนี้เราจะยังไม่พิจารณาสปินของอิเล็กตรอนที่จะต้องสอดคล้องและไม่ขัดกับหลักกีดกันของเพาลี (Pauli Exclusion)
ซึ่งจะมีการรวม Spin-orbital สำหรับ Molecular Orbital $m$ $(\varphi_{m})$ เข้าไปด้วย}
\idxboth{หลักกีดกันของเพาลี}{Pauli Exclusion}

สำหรับกระบวนการหรือขั้นตอนที่เราจะนำมาใช้ในการแก้สมการของระบบอิเล็กตรอนหลายตัวนั้น เราจะพิจารณาสมการรูทฮาน (Roothaan Equation) 
เป็นหลัก ซึ่งเป็นวิธีหนึ่งในการแก้สมการ Hartree-Fock (HF) ซึ่งมีการกำหนดตัวดำเนินการใหม่ขึ้นมาใช้แทน Hamiltonian นั่นก็คือ Fock Operator 
โดยที่ Fock Operator $(f_{1})$ ถูกนิยามในเทอมของ Coulomb Operator และ Exchange Operator ขึ้นมา นั่นก็คือ Fock Operator 
ซึ่งเขียนสมการสำหรับอิเล็กตรอน 1 ตัวได้เป็น

\begin{equation}\label{eq:fock}
    f_{1} \psi_{m}(1) = \varepsilon_{n} \psi_{m}(1)
\end{equation}

%--------------------------
\subsection{สมการ Roothaan}
\label{ssec:roothaan}
\idxen{Roothaan Equation}
%--------------------------

สำหรับการแก้สมการ HF ตรง ๆ โดยใช้ SCF นั้นสามารถทำได้ตรง ๆ ด้วยวิธีการเชิงตัวเลข (Numerical Method) แต่ว่าผลเฉลยที่ได้มานั้นมีความ%
ซับซ้อนมาก โดยในเวลาต่อมานักฟิสิกส์และนักเคมีชาวดัตช์ที่ชื่อว่า Clemens C.J. Roothaan จึงได้เสนอวิธีการใหม่สำหรับการอธิบาย MO โดยเรียก%
วิธีนั้นว่าผลรวมเชิงเส้น (Linear Combination of Atomic Orbitals หรือ LCAO)\autocite{atkins2010} เรามาดูรายละเอียดของ LCAO 
กันครับ 
\idxen{Linear Combination of Atomic Orbitals (LCAO)}

เริ่มต้นเราจะนิยามฟังก์ชันพื้นฐาน (Basis Function) สำหรับระบบที่มีอิเล็กตรอน $N$ ตัวขึ้นมาก่อน ซึ่งเขียนแทนด้วย $\chi_{o}$
ซึ่งไอเดียตอนนี้ก็คือเราจะมองว่า Basis Function แบบที่ง่ายที่สุดที่เราสามารถนำมาใช้ได้นั้นก็คือ AO โดยเราสามารถเขียนฟังก์ชันคลื่นเชิงพื้นที่ 
(Spatial Wavefunction) ซึ่งเป็น Wavefunction ที่ขึ้นกับตำแหน่งของ AO ให้อยู่ในผลรวมเชิงเส้นของการคูณระหว่างสัมประสิทธิ์เชิงเส้นที่เรา%
ยังไม่ทราบค่า $(c_{om})$ กับ Basis Function $\chi_{o}$ ได้ดังนี้
\idxboth{ออร์บิทัลเชิงอะตอม}{Atomic Orbital}
\idxboth{ฟังก์ชันคลื่นเชิงพื้นที่}{Spatial Wavefunction}
\idxboth{สัมประสิทธิ์ที่ยังไม่ทราบค่า}{Unknown Coefficients} 

\begin{equation}\label{eq:lcao}
    \psi_{m} = \sum^{N_{o}}_{o=1} c_{om} \chi_{o} 
\end{equation}

\noindent เมื่อเราแทนสมการ \ref{eq:lcao} เข้าไปในสมการ \ref{eq:fock} เราจะได้

\begin{equation}\label{eq:lcao_in_fock}
    f_{1} \sum^{N_{o}}_{o=1} \chi_{o}(1) = \varepsilon \sum^{N_{o}}_{o=1} c_{om} \chi_{o}(1)
\end{equation}

\noindent แล้วทำการคูณสมการ \ref{eq:lcao_in_fock} ทั้งสองข้างด้วย $\chi^{*}_{o}(1)$ และทำการอินทิเกรตทั่วทั้ง Space 
ซึ่งจะทำให้เราได้ความสัมพันธ์ต่อไปนี้

\begin{equation}\label{eq:lcao_in_fock_int}
    \sum^{N_{o}}_{o=1} c_{om} \int \chi^{*}_{o}(1) f_{1} \chi_{o}(1) d\tau_{1} =
    \varepsilon_{m} \sum^{N_{o}}_{o=1} c_{om} \int \chi^{*}_{o}(1) \chi_{o}(1) d\tau_{1}
\end{equation}

\noindent จากสมการข้างต้นเราจะพบว่าจะมีผลคูณของ Basis Function ทั้งสองฝั่ง โดยทางฝั่งซ้ายนั้นเราสามารถนิยาม Fock Matrix (F) ได้

\begin{equation}\label{eq:matrix_fock}
    F_{o'o} = \int \chi^{*}_{o'}(1) f_{1} \chi_{o}(1) d\tau_{1}
\end{equation}

\noindent และทางฝั่งขวา เรานิยามสิ่งที่เรียกว่า Overlap Matrix (S) ซึ่งเป็น Matrix ที่อธิบายถึงการซ้อนทับกันระหว่างสถานะ 2 สถานะ

\begin{equation}\label{eq:matrix_overlap}
    S_{o'o} = \int \chi^{*}_{o'}(1) \chi_{o}(1) d\tau_{1}
\end{equation}

\noindent ซึ่งเราสามารถเขียนสมการ \ref{eq:lcao_in_fock_int} ให้อยู่ในรูปของสมการที่เรียกว่า Roothaan Equation ได้กระชับ ๆ ดังนี้

\begin{equation}\label{eq:roothaan}
    F c = \varepsilon S c
\end{equation}

\noindent โดยที่ $c$ คือเมทริกซ์ขนาด $N_{o} \times N_{o}$ ซึ่งประกอบไปด้วยสมาชิกของ Coefficient $c_{om}$ และ $\varepsilon$ 
คือเมทริกซ์ที่มีขนาด $N_{o} \times N_{o}$ เช่นเดียวกันซึ่งเป็นเมทริกซ์แบบ Diagonal Matrix (สมาชิกที่ไม่ใช่แนวทแยงมีค่าเป็น 0 ทั้งหมด) 
ซึ่งก็คือพลังงานของ Orbital นั่นเอง ซึ่งตรงจุดนี้เราต้องไม่ลืมว่า Fock Operator $(f_{1})$ นั้นถูกกำหนดให้อยู่ในรูปของ Integral บน MO 
และขึ้นอยู่กับค่าของ Coefficient $c_{om}$ ด้วย

สำหรับการแก้สมการ \ref{eq:roothaan} นั้นสามารถทำได้ผ่าน Determinant ดังนี้

\begin{equation}\label{eq:scf_secular}
    det|F - \varepsilon S| = 0
\end{equation}

\noindent ซึ่งสมการด้านบนไม่สามารถแก้ได้แบบตรงไปตรงมาเพราะว่าสมาชิกของเมทริกซ์ $F_{o'o}$ นั้นเกี่ยวเนื่องโดยตรงกับ Integral ของ 
Coulomb Operator และ Exchange Operator ซึ่งขึ้นอยู่กับ Spatial Wavefunction นั่นจึงทำให้เป็นปัญหาแบบงูกินหาง ดังนั้นเราจึงต้อง%
ใช้กระบวนการวนซ้ำ (Iterative Method) ในการแก้ปัญหาจนกว่าคำตอบหรือผลลัพธ์ที่เราต้องการจากสมการ (พลังงาน) จะลู่เข้านั่นเอง
\idxboth{โอเปอเรเตอร์คูลอมป์}{Coulomb Operator}
\idxboth{โอเปอเรเตอร์แลกเปลี่ยน}{Exchange Operators}
\idxboth{ฟังก์ชันคลื่นเชิงพื้นที่}{Spatial Wavefunction}

%--------------------------
\subsection{การแก้สมการ Roothaan ด้วย Self-consistent Field}
\label{ssec:roothaan_scf}
%--------------------------

\begin{figure}[htbp]
    \centering
    \includegraphics[width=0.9\linewidth]{fig/scf.png}
    \caption{แผนผังขั้นตอนของการประมาณค่าหาพลังงานของออร์บิทัลด้วยวิธี SCF}
    \label{fig:scf}
\end{figure}

ภาพที่ \ref{fig:scf} แสดงแผนผงอัลกอริทึมของวิธี SCF โดยเริ่มจากการเลือก Atomic Basis Function ซึ่งถือว่าเป็นองค์ประกอบหลักของ%
การนำไปสร้าง (Formulate) $S$ โดยใช้สมการ \ref{eq:matrix_overlap} กับ $c_{om}$ ซึ่งเราจะใช้วิธีการสร้างค่าเริ่มต้นด้วยวิธี Guess 
ซึ่งมีด้วยกันหลายวิธี เช่น

\begin{enumerate}[topsep=0pt]
    \item \textbf{H{\"u}ckel guess} : ใช้ H{\"u}ckel Orbital\autocite{jensen2017}
    
    \item \textbf{Superposition of Atomic Densities (SAD)} : ใช้ผลรวมของ Atomic Density ในการสร้าง Density Matrix
    
    \item \textbf{Generalized Wolfsberg-Helmholtz (GWH)} : เป็นวิธีการที่อาศัย H{\"u}ckel Theory โดยการใช้ Overlap 
    Matrix และ Core Hamiltonian\autocite{wolfsberg1952}
    
    \item \textbf{CORE} : ทำการทำ Core Hamiltonian ให้เกิดเมทริกซ์รูปทแยง (Diagonalization)
    
    \item \textbf{Harris} : ใช้ Harris Functional ซึ่งเป็น Non-self-consistent Approximation สำหรับ Kohn-Sham 
    Orbital\autocite{harris1985}
\end{enumerate}

ซึ่งโปรแกรมเคมีเชิงคำนวณต่างก็มีการเลือกใช้ Guess Method สำหรับการเดา Coefficient หรือ Wavefunction เริ่มต้นในการแก้ SCF แตกต่างกันไป
โปรแกรม Gaussian ใช้วิธี Harris สำหรับการคำนวณ HF และ DFT และใช้ H{\"u}ckel หรือ CORE สำหรับ Semiempirical Methods, 
โปรแกรม Q-Chem และ Psi4 ใช้วิธี SAD กับ GWH เป็นวิธีเริ่มต้นโดยอัตโนมัติ เป็นต้น

หลังจากสร้าง Coefficient Matrix ขั้นตอนต่อไปคือการสร้าง Fock Matrix $F$ โดยใช้สมการ \ref{eq:matrix_fock} 
หลังจากนั้นเราจะทำการแก้สมการลักษณะเฉพาะ (Secular Equation) สมการที่ \ref{eq:scf_secular} เพื่อหา Energy Matrix 
แล้วก็ทำการวนซ้ำขั้นตอนการสร้าง $S$ กับ $F$ ไปปรับหาค่าพลังงานไปเรื่อย ๆ จนกว่าค่าความคลาดเคลื่อนหรือ Error จะมีค่าน้อยกว่าค่าที่กำหนดไว้ 
(Threshold) แล้วจึงสิ้นสุดกระบวนการ SCF เมื่อค่าพลังงานนั้นลู่เข้า

%--------------------------
\subsection{การคำนวณอนุพันธ์ของพลังงานและเมทริกซ์เฮสเซียน}
\label{ssec:ener_der}
\idxboth{อนุพันธ์ของพลังงาน}{Energy Derivative}
\idxboth{เมทริกซ์เฮสเซียน}{Hessian Matrix}
%--------------------------

หลังจากที่เราสามารถหาพลังงานเชิงอิเล็กทรอนิกส์ (Electronic Energy) ได้แล้ว ลำดับถัดไปที่เราสามารถคำนวณได้ก็คือคุณสมบัติต่าง ๆ ของโมเลกุล
สิ่งแรกที่เราทำได้และถือว่าสำคัญมาก ๆ ในงานวิจัยทางด้านเคมีควอนตัมก็คือการหาโครงสร้างที่เหมาะสมหรือเสถียรที่สุดของโมเลกุลโดยใช้หลักเกณฑ์%
พลังงานรวมที่ต่ำที่สุด ซึ่งการที่เราทราบโครงสร้างที่เหมาะสมที่สุดนั้นมีประโยชน์อย่างมากเพราะเราสามารถนำผลการคำนวณไปเทียบกับผลจากการทดลอง%
ด้วยเทคนิค X-ray Crystallography, Electron Diffractiom, หรือ Microwave Spectroscopy เป็นต้น โดยการหาโครงสร้างที่สภาวะ%
เหมาะสมหรือสมดุล (Equilibrium Structure) นั้นสามารถทำได้โดยหาอนุพันธ์ของพลังงานศักย์ของโมเลกุลเทียบกับพิกัดนิวเคลียร์ ซึ่งวิธีการที่%
เราสามารถนำมาหาอนุพันธ์เพื่อให้ได้ผลลัพเชิงวิเคราะห์ (Analytical Method) เรียกว่า Gradient Method ซึ่งเร็วและให้ผลลัพธ์ที่แม่นยำกว่า%
ระเบียบวิธีเชิงตัวเลข (Numerical Method)

สำหรับอนุพันธ์ของพลังงานนั้นเราจะเริ่มต้นพิจารณากรณีแบบง่ายก่อนนั่นก็คือโมเลกุลที่มีอะตอมสองอะตอม โดยเราจะเขียนพลังงานศักย์ของโมเลกุลเป็น 
$E$ ซึ่งจะมีเทอมที่เป็นแรงผลักระหว่างนิวเคลียสของทั้งสองอะตอมด้วย ซึ่งแรงผลักนี้จะขึ้นกับระยะห่างระหว่างนิวเคลียส (Internuclear Distance) 
หรือ $R$ นอกจากนี้เรายังทราบอีกด้วยว่าสำหรับโครงสร้างที่อยู่ในสมดุลนั้น แรง (Force) ที่กระทำต่อนิวเคลียสโดยอิเล็กตรอนนั้นจะเท่ากับศูนย์ 
ซึ่งแรงดังกล่าวเป็นแรงย่อยมีนิยามคืออนุพันธ์อันดับที่หนึ่งของพลังงานศักย์เทียบกับพิกัดของนิวเคลียสที่ $i$

\begin{align}
    f_{i} &= - \pdv{E}{q_{i}} \\
    &= 0
\end{align}

โดยการคำนวณหาอนุพันธ์ข้างต้นด้วยวิธีการวิเคราะห์หรือ Analytical Method นั้นเราจะต้องทำการคำนวณหาอนุพันธ์ของอินทิกรัลของอิเล็กตรอน%
หนึ่งตัวและอิเล็กตรอนสองตัว (One-electron กับ Two-electron Integrals) เทียบกับพิกัดนิวเคลียร์ นั่นคือเราจะต้องทำการหาอนุพันธ์ของ 
Basis Function นั่นเอง\footnote{Basis Function ก็คือ Basis ที่เกิดขึ้นมาจาก Atomic Orbtials ที่ถูก centered หรือมีตำแหน่ง%
อยู่ที่จุดอ้างอิงของนิวเคลียสของอะตอมในโมเลกุล} ซึ่งเราสามารถทำได้ผ่านการใช้กฎลูกโซ่ (Chain Rule) โดยทำการหาอนุพันธ์ของพลังงานศักย์%
เทียบกับ Expansion Coefficient

ลำดับถัดมาคือการหาเมทริกซ์เฮสเซียน (Hessian Matrix) ซึ่งสามารถทำได้โดยการหาอนุพันธ์ย่อยอันดับที่สองของพลังงานศักย์เทียบกับนิวเคลียส%
ของอะตอมตัวที่ $i$ และ $j$ $(\pdv{E}{q_{i}}{q_{j}})$ ซึ่งช่วยให้เราสามารถระบุได้ว่าค่าพลังงานที่คำนวณออกมาได้นั้นสอดคล้องกับ%
จุดต่ำสุดหรือสูงสุดบนพื้นผิวพลังงานศักย์ (Potential Energy Surface หรือ PES) โดยจะสอดคล้องกับอนุพันธ์อันดับที่สองที่ได้ค่าออกมาเป็นบวก 
(สำหรับ Minimum Point) และลบ (สำหรับ Maximum Point) ตามลำดับ

%--------------------------
\subsection{จากอนุพันธ์ของพลังงานสู่คุณสมบัติเชิงโมเลกุล}
\label{ssec:ener_der_mol_prop}
\idxboth{อนุพันธ์ของพลังงาน}{Energy Derivative}
\idxboth{คุณสมบัติเชิงโมเลกุล}{Molecular Properties}
%--------------------------

คุณสมบัติเชิงโมเลกุลที่เกี่ยวข้องโครงสร้างเชิงอิเล็กทรอนิกส์นั้นเป็นสิ่งที่สำคัญและจำเป็นมากในการคำนวณทางด้านเคมีควอนตัม เพราะว่าคุณสมบัติหรือ%
ปริมาณเหล่านี้เป็นสิ่งที่เรานำไปใช้ในการศึกษาโมเลกุลและปฏิกิริยาเคมี แล้วเราสามารถนำผลการคำนวณไปเปรียบเทียบกับค่าที่วัดได้จากการทดลองเพื่อ%
ตรวจสอบและยืนยันความถูกต้องของทฤษฎีที่ใช้ในการคำนวณคุณสมบัตินั้น ๆ ด้วย ตามที่ได้อธิบายไปก่อนหน้านี้ว่าอนุพันธ์ของพลังงานนั้นเปรียบเสมือน%
เป็นกุญแจที่สามารถนำไปไขกล่องที่เก็บซ่อนความลับของโมเลกุลได้ โดยเราสามารถแบ่งความสำคัญของคุณบัติเชิงโมเลกุลออกได้เป็น 3 ประเภท ดังนี้

\begin{enumerate}[topsep=0pt]
    \item ความแตกต่างของพลังงาน (Energy Differences) เช่น พลังงานของปฏิกิริยา (Reaction Energies), พลังงานในการทำให้%
    กลายเป็นอะตอม (Atomization Energies), พลังงานที่ใช้ในการสลายโมเลกุล (Dissociation Energies), พลังงานที่แตกต่างกัน%
    ระหว่างคอนฟอร์เมอร์หรือไอโซเมอร์

    \item คุณสมบัติเชิงโมเลกุลสำหรับสถานะเชิงอิเล็กทรอนิกส์ เช่น โครงสร้าง ณ สภาวะสมดุล (Equilibrium Structure), ไดโพลโมเมนต์ 
    (Dipole Moment), ความสามารถในการมีสภาพขั้ว (Polarizability), ความถี่เชิงการสั่น (Vibrational Frequencies), 
    ความสามารถในการมีสภาพแม่เหล็ก (Magnetazibility), NMR Chemical Shifts

    \item คุณสมบัติที่บ่งบอกการทรานซิชั่นระหว่างสถานะเชิงอิเล็กทรอนิกส์ที่แตกต่างกันได้ เช่น พลังงานกระตุ้นเชิงอิเล็กทรอนิกส์ (Electronic 
    Excitation Energies), ความเข้มของการทรานซิชั่นของโฟตอน 1 ตัวและ 2 ตัว (One- and two-photon Transition Strengths), 
    ระยะเวลาชีวิตในการแผ่รังสี (Radiative Life Times), พลังงานศักย์ในการทำให้เกิดไอออน (Ionization Potentials) 
\end{enumerate}

โดยในหัวข้อนี้เราจะสนใจคุณสมบัติเชิงโมเลกุลประเภทที่ 2 ซึ่งเกี่ยวกับสถานะเชิงอิเล็กทรอนิกส์เป็นพิเศษ โดยต้องเกริ่นก่อนว่าคุณสมบัติเชิงโมเลกุลนั้น%
เกิดขึ้นจากการที่โมเลกุลมีการตอบสนอง (Response) ต่อสนาม (Feid) ที่กระทำต่อโมเลกุลซึ่งมองได้ในรูปของอนุพันธ์อันดับต่าง ๆ ของพลังงาน 
เช่น อนุพันธ์สามอันดับแรก ดังนี้

\begin{itemize}[topsep=0pt]
    \item อนุพันธ์อันดับหนึ่ง: แรง (Force), ความเครียด (Stress), Dipole Moment เป็นต้น
    \item อนุพันธ์อันดับสอง: Dielectric Susceptibility, Polarizability, Born Effective Charges เป็นต้น
    \item อนุพันธ์อันดับสาม: Nonlinear Dielectric Susceptibility, (First-order) Hyperpolarizability เป็นต้น  
\end{itemize}

โดยเราสามารถเขียนพลังงานที่อยู่ภายใต้สนามภายนอก (External Field) ในรูปของฟังก์ชันการกระจายของเทเลอร์ (Taylor Expansion) 
รอบ ๆ ตำแหน่งที่ไม่มีสนาม (Field-free) ได้ดังนี้

\begin{equation}\label{eq:ener_taylor}
    E(\epsilon) = E(\epsilon = 0) 
    + \underbrace{\evaluated{\dv{E}{\epsilon}}_{\epsilon = 0} \epsilon}_{%
    \text{First Response}}
    + \underbrace{\frac{1}{2} \evaluated{\dv[2]{E}{\epsilon}}_{\epsilon = 0} \epsilon^{2}}_{%
    \text{Second Response}}
    + \dots
\end{equation}

\noindent สำหรับเทอมที่สองที่เป็นอนุพันธ์อันดับสองของพลังงานนั้นคือ Gradient ซึ่งเป็นฟังก์ชันแบบเส้นตรงสำหรับ (Linear) เราจึงเรียกคุณสมบัติ%
ที่ได้จากเทอมนี้ว่า Linear Response Properties ส่วนเทอมอื่น ๆ เช่น เทอมที่สามนั้นเป็นอนุพันธ์อันดับสองซึ่งจะเกี่ยวข้องกับฟังก์ชัน Quadratic  
นอกจากนี้เรายังสามารถสรุปได้แบบนี้ว่า

\begin{framed}
    \begin{align*}
        &\text{Dipole Moment}\,(\mu) &&= &&-\evaluated{\dv{E}{\epsilon}}_{\epsilon = 0} \\
        &\text{Polarizability}\,(\alpha) &&= &&-\evaluated{\dv[2]{E}{\epsilon}}_{\epsilon = 0} \\
        &\text{First Hyperpolarizability}\,(\beta) &&= &&-\evaluated{\dv[3]{E}{\epsilon}}_{\epsilon = 0}
    \end{align*}
\end{framed}

โดยเราสามารถนำคำนวณคุณสมบัติเชิงโมเลกุลต่าง ๆ เหล่านี้ด้วยวิธีการคำนวณทั่วไป (เช่น ใช้วิธี HF หรือ DFT) เพื่อนำไปใช้เป็น Feature 
สำหรับการฝึกสอนโมเดล ML หรือนำมาใช้เป็นเอาต์พุตสำหรับการทำนายก็ได้

%--------------------------
\section{ทฤษฎีฟังก์ชันนอลความหนาแน่น}
\label{sec:dft}
\idxboth{ทฤษฎีฟังก์ชันนอลความหนาแน่น}{Density Functional Theory}
%--------------------------

ตามที่ผู้เขียนได้พูดถึง \textit{\enquote{ทฤษฎีฟังก์ชันนอลความหนาแน่น}} หรือ \textit{\enquote{Density Functional Theory 
(DFT)}} ในบทก่อนหน้านี้แล้วว่าเป็นทฤษฎีที่มีความสำคัญมากในวงการวิทยาศาสตร์นั่นก็เพราะว่าทฤษฎี DFT ได้พลิกโฉมงานวิจัยที่เกี่ยวข้องกับการ%
ศึกษาโครงสร้างเชิงอิเล็กทรอนิกส์ของโมเลกุลไปอย่างสิ้นเชิง แทนที่จะมองโมเลกุลเป็นระบบที่อิเล็กตรอนมีอันตรกิริยากันตรง ๆ แล้วใช้ 
Wavefunction ในการอธิบายระบบนั้น วิธี DFT จะมองโมเลกุลว่าเป็นกลุ่มก้อนของอิเล็กตรอนและใช้ความหนาแน่นในการอธิบายแทน จึงทำให้เรา%
ไม่จำเป็นที่จะต้องแก้สมการเพื่อหาผลเฉลยแบบแม่นตรง (Exact Solution) ของ Wavefunction (ซึ่งเราไม่สามารถคำนวณหาบางเทอมของ 
Wavefunction ได้สำหรับกรณีที่ระบบมีอิเล็กตรอนมากกว่าหนึ่งตัว)

%--------------------------
\subsection{ฟังก์ชันและฟังก์ชันนอล}
\label{ssec:function_functional}
\idxboth{ฟังก์ชัน}{Function}
\idxboth{ฟังก์ชันนอล}{Functional}
%--------------------------

ก่อนที่ผู้อ่านจะได้ศึกษาในหัวข้อต่อไปซึ่งจะลงรายละเอียดมากกว่านี้ ผู้เขียนขอเริ่มด้วยการอธิบายความหมายและการใช้งานของสิ่งที่เรียกว่าฟังก์ชันและ%
ฟังก์ชันนอลก่อนครับ เพราะว่าการที่เราเข้าใจความหมายและความแตกต่างของคำศัพท์สองคำนี้จะเป็นพื้นฐานสำคัญในการเข้าใจทฤษฎี DFT ที่ว่าด้วย%
เรื่องของความหนาแน่นของอิเล็กตรอนที่ผู้อ่านจะได้ศึกษาในหัวข้อที่ \ref{ssec:elec_density} โดยฟังก์ชันกับฟังก์ชันนอลนั้นต่างกันตรงที่อินพุต 
ดังนี้

\begin{description}
    \item[$\bullet$ ฟังก์ชัน (Function)] รับอินพุตที่เป็นตัวเลขและให้เอาต์พุตที่เป็นตัวเลขเช่นเดียวกัน โดยสามารถเขียนการ Mapping 
    ได้เป็น $x_0 \mapsto f(x_0)$ โดยที่ $x_{0}$ คืออาร์กิวเมนต์หรืออินพุตของฟังก์ชัน $f$ 
    
    ตัวอย่างของฟังก์ชัน เช่น
    \begin{align*}
        f(x) &= x^{2} \\
        g(x,y) &= \cos(x) + e^{-3\sqrt{x^{2} + y^{2}}}
    \end{align*}

    \item[$\bullet$ ฟังก์ชันนอล (Functional)] เป็นฟังชันก์ชนิดหนึ่งซึ่งรับอินพุตที่เป็นฟังก์ชันและให้เอาต์พุตที่เป็นตัวเลข ซึ่งสรุปได้ง่าย ๆ 
    ว่า \enquote{ฟังก์ชันนอลนั้นก็คือฟังก์ชันของฟังก์ชัน} โดยสามารถเขียนการ Mapping ได้เป็น $f \mapsto f(x_0)$ โดยที่ $x_{0}$ 
    คือพารามิเตอร์ 
    
    ตัวอย่างของฟังก์ชันนอล เช่น 
    \begin{align*}
        F[f] &= \int^{\infty}_{-\infty} f^{3}(x) \, dx \\
        H[g] &= \int^{3}_{2} \int^{4}_{-10} \left ( \pdv[2]{g(x,y)}{x} - 2.3 g(x,y) \right ) \, dx dy
    \end{align*}
\end{description}

เมื่อทราบความแตกต่างแล้วผู้อ่านก็น่าจะพอเดาออกแล้วว่าคำว่า \enquote{Functional} ในชื่อของทฤษฎี Density Functional Theory 
นั้นหมายถึงว่าเป็นทฤษฎีที่ขึ้นอยู่กับฟังก์ชันที่สามารถอธิบายความหนาแน่นของอิเล็กตรอนได้

%--------------------------
\subsection{จากฟังก์ชันคลื่นสู่ความหนาแน่นเชิงอิเล็กทรอนิกส์}
\label{ssec:elec_density}
\idxboth{ความหนาแน่นเชิงอิเล็กทรอนิกส์}{Electronic Density}
%--------------------------

ในการพิจารณาฟังก์ชันคลื่นของอิเล็กตรอนนั้นเราไม่สามารถพิจารณาแค่พิกัดหรือตำแหน่งของอิเล็กตรอนเชิงพื้นที่ (Spatial Coordinates) หรือ
$(x, y, z)$ แต่ยังต้องพิจารณาตำแหน่งของสปิน (Spin Coordinates) หรือ $\omega$ ด้วย ดังนั้นจำนวนตัวแปรที่ส่งผลต่อ Wavefunction 
มีทั้งหมด 4 ตัวแปรต่อหนึ่งอิเล็กตรอน ถ้าหากระบบของเรามี $N$ อิเล็กตรอน จำนวนตัวแปรของ Wavefunction ก็จะเท่ากับ 
$4 N_{\text{electrons}}$

เพื่อทำให้ชีวิตง่ายขึ้น แนวคิดในการใช้ความหนาแน่น (Density) สำหรับศึกษาโครงสร้างเชิงอิเล็กทรอนิกส์ของโมเลกุลแทนที่จะใช้ Wavefunction 
โดยตรงนั้นจึงได้รับความสนใจและถูกพัฒนาเรื่อยมาจนถึงปัจจุบัน ข้อดีของการอธิบายระบบ (โมเลกุล) ด้วยความหนาแน่นนั้นง่ายกว่า Wavefunction 
มาก เพราะความหนาแน่นนั้นสามารถถูกเขียนให้เป็นฟังก์ชันที่ขึ้นอยู่กับตัวแปรเพียงแค่ 3 ตัวเท่านั้น (สำหรับกรณีที่ไม่พิจารณาสปินของอิเล็กตรอน) 
กล่าวคือสำหรับ Wavefunction ที่เขียนด้วย Schr\"{o}dinger Equation นั้นจะเป็นฟังก์ชันที่มีจำนวนมิติคือ $3N_{\text{electron}}$  
แต่สำหรับความหนาแน่นนั้นเราจะได้สมการที่มีจำนวนมิติคือ 3 มิติด้วยกันทั้งหมดจำนวน $N$ สมการ (ตามจำนวนอิเล็กตรอน) ซึ่งจะเห็นได้ว่าความ%
ซับซ้อนในการคำนวณจะต่างกันอย่างมาก โดยสรุปเป็นความสัมพันธ์ได้ดังนี้

\begin{framed}
    \centering
    $3N$-dimensional Schr\"{o}dinger Equation \\
    $\Psi(\bm{r}_{1}, \bm{r}_{2}, \bm{r}_{3}, \dots, \bm{r}_{N})$ \\
    $\downarrow$ \\ 
    $N$ $3$-dimensional Single Particle Equation \\
    $n({\bm{r}})$
\end{framed}

\noindent โดยที่ Single Particle Equation ในที่นี้คือสมการที่ใช้ในการอธิบายอนุภาค 1 ตัวซึ่งก็คืออิเล็กตรอนนั่นเอง

จริง ๆ แล้วความหนาแน่นเชิงอิเล็กทรอนิกส์ก็คือความหนาแน่นของอิเล็กตรอน (Electron Density) หรือ $n(\bm{r})$ ซึ่งเป็นหัวใจสำคัญของ%
ทฤษฎี DFT เลยก็ว่าได้ โดยความน่าจะเป็นของโอกาสที่จะพบอิเล็กตรอนตัวที่ 1 ของระบบหรือโมเลกุลที่มี $N$ อิเล็กตรอนนั้นสามารถคำนวณได้%
จากการใช้ปริพันธ์ (Integral) ตามสมการต่อไปนี้

\begin{equation}\label{eq:elec_prob_density_1e}
    P(\bm{1}) = \left [ \int {d}^{3} \bm{r}_{2} \cdots \int {d}^{3} \bm{r}_{N} \, 
                \psi^*(\bm{r}_{1}, \bm{r}_{2}, \dots, \bm{r}_{N}) \psi(\bm{r}_{1}, \bm{r}_2, 
                \dots, \bm{r}_N) \right ] {d}^{3} \bm{r}_{1}
\end{equation}

เนื่องจากว่าอิเล็กตรอนทุก ๆ ตัวนั้นมีคุณสมบัติเหมือนกันหมดทุกประการเรา (Indistinguishable) ดังนั้นความหนาแน่นของความน่าจะเป็น 
(Probability Density) ของอิเล็กตรอนแต่ละตัวก็จะเท่ากันด้วย หมายความว่าความน่าจะเป็นของความหนาแน่นของอิเล็กตรอนตัวที่ 1 ก็เท่ากับ%
ของตัวที่ 2, ตัวที่ 3, ไปจนถึงตัวที่ $N$ ดังนั้นในการคำนวณหาความหนาแน่นของความน่าจะเป็น (Probability Density) ของอิเล็กตรอน%
ทั้งหมดนั้นจึงสามารถทำได้โดยการรวมแบบเชิงเส้น นั่นก็คือเราคูณความน่าจะเป็นของโอกาสที่จะพบอิเล็กตรอน 1 ตัวด้วย $N$ ดังนี้
\idxboth{ความหนาแน่นของอิเล็กตรอน}{Electron Density}
\idxboth{ความหนาแน่นของความน่าจะเป็น}{Probability Density}

\begin{equation}\label{eq:elec_density_all}
    n(\bm{r}) = N \underbrace{\int {d}^{3} \bm{r}_{2} \cdots \int {d}^{3} \bm{r}_{N} \, 
                \psi^*(\bm{r}, \bm{r}_{2}, \dots, \bm{r}_{N}) \psi(\bm{r}, \bm{r}_2, 
                \dots, \bm{r}_N)}_{\textstyle \text{ความน่าจะเป็นที่จะพบอิเล็กตรอน 1 ตัว}}
\end{equation}

\noindent โดยที่ $\psi$ คือฟังก์ชันคลื่นที่ผ่านการถูกทำให้เป็นปกติ (Normalized Wavefunction) มาแล้ว ซึ่งความหมายของการทำให้เป็น%
ปกติ (Normalization) ก็คือการหารูปแบบ (Form) ของ Wavefunction เพื่อให้สอดคล้องกับเงื่อนไขดังต่อไปนี้%
\footnote{ไม่ใช่ทุก Wavefunction ที่สามารถ Normalization ได้ ตัวอย่างของฟังก์ชันที่เป็นข้อยกเว้น เช่น Planewave Wavefunction 
ซึ่งเป็นฟังก์ชันที่ขึ้นกับพิกัดและเวลา ดังนี้ $\psi(x,t) = \psi_0 {\rm e}^{ {\rm i} (k x-\omega t)}$ ไม่เป็น Square-integrable 
Function}

\begin{equation}\label{eq:square_integrable}
    \int^\infty_{-\infty} \psi^* \psi dx = 1
\end{equation}

%--------------------------
\subsection{จากความหนาแน่นเชิงอิเล็กทรอนิกส์สู่พลังงานของระบบ}
\label{ssec:ener_density}
\idxen{Density Functional Theory!Electronic Density}
%--------------------------

เมื่อเราเข้าใจนิยามและไอเดียของความหนาแน่นเชิงอิเล็กทรอนิกส์หรือความหนาแน่นของอิเล็กตรอนแล้ว ลำดับต่อไปก็คือเราจะคำนวณพลังงานของระบบ
(โมเลกุล) โดยใช้ความหนาแน่นได้อย่างไร ซึ่งตามทฤษฎีนั้นเราสามารถคำนวณพลังงานได้แบบอ้อม ๆ ผ่าน Wavefunction 

\begin{figure}[htbp]
    \centering
    \includegraphics[width=\linewidth]{fig/density_wavefunc_ener.png}
    \caption{ความเชื่อมโยงแบบตรงและแบบอ้อมระหว่างความหนาแน่นของอิเล็กตรอนและพลังงานเชิงอิเล็กทรอนิกส์ของระบบ}
    \label{fig:density_wavefunc_ener}
\end{figure}

จากไดอะแกรมที่แสดงในภาพที่ \ref{fig:density_wavefunc_ener} นั้นสามารถตีความได้ว่าเราสามารถคำนวณพลังงานของระบบโดยผ่าน 
Hamiltonian และ Wavefunction ได้ซึ่งก็จะมีความซับซ้อนในเชิงคำนวณ ดังนั้นคำถามสำคัญที่ตามมาก็คือจะเป็นไปได้ไหมที่เราจะคำนวณพลังงาน%
จากความหนาแน่นของอิเล็กตรอนตรง ๆ ซึ่งคำตอบก็คือจริง ๆ แล้วไม่สามารถหาได้ตรง ๆ แต่เรามีทริคที่สามารถทำได้ ดังนี้

เริ่มจากการกำหนดให้พลังงานเชิงอิเล็กทรอนิกส์ได้จากการคำนวณค่า Expectation Value (ค่าเฉลี่ย) ของ Hamiltonian Operator 

\begin{equation}\label{eq:ener_expect_value}
    E_{\text{el}} = \int \cdots \int \Psi^{\ast} \hat{H}_{\text{el}} \Psi \, d\bm{x}_{1} \dots \, 
    d\bm{x}_{N_{\text{el}}}
\end{equation}

\noindent โดยที่ Hamiltonian Operator สำหรับอิเล็กตรอนมีสมการดังต่อไปนี้

\begin{equation}\label{eq:hamil_one_elec}
    \hat{H}_{\text{el}} = \sum^{N_{\text{el}}}_{i=1} -\frac{1}{2} \nabla^{2}_{i} 
    + \sum^{N_{\text{el}}}_{i=1} \sum^{N_{\text{el}}}_{j=i+1} \frac{1}{|\bm{r}_{i}-\bm{r}_{j}|}
    + \underbrace{\sum^{N_{\text{el}}}_{i=1} \sum^{N_{\text{nu}}}_{A=1} \frac{-Z_{A}}{|\bm{r}_{i}-\bm{R}_{A}|}}%
    _{\text{Nuclear Attraction Energy}}
\end{equation}

\noindent ซึ่งเทอมที่ 3 ของสมการที่ \ref{eq:hamil_one_elec} นั้นคือพลังงานดึงดูดระหว่างอิเล็กตรอนกับนิวเคลียสซึ่งมีชื่อเรียกอีกชื่อว่า 
\enquote{ศักย์ภายนอก} (External Potential) โดยเป็นคำศัพท์ที่ใช้ในทฤษฎี DFT ซึ่งคำว่า External นี้มาจากการที่เราใช้การประมาณ%
ของ Born-Oppenheimer แล้วทำให้นิวเคลียสนั้นเป็นวัตถุที่ถูกตรึงอยู่กับที่ (Fixed) ซึ่งทำให้เกิดพลังงานศักย์คูลอมป์ (Coulomb Potential) 
ต่ออิเล็กตรอน ดังนั้นจากสมการที่ \ref{eq:hamil_one_elec} จะได้เป็น 
\idxboth{ศักย์ภายนอก}{External Potential}

\begin{align}\label{eq:hamil_ext_pot}
    \hat{H}_{\text{el}} &= \sum^{N_{\text{el}}}_{i=1} -\frac{1}{2} \nabla^{2}_{i} 
    + \sum^{N_{\text{el}}}_{i=1} \sum^{N_{\text{el}}}_{j=i+1} \frac{1}{|\bm{r}_{i}-\bm{r}_{j}|}
    + \sum^{N_{\text{el}}}_{i=1} 
    \underbrace{\left ( \sum^{N_{\text{nu}}}_{A=1} \frac{-Z_{A}}{|\bm{r}_{i}-\bm{R}_{A}|} \right )}%
    _{\text{Nuclear Attraction Energy}} \nonumber \\
    &= \sum^{N_{\text{el}}}_{i=1} -\frac{1}{2} \nabla^{2}_{i} 
    + \sum^{N_{\text{el}}}_{i=1} \sum^{N_{\text{el}}}_{j=i+1} \frac{1}{|\bm{r}_{i}-\bm{r}_{j}|}
    + \sum^{N_{\text{el}}}_{i=1} V_{\text{ext}}(\bm{r}_{i})
\end{align}

\noindent โดยที่มี External Potential $V_{\text{ext}}(\bm{r}_{i})$ กระทำต่ออิเล็กตรอนทุกตัวในโมเลกุล 

ลำดับต่อไปก็คือเราลองมาทำการกระจาย Expectation Value ของพลังงานเชิงอิเล็กทรอนิกส์โดยการแทนสมการที่ \ref{eq:hamil_ext_pot} 
เข้าไปในสมการที่ \ref{eq:ener_expect_value} ซึ่งเราจะได้พลังงานที่ประกอบไปด้วย 3 เทอม ดังนี้

\begin{align}\label{eq:ener_express_ext_pot}
    E_{\text{el}} =& \int \cdots \int \Psi^{\ast} 
    \left ( \sum^{N_{\text{el}}}_{i=1} -\frac{1}{2} \nabla^{2}_{i} \right ) 
    \Psi \, d\bm{x}_{1} \dots \, d\bm{x}_{N_{\text{el}}} \nonumber \\
    &+ \int \cdots \int \Psi^{\ast} 
    \left ( \sum^{N_{\text{el}}}_{i=1} \sum^{N_{\text{el}}}_{j=i+1} \frac{1}{|\bm{r}_{i}-\bm{r}_{j}|} \right ) 
    \Psi \, d\bm{x}_{1} \dots \, d\bm{x}_{N_{\text{el}}} \nonumber \\
    &+ \underbrace{\int \cdots \int \Psi^{\ast} 
    \left ( \sum^{N_{\text{el}}}_{i=1} V_{\text{ext}}(\bm{r}_{i}) \right ) 
    \Psi \, d\bm{x}_{1} \dots \, d\bm{x}_{N_{\text{el}}}%
    }_{\textstyle \int V_{\text{ext}}(\bm{r}) n(\bm{r}) \, d\bm{r}}
\end{align}

\noindent โดยสมการที่ \ref{eq:ener_express_ext_pot} มีเพียงแค่เทอมที่ 3 เท่านั้นที่สามารถเขียนให้อยู่ในรูปของฟังก์ชันนอลของ%
ความหนาแน่นได้ (Explicit Functional of Density)

%--------------------------
\subsection{ความสัมพันธ์ระหว่างความหนาแน่นของอิเล็กตรอนและศักย์ภายนอก}
\label{ssec:ener_density_ext_pot}
%--------------------------

\begin{figure}[H]
    \centering
    \frame{\includegraphics[width=\linewidth]{fig/hohenberg_kohn_abstract.png}}
    \caption{บทคัดย่อของบทความงานวิจัยที่นำเสนอทฤษฎีบท Hohenberg-Kohn ในปี ค.ศ. 1964}
    \label{fig:hohenberg_kohn_abs}
\end{figure}

สำหรับระบบที่มีจำนวนอิเล็กตรอน $N$ ตัวนั้น ศาสตราจารย์ Pierre Hohenberg (New york University) และศาสตราจารย์ Walter Kohn 
(University of California at Santa Barbara) ได้เสนอทฤษฎีบทที่เป็นรากฐานสำคัญของทฤษฎี DFT ในปี ค.ศ. 1964 นั่นก็คือทฤษฎี%
โฮเฮนเบิร์ค-โคห์น (Hohenberg-Kohn Theorem)\autocite{hohenberg1964} ซึ่งเป็นทฤษฎีที่ว่าด้วยการพิสูจน์ความสัมพันธ์ระหว่างความ%
หนาแน่นและศักย์ภายนอกว่าเป็นแบบหนึ่งต่อหนึ่ง (One-to-one) โดยใช้หลักการแปรค่า (Variational Principle) โดยบทความงานวิจัย%
ฉบับนี้ถือว่ามีความสำคัญอย่างมากต่อวงการวิทยาศาสตร์โดยเฉพาะสาขาฟิสิกส์และเคมีเชิงโมเลกุล

\begin{framed}
    \centering
    \begin{align*}
        &n^{(1)}(\bm{r}) &\underset{\text{H-K}}{\rightleftarrows} &&V_{\text{ext}^{(1)}}(\bm{r}) \\[0.5ex]
        &n^{(2)}(\bm{r}) &\underset{\text{H-K}}{\rightleftarrows} &&V_{\text{ext}^{(2)}}(\bm{r}) \\[0.5ex]
        &n^{(3)}(\bm{r}) &\underset{\text{H-K}}{\rightleftarrows} &&V_{\text{ext}^{(3)}}(\bm{r}) \\[0.5ex]
        &\cdots & &&\cdots 
    \end{align*}
\end{framed}

นอกจากนี้แล้วเรา Hohenberh และ Kohn ยังได้นำเสนอพลังงานเชิงอิเล็กทรอนิกส์ที่เขียนให้อยู่ในรูปของฟังก์ชันทั่วไป ดังนี้

\begin{align}\label{eq:ener_univer_ext_pot}
    E_{\text{el}} =& \underbrace{\int \Psi^{\ast} 
    \left ( \sum^{N_{\text{el}}}_{i=1} -\frac{1}{2} \nabla^{2}_{i} \right ) 
    \Psi \, d\bm{X} 
    + \int \Psi^{\ast} 
    \left ( \sum^{N_{\text{el}}}_{i=1} \sum^{N_{\text{el}}}_{j=i+1} \frac{1}{|\bm{r}_{i}-\bm{r}_{j}|} \right ) 
    \Psi \, d\bm{X}}_{\textstyle F_{\text{H-K}}[n]} \nonumber \\
    &+ \underbrace{\int \Psi^{\ast}  
    \left ( \sum^{N_{\text{el}}}_{i=1} V_{\text{ext}}(\bm{r}_{i}) \right ) 
    \Psi \, d\bm{X}%
    }_{\textstyle \int V_{\text{ext}}(\bm{r}) n(\bm{r}) \, d\bm{r}} \\
    =& E_{\text{el}}[n]
\end{align}

\noindent โดยผลรวมของสองเทอมแรกนั้นคือ \enquote{ฟังก์ชันนอลสากล (Universal Functional)} หรือ $F_{\text{H-K}}[n]$ 
ซึ่งไม่ขึ้นกับศักย์ภายนอก อย่างไรก็ตาม ปัญหาก็คือเราไม่ทราบหน้าตาหรือผลเฉลยแบบแม่นตรงของฟังก์ชันนอลสากลแต่ว่าเรายังคงต้องการฟังก์ชัน%
นอลนี้สำหรับการคำนวณพลังงาน ซึ่งสิ่งที่เราทำได้ก็คือการหาฟังก์ชันสากลโดยใช้วิธีการประมาณนั่นเอง
\idxboth{ฟังก์ชันนอลสากล}{Universal Functional}

%--------------------------
\subsection{ฟังก์ชันนอลสากลและทฤษฎีฟังก์ชันนอลความหนาแน่นแบบไร้ออร์บิทัล}
\label{ssec:univer_functional}
%--------------------------

สำหรับพลังงานเชิงอิเล็กทรอนิกส์ที่สามารถเขียนได้จากองค์ประกอบ 3 ส่วนคือ

\begin{equation}\label{eq:ener_elec_simplified}
    E_{\text{el}}[n] = E_{\text{kin}}[n] + E_{\text{pot}}[n] + E_{\text{ext}}[n]
\end{equation}
\idxboth{พลังงานเชิงอิเล็กทรอนิกส์}{Electronic Energy}

\noindent เราสามารถเขียนเทอมที่ 2 ของฟังก์ชัน \ref{eq:ener_elec_simplified} ให้อยู่ในรูปของผลรวมของพลังงานศักย์คูลอมป์ 
(Coulomb Energy) หรือ $E_{\text{Col}}[n]$ และพลังงานของอินตรกิริยะระหว่างอิเล็กตรอนซึ่งก็คือพลังงานแลกเปลี่ยน (Exchange Energy) 
หรือ $E_{\text{X}}[n]$ และพลังงานสหสัมพันธ์ (Correlation Energy) หรือ $E_{\text{C}}[n]$ ได้ดังนี้
\idxboth{พลังงานศักย์คูลอมป์}{Coulomb Energy}
\idxboth{พลังงานแลกเปลี่ยน}{Exchange Energy}
\idxboth{พลังงานสหสัมพันธ์}{Correlation Energy}

\begin{equation}\label{eq:ener_elec_full}
    E_{\text{el}}[n] = E_{\text{kin}}[n] + \underbrace{E_{\text{Col}}[n] + E_{\text{X}}[n] + E_{\text{C}}[n]}%
    _{\textstyle E_{\text{pot}}[n]} + E_{\text{ext}}[n]
\end{equation}

\noindent ซึ่งเทอมที่ 2 ที่เป็นพลังงานศักย์คูลอมป์กับเทอมที่ 5 ที่เป็นศักย์ภายนอกนั้นเรารู้สมการของผลเฉลยแบบแม่นตรง ดังนี้

\begin{multline}\label{eq:ener_elec_full_exact}
    E_{\text{el}}[n] = \overbrace{E_{\text{kin}}[n]}^{\textstyle \text{ไม่รู้}} 
    + \overbrace{\frac{1}{2} \int \int \frac{n(\bm{r})n(\bm{r'})}{|\bm{r}-\bm{r'}|} \, d\bm{r} d\bm{r'}}^{%
    \textstyle \text{รู้}} + \overbrace{E_{\text{X}}[n]}^{\textstyle \text{ไม่รู้}} \\ 
    + \underbrace{E_{\text{C}}[n]}_{\textstyle \text{ไม่รู้}} 
    + \underbrace{\int V_{\text{ext}}(\bm{r}) n(\bm{r}) \, d\bm{r}}_{\textstyle \text{รู้}}
\end{multline}

\noindent ส่วนเทอมที่ 1 (พลังงานจลน์), เทอมที่ 3 (พลังงานแลกเปลี่ยน), และเทอมที่ 4 (พลังงานสหสัมพันธ์) นั้นเราไม่รู้สมการที่แน่นอน% 
ซึ่งเป็นสิ่งที่ต้องประมาณค่าเอง และการประมาณค่าเพื่อหาฟังก์ชันของพลังงานทั้ง 3 เทอมนี้ที่แม่นยำที่สุดเท่าที่จะเป็นไปได้ก็เป็นหนึ่งในงานวิจัยที่ได้รับ%
ความสนใจจนถึงปัจจุบัน\autocite{peverati2014} เรียกได้ว่าตั้งแต่อดีตจนปัจจุบันได้มี Exchange-Correlation Functional ที่ถูกพัฒนา%
ขึ้นมาและถูกทดสอบหลายร้อย Functional\autocite{ernzerhof1999,dev2012,peverati2012,zhang2013,kanungo2019} สำหรับการ%
ศึกษาคุณสมบัติประเภทต่าง ๆ ของอะตอมและโมเลกุล\autocite{han2018,sharma2018,borlido2019,fabiano2019,cardeynaels2020,%
deoliveira2021,moldabekov2022}

วิธีข้างต้นที่คำนวณพลังงานเชิงอิเล็กทรอนิกส์โดยผ่านฟังก์ชันสากล (Universal Functional) นั้นจะเรียกว่าฟังก์ชันนอลความหนาแน่นแบบบริสุทธิ์ 
(Pure DFT) ก็ได้เพราะว่าไม่มีการพิจารณาออร์บิทัล (Orbital-free) ซึ่งเป็นการคำนวณพลังงานของระบบที่อิเล็กตรอนมีอันตรกิริยาต่อกัน 
(Interacting Electrons) ด้วยฟังก์ชันนอลของความหนาแน่น\autocite{ligneres2005} ข้อดีของวิธี Orbital-free DFT คือมีความ%
เรียบง่ายและไม่ซับซ้อนมากนัก (Simplicity) แต่ข้อด้อยก็คือมีความแม่นยำในการคำนวณที่ต่ำมากนั่นก็เพราะว่าการประมาณค่าของเทอมพลังงานจลน์ 
(เทอมแรกของสมการที่ \ref{eq:ener_elec_full_exact}) นั้นทำได้ยากมากและขาดความแม่นยำในการประมาณ (เพราะว่าเทอม 
$\frac{1}{\bm{r}_{i} - \bm{r}_{j}}$ นั้นไม่สามารถถูกแยกแบ่งออกเป็นผลรวมของ $\bm{r}_{i}$ และ $\bm{r}_{j}$ ได้) 
เมื่อเราไม่สามารถประมาณค่าพลังงานจลน์ได้อย่างแม่นยำจึงทำให้พลังงานเชิงอิเล็กทรอนิกส์ที่คำนวณออกมานั้นมีความแม่นยำต่ำตามไปด้วย 

\begin{figure}[H]
    \centering
    \frame{\includegraphics[width=\linewidth]{fig/kohn_sham_abstract.png}}
    \caption{บทคัดย่อของบทความงานวิจัยที่นำเสนอทฤษฎีบท Kohn-Sham ในปี ค.ศ. 1965}
    \label{fig:kohn_sham_abs}
\end{figure}

สำหรับการแก้ปัญหาดังกล่าวนั้น ในปี ค.ศ. 1965 ศาสตราจารย์ Walter Kohn และศาสตราจารย์ Lu Jeu Sham (University of California, 
San Diego) ก็ได้นำเสนอบทความงานวิจัย (หนึ่งปีหลังจากนำเสนอทฤษฎี Pure (Orbital-free) DFT) โดยได้เสนอการคำนวณฟังก์ชัน%
ของพลังงานโดยใช้ระบบที่อิเล็กตรอนไม่มีอันตรกิริยาต่อกัน (Non-interacting Electrons) แทนการแก้ผ่านระบบที่อิเล็กตรอนมีอันตรกิริยาต่อกัน%
\autocite{kohn1965} ซึ่งมีข้อดีคือทำให้ DFT มีความแม่นยำมากขึ้นเพราะว่าพลังงานจลน์ของระบบที่อิเล็กตรอนแต่ละตัวไม่ขึ้นหรือมีความสัมพันธ์%
กับอิเล็กตรอนตัวอื่น ๆ นั้นมีสมการที่เรารู้หน้าตาแน่นอน จึงไม่มีความจำเป็นที่จะต้องประมาณค่าฟังก์ชันนอลของพลังงานจลน์ในรูปของความหนาแน่น%
อีกต่อไป โดยในเวลาต่อมาทฤษฎีนี้คือ Kohn-Sham DFT นั่นเอง
\idxen{Density Functional Theory!Kohn-Sham Theorem}

%--------------------------
\subsection{จาก Hohenberg-Kohn สู่ Kohn-Sham}
\label{ssec:from_hk_to_ks}
%--------------------------

ในหัวข้อนี้เราจะมารู้จักกับความแตกต่างระหว่างระบบที่อิเล็กตรอนมีอันตรกิริยาต่อกัน (Interacting Electrons) และไม่มีอันตรกิริยาต่อกัน 
(Non-interacting Electrons) กันให้มากขึ้น เพราะว่าเป็นระบบที่ถูกนำมาใช้ในการพิจารณาความหนาแน่นของโมเลกุล

ตามที่ Kohn กับ Sham ได้เสนอการแก้ปัญหาของ Pure DFT โดยการเปลี่ยนมาพิจารณาระบบที่อิเล็กตรอนไม่มีอันตรกิริยาต่อกันแทนนั้น จริง ๆ 
แล้ว Wavefunction และความหนาแน่นของทั้งสองระบบนั้นแตกต่างกันอย่างสิ้นเชิง แต่ว่าทริคของวิธี Kohn-Sham นั้นคือทำการจำลองหรือสร้าง%
ระบบอิเล็กตรอนที่ไม่มีอันตรกิริยาต่อกันแบบปลอม ๆ ขึ้นมา (Fictitious Non-interacting Electron System) หรืออาจจะเรียกว่าระบบ%
อิเล็กตรอนแบบเสริมก็ได้ (Auxiliary Non-interacting Electron System)\autocite{martin2020} โดยบังคับให้ความหนาแน่นของ%
ระบบนี้มีค่าเท่ากันกับความหนาแน่นของระบบที่อิเล็กตรอนมีอันตรกิริยาต่อกัน ดังนั้นความท้าทายจึงเปลี่ยนจากการหาฟังก์ชันสากลสำหรับ Pure DFT 
มาเป็นการหาระบบที่อิเล็กตรอนไม่มีอันตรกิริยากันแบบปลอม ๆ ที่มีความหนาแน่นเท่ากัน ซึ่งการใช้ทริคนี้ทำให้เราไม่ต้องมาประมาณค่าพลังงานจลน์%
และทำให้การคำนวณ DFT นั้นมีความแม่นยำมากขึ้นเพราะว่าเรามีผลเฉลยที่แน่นอนของพลังงานตามที่ได้อธิบายไว้ก่อนหน้านี้ในย่อหน้าสุดท้ายของ%
หัวข้อที่ \ref{ssec:univer_functional}
\idxboth{ระบบอิเล็กตรอน}{Electron System}
\idxboth{ระบบอิเล็กตรอนที่ไม่มีอันตรกิริยาต่อกัน}{Non-interacting Electron System}
\idxboth{ระบบอิเล็กตรอน!แบบเสริม}{Auxiliary}

\begin{figure}[htbp]
    \centering
    \includegraphics[width=\linewidth]{fig/electron_system.png}
    \caption{การเปรียบเทียบแบบจำลองของ (1) ระบบที่อิเล็กตรอนมีอันตรกิริยาต่อกัน, (2) ระบบที่อิเล็กตรอนไม่มีอันตรกิริยาต่อกัน, และ 
    (3) ระบบที่อิเล็กตรอนไม่มีอันตรกิริยาต่อกันของโมเลกุลปลอมตามทฤษฎี Kohn-Sham}
    \label{fig:electron_system}
\end{figure}

เพื่อให้ผู้อ่านเข้าใจได้ง่ายขึ้นว่าทำไมระบบที่อิเล็กตรอนไม่มีอันตรกิริยาต่อกันของ Kohn-Sham นั้นถึงมีความหนาแน่นเท่ากันกับระบบที่อิเล็กตรอนมี%
อันตรกิริยาต่อกัน ให้ผู้อ่านเริ่มด้วยการศึกษาภาพที่ \ref{fig:electron_system} ซึ่งเป็นการเปรียบเทียบระหว่างโมเดลของอิเล็กตรอนที่แตกต่างกัน

\begin{itemize}[topsep=0pt]
    \item ระบบที่ 1 คือระบบที่อิเล็กตรอนมีอันตรกิริยาต่อกัน ซึ่งเราสามารถหาผลเฉลยแบบแม่นตรงของ Wavefunction ของระบบนี้ได้ 
    และความหนาแน่นของโมเดลนี้จะเท่ากับความหนาแน่นของโมเลกุลจริงด้วย

    \item ระบบที่ 2 ระบบที่อิเล็กตรอนไม่มีอันตรกิริยาต่อกัน หมายความว่าอิเล็กตรอนแต่ละตัวจะมี Hamiltonian Operator เป็นของตัวเอง 
    ซึงอิเล็กตรอนแต่ละตัวจะวิ่งอยู่ภายในสนามของศักย์เฉลี่ย (Average Potential) ที่เกิดจากอิเล็กตรอนตัวอื่นในระบบ สำหรับระบบนี้เราจะทำ%
    การรวม Hamiltonian ของอิเล็กตรอนแต่ละตัวเข้าด้วยกันเพื่อประมาณค่า Wavefunction สำหรับอิเล็กตรอนทุกตัว

    \item ระบบที่ 3 จะคล้ายกับระบบที่ 2 แต่จะมีความแตกต่างกันที่ศักย์เฉลี่ยที่กระทำต่ออิเล็กตรอน กล่าวคือในระบบนี้ (เรียกว่าระบบอิเล็กตรอน 
    ของ Kohn-Sham ก็ได้) ศักย์เฉลี่ยที่เกิดขึ้นจะมาจากระบบของอิเล็กตรอนแบบปลอม ๆ (Fictitious System of Electrons) โดยเรา%
    จะทำการรวม Wavefunction ของอิเล็กตรอน (Molecular Orbitals) เข้าด้วยกันเพื่อประมาณค่า Wavefunction สำหรับอิเล็กตรอนทุกตัว
    ซึ่งความหนาแน่นของระบบ Kohn-Sham นี้จะมีค่าเท่ากับความหนาแน่นของระบบที่เป็นโมเลกุลจริง ๆ นั่นคือเท่ากับความหนาแน่นของระบบที่ 1 
    ด้วย
\end{itemize}

%--------------------------
\subsection{แฮมิลโทเนียนสำหรับอิเล็กตรอนที่ไม่มีอันตรกิริยาต่อกัน}
\label{ssec:hamil_noninter_elec}
%--------------------------

การที่เราเปลี่ยนมาพิจารณาระบบที่อิเล็กตรอนไม่มีอันตรกิริยาต่อกันแทนนั้น เราสามารถเปลี่ยนเทอมที่เป็นพลังงานที่เกิดจากการผลักกันของอิเล็กตรอน 
(Direct Interaction) ให้เป็นโอเปอเรเตอร์สำหรับอิเล็กตรอน 1 ตัวได้ (เพราะว่าอิเล็กตรอนทุกตัวเป็นอิสระและไม่ขึ้นต่อกันอีกต่อไปแล้ว) 
ซึ่งโอเปอเรเตอร์ที่ว่านั้นคือพลังงานศักย์ที่อธิบายค่าเฉลี่ยของผลที่เกิดจากอันตรกิริยาระหว่างอิเล็กตรอน (Average Effect) อธิบายง่าย ๆ คือ%
เราใช้เทอมนี้เป็นตัวแทนของอันตรกิริยาระหว่างอิเล็กตรอนในระบบที่อิเล็กตรอนไม่มีอันตรกิริยาต่อกัน (เป็นเทอมที่เป็นส่วนเติมเต็ม) โดยเราสามารถ%
พิสูจน์แฮมิลโทเนียน (Hamiltonian) ของ Effective Interaction สำหรับอิเล็กตรอนที่ไม่มีอันตรกิริยาได้จาก Hamiltonian แบบดั้งเดิม 
ดังต่อไปนี้
\idxen{Density Functional Theory!Effective Interaction}

\begin{align}\label{eq:hamil_inter_elec}
    \hat{H}_{\text{el}} &= \sum^{N_{\text{el}}}_{i=1} -\frac{1}{2} \nabla^{2}_{i} 
    + \sum^{N_{\text{el}}}_{i=1} \sum^{N_{\text{el}}}_{j=i+1} \frac{1}{|\bm{r}_{i}-\bm{r}_{j}|}
    + \sum^{N_{\text{el}}}_{i=1} V_{\text{ext}}(\bm{r}_{i})
\end{align}

\noindent สมการนี้จะเปลี่ยนเป็นสมการสำหรับ Effective Interaction

\begin{align}\label{eq:hamil_noninter_eff_full}
    \hat{H}_{\text{eff}} &= \sum^{N_{\text{el}}}_{i=1} -\frac{1}{2} \nabla^{2}_{i} 
    + \sum^{N_{\text{el}}}_{i=1} V_{\text{aver}}(\bm{r}_{i})
    + \sum^{N_{\text{el}}}_{i=1} V_{\text{ext}}(\bm{r}_{i}) \nonumber \\
    &= \sum^{N_{\text{el}}}_{i=1} -\frac{1}{2} \nabla^{2}_{i} 
    + \sum^{N_{\text{el}}}_{i=1} \{ V_{\text{aver}}(\bm{r}_{i}) + V_{\text{ext}}(\bm{r}_{i}) \} \nonumber \\
    &= \sum^{N_{\text{el}}}_{i=1} -\frac{1}{2} \nabla^{2}_{i} 
    + \sum^{N_{\text{el}}}_{i=1} V_{\text{eff}}(\bm{r}_{i}) \nonumber \\
    &= \sum^{N_{\text{el}}}_{i=1} \{ -\frac{1}{2} \nabla^{2}_{i} + V_{\text{eff}}(\bm{r}_{i}) \} \nonumber \\
    &= \sum^{N_{\text{el}}}_{i=1} \hat{h}(\bm{r}_{i})
\end{align}

จากการพิสูจน์ข้างต้นจะได้ว่าสุดท้ายแล้ว Hamiltonian ทั้งหมดของระบบที่อิเล็กตรอนไม่มีอันตรกิริยาต่อกันนั้นคือผลรวมของ Hamiltonian ของ%
อิเล็กตรอนหนึ่งตัว (1 Hamiltonian ต่ออิเล็กตรอน 1 ตัว) กล่าวคือจากการพิสูจน์ของสมการที่ \ref{eq:hamil_noninter_eff_full} 
เราจะได้ความสัมพันธ์ที่สั้นและกระชับ ดังนี้

\begin{equation}\label{eq:hamil_noninter_eff}
    \hat{H}_{\text{eff}} = \sum^{N_{\text{el}}}_{i=1} \hat{h}(\bm{r}_{i})
\end{equation}

นอกจากนี้เราสามารถเขียนและแก้ Schr\"{o}dinger Equation สำหรับ Hamiltonian ของอิเล็กตรอนแต่ละตัวแยกกันได้ดังนี้

\begin{equation}\label{eq:hamil_one_elec_mo}
    \hat{h}(\bm{r}_{i}) \underbrace{\psi_{a}(\bm{r}_{i})}_{\text{MOs}} = 
    \underbrace{\epsilon_{a}}_{\text{Energy}} \psi_{a}(\bm{r}_{i})
\end{equation}

\noindent ซึ่งจากตรงนี้เราสามารถคำนวณหาออร์บิทัลเชิงโมเลกุล (Molecular Orbitals หรือ MOs) และพลังงานที่สอดคล้องกันได้

%--------------------------
\subsection{ออร์บิทัลเชิงโมเลกุลและผลคูณฮาร์ทรี}
\label{ssec:mol_orb_hartree_prod}
\idxboth{ออร์บิทัลเชิงโมเลกุล}{Molecular Orbital}
%--------------------------

เนื่องจากว่า Kohn-Sham DFT นั้นอ้างอิงอยู่กับออร์บิทัลเชิงโมเลกุล (Molecular Orbital หรือ MO) เราจึงควรที่จะเข้าใจเกี่ยวกับ MO ด้วย ซึ่ง 
MO นั้นจริง ๆ แล้วคือฟังก์ชันคลื่นของอิเล็กตรอนหนึ่งตัว (One-electron Wavefunction) โดย MO ที่ประกอบไปด้วยพิกัดเชิงพื้นที่ (Spatial 
Coordinates) และพิกัดเชิงสปิน (Spin Coordinates) นั้นจะมีชื่อเรียกว่าออร์บิทัลเชิงสปิน (Spin Orbital) ซึ่งเป็นผลคูณของทั้ง 
Spatial Function และ Spin Function ตามสมการดังต่อไปนี้
\idxboth{ออร์บิทัลเชิงโมเลกุล}{Molecular Orbital}
\idxboth{พิกัดเชิงพื้นที่}{Spatial Coordinates}
\idxboth{พิกัดเชิงสปิน}{Spin Coordinates}
\idxboth{ออร์บิทัลเชิงสปิน}{Spin Orbital}

\noindent สปินขึ้น (Up Spin):

\begin{equation}\label{eq:hamil_spin_orb_up}
    \underbrace{\chi^{\uparrow}(\bm{x})}_{\text{Spin Orbital}} = 
    \underbrace{\psi(\bm{r})}_{\text{Spatial}} \underbrace{\alpha(\omega)}_{\text{Spin}}
\end{equation}

\noindent สปินลง (Down Spin)

\begin{equation}\label{eq:hamil_spin_orb_down}
    \underbrace{\chi^{\downarrow}(\bm{x})}_{\text{Spin Orbital}} = 
    \underbrace{\psi(\bm{r})}_{\text{Spatial}} \underbrace{\beta(\omega)}_{\text{Spin}}
\end{equation}

\noindent ถึงแม้ว่า Hamilnotian สำหรับอิเล็กตรอนหนึ่งตัว (สมการที่ \ref{eq:hamil_one_elec_mo}) เป็นฟังก์ชันที่ขึ้นอยู่กับเพียงแค่
Spatial Coordinates เท่านั้น เราก็สามารถใช้ Spin Orbitals ซึ่งก็เป็น Eigenfunction ได้เช่นกัน ดังนั้นจากสมการที่ 
\ref{eq:hamil_one_elec_mo} เราจะได้ Schr\"{o}dinger Equation ที่ใช้ Spin Orbitals สำหรับอิเล็กตรอนตัวที่ $i$ ดังนี้

\begin{equation}\label{eq:hamil_spin_orb}
    \hat{h}(\bm{r}_{i}) \underbrace{\chi_{a}(\bm{r}_{i})}_{\text{MOs}} = 
    \underbrace{\epsilon_{a}}_{\text{Energy}} \chi_{a}(\bm{r}_{i})
\end{equation}

คราวนี้มาดูตัวอย่างง่าย ๆ นั่นคือระบบที่มีอิเล็กตรอนสองตัวที่ไม่มีอัตรกิริยาต่อกัน (Two Non-interacting Electrons) โดยเราสามารถเขียน 
Wavefunction สำหรับทั้งสองอิเล็กตรอนได้ดังนี้

\begin{align}\label{eq:hamil_spin_orb_two_elec}
    \hat{h}(\bm{r}_{1}) \underbrace{\chi_{a}(\bm{r}_{1})}_{\text{MOs}} &= 
    \underbrace{\epsilon_{a}}_{\text{Energy}} \chi_{a}(\bm{r}_{1}) \\
    \hat{h}(\bm{r}_{2}) \underbrace{\chi_{a}(\bm{r}_{2})}_{\text{MOs}} &= 
    \underbrace{\epsilon_{a}}_{\text{Energy}} \chi_{a}(\bm{r}_{2})
\end{align}

ถ้าเป็นกรณีหลายวัตถุ (Many-body) แบบที่มีอิเล็กตรอน $N_{\text{el}}$ ตัวในระบบที่อิเล็กตรอนไม่มีอัตรกิริยาต่อกัน เราสามารถเขียนผล%
เฉลยแบบแม่นตรง (Exact Solution) ของ Schr\"{o}dinger Equation ได้ดังนี้

\begin{multline}\label{eq:schro_eq_spin_orb_n_elec}
    \left [ \sum_{i=1}^{N_{\text{el}}} \hat{h}(\bm{r}_{i}) \right ] \chi_{a}(\bm{x}_{1}) \chi_{b}(\bm{x}_{2}) 
    \cdots \chi_{z}(\bm{x}_{N_{el}}) = \\
    (\epsilon_{a} + \epsilon_{b} \dots \epsilon_{z}) \chi_{a}(\bm{x}_{1}) \chi_{b}(\bm{x}_{2}) \cdots 
    \chi_{z}(\bm{x}_{N_{\text{el}}})
\end{multline}

\noindent ซึ่งเราสรุปได้ว่า \textbf{\textit{พลังงานรวม (Eigenvalue) ของระบบหรือโมเลกุลของเรานั้นจริง ๆ แล้วมีค่าเท่ากับผลรวม%
ของพลังงานของ Spin Orbital ของอิเล็กตรอนแต่ละตัวรวมกัน}}

ประเด็นต่อมาก็คือตามสมการที่ \ref{eq:schro_eq_spin_orb_n_elec} นั้น เราสามารถเขียนผลคูณของ Spin Orbitals แต่ละตัวให้เป็น
Wavefunction ของระบบที่อิเล็กตรอนเป็นอิสระต่อกันได้ ดังนี้

\begin{equation}\label{eq:hartree_product}
    \Psi_{\text{eff}} = \chi_{a}(\bm{x}_{1}) \chi_{b}(\bm{x}_{2}) \cdots \chi_{z}(\bm{x}_{N_{\text{el}}})
\end{equation}

\noindent โดยเราเรียก Wavefunction ในสมการที่ \ref{eq:hartree_product} นี้ว่า \enquote{ผลคูณฮาร์ทรี (Hartree Product)}
ซึ่งการเขียน Wavefunction แบบนี้ถูกต้องตามหลักคณิตศาสตร์ทุกประการ แต่ทว่าตามหลักกลศาสตร์ควอนตัม Hartree Product นั้นขัดแย้งกับ%
คุณสมบัติข้อหนึ่งของ Wavefunction นั่นก็คือ Wavefunction จะต้องมีปฏิสมมาตร (Antisymmetry) กล่าวคือถ้าเรามีการสลับตำแหน่งของ 
Spatial Coordinate หนึ่งครั้ง เครื่องหมายของ Wavefunction ก็จะเปลี่ยนไป (จากลบเป็นบวกหรือจากบวกเป็นลบ) แต่ว่าในกรณีของ 
Hartree Product นั้นเนื่องจากว่าอิเล็กตรอนทุกตัวเป็นอิสระต่อกัน จึงทำให้ Wavefunction ที่เป็นผลคูณระหว่าง Spin Orbitals นั้นไม่มี%
คุณสมบัติ Antisymmetry ดังนั้นเป้าหมายต่อไปของเราก็คือการหารูปแบบ (Form) ทางคณิตศาสตร์แบบอื่นที่สามารถอธิบาย Wavefunction 
และมีคุณสมบัติของ Antisymmetry อยู่ด้วย 
\idxboth{ผลคูณฮาร์ทรี}{Hartree Product}
\idxboth{ปฏิสมมาตร}{Antisymmetry}

%--------------------------
\subsection{จากผลคูณฮาร์ทรีสู่ดีเทอร์มิแนนต์ของสเลเตอร์}
\label{ssec:hartree_prod_to_slater_deter}
\idxboth{ออร์บิทัลเชิงโมเลกุล}{Molecular Orbital}
%--------------------------

เมื่อผลคูณฮาร์ทรี (Hartree Product) ไม่เหมาะสมที่จะถูกนำมาใช้ในการสร้าง Wavefunction ดังนั้นจึงได้มีพัฒนาสิ่งที่เรียกว่าดีเทอร์มิแนนต์ของ%
สเลเตอร์ (Slater Determinant) ขึ้นมา โดยตั้งตามนามสกุลของศาสตราจารย์ John Clarke Slater นักฟิสิกส์ชาวอเมริกา ซึ่งแนวคิดของ
Slater Determinant นั้นก็คือการใช้ผลรวมเชิงเส้น (Linear Combination) ของ Hartree Product นั่นเอง\autocite{slater1929} 
ตัวอย่างเช่นกรณีที่ระบบมีอิเล็กตรอนสองตัว เราจะได้ว่าฟังก์ชันคลื่นที่เขียนในรูปของ Linear Combination ของ Hartree Product ที่มีคุณสมบัติ 
Antisymmetry มีดังนี้

\begin{equation}\label{eq:lin_com_hartree_prod}
    \Psi(\bm{x}_{1}, \bm{x}_{2}) = \chi_{1}(\bm{x}_{1})\chi_{2}(\bm{x}_{2}) - 
    \chi_{2}(\bm{x}_{1})\chi_{1}(\bm{x}_{2})
\end{equation}

\noindent ซึ่งสามารถเขียนเป็นดีเทอร์มีแนนต์ (Determinant) ของเมทริกซ์ออร์บิทัลเชิงสปินได้ดังนี้

\begin{align}\label{eq:slater_det_2e}
    \Psi(\bm{x}_{1}, \bm{x}_{2}) &= 
    \det \begin{pmatrix}
        \chi_{1}(\bm{x}_{1}) & \chi_{2}(\bm{x}_{1}) \\
        \chi_{1}(\bm{x}_{2}) & \chi_{2}(\bm{x}_{2})
    \end{pmatrix} \nonumber \\
    &= \begin{vmatrix}
        \chi_{1}(\bm{x}_{1}) & \chi_{2}(\bm{x}_{1}) \\
        \chi_{1}(\bm{x}_{2}) & \chi_{2}(\bm{x}_{2})
    \end{vmatrix}
\end{align}

\noindent โดยเราเรียกสมการที่ \ref{eq:slater_det_2e} ที่เป็น Wavefunction นี้ว่า Slater Determinant ซึ่งสามารถที่จะทำให้มี%
รูปแบบทั่วไป (Generalized) สำหรับระบบที่มีอิเล็กตรอนกี่ตัวก็ได้ ($N$-electron System) นอกจากนี้ ตามที่ได้อธิบายไว้ก่อนหน้านี้ว่า Slater 
Determinant นั้นมีคุณสมบัติ Antisymmetry ซึ่งทำให้ Wavefunction นั้นจะมีการเปลี่ยนเครื่องหมายทุกครั้งที่เราทำการสลับแถวหรือสลับหลัก

สำหรับ Slater Determinant ของระบบที่มี $N$ อิเล็กตรอนนั้นเราสามารถกำหนดได้ดังนี้\autocite{szabo1996}

\begin{equation}\label{eq:slater_det_N_elec}
    \Psi(\bm{x}_{1}, \bm{x}_{2}, \dots, \bm{x}_{N}) = 
    \frac{1}{\sqrt{N!}}
    \begin{vmatrix}
        \chi_{1}(\bm{x}_{1}) & \chi_{2}(\bm{x}_{1}) & \cdots & \chi_{N}(\bm{x}_{1}) \\
        \chi_{1}(\bm{x}_{2}) & \chi_{2}(\bm{x}_{2}) & \cdots & \chi_{N}(\bm{x}_{2}) \\
        \vdots & \vdots & & \vdots \\
        \chi_{1}(\bm{x}_{N}) & \chi_{2}(\bm{x}_{N}) & \cdots & \chi_{N}(\bm{x}_{N})
    \end{vmatrix}
\end{equation}

\noindent โดยที่ $1/\sqrt{N!}$ นั้นคือค่าคงที่ของการทำ Normalization นอกจากนี้เรายังสามารถเขียนสมการที่ 
\ref{eq:slater_det_N_elec} ให้สั้นลงได้โดยใช้สัญกรณ์เค็ท (Ket Notation) ดังนี้

\begin{equation}\label{eq:slater_det_ket}
    \Psi(\bm{x}_{1}, \bm{x}_{2}, \cdots, \bm{x}_{N}) = 
    \ket{\chi_{1}(\bm{x}_{N}) \chi_{2}(\bm{x}_{N}) \cdots \chi_{N}(\bm{x}_{N})}
\end{equation}

\noindent หรือจะเขียนให้สั้นและกระชับกว่านี้ก็ได้ ดังนี้

\begin{equation}\label{eq:slater_det_ket_short}
    \Psi(\bm{x}_{1}, \bm{x}_{2}, \dots, \bm{x}_{N}) = 
    \ket{1, 2, \cdots, N}
\end{equation}

คุณสมบัติอีกอย่างหนึ่งที่ Slater Determinant มีนั่นก็คือประพฤติตัวตามหลักของเพาลี (Pauli Principle) นั่นคือ Spatial Orbital 
นั้นจะมีอิเล็กตรอนได้สูงสุดไม่เกิน 2 ตัว และอิเล็กตรอนทั้งสองตัวนั้นจะต้องมีสปินที่ตรงข้ามกัน\autocite{atkins2010}

ณ จุด ๆ นี้เมื่อเราสามารถใช้ Slater Determinant เป็น Wavefunction ได้แล้ว เราก็สามารถหาความหนาแน่นของอิเล็กตรอนได้เช่นกัน
โดยสำหรับระบบที่มีจำนวนอิเล็กตรอนเป็นเลขคู่ (Closed-shell) และอิเล็กตรอนทุกตัวถูกบรรจุอยู่ใน Spatial Orbitals จะได้ว่า

\begin{equation}\label{eq:density_slater}
    n(\bm{r}) = 2 \sum_{i=1}^{N_{el}/2} |\psi_{i}(\bm{r})|^{2}
\end{equation}

ดังนั้นเราจึงสรุปได้ว่า  \textbf{\textit{ความหนาแน่นของอิเล็กตรอนของฟังก์ชันคลื่นแบบ Slater Determinant นั้นมีค่าเท่ากับผลรวม%
ของยกกำลังสองของออร์บิทัลที่บรรจุอิเล็กตรอน (Occupied Orbital)}}

%--------------------------
\subsection{Orbital-free DFT ปะทะ Kohn-Sham DFT}
\label{ssec:orb_free_vs_kohn_sham_dft}
\idxen{Density Functional Theory!Orbital-free}
\idxen{Density Functional Theory!Kohn-Sham}
%--------------------------

\begin{figure}[htbp]
    \centering
    \includegraphics[width=\linewidth]{fig/orb_free_vs_kohn_sham.png}
    \caption{การเปรียบเทียบขั้นตอนการคำนวณพลังงานของระบบด้วยวิธี Orbital-free DFT และวิธี Kohn-Sham DFT}
    \label{fig:orb_free_vs_kohn_sham}
\end{figure}

ภาพที่ \ref{fig:orb_free_vs_kohn_sham} แสดงการเปรียบเทียบขั้นตอนการคำนวณพลังงานของระบบที่อิเล็กตรอนไม่มีอันตรกิริยาต่อกันด้วย%
วิธี Orbital-free DFT ซึ่งเป็นการคำนวณผ่าน Universal Function ของ Hohenberg-Kohn และวิธี Kohn-Sham DFT ซึ่งเป็นการคำนวณ%
ผ่านความหนาแน่นซึ่งอ้างอิงกับ Molecular Orbitals ผู้อ่านสามารถศึกษาโค้ดของ Kohn-Sham DFT และการเขียนนำฟังก์ชันนอลจาก LibXC 
มาใช้เพื่อคำนวณพลังงานได้ที่ซอร์สโค้ดของโปรแกรม PySCF \url{https://github.com/pyscf/pyscf/tree/master/pyscf/dft} 
ตัวอย่างเช่น โค้ดของฟังก์ชัน \pyinline{get_veff} 
\url{https://github.com/pyscf/pyscf/blob/master/pyscf/dft/uks.py#L30-L105} สำหรับ Unrestricted Kohn-Sham 
ซึ่งเป็นฟังก์ชันที่คำนวณผลรวมของพลังงานศักย์คูลอมป์ (Coulomb Energy) และพลังงานแลกเปลี่ยน-สหสัมพันธ์ (Exchange-Correlation 
Energy)

จากที่ได้อธิบายมาตั้งแต่ต้นเราสามารถสรุปเปรียบเทียบฟังก์ชันของพลังงานที่คำนวณด้วยวิธี Orbital-free DFT และวิธี Kohn-Sham DFT ได้ดังนี้

\noindent $\bullet$ \textbf{Orbital-free DFT}

\begin{equation}\label{eq:ener_orb_free}
    E[n] = E_{\text{kin}}[n] + E_{\text{Coul}}[n] + E_{\text{XC}}[n] + E_{\text{Ext}}[n]
\end{equation}

\noindent อุปสรรคในการใช้สมการที่ \ref{eq:ener_orb_free} ก็คือเราไม่มีผลเฉลยที่ถูกต้องของฟังก์ชันนอลของพลังงานจลน์

\noindent $\bullet$ \textbf{Kohn-Sham DFT}

\begin{multline}\label{eq:ener_kohn_sham}
    E[n] = E_{\text{kin,KS}}[n] + (E_{\text{kin}}[n] - E_{\text{kin,KS}}[n]) + E_{\text{Coul}}[n] \\
    + E_{\text{XC}}[n] + E_{\text{Ext}}[n]
\end{multline}

\noindent โดยที่ $E_{\text{kin,KS}}[n]$ คือพลังงานของระบบที่อิเล็กตรอนไม่มีอันตรกิริยาต่อกัน (ระบบอิเล็กตรอนของ Kohn-Sham) 
ซึ่งเรารู้ผลเฉลยแบบแม่นตรงของ $E_{\text{kin,KS}}[n]$ ถึงแม้ว่าจะอยู่ในเทอมของ Molecular Orbitals และไม่ใช่ความหนาแน่นก็ตาม
นอกจากนี้เทอมที่น่าสนใจอีกเทอมหนึ่งก็คือความแตกต่างระหว่างพลังงานจลน์จริง ๆ ของระบบกับพลังงานจลน์ของ Kohn-Sham 
$(E_{\text{kin}}[n] - E_{\text{kin,KS}}[n])$ นั้นมีค่าน้อยกว่าความคลาดเคลื่อนในประมาณค่าของ $E_{\text{kin,KS}}[n]$ ของ
Orbital-free DFT มาก ดังนั้น Kohn-Sham DFT จึงสามารถคำนวณพลังงานเชิงอิเล็กทรอนิกส์ได้แม่นยำมากกว่าวิธี Orbital-free DFT

%--------------------------
\subsection{พลังงานของ Kohn-Sham}
\label{ssec:kohn_sham_ener_expr}
\idxth{ทฤษฎีฟังก์ชันนอลความหนาแน่น!พลังงาน Kohn-Sham}
\idxen{Density Functional Theory!Kohn-Sham Energy}
%--------------------------

เพื่อไม่ให้ผู้อ่านสับสนเราสามารถสรุปสมการที่ใช้ในการคำนวณพลังงานของระบบอิเล็กตรอนของ Kohn-Sham ดังนี้

\begin{framed}
    ฟังก์ชันนอลของพลังงานรวม $=$ พลังงานจลน์ที่เป็นฟังก์ชันนอลของ MOs $+$ พลังงานเทอมอื่น ๆ ที่เหลือที่เป็นฟังก์ชันนอลของความหนาแน่น 
\end{framed}

\noindent ซึ่งเขียนได้เป็นสมการดังนี้

\begin{align}
    \label{eq:kohn_sham_ener_1}
    E_{\text{KS}}[n] &= E_{\text{kin,KS}}[n] + E_{\text{Coul}}[n] + E_{\text{Ext}}[n] + 
    \underbrace{E_{\text{XC}}[n] + (E_{\text{kin}}[n] - E_{\text{kin,KS}}[n])}_{\text{นำมารวมกันได้}} \\
    \label{eq:kohn_sham_ener_2}
    &= 2 \sum^{N_{\text{el}}/2}_{i=1} \int \psi^{\ast}_{i}(\bm{r}) \left ( -\frac{1}{2}\nabla^{2} \right ) 
    + E_{\text{Coul}}[n] + E_{\text{Ext}}[n] + {E'}_{\text{XC}}[n]
\end{align}

\noindent นั่นคือความหนาแน่นของระบบที่อิเล็กตรอนที่มีอันตรกิริยาต่อกันถูกสร้างขึ้นจากออร๊บิทัลเชิงโมเลกุลของระบบที่อิเล็กตรอนไม่มีอันตรกิริยา%
ต่อกันของ Kohn-Sham (Kohn-Sham MOs) ดังนั้นท้ายที่สุดแล้วเราสามารถสรุปได้อีกว่า \enquote{\textit{พลังงานของระบบของ Kohn-Sham 
นั้นเป็นฟังก์ชันนอลของ Kohn-Sham MOs นั่นเอง}}

ประเด็นก็คือเราจะต้องทำการประมาณค่าของเทอม ${E'}_{\text{XC}}[n]$ ซึ่งถึงแม้ว่าเทอมนี้จะมีการรวม Contribution จากพลังงานจลน์%
ของระบบเข้ามาด้วย (จากสมการที่ \ref{eq:kohn_sham_ener_1} มาเป็นสมการที่ \ref{eq:kohn_sham_ener_2}) แต่เราก็ยังเรียกเทอมนี้ว่า 
Exchange-Correlation Functional อยู่ครับ ซึ่งการหาเทอมนี้เป็นหนึ่งในหัวข้องานวิจัยที่สำคัญและมีกลุ่มวิจัยหลายกลุ่มให้ความสนใจเป็นอย่างมาก

%--------------------------
\subsection{การปรับหาค่าพลังงานให้ต่ำที่สุด}
\label{ssec:kohn_sham_ener_minimize}
\idxth{ทฤษฎีฟังก์ชันนอลความหนาแน่น!การปรับลดค่าพลังงาน Kohn-Sham}
\idxen{Density Functional Theory!Kohn-Sham Energy Minimization}
%--------------------------

เมื่อเรารู้แล้วว่าพลังงานรวมของระบบที่ถูกคำนวณด้วย Kohn-Sham DFT นั้นเป็นฟังก์ชันนอลของ MOs ของระบบที่อิเล็กตรอนไม่มีอันตรกิริยาต่อกันของ 
คำถามต่อมาที่เราจะต้องมาหาคำตอบก็คือเราจะคำนวณ Kohn-Sham MOs ได้อย่างไร ในหัวข้อก่อนหน้านี้ผู้อ่านได้ศึกษาไปแล้วว่าในการปรับลดค่าของ%
พลังงานนั้นสามารถทำได้โดยการใช้หลักการแปรค่า (Variation Principle) ซึ่งจะเป็นการปรับลดค่าพลังงานรวมโดยเทียบกับความหนาแน่นเพื่อ%
หาพลังงานของระบบ ณ สภาวะพื้นโดยเราเรียกกระบวนการนี้ว่า Energy Minimization อย่างไรก็ตามเราไม่สามารถใช้เทคนิคเดียวกันนี้กับกรณีของ%
Kohn-Sham DFT เพราะว่าพลังงานรวมของ Kohn-Sham นั้นเป็นฟังก์ชันนอลของ MOs ดังนั้นเราจะต้องทำการ Minimize พลังงานโดยเทียบกับ MOs 
แทนซึ่งความหนาแน่นนั้นก็ขึ้นอยู่กับ MOs ด้วย 

เนื่องจากว่าปัญหาการปรับลดค่าพลังงานรวมนั้นเป็นปัญหาค่าต่ำสุดที่มีหลายตัวแปรซึ่งเราสามารถใช้ตัวคูณลากรองจ์ (Langrange Multiplier) ได้
เริ่มต้นเรากำหนดชุดของ Langrange Multiplier แล้วทำการ Minimize ดังนี้

\begin{equation}\label{eq:langrange_mult}
    \Omega_{\text{KS}[n]} = E_{\text{KS}}[n] - 2 \sum^{N_{\text{el}}/2}_{i=1} \sum^{N_{\text{el}}/2}_{j=1}
    \epsilon_{ij} \left ( \int \phi^{\ast}_{i}(\bm{r}) \phi_{j}(\bm{r}) d\bm{r} - \delta_{ij} \right )
\end{equation}

\noindent ในการหาอนุพันธ์เชิงฟังก์ชันนอล (Functional Derivative) ของสมการที่ \ref{eq:langrange_mult} เทียบกับออร์บิทัลเรา%
สามารถใช้กฎลูกโซ่ได้ ดังนี้

\begin{equation}\label{eq:func_der_ener_chain_1}
    \fdv{\Omega_{\text{KS}[n]}}{\phi^{\ast}_{j}(\bm{r})} = \fdv{\Omega_{\text{KS}[n]}}{n(\bm{r})}
    \fdv{n(\bm{r})}{\phi^{\ast}_{j}(\bm{r})}
\end{equation}

\noindent เนื่องจากว่าเราทราบค่าของอนุพันธ์ของความหนาแน่นเทียบกับออร์บิทัล ดังต่อไปนี้

\begin{equation}\label{eq:func_der_den_chain}
    \fdv{n(\bm{r})}{\phi^{\ast}_{j}(\bm{r})} = 2 \phi^{\ast}_{j}(\bm{r})
\end{equation}

\noindent เมื่อเราแทนสมการ \ref{eq:func_der_den_chain} เข้าไปในสมการที่ \ref{eq:func_der_ener_chain_1} เราจะได้สมการคือ

\begin{equation}\label{eq:func_der_ener_chain_2}
    \fdv{\Omega_{\text{KS}[n]}}{\phi^{\ast}_{j}(\bm{r})} = \fdv{\Omega_{\text{KS}[n]}}{n(\bm{r})}
    \fdv{n(\bm{r})}{\phi^{\ast}_{j}(\bm{r})} = \fdv{\Omega_{\text{KS}[n]}}{n(\bm{r})} 2 \phi^{\ast}_{j}(\bm{r})
\end{equation}

\noindent นอกจากนี้เรามีเงื่อนไขเพิ่มเติมว่าอนุพันธ์เชิงฟังก์ชันนอลของพลังงานเมื่อเทียบกับออร์บิทัลนั้นจริง ๆ แล้วเท่ากับศูนย์ ดังนี้

\begin{equation}\label{eq:func_der_ener_chain_zero}
    \fdv{\Omega_{\text{KS}[n]}}{\phi^{\ast}_{j}(\bm{r})} = 0
\end{equation}

เมื่อรวมทุกอย่างเข้าด้วยกันเราจะพบว่าอนุพันธ์เชิงฟังก์ชันนอลของพลังงานนั้นสามารถเขียนกระจายได้เป็น

\begin{align*}
    \fdv{\Omega_{\text{KS}[n]}}{\phi^{\ast}_{j}(\bm{r})} 
    &= \begin{multlined}[t]
        2 \left ( -\frac{1}{2}\nabla^{2} \phi_{j}(\bm{r}) \right ) \\
        + 2 \left ( \fdv{E_{\text{Coul}}}{n} + \fdv{E_{\text{ext}}}{n} + \fdv{E_{\text{XC}}}{n} \right ) 
        \phi_{j}(\bm{r}) \\
        - 2 \sum^{N_{\text{el}}/2}_{i=1} \epsilon_{ij} \phi_{j}(\bm{r})
        \end{multlined} \\
    &= 0
\end{align*}

เมื่อถึงขั้นตอนนี้แล้วขั้นตอนต่อไปก็คือเราสามารถทำการรวมออร์บิทัล $\{\phi\}$ ทั้งหมดเข้าด้วยกัน (โดยใช้ Summation สำหรับทุกดัชนี $i$) 
ให้เป็นออร์บิทัลใหม่โดยกำหนดด้วยตัวแปร $\{\psi\}$ ซึ่งเราจะได้สมการดังต่อไปนี้

\begin{equation}
    \left ( -\frac{1}{2}\nabla^{2} \psi_{j}(\bm{r}) \right )
    + \left ( \fdv{E_{\text{Coul}}}{n} + \fdv{E_{\text{ext}}}{n} + \fdv{E_{\text{XC}}}{n} \right ) 
    \psi_{j}(\bm{r}) - \epsilon_{j} \psi_{j}(\bm{r})
    = 0
\end{equation}

\noindent เมื่อทำการจัดรูปสมการโดยย้ายเทอมที่มีพลังงานของระบบมาทางด้านขวาของสมการแล้วทำการรวมเทอมฝั่งซ้ายซึ่งเป็นโอเปอเรเตอร์ดังนี้

\begin{align}    
    \left \{ -\frac{1}{2}\nabla^{2} \left (
    + \fdv{E_{\text{Coul}}}{n} + \fdv{E_{\text{ext}}}{n} + \fdv{E_{\text{XC}}}{n} \right ) 
    \right \} \psi_{j}(\bm{r})
    = \epsilon_{j} \psi_{j}(\bm{r}) \nonumber \\
    \left ( -\frac{1}{2}\nabla^{2} + V_{\text{KS}}(\bm{r}) \right ) \psi_{j}(\bm{r})
    = \epsilon_{j} \psi_{j}(\bm{r}) \label{eq:func_der_ener_chain_full}
\end{align}

\noindent ซึ่งสมการที่ \ref{eq:func_der_ener_chain_full} นี้ก็คือ Schr\"{o}dinger Equation สำหรับหนึ่งอิเล็กตรอนนั่นเอง 
โดยเราสามารถแก้สมการเพื่อหา Kohn-Sham MOs ได้ อย่างไรก็ตาม เนื่องจากว่าเราไม่รู้ว่า MOs เริ่มต้นนั้นมีหน้าตาเป็นอย่างไร (Unknown)
เราจึงไม่สามารถหาเฉลยแม่นตรงได้ ดังนั้นเราจึงจำเป็นต้องใช้วิธี Self-consistent Field (SCF) ซึ่งเป็นวิธีวนซ้ำ (Iterative Method) 
โดนมีขั้นตอนคร่าว ๆ ดังต่อไปนี้

\begin{framed}
    \centering
    1. ทำการสร้างออร์บิทัลเชิงโมเลกุลเริ่มต้น (Initial Guess of MOs)
    \\ $\downarrow$ \\
    2. สร้าง Kohn-Sham Operator 
    \\ $\downarrow$ \\
    3. คำนวณพลังงาน 
    \\ $\downarrow$ \\
    4. เปรียบเทียบพลังงานของรอบปัจจุบันกับพลังงานของรอบที่แล้ว 
    \\ $\downarrow$ \\
    5. ถ้าความแตกต่างของพลังงานและความหนาแน่น
    \\
    \vspace{0.5em}
    \begin{minipage}{0.6\linewidth}
        \begin{itemize}
            \item $>$ ค่า Cutoff $\rightarrow$ ขั้นตอนที่ 2
            \item $<$ ค่า Cutoff $\rightarrow$ สิ้นสุดการคำนวณ
        \end{itemize} 
    \end{minipage}
\end{framed}

รายละเอียดของทฤษฎี DFT และการคำนวณด้วยคอมพิวเตอร์นั้นมีอีกเยอะมาก เช่น Basis Set, Exchange-Correlation Functional 
แบบต่าง ๆ, Spin-polarised DFT, Geometry Optimization รวมไปถึงเทคนิคต่าง ๆ ที่ได้มีการพัฒนาเพื่อปรับปรุง DFT ให้มีความแม่นยำ%
ในการคำนวณคุณสมบัติเชิงอิเล็กทรอนิกส์ของโมเลกุลมากขึ้น 

ผู้อ่านที่สนใจศึกษาเพิ่มเติมสามารถศึกษาได้จากหนังสือดังต่อไปนี้

\begin{enumerate}[topsep=0pt]
    \item Introduction to Computational Chemistry \\ ผู้เขียน Frank Jensen
    
    \item Electronic Structure - Basic Theory and Practical Methods \\ ผู้เขียน Richard M. Martin
    
    \item Density Functional Theory - An Advanced Course \\ ผู้เขียน Eberhard Engel และ Reiner M. Dreizler
\end{enumerate}

%--------------------------
\section{การคำนวณพลังงานของระบบอิเล็กตรอนด้วย Kohn-Sham DFT}
\label{sec:calc_ener_kohn_sham}
\idxth{ทฤษฎีฟังก์ชันนอลความหนาแน่น!การคำนวณพลังงานของระบบอิเล็กตรอน Kohn-Sham}
\idxen{Density Functional Theory!Kohn-Sham Energy Calculation}
%--------------------------

ในหัวข้อก่อนหน้านี้เป็นการอธิบายทฤษฎีของ Kohn-Sham DFT ซึ่งมีเคล็ดลับคือกำหนดให้อิเล็กตรอนในระบบนั้นไม่มีอันตรกิริยาต่อกัน เพื่อให้ผู้อ่าน%
เห็นภาพมากขึ้น ในหัวข้อนี้ผู้อ่านจะได้ศึกษาการเขียนโปรแกรมสำหรับการคำนวณพลังงานของระบบหลายอิเล็กตรอนโดยใช้ Kohn-Sham DFT โดย%
เราจะสนใจกรณีที่เป็น 1 มิติเท่านั้น (อิเล็กตรอนมีการเคลื่อนที่ตามแกน $x$ เพียงอย่างเดียว)

ในการเขียนโค้ดของ Kohn-Sham (KS) DFT นั้นเราจะใช้ใช้ Hamiltonian ตามที่เราได้ศึกษาไปแล้วตามสมการที่ \ref{eq:kohn_sham_ener_1} 
โดยเราสามารถเขียนให้สั้นและกระชับมากขึ้นได้ ดังนี้

\begin{equation}
    \hat{H} = -\frac{1}{2} \frac{d^2}{dx^2} + v_{Coul}(x) + v_{LDA}(x) + v_{ext}
\end{equation}

\noindent โดยแต่ละเทอมก็คือ

\begin{enumerate}[topsep=0pt,noitemsep]
    \item พลังงานจลน์ (Kinetic Energy)
    \item พลังงานศักย์คูลอมป์ (Coulomb Energy) หรือแรงผลักไฟฟ้าสถิตย์ระหว่างอิเล็กตรอน
    \item พลังงานแลกเปลี่ยน (Exchange Energy) ซึ่งเราจะใช้การประมาณค่าความหนาแน่นแบบพื้นที่ (Local Density Approximation) 
    \item พลังงานภายนอก (External Potential) ซึ่งเราจะใช้ฟังก์ชัน Harmonic Oscillator
\end{enumerate}

\noindent หมายเหตุ: เราจะไม่พิจารณา Correlation Energy เนื่องจากว่ามีความซับซ้อนมากเกิน

โดยเราจะใช้ภาษา Python ในการเขียน โดยสิ่งที่เราต้องทำหลัก ๆ มีดังนี้

\begin{enumerate}[topsep=0pt,noitemsep]
    \item สร้าง Hamiltonian
    \item คำนวณฟังก์ชันคลื่นของ Kohn-Sham (KS Wavefunction)
    \item คำนวณความหนาแน่น (Density)
    \item คำนวณพลังงานอิเล็กทรอนิกส์
\end{enumerate}

\noindent เมื่อพร้อมแล้วเราก็มาเริ่มกันได้เลย

\noindent \textbf{1. นำเข้าไลบรารี่}

\begin{lstlisting}[style=MyPython]
import numpy as np
import matplotlib.pyplot as plt
import seaborn as sns
\end{lstlisting}

\vspace{1em}
\noindent \textbf{2. กำหนดโอเปอเรเตอร์เชิงอนุพันธ์สำหรับการสร้าง Hamiltonian ของพลังงานจลน์}

\begin{lstlisting}[style=MyPython]
# Define a real-space grid
n_grid = 200
x = np.linspace(-5, 5, n_grid)
y = np.sin(x)

# First derivative
h = x[1]-x[0]
D = -np.eye(n_grid) + np.diagflat(np.ones(n_grid-1),1)
D = D / h

# Second derivative
D2 = D.dot(-D.T)
D2[-1,-1] = D2[0,0]
\end{lstlisting}

\vspace{1em}
\noindent \textbf{3. คำนวณพลังงานจลน์}

แก้สมการ Kohn-Sham เฉพาะของพลังงานจลน์โดยการทำ Diagonalization (เป็นขั้นตอนที่กำหนด Computational Complexity ของ DFT 
นั่นคือ $\mathcal{O}(n^{3})$

\begin{lstlisting}[style=MyPython]
# Solve Kohn-Sham equation
eig_non, psi_non = np.linalg.eigh(-D2/2)
\end{lstlisting}

\vspace{1em}
\noindent \textbf{4. คำนวณพลังงานศักย์ภายนอก}

ลำดับต่อไปคือการพิจารณาศักย์ภายนอก (External Potential) ซึ่งเราสามารถใช้ฟังก์ชัน Harmonic Oscillator ง่าย ๆ ได้ ในตัวอย่างนี้%
ผู้เขียนเลือกใช้ External Potential เป็นฟังก์ชันพหุนาม คือ $v_{ext}=x^2$:

\begin{lstlisting}[style=MyPython]
# Define external potential with a matrix
X = np.diagflat(x*x)

# Solve Kohn-Sham equation
eig_harm, psi_harm = np.linalg.eigh(-D2/2+X)
\end{lstlisting}

\vspace{1em}
\noindent \textbf{5. คำนวณพลังงานแลกเปลี่ยน}

ลำดับต่อมาคือการคำนวณพลังงานแลกเปลี่ยน (Exchange Energy) โดยเราจะพิจารณาฟังก์ชันนอลแลกเปลี่ยน (Exchange Functional) โดยใช้ 
Local Density Approximation (LDA) ซึ่งมีสมการดังต่อไปนี้ (จริง ๆ แล้ว LDA มี Correlation Functional ด้วยแต่ว่าเราจะไม่สนใจ)

\begin{equation}
    E_X^{LDA}[n] = -\frac{3}{4} \left(\frac{3}{\pi}\right)^{1/3} \int n^{4/3} dx
\end{equation}

\noindent โดยที่ Potential นั้นสามารถคำนวณได้จากอนุพันธ์ของ Exchange Energy เทียบกับความหนาแน่น

\begin{align}
    v_X^{LDA}[n] &= \frac{\partial E_X^{LDA}}{\partial n} \nonumber \\ 
    &= - \left(\frac{3}{\pi}\right)^{1/3} n^{1/3}
\end{align}

\vspace{1em}
\begin{lstlisting}[style=MyPython]
def get_exchange(nx, x):
    energy = -3./4.*(3./np.pi)**(1./3.)*integral(x, nx**(4./3.))
    potential = -(3./np.pi)**(1./3.)*nx**(1./3.)
    return energy, potential
\end{lstlisting}

\vspace{1em}
\noindent \textbf{6. คำนวณพลังงานคูลอมป์}

ลำดับต่อมาคือพลังงานคูลอมป์ซึ่งเป็นพลังงานทางไฟฟ้าสถิตย์ (Electrostatic Energy) หรืออาจจะเรียกเรียกว่าพลังงานฮาร์ทรี Hartree Energy 
ก็ได้ อย่างไรก็ตาม ตามทฤษฎีนั้นพลังงานคูลอมป์สำหรับนั้นลู่เข้า (Converged) เฉพาะกรณี 3 มิติเท่านั้นซึ่งมีสมการดังต่อไปนี้

\begin{equation}
    E^{3D}_{Coul} = \frac{1}{2}\iint \frac{n(r)n(r')}{\sqrt{(r-r')^2}}drdr'
\end{equation}

\noindent ดังนั้นในกรณี 1 มิติเราจะต้องทำการโกงนิดหน่อยเพื่อทำให้พลังงานนั้นลู่เข้าโดยการปรับสมการ ดังนี้

\begin{equation}
    E^{1D}_{Coul} = \frac{1}{2}\iint \frac{n(x)n(x')}{\sqrt{(x-x')^2+\varepsilon}}dxdx'
\end{equation}

\noindent โดยที่ $\varepsilon$ คือคงที่ที่เป็นบวกโดยมีค่าน้อย ๆ ซึ่งทำให้ฟังก์ชันนี้ลู่เข้าได้ง่ายขึ้น

\noindent ดังนั้นพลังงานศักย์จึงมีสมการดังต่อไปนี้:

\begin{equation}
    v_{Coul} = \int \frac{n(x')}{\sqrt{(x-x')^2+\varepsilon}}dx'
\end{equation}

\vspace{1em}
\begin{lstlisting}[style=MyPython]
def get_coulomb(nx, x, eps=1e-1):
    h = x[1]-x[0]
    energy = np.sum(nx[None,:]*nx[:,None]*h**2 / np.sqrt((x[None,:]-x[:,None])**2 + eps)/2)
    potential = np.sum(nx[None,:]*h/np.sqrt((x[None,:]-x[:,None])**2+eps), axis=-1)
    return energy, potential
\end{lstlisting}

\vspace{1em}
\noindent \textbf{7. คำนวณความหนาแน่น}

เนื่องจากว่าเราจะต้องทำการรวม Coulomb Energy และ LDA Exchange โดยที่ทั้งคู่นั้นเป็นฟังก์ชันนอลของความหนาแน่น ดังนั้นเราจึงจำเป็นต้อง%
คำนวณความหนาแน่นของอิเล็กตรอน (Electron Density) โดยเรามีเงื่อนไขของการทำ Normalization ดังนี้

\begin{equation}
    \int \lvert \psi \rvert ^2 dx = 1
\end{equation}

\noindent ซึ่งเราสามารถเขียนความหนาแน่นให้อยู่ในรูปของผลรวมเชิงเส้นของออร์บิทัลยกกำลังสองได้ ดังนี้ 

\begin{equation}
    n(x)=\sum_n f_n \lvert \psi(x) \rvert ^2
\end{equation}

\noindent โดยที่ $f_n$ คือ Occupation Number (จำนวนอิเล็กตรอนในออร์บิทัลที่ $n$) ซึ่งแต่ละ State นั้นจะมีอิเล็กตรอนที่มีสปินขึ้นและ%
สปินลง โดยใน DFT นั้นเราคำนวณสถานะพื้นของระบบ

\begin{lstlisting}[style=MyPython]
def integral(x, y, axis=0):
    dx = x[1]-x[0]
    return np.sum(y*dx, axis=axis)
\end{lstlisting}

\vspace{1em}
\noindent กำหนดจำนวนอิเล็กตรอน

\begin{lstlisting}[style=MyPython]
num_electron = 17
\end{lstlisting}

\vspace{1em}
\noindent ทำการคำนวณความหนาแน่น

\begin{lstlisting}[style=MyPython]
def get_nx(num_electron, psi, x):
    # Normalization
    I = integral(x, psi**2, axis=0)
    normed_psi = psi/np.sqrt(I)[None, :]
    
    # Occupation Number
    fn=[2 for _ in range(num_electron//2)]
    if num_electron % 2:
        fn.append(1)

    # Density
    res = np.zeros_like(normed_psi[:,0])
    for ne, psi in zip(fn, normed_psi.T):
        res += ne*(psi**2)

    return res
\end{lstlisting}

\vspace{1em}
\noindent \textbf{8. คำนวณพลังงานอิเล็กทรอนิกส์ของระบบ}

เมื่อเราเตรียมองค์ประกอบทุกอย่างพร้อมแล้ว ขั้นตอนต่อไปนี้สำคัญมากเพราะว่าเป็นขั้นตอนสุดท้ายที่เราจะนำฟังก์ชันทั้งหมดที่เราได้เขียนไว้มาแก้สมการ 
Kohn-Sham โดยการวนซ้ำเทียบกับตัวเอง (Self-Consistency) มีขั้นตอนดังนี้

\begin{enumerate}[topsep=0pt]
    \item เริ่มต้นด้วยการ Initialize ความหนาแน่น (เราสามารถใช้ค่าคงที่อะไรก็ได้)
    \item คำนวณพลังงานศักย์แลกเปลี่ยน (Exchange) และศักย์คูลอมป์ (Coulomb Potential)
    \item คำนวณ Hamiltonian
    \item แก้สมการ KS เพื่อคำนวณหา Wavefunctions และ Eigenvalues (พลังงาน)
    \item ตรวจสอบการลู่เข้า ถ้าไม่ลู่เข้า ให้อัพเดทความหนาแน่นและกลับไปที่ขั้นตอนที่ 2
\end{enumerate}

ก่อนอื่นให้สร้างฟังก์ชันสำหรับเก็บข้อมูลเพื่อแสดงผลในระหว่างการทำ Iteration

\begin{lstlisting}[style=MyPython]
def print_log(i,log):
    print(f"step: {i:<5} energy: {log['energy'][-1]:<10.4f} energy_diff: {log['energy_diff'][-1]:.10f}")
\end{lstlisting}

\vspace{1em}

กำหนดพารามิเตอร์เพิ่มเติม เช่น จำนวนรอบสูงสุดในการวนซ้ำและค่า Cutoff ของความแตกต่างระหว่างพลังงานจากรอบที่ $n$ และรอบที่ $n+1$

\begin{lstlisting}[style=MyPython]
max_iter = 1000
energy_tolerance = 1e-5
log={"energy":[float("inf")], "energy_diff":[float("inf")]}
\end{lstlisting}

\vspace{1em}

กำหนดค่าความหนาแน่นเริ่มต้นซึ่งจะถูกมานำใช้เป็นค่าเริ่มต้นในการประมาณค่าหาความหนาแน่นโดยการทำปรับค่าเทียบค่าความหนาแน่นที่ได้จากลูป%
ในรอบก่อนหน้า โดยค่าความหนาแน่นเริ่มต้นนั้นเราจะกำหนดโดยใช้ค่าคงที่อะไรก็ได้ ในตัวอย่างนี้ผู้เขียนใช้ความหนาแน่นเท่ากับ 0 และสิ่งที่เกิดขึ้น%
ภายในลูปนั้นเราจะทำการคำนวณพลังงาน Exchange และพลังงาน Coulomb ก่อนแล้วก็สร้าง Hamiltonian ขึ้นมาแล้วก็ทำการ Diagonalize 
Hamiltonian เพื่อให้ได้ Eigenvalue ออกมาซึ่งนั่นก็คือพลังงานของเรานั่นเอน หลังจากนั้นเราจะทำการเก็บค่าพลังงานที่ได้แล้วก็ตรวจสอบว่า%
ส่วนต่างของพลังงานที่ได้จากการวนลูปรอบปัจจุบันที่ $n$ กับรอบที่ $n-1$ นั้นต่ำกว่าค่า Cutoff แล้วหรือยัง ถ้าหากว่ายังก็ให้ทำการอัพเดทค่า%
ความหนาแน่นแล้วทำการคำนวณพลังงานอีกรอบ

\begin{lstlisting}[style=MyPython]
# Initialize density
nx = np.zeros(n_grid)

for i in range(max_iter):
    ex_energy, ex_potential = get_exchange(nx, x)
    ha_energy, ha_potential = get_coulomb(nx, x)
    
    # Hamiltonian
    H = -D2/2 + np.diagflat(ex_potential + ha_potential + x*x)
    
    energy, psi= np.linalg.eigh(H)
    
    # Collect energy and eenrgy difference
    log["energy"].append(energy[0])
    energy_diff = energy[0] - log["energy"][-2]
    log["energy_diff"].append(energy_diff)
    print_log(i, log)
    
    # Check if the calculation is converged
    if abs(energy_diff) < energy_tolerance:
        print("Converged!   :)")
        break
    
    # Update the density
    nx = get_nx(num_electron, psi, x)
else:
    print("Not Converged   :(")
\end{lstlisting}

\vspace{1em}

\noindent เมื่อทำการรันโค้ดด้านบนแล้วจะได้เอาต์พุตดังต่อไปนี้

\begin{lstlisting}[style=MyPython]
step: 0     energy: 0.7069     energy_diff: -inf
step: 1     energy: 16.3625    energy_diff: 15.6555321919
step: 2     energy: 13.8021    energy_diff: -2.5603559494
step: 3     energy: 15.3002    energy_diff: 1.4980525863
step: 4     energy: 14.4119    energy_diff: -0.8882287680
step: 5     energy: 14.9470    energy_diff: 0.5350438262
step: 6     energy: 14.6242    energy_diff: -0.3228271880
step: 7     energy: 14.8201    energy_diff: 0.1959328656
step: 8     energy: 14.7011    energy_diff: -0.1190355457
step: 9     energy: 14.7735    energy_diff: 0.0724651058
step: 10    energy: 14.7294    energy_diff: -0.0441312736
step: 11    energy: 14.7563    energy_diff: 0.0268946713
step: 12    energy: 14.7399    energy_diff: -0.0163922405
step: 13    energy: 14.7499    energy_diff: 0.0099933983
step: 14    energy: 14.7438    energy_diff: -0.0060926001
step: 15    energy: 14.7475    energy_diff: 0.0037147279
step: 16    energy: 14.7452    energy_diff: -0.0022649307
step: 17    energy: 14.7466    energy_diff: 0.0013810031
step: 18    energy: 14.7458    energy_diff: -0.0008420446
step: 19    energy: 14.7463    energy_diff: 0.0005134280
step: 20    energy: 14.7460    energy_diff: -0.0003130574
step: 21    energy: 14.7462    energy_diff: 0.0001908842
step: 22    energy: 14.7461    energy_diff: -0.0001163900
step: 23    energy: 14.7461    energy_diff: 0.0000709679
step: 24    energy: 14.7461    energy_diff: -0.0000432721
step: 25    energy: 14.7461    energy_diff: 0.0000263849
step: 26    energy: 14.7461    energy_diff: -0.0000160880
step: 27    energy: 14.7461    energy_diff: 0.0000098095
Converged!   :)
\end{lstlisting}

\vspace{1em}

เมื่อทำการแก้หาค่าพลังงานไปทั้งหมด 27 รอบจะพบว่าพลังงานนั้นลู่เข้า โดยค่าพลังงานสุดท้ายที่ได้คือ 14.7461 และมีค่าความแตกต่างระหว่างพลังงาน%
ของรอบที่ 26 กับพลังงานของรอบที่ 27 เท่ากับ 0.0000098095 ซึ่งน้อยกว่า Cutoff ที่กำหนดไว้คือ 0.00001

นอกจากนี้เราสามารถพลอต Wavefunction ซึ่งเป็นฟังก์ชันของ Real-space Grid และระบุพลังงานได้ด้วย ดังนี้

\begin{lstlisting}[style=MyPython]
for i in range(5):
    plt.plot(x, psi[:,i], label=f"{energy[i]:.4f}")
    plt.legend(loc=1)
\end{lstlisting}

\vspace{1em}

\begin{figure}[H]
    \centering
    \includegraphics[width=0.9\linewidth]{fig/ks_dft_1d_wfn_ener.png}
    \caption{Wavefunction และพลังงานที่ได้จากการคำนวณ Kohn-Sham DFT สำหรับกรณี 1 มิติ}
    \label{fig:ks_dft_1d_wfn_ener}
\end{figure}

ผู้อ่านที่ต้องการศึกษาโค้ดฉบับสมบูรณ์สามารถดูได้ที่ไฟล์ \inlinehighlight{6_1D_DFT.ipynb} ใน Code Repository ของหนังสือที่ 
\url{https://github.com/rangsimanketkaew/ml-qm-book-code}
