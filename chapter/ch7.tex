% LaTeX source for ``การเรียนรู้ของเครื่องสำหรับเคมีควอนตัม (Machine Learning for Quantum Chemistry)''
% Copyright (c) 2022 รังสิมันต์ เกษแก้ว (Rangsiman Ketkaew).

% License: Creative Commons Attribution-NonCommercial-NoDerivatives 4.0 International (CC BY-NC-ND 4.0)
% https://creativecommons.org/licenses/by-nc-nd/4.0/

\chapter{คุณสมบัติเชิงอิเล็กทรอนิกส์ของโมเลกุล}
\label{ch:el_prop}

\begin{figure}[htbp]
    \centering
    \includegraphics[width=0.9\linewidth]{fig/mol_properties.png}
    \caption{สนามเวกเตอร์ของโมเลกุลเมื่ออยู่ในสนามแม่เหล็ก}
    \label{fig:mol_prop}
\end{figure}

ส่วนที่สองของหนังสือเล่มนี้จะเกี่ยวข้องกับเคมีควอนตัมเป็นหลัก เคมีควอนตัมเป็นพื้นฐานสำคัญของการพัฒนาเทคนิคสำหรับการวิเคราะห์คุณสมบัติของโมเลกุล
โดยนักเคมีนั้นส่วนใหญ่แล้วจะใช้เทคนิคทางสเปกโทรสโกปี เช่น Infrared (IR) Spectroscopy, Nuclear Magnetic Resonance (NMR) 
Spectroscopy, และ Scanning Probe Microscopy ซึ่งเทคนิคเหล่านี้ล้วนเกี่ยวข้องกับการคำนวณหาพลังงานในระดับโมเลกุล นอกจากนี้แล้ว%
เคมีควอนตัมยังเกี่ยวข้องกับการศึกษาสถานะพื้น (Ground State) และสถานะกระตุ้น (Excited State) ของอะตอมแต่ละตัว รวมไปถึงการศึกษา%
กลไกการเกิดปฏิริยาเคมีและสถานะทรานซิชั่น (Transition State) ที่เป็นสถานะที่เกิดขึ้นในการเปลี่ยนแปลงโครงสร้างของโมเลกุล การที่เรา%
เข้าใจองค์ความรู้ขั้นพื้นฐานในระดับอะตอมและโมเลกุลนั้นทำให้เราประยุกต์ใช้และนำไปสู่การศึกษาคุณสมบัติของโมเลกุลในระดับที่ใหญ่ขึ้นได้ เช่น 
เทอร์โมไดนามิกส์ (Thermodynamics) และจลนศาสตร์เชิงเคมี (Chemical Kinetics) ซึ่งนำไปสู่การพัฒนาแบบจำลองทางคณิตศาสตร์ต่าง ๆ 
ซึ่งสามารถใช้คอมพิวเตอร์ในการศึกษาระบบที่เราสนใจได้ก่อนที่จะไปศึกษาจริงในห้องทดลอง

เนื้อหาในบทนี้จะเป็นการอธิบายทฤษฎีของคุณสมบัติเชิงอิเล็กทรอนิกส์ (Electronic Properties) ของโมเลกุลซึ่งผู้เขียนคิดว่าเป็นสิ่งที่สำคัญมาก 
นั่นก็เพราะว่าถ้าหากเราต้องการที่จะเชื่อมโยง ML และเคมีควอนตัมเข้าด้วยกันเราควรจะต้องเข้าใจถึงทฤษฎีของคุณสมบัติของโมเลกุลที่เราต้องการที่%
จะศึกษาเสียก่อน เมื่อเราเข้าใจทฤษฎีแล้วก็จะทำให้เกิดไอเดียที่เราสามารถประยุกต์ใช้ ML ได้อย่างถูกต้อง

โมเลกุลเป็นหน่วยพื้นฐานของสิ่งต่าง ๆ รอบตัวเรา โมเลกุลก็คือกลุ่มของอะตอมหลาย ๆ อะตอมมารวมกัน และในอะตอมนั้นเราสนใจพฤติกรรมของ%
อิเล็กตรอนเป็นพิเศษ ในวิชากลศาสตร์ควอนตัมนั้นเราจะอธิบายพฤติกรรมของโมเลกุลโดยมุ่งเน้นไปที่อิเล็กตรอนซึ่งสามารถที่จะถูกอธิบายได้ด้วย%
ฟังก์ชันทางคณิตศาสตร์ที่เรียกว่า \enquote{\textit{ฟังก์ชันคลื่น (Wavefunction)}} ซึ่งถูกพัฒนาขึ้นมาเพื่อเป็นแนวคิดสำหรับการอธิบาย%
อิเล็กตรอนและระบบที่ประกอบไปด้วยอิเล็กตรอนหลายตัว โดยหนึ่งในสมการที่โด่งดังที่สุดสมการหนึ่งของวงการวิทยาศาสตร์นั่นคือสมการชโรดิงเงอร์ 
(Schr\"{o}dinger Equation)\autocite{schleich2013} ซึ่งนำเสนอโดยศาสตราจารย์ Erwin R. J. A. Schr\"{o}dinger 
(นักฟิสิกส์เชื้อสายออสเตรีย-ไอริช ซึ่งในขณะนั้นดำรงตำแหน่งอยู่ที่ University of Zurich) สำหรับการอธิบาย Wavefunction 
โดย Schr\"{o}dinger ได้ตีพิมพ์บทความงานวิจัยในวารสาร Annalen der Physik\footnote{ปัจจุบันนี้วารสาร Annalen der Physik 
ยังตีพิมพ์บทความวิชาการอย่างต่อเนื่อง} ในปี ค.ศ. 1926 ที่ต่อเนื่องกันเป็นจำนวน 4 บทความในซีรีย์ที่ชื่อว่า \textit{Quantisierung als 
Eigenwertproblem} โดยบทความฉบับแรกนั้นเป็นการนำเสนอสมการ \textbf{Time-independent Schr\"{o}dinger Equation}%
\autocite{schrodinger1926} และในเวลาต่อมา Schr\"{o}dinger ก็สามารถพิสูจน์หารูปแบบของสมการ \textbf{Time-dependent 
Schr\"{o}dinger Equation} และตีพิมพ์ในบทความฉบับที่ 4 ได้สำเร็จ\autocite{schrodinger1926a} โดยสมการชโรดิงเงอร์ถูกนำมาใช้%
ในการศึกษาระบบทางกลศาสตร์ควอนตัมซึ่งการแก้สมการชโรดิงเงอร์ได้นั้นจะทำให้ได้มาซึ่งผลเฉลยของสมการคณิตศาสตร์ที่อธิบาย Wavefunction 
ได้นั่นเอง

จากผลงานดังกล่าวทำให้ศาสตราจารย์ Erwin Schr\"{o}dinger ได้รับรางวัลโนเบลสาขาฟิสิกส์ ค.ศ. 1933 ร่วมกับศาสตราจารย์ Paul A. M. 
Dirac (ศาสตราจารย์ที่ University of Cambridge) ซึ่งเป็นหนึ่งในผู้บุกเบิกกลศาสตร์ควอนตัมและการพัฒนาสมการดิแรก (Dirac Equation) 
ซึ่งถูกนำมาใช้อธิบายพฤติกรรมของแฟร์มิออน (Fermions)%
\footnote{อ้างอิง \url{https://www.nobelprize.org/prizes/physics/1933/summary}}

\begin{figure}[htbp]
    \centering
    \includegraphics[width=0.8\linewidth]{fig/time-inde-schrodinger-eq.png}
    \caption{ส่วนหนึ่งของบทความฉบับแรกที่ตีพิมพ์โดย Erwin Schr\"{o}dinger ในเดือนมกราคมของปี ค.ศ. 1926 โดยเสนอสมการ 
    Time-independent Schr\"{o}dinger Equation (สมการที่ (1") และ (5))}
    \label{fig:schrodinger_paper_1}
\end{figure}

\begin{figure}[htbp]
    \centering
    \includegraphics[width=0.8\linewidth]{fig/time-dep-schrodinger-eq.png}
    \caption{ส่วนหนึ่งของบทความฉบับที่ 4 ที่ตีพิมพ์โดย Erwin Schr\"{o}dinger โดยเสนอสมการ Time-dependent Schr\"{o}dinger 
    Equation}
    \label{fig:schrodinger_paper_4}
\end{figure}

สมการชโรดิงเงอร์สามารถแบ่งออกได้เป็นสองแบบคือแบบที่ไม่ขึ้นกับเวลาและแบบที่ขึ้นกับเวลา ดังนี้

\noindent $\bullet$ \textbf{1. Time-independent Schr\"{o}dinger Equation}

\fbox{%
\begin{minipage}{0.9\linewidth}
    \begin{equation}\label{eq:tise}
        \hat{H}\ket{\Psi} = E \ket{\Psi}
    \end{equation}
\end{minipage}}

\noindent $\bullet$ \textbf{2. Time-dependent Schr\"{o}dinger Equation}

\fbox{%
\begin{minipage}{0.9\linewidth}
    \begin{equation}\label{eq:tdse}
        i \hbar \frac{d}{d t} \ket{\Psi(t)} = \hat{H} \ket{\Psi(t)}
    \end{equation}
\end{minipage}}

โดย Wavefunction ($\Psi(t)$) ที่เป็น Eigenfunction นั้นจะบรรจุข้อมูลเชิงอิเล็กทรอนิกส์ทุกอย่างเกี่ยวกับระบบของเราเอาไว้%
\autocite{szabo1996,cramer2004,jensen2017} ซึ่งระบบในที่นี้ก็คือโมเลกุล โดยสมการข้างต้นเป็นการคำนวณหาพลังงานของระบบโดยใช้ 
Hamiltonian Operator ($\hat{H}$) ซึ่งเป็น Operator ที่สอดคล้องกับพลังงาน ซึ่งจริง ๆ แล้ว Eigenvalue ของสมการข้างต้น (สมการที่ 
\ref{eq:tdse} และ \ref{eq:tise}) จะเป็นคุณสมบัติของโมเลกุลอะไรก็ได้ ตราบใดที่เราใช้ Operator ที่สอดคล้องกับคุณสมบัตินั้น ๆ 

%--------------------------
\section{แฮมิลโทเนียน}
\label{sec:hamiltonian}
%--------------------------

Hamiltonian เป็นสิ่งที่สำคัญมากในเคมีควอนตัมเพราะเปรียบเสมือนเป็นกุญแจที่สามารถไขรหัสหาคำตอบหรือความลับจาก Wavefunction ได้
โดย Hamiltonian Operator ที่เรานำมาใช้งานนั้นจริง ๆ แล้วก็คือ Operator สำหรับการหาพลังงานรวมนั่นเอง โดยเป็นผลรวมของ Operator 
พลังงานจลน์และพลังงานศักย์
\idxen{Operator}
\idxboth{แฮมิลโทเนียน}{Hamiltonian}

\begin{equation}\label{eq:hamil}
    \hat{H} = \hat{T} + \hat{V}
\end{equation}

\noindent โดยที่พลังงานจลน์นั้นสามารถเขียนให้อยู่ในรูปของ Momentum Operator ได้โดยพิสูจน์จากพลังงานจลน์ในกรณีแบบดั้งเดิม ดังนี้
\idxboth{โมเมนตัม}{Momentum}
\idxboth{พลังงานจลน์}{Kinetic Energy}

\begin{align}
    T &= \frac{1}{2}mv^{2}_{x} \\
      &= \frac{(mv_{x})^{2}}{2m}
\end{align}

\noindent ทำการจัดรูปใหม่แล้วทำการแทนเทอม $mv_{x}$ ด้วย Momentum Opeator ในทางกลศาสตร์ควอนตัม ($-ih\frac{d}{dx}$) 
จะได้ Operator ใหม่ดังนี้

\begin{equation}
    \hat{T} = -\frac{\hbar^{2}}{2m}\frac{d^{2}}{dx^{2}}
\end{equation}

สำหรับพลังงานศักย์นั้นตรงไปตรงมา นั่นคือเราสามารถเขียนพลังงานศักย์ในทางควอนตัมได้แบบเดียวกับกรณีกลศาสตร์ดั้งเดิมได้เลย ดังนี้
\idxboth{พลังงานศักย์}{Potential Energy}

\begin{equation}
    \hat{V} = V(x)
\end{equation}

เมื่อเรานำ Operator ของทั้งสองพลังงานมารวมกันเราจะได้ Hamiltonian Operator ดังนี้

\begin{equation}
    \hat{H} = -\frac{\hbar^{2}}{2m}\frac{d^{2}}{dx^{2}} + V(x)
\end{equation}

ลำดับต่อมาคือเราจะมาทำการพิจารณาพลังงานศักย์กันก่อนเพราะว่าไม่ซับซ้อนเหมือนกับกรณีของพลังงานจลน์ โดยพลังงานศักย์ที่เราจะพิจารณาก็คือ%
พลังงานงานศักย์คูลอมบ์ (Coulomb Potential Energy หรือ Operator นั่นเอง) โดยมีสมการดังต่อไปนี้
\idxboth{พลังงานงานศักย์!พลังงานคูลอมบ์}{Potential Energy!Coulomb Energy}

\begin{equation}
    E_{q_{1}q_{2}} = q_{1}\frac{q_{2}}{4\pi\epsilon_{0}|\bm{R}|}
\end{equation}

โดยเมื่อเราพิจารณาระบบง่าย ๆ เช่น อะตอมไฮโดรเจนซึ่งมี 1 อิเล็กตรอนและ 1 นิวเคลียส แล้วกำหนดจุดกำเนิด (Origin Point) ซึ่งมีระยะห่าง%
จากอิเล็กตรอนเท่ากับ $แผ่{r}$ หน่วยและมีระยะห่างจากนิวเคลียสเท่ากับ $\bm{R}$ หน่วย จะได้ว่าระยะห่างระหว่างอิเล็กตรอนและนิวเคลียสคือ 
$\bm{r}-\bm{R}$ หน่วย ดังนั้นเราสามารถเขียน Hamiltonian Operator ได้ดังนี้

\begin{multline}\label{eq:hamil_hydrogen}
    \hat{H} = -\underbrace{\frac{\hbar^{2}}{2M} \left( \pdv[2]{X} + \pdv[2]{Y} + \pdv[2]{Z} \right)}_{%
                \text{Nuclear Kinetic Energy}} 
              \\
              -\underbrace{\frac{\hbar^{2}}{2m} \left( \pdv[2]{x} + \pdv[2]{y} + \pdv[2]{z} \right)}_{%
                \text{Electronic Kinetic Energy}}
              \\
              -\underbrace{\frac{1}{4\pi\epsilon_{0}}\frac{e^{2}}{|\bm{r}-\bm{R}|}}_{%
                \text{Electron-Nucleus Attraction}}
\end{multline}

\noindent โดยเราสามารถใช้สัญลักษณ์ $\nabla^{2}$ หรือ Laplace Operator ($\nabla$ อ่านว่า Nabla) ซึ่งเป็นอนุพันธ์อันดับที่สอง%
ของพลังงานจลน์ของนิวเคลียส (เทอมแรก) และของพลังงานจลน์ของอิเล็กตรอน (เทอมที่สอง) ของสมการที่ \ref{eq:hamil_hydrogen} 
โดยสามารถเขียนสมการใหม่ได้ดังนี้

\begin{equation}\label{eq:hamil_reduced}
    \hat{H} = -\frac{\hbar^{2}}{2M} \nabla^{2}_{\bm{R}} - \frac{\hbar^{2}}{2m} \nabla^{2}_{\bm{r}}
              -\frac{1}{4\pi\epsilon_{0}}\frac{e^{2}}{|\bm{r}-\bm{R}|}
\end{equation}

ถึงแม้ว่าสมการที่ \ref{eq:hamil_reduced} มีความเรียบง่ายแล้วแต่ว่าในเคมีควอนตัมนั้นเราจะไม่ได้ใช้สมการของ Operator ที่อยู่ในหน่วย
SI (SI Units) แต่เราใช้หน่วยอะตอม (Atomic Units หรือย่อได้เป็น a.u. หรือบางครั้งก็เขียนแค่ au)\footnote{อ่านรายละเอียดเกี่ยวกับ
Atomic Units ได้ที่ \url{https://en.wikipedia.org/wiki/Hartree_atomic_units}} แทน ซึ่งเมื่อเราเขียนสมการในรูปของ 
Atomic Units แล้วจะได้สมการที่เรียบง่ายกว่าเดิม ดังนี้

\begin{equation}\label{eq:hamil_au}
    \hat{H} = -\frac{1}{2M} \nabla^{2}_{\bm{R}} 
              -\frac{1}{2} \nabla^{2}_{\bm{r}}
              -\frac{1}{|\bm{r}-\bm{R}|}
\end{equation}

\noindent โดยจะสังเกตได้ว่าตัวแปรที่เกี่ยวข้องกับอิเล็กตรอนนั้นจะถูกลดรูปไป ปริมาณที่กำหนดให้มีหน่วยเป็น Atomic Units ได้มีดังนี้
\idxen{Atomic Units}
\idxen{SI Units}

\begin{table}[H]
    \centering
    \caption{เปรียบเทียบปริมาณทางเคมีควอนตัมในหน่วย Atomic Units และ SI Units}
    \label{tab:atomic_units}
    \begin{tabular}{lll}\toprule
    \textbf{ปริมาณ} &\textbf{Atomic Unit} &\textbf{ค่าในหน่วย SI} \\\midrule
    พลังงาน & $\hbar^{2}/m_{e}a_{0}$ (Hartree) & $4.36 \times 10^{-18} J$ \\
    ประจุ & $e$ & $1.60 \times 10^{-19} C$ \\
    ความยาว & $a_{0}$ & $5.29 \times 10^{-11} m$ \\
    มวล & $m_{e}$ & $9.11 \times 10^{-31} kg$ \\
    \bottomrule
    \end{tabular}
\end{table}

สำหรับกรณีที่เป็นระบบมากกว่า 1 อิเล็กตรอน เช่น อะตอมฮีเลียมที่มี 2 อิเล็กตรอน เราสามารถกระจายเทอมของ Hamiltonian ได้ดังนี้

\begin{multline}\label{eq:hamil_he_au}
    \hat{H} = -\frac{1}{2M} \nabla^{2}_{\bm{R}} 
              -\frac{1}{2} \nabla^{2}_{\bm{r_{1}}}
              -\frac{1}{2} \nabla^{2}_{\bm{r_{2}}}
              -\frac{2}{|\bm{r_{1}}-\bm{R}|}
              \\
              -\frac{2}{|\bm{r_{2}}-\bm{R}|}
              +\frac{1}{|\bm{r_{1}}-\bm{r_{1}}|}
\end{multline}

\noindent โดยทั้ง 6 เทอมคือพลังงานจลน์ของนิวเคลียส, พลังงานจลน์ของอิเล็กตรอนตัวที่ 1, พลังงานจลน์ของอิเล็กตรอนตัวที่ 2, 
แรงดึงดูดระหว่างอิเล็กตรอนตัวที่ 1 และนิวเคลียส, แรงดึงดูดระหว่างอิเล็กตรอนตัวที่ 2 และนิวเคลียส, และแรงผลักระหว่างอิเล็กตรอน ตามลำดับ 

นอกจากเราสามารถใช้การประมาณของบอร์น-ออปเพนไฮเมอร์ (Born-Oppenheimer (BO) Approximation) ซึ่งเป็นเทคนิคที่นำมาใช้เพื่อการ%
ประมาณว่า Wavefunction ของโมเลกุลนั้นขึ้นอยู่กับตำแหน่งของอิเล็กตรอนเพียงอย่างเดียวและไม่ขึ้นกับตำแหน่งของนิวเคลียสเนื่องจากว่ามวลของ%
นิวเคลียวนั้นเยอะกว่ามวลของอิเล็กตรอนมาก ซึ่งถ้าหากใช้ BO Approximation กับ Hamiltonian ของอะตอมฮีเลียมนั้น เทอมแรกของสมการที่ 
\ref{eq:hamil_he_au} จะไม่ถูกนำมาพิจารณาในการคำนวณพลังงานของระบบ

%--------------------------
\section{การแก้สมการฟังก์ชันคลื่นเพื่อคำนวณพลังงาน}
\label{sec:wavefunc_ener}
%--------------------------

\begin{figure}[htbp]
    \centering
    \includegraphics[width=0.9\linewidth]{fig/hydrogen_density_plots.png}
    \caption{แบบจำลองของออร์บิทัลเชิงอะตอม (Atomic Orbitals) ของอิเล็กตรอนของอะตอมไฮโดรเจนที่ระดับพลังงานที่แตกต่างกัน
    โดยความเข้มของสีที่ไฮไลท์บ่งบอกถึงโอกาสที่จะพบอิเล็กตรอน ณ ตำแหน่งนั้น 
    (เครดิตภาพ: \url{https://en.wikipedia.org/wiki/Atomic_orbital})}
    \label{fig:hydrogen_density}
\end{figure}

หนึ่งในเป้าหมายสำคัญของกลศาสตร์ควอนตัมเชิงโมเลกุล (Molecular Quantum Mechanics) ก็คือการหาวิธีแก้สมการ Time-independent 
Schr\"{o}dinger Equation เพื่อให้ได้มาซึ่งคำตอบหรือผลเฉลยที่แม่นยำมากที่สุด ซึ่งจะช่วยให้นักเคมีคำนวณสามารถคำนวณคุณสมบัติโครงสร้าง%
เชิงอิเล็กทรอนิกส์ (Electronic Structure) ของโมเลกุล โดยหัวข้อแรกของบทนี้ที่เราจะมาดูกันแบบละเอียดก็คือการใช้เทคนิคควอนตัมเชิงคำนวณ%
และอาศัยการประมาณค่าในการแก้สมการดังกล่าว โดยทั่วไปนั้นจะมีวิธีการหลัก ๆ 2 วิธีที่สามารถช่วยให้เราหาคำตอบของสมการชโรดิงเงอร์ ได้นั่นคือ 
\textbf{\textit{ab initio} method} ซึ่งเป็นวิธีที่ความแม่นยำของผลลัพธ์ที่ได้จากการแก้สมการนั้นจะขึ้นอยู่กับโมเดลที่เรานำมาใช้ในการอธิบาย 
Wavefunction ของระบบของเรา (โมเลกุลจะถูกมองเป็น Many-body System) วิธี \textit{ab initio} ที่ได้มีการพัฒนากันมาตั้งแต่อดีตจน%
ถึงปัจจุบันนั้นมีหลายวิธีมาก\autocite{friesner2005,helgaker2014,jensen2017} วิธีทีได้รับความนิยมมีดังไปนี้
\idxboth{โครงสร้างเชิงอิเล็กทรอนิกส์}{Electronic Structure}

\noindent \textbf{วิธี Hartree-Fock}
\begin{itemize}[topsep=0pt,noitemsep]
    \item Hartree-Fock (HF)
    \item Restricted open-shell Hartree-Fock (ROHF)
    \item Unrestricted Hartree-Fock (UHF)
\end{itemize}

\noindent \textbf{วิธี Post-Hartree-Fock}
\begin{itemize}[topsep=0pt,noitemsep]
    \item Møller-Plesset Perturbation Theory (MPn)
    \item Configuration Interaction (CI)
    \item Coupled Cluster (CC)
    \item Quadratic Configuration Interaction (QCI)
    \item Quantum Chemistry Composite Methods
\end{itemize}

\noindent \textbf{วิธี Multi-reference}
\begin{itemize}[topsep=0pt,noitemsep]
    \item Multi-configurational Self-consistent Field (MCSCF) รวมถึงวิธี CASSCF and RASSCF
    \item Multi-reference Configuration Interaction (MRCI)
    \item n-electron Valence State Perturbation Theory (NEVPT)
    \item Complete Active Space Perturbation Theory (CASPTn)
    \item State Universal Multi-reference Coupled-cluster Theory (SUMR-CC)
\end{itemize}

นอกจากนี้เป็นที่ทราบกันดีว่าสำหรับโมเลกุลที่มีขนาดใหญ่นั้นการคำนวณด้วยวิธี \textit{ab initio} มีความสิ้นเปลืองสูงมาก (Computationally 
Expensive)\autocite{grabowski2011} ดังนั้นจึงเป็นที่มาของการพัฒนาวิธีการที่สองนั่นคือ \textbf{Semiempirical method}%
\autocite{thiel2014,christensen2016,kriz2020} ซึ่งจะใช้แนวคิดในการตีความ Hamiltonian ในรูปแบบที่ง่ายกว่าซึ่งอ้างอิงด้วยออร์บิทัล%
เชิงโมเลกุล (Molecular Orbital หรือ MO) และอาศัยค่าพารามิเตอร์ที่ได้จากการทดลองเพื่อเพิ่มความแม่นยำ อย่างไรก็ตาม วิธี Density 
Functional Theory (DFT) ก็ถูกพัฒนาขึ้นมาเพื่อแก้ปัญหาที่เราจะต้องมาแก้หรือประมาณค่า Wavefunction ตรง ๆ ซึ่งทำได้ยากโดยเฉพาะกรณี%
ที่ระบบมีหลายอิเล็กตรอน ดังนั้นในปัจจุบันการคำนวณเชิงควอนตัมส่วนใหญ่จึงเป็นการใช้ DFT เพราะว่ามีความสิ้นเปลืองของการคำนวณที่ต่ำมากเมื่อ%
เทียบกับสองวิธีข้างต้นทีได้กล่าวไปนั่นเอง
\idxen{Semiempirical Method}
\idxboth{ออร์บิทัลเชิงโมเลกุล}{Molecular Orbital}

ตัวอย่างของความสำเร็จในการแก้สมการ Wavefunction ก็คือผลลัพธ์ที่แน่นอนของคุณสมบัติของระบบ กรณีที่เราสามารถหาผลเฉลยได้แน่นอนก็คือ%
ระบบที่มีอิเล็กตรอน 1 ตัว ตามแสดงในภาพที่ \ref{fig:hydrogen_density} ซึ่งเป็นแบบจำลองของออร์บิทัลเชิงอะตอม (Atomic Orbital 
หรือ AO) ของอิเล็กตรอนของอะตอมไฮโดรเจน ผู้อ่านสามารถศึกษาการเขียนโค้ดสำหรับพล็อตออร์บิทัลของอะตอทไฮโดรเจนได้ที่ภาพผนวกหัวข้อที่ 
\ref{ap:hydro_orbitals}
\idxboth{ออร์บิทัลเชิงอะตอม}{Atomic Orbital}

%--------------------------
\subsection{วิธี Self-Consistent Field}
\label{ssec:scf}
\idxen{Self-Consistent Field}
%--------------------------

ในหัวข้อนี้เราจะมาพูดถึงการแก้สมการชโรดิงเงอร์โดยใช้วิธีที่ชื่อว่า Self-Consistent Field (SCF) ซึ่งเป็นการประมาณค่า Hamiltonian 
แบบวนซ้ำ (เป็นที่มาของคำว่า \textit{Self-Consistent} ซึ่งมีความหมายประมาณว่าเป็นดำเนินการเปรียบเทียบพารามิเตอร์ใหม่กับพารามิเตอร์%
เดิมโดยที่ยังคงใช้โมเดลอันเดียวกัน) เริ่มต้นเราจะต้องมาดูกันก่อนว่าการมอง Wavefunction ของระบบหลายอิเล็กตรอนสำหรับวิธี SCF นั้นจะมีการ%
ตัดสิ่งที่ซับซ้อนออกไปนั่นก็คืออันตรกิริยาแรงผลักระหว่างอิเล็กตรอน (Electron-electron Repulsion) โดย Wavefunction สามารถถูกอธิบาย%
ได้ด้วยสมการชโรดิงเงอร์ที่ไม่ขึ้นกับเวลา ดังต่อไปนี้\autocite{cramer2004}

\begin{equation}\label{eq:tise_elec}
    H^{\circ} \Psi^{\circ} = E^{\circ} \Psi^{\circ}
\end{equation}

โดยกำหนดให้ $H^{\circ} = \sum^{N}_{i=1} h_{i}$ เมื่อ $h$ คือ Hamiltonian สำหรับหนึ่งอิเล็กตรอน (อิเล็กตรอนตัวที่ $i$) 
ในระบบที่มีอิเล็กตรอน $N$ ตัว นั่นคือสมการสำหรับระบบที่มีอิเล็กตรอน $N$ ตัวนั้น จะสามารถถูกแยกออกมาได้เป็นสมการของระบบหนึ่งอิเล็กตรอนได้ 
$N$ สมการและ Wavefunction ของอิเล็กตรอนหนึ่งตัวนั้นจริง ๆ แล้วก็คือออร์บิทัล (Orbital) เราจึงสามารถเขียนสมการของอิเล็กตรอนหนึ่งตัว%
โดยอ้างอิงจากสมการที่ \ref{eq:tise_elec} ได้เป็นสมการที่จำเพาะเจาะจงมากขึ้น ดังนี้
\idxboth{ออร์บิทัล}{Orbital}

\begin{equation}\label{eq:tise_elec_i}
    h_{i} \Psi^{\circ}(i) = E^{\circ}_{m} \Psi^{\circ}(i)
\end{equation}

\noindent โดยที่ $E^{\circ}_{m}$ คือพลังงานของอิเล็กตรอนหนึ่งตัวใน MO ซึ่งเขียนแทนด้วย $m$ นั่นเอง สำหรับระบบที่อิเล็กตรอนไม่ขึ้นต่อกัน
\idxboth{ออร์บิทัลเชิงโมเลกุล}{Molecular Orbital}

ด้วยเหตุนี้ Wavefunction รวมของระบบ ($\Psi^{\circ}$) จึงสามารถเขียนให้อยู่ในรูปของ Wavefunction ของอิเล็กตรอนหนึ่งตัวได้ดังนี้

\begin{equation}
    \Psi^{\circ} = \psi^{\circ}_{a}(1) \psi^{\circ}_{b}(1) \dots \psi^{\circ}_{z}(N)
\end{equation}

\noindent ซึ่ง Wavefunction ด้านบนนี้จะขึ้นอยู่กับพิกัดของอิเล็กตรอนทุกตัวและขึ้นกับตำแหน่งของนิวเคลียสหรืออะตอมด้วย%
\footnote{ตอนนี้เราจะยังไม่พิจารณาสปินของอิเล็กตรอนที่จะต้องสอดคล้องและไม่ขัดกับหลักกีดกันของเพาลี (Pauli Exclusion)
ซึ่งจะมีการรวม Spin-orbital สำหรับ Molecular Orbital $m$ ($\varphi_{m}$) เข้าไปด้วย}
\idxboth{หลักกีดกันของเพาลี}{Pauli Exclusion}

สำหรับกระบวนการหรือขั้นตอนที่เราจะนำมาใช้ในการแก้สมการของระบบอิเล็กตรอนหลายตัวนั้น เราจะพิจารณาสมการรูทฮาน (Roothaan Equation) 
เป็นหลัก ซึ่งเป็นวิธีหนึ่งในการแก้สมการ Hartree-Fock (HF) ซึ่งมีการกำหนดตัวดำเนินการใหม่ขึ้นมาใช้แทน Hamiltonian นั่นก็คือ Fock Operator 
โดยที่ Fock Operator ($f_{1}$) ถูกนิยามในเทอมของ Coulomb Operator และ Exchange Operator ขึ้นมา นั่นก็คือ Fock Operator 
ซึ่งเขียนสมการสำหรับอิเล็กตรอน 1 ตัวได้เป็น

\begin{equation}\label{eq:fock}
    f_{1} \psi_{m}(1) = \varepsilon_{n} \psi_{m}(1)
\end{equation}

%--------------------------
\subsection{สมการ Roothaan}
\label{ssec:roothaan}
\idxen{Roothaan Equation}
%--------------------------

สำหรับการแก้สมการ HF ตรง ๆ โดยใช้ SCF นั้นสามารถทำได้ตรง ๆ ด้วยวิธีการเชิงตัวเลข (Numerical Method) แต่ว่าผลเฉลยที่ได้มานั้นมีความ%
ซับซ้อนมาก โดยในเวลาต่อมานักฟิสิกส์และนักเคมีชาวดัตช์ที่ชื่อว่า Clemens C.J. Roothaan จึงได้เสนอวิธีการใหม่สำหรับการอธิบาย MO โดยเรียก%
วิธีนั้นว่าผลรวมเชิงเส้น (Linear Combination of Atomic Orbitals หรือ LCAO)\autocite{atkins2010} เรามาดูรายละเอียดของ LCAO 
กันครับ 
\idxen{Linear Combination of Atomic Orbitals (LCAO)}

เริ่มต้นเราจะนิยามฟังก์ชันพื้นฐาน (Basis Function) สำหรับระบบที่มีอิเล็กตรอน $N$ ตัวขึ้นมาก่อน ซึ่งเขียนแทนด้วย $\chi_{o}$
ซึ่งไอเดียตอนนี้ก็คือเราจะมองว่า Basis Function แบบที่ง่ายที่สุดที่เราสามารถนำมาใช้ได้นั่นก็คือ AO ซึ่งสามารถที่จะเขียน Spatial Wavefunction 
(ฟังก์ชันคลื่นที่ขึ้นกับตำแหน่งของ AO) ให้อยู่ในผลรวมเชิงเส้นของการคูณระหว่างสัมประสิทธิ์เชิงเส้นที่เรายังไม่ทราบค่า (Unknown Coefficients, 
$c_{om}$) กับ $\chi_{o}$ ดังนี้
\idxboth{ออร์บิทัลเชิงอะตอม}{Atomic Orbital}

\begin{equation}\label{eq:lcao}
    \psi_{m} = \sum^{N_{o}}_{o=1} c_{om} \chi_{o} 
\end{equation}

\noindent เมื่อเราแทนสมการ \ref{eq:lcao} เข้าไปในสมการ \ref{eq:fock} เราจะได้

\begin{equation}\label{eq:lcao_in_fock}
    f_{1} \sum^{N_{o}}_{o=1} \chi_{o}(1) = \varepsilon \sum^{N_{o}}_{o=1} c_{om} \chi_{o}(1)
\end{equation}

\noindent แล้วทำการคูณสมการ \ref{eq:lcao_in_fock} ทั้งสองข้างด้วย $\chi^{*}_{o}(1)$ และทำการอินทิเกรตทั่วทั้ง Space 
ซึ่งจะทำให้เราได้ความสัมพันธ์ต่อไปนี้

\begin{equation}\label{eq:lcao_in_fock_int}
    \sum^{N_{o}}_{o=1} c_{om} \int \chi^{*}_{o}(1) f_{1} \chi_{o}(1) d\tau_{1} =
    \varepsilon_{m} \sum^{N_{o}}_{o=1} c_{om} \int \chi^{*}_{o}(1) \chi_{o}(1) d\tau_{1}
\end{equation}

\noindent จากสมการข้างต้นเราจะพบว่าจะมีผลคูณของ Basis Function ทั้งสองฝั่ง โดยทางฝั่งซ้ายนั้นเราสามารถนิยาม Fock Matrix (F) ได้

\begin{equation}\label{eq:matrix_fock}
    F_{o'o} = \int \chi^{*}_{o'}(1) f_{1} \chi_{o}(1) d\tau_{1}
\end{equation}

\noindent และทางฝั่งขวา เรานิยามสิ่งที่เรียกว่า Overlap Matrix (S) ซึ่งเป็น Matrix ที่อธิบายถึงการซ้อนทับกันระหว่างสถานะ 2 สถานะ

\begin{equation}\label{eq:matrix_overlap}
    S_{o'o} = \int \chi^{*}_{o'}(1) \chi_{o}(1) d\tau_{1}
\end{equation}

\noindent ซึ่งเราสามารถเขียนสมการ \ref{eq:lcao_in_fock_int} ให้อยู่ในรูปของสมการที่เรียกว่า Roothaan Equation ได้กระชับ ๆ ดังนี้

\begin{equation}\label{eq:roothaan}
    F c = \varepsilon S c
\end{equation}

\noindent โดยที่ $c$ คือเมทริกซ์ขนาด $N_{o} \times N_{o}$ ซึ่งประกอบไปด้วยสมาชิกของ Coefficient $c_{om}$ และ $\varepsilon$ 
คือเมทริกซ์ที่มีขนาด $N_{o} \times N_{o}$ เช่นเดียวกันซึ่งเป็นเมทริกซ์แบบ Diagonal Matrix (สมาชิกที่ไม่ใช่แนวทแยงมีค่าเป็น 0 ทั้งหมด) 
ซึ่งก็คือพลังงานของ Orbital นั่นเอง ซึ่งตรงจุดนี้เราต้องไม่ลืมว่า Fock Operator ($f_{1}$) นั้นถูกกำหนดให้อยู่ในรูปของ Integral บน MO 
และขึ้นอยู่กับค่าของ Coefficient $c_{om}$ ด้วย

สำหรับการแก้สมการ \ref{eq:roothaan} นั้นสามารถทำได้ผ่าน Determinant ดังนี้

\begin{equation}\label{eq:scf_secular}
    det|F - \varepsilon S| = 0
\end{equation}

\noindent ซึ่งสมการด้านบนไม่สามารถแก้ได้แบบตรงไปตรงมาเพราะว่าสมาชิกของเมทริกซ์ $F_{o'o}$ นั้นเกี่ยวเนื่องโดยตรงกับ Integral ของ 
Coulomb Operator และ Exchange Operators ซึ่งขึ้นอยู่กับ Spatial Wavefunction นั่นจึงทำให้เป็นปัญหาแบบงูกินหาง ดังนั้นเราจึงต้อง%
ใช้กระบวนการวนซ้ำ (Iterative Method) ในการแก้ปัญหาจนกว่าคำตอบหรือผลลัพธ์ที่เราต้องการจากสมการ (พลังงาน) จะลู่เข้านั่นเอง

%--------------------------
\subsection{การแก้สมการ Roothaan ด้วย Self-Consistent Field}
\label{ssec:roothaan_scf}
%--------------------------

\begin{figure}[htbp]
    \centering
    \includegraphics[width=0.9\linewidth]{fig/scf.png}
    \caption{แผนผังขั้นตอนของการประมาณค่าหาพลังงานของออร์บิทัลด้วยวิธี SCF}
    \label{fig:scf}
\end{figure}

ภาพที่ \ref{fig:scf} แสดงแผนผงอัลกอริทึมของวิธี SCF โดยเริ่มจากการเลือก Atomic Basis Function ซึ่งถือว่าเป็นองค์ประกอบหลักของ%
การนำไปสร้าง (Formulate) $S$ โดยใช้สมการ \ref{eq:matrix_overlap} กับ $c_{om}$ ซึ่งเราจะใช้วิธีการสร้างค่าเริ่มต้นด้วยวิธี Guess 
ซึ่งมีด้วยกันหลายวิธี เช่น

\begin{enumerate}[topsep=0pt]
    \item \textbf{H{\"u}ckel guess} : ใช้ H{\"u}ckel Orbital\autocite{jensen2017}
    
    \item \textbf{Superposition of Atomic Densities (SAD)} : ใช้ผลรวมของ Atomic Density ในการสร้าง Density Matrix
    
    \item \textbf{Generalized Wolfsberg-Helmholtz (GWH)} : เป็นวิธีการที่อาศัย H{\"u}ckel Theory โดยการใช้ Overlap 
    Matrix และ Core Hamiltonian\autocite{wolfsberg1952}
    
    \item \textbf{CORE} : ทำการทำ Core Hamiltonian ให้เกิดเมทริกซ์รูปทแยง (Diagonalization)
    
    \item \textbf{Harris} : ใช้ Harris Functional ซึ่งเป็น Non-self-consistent Approximation สำหรับ Kohn-Sham 
    Orbital\autocite{harris1985}
\end{enumerate}

ซึ่งโปรแกรมเคมีเชิงคำนวณต่างก็มีการเลือกใช้ Guess Method สำหรับการเดา Coefficient หรือ Wavefunction เริ่มต้นในการแก้ SCF แตกต่างกันไป
โปรแกรม Gaussian ใช้วิธี Harris สำหรับการคำนวณ HF และ DFT และใช้ H{\"u}ckel หรือ CORE สำหรับ Semiempirical Methods, 
โปรแกรม Q-Chem และ Psi4 ใช้วิธี SAD กับ GWH เป็นวิธีเริ่มต้นโดยอัตโนมัติ เป็นต้น

หลังจากสร้าง Coefficient Matrix ขั้นตอนต่อไปคือการสร้าง Fock Matrix $F$ โดยใช้สมการ \ref{eq:matrix_fock} 
หลังจากนั้นเราจะทำการแก้สมการลักษณะเฉพาะ (Secular Equation) สมการที่ \ref{eq:scf_secular} เพื่อหา Energy Matrix 
แล้วก็ทำการวนซ้ำขั้นตอนการสร้าง $S$ กับ $F$ ไปปรับหาค่าพลังงานไปเรื่อย ๆ จนกว่าค่าความคลาดเคลื่อนหรือ Error จะมีค่าน้อยกว่าค่าที่กำหนดไว้ 
(Threshold) แล้วจึงสิ้นสุดกระบวนการ SCF เมื่อค่าพลังงานนั้นลู่เข้า

%--------------------------
\subsection{การคำนวณอนุพันธ์ของพลังงานและเมทริกซ์เฮสเซียน}
\label{ssec:ener_der}
\idxboth{อนุพันธ์ของพลังงาน}{Energy Derivative}
\idxboth{เมทริกซ์เฮสเซียน}{Hessian Matrix}
%--------------------------

หลังจากที่เราสามารถหาพลังงานเชิงอิเล็กทรอนิกส์ (Electronic Energy) ได้แล้ว ลำดับถัดไปที่เราสามารถคำนวณได้ก็คือคุณสมบัติต่าง ๆ ของโมเลกุล
สิ่งแรกที่เราทำได้และถือว่าสำคัญมาก ๆ ในงานวิจัยทางด้านเคมีควอนตัมก็คือการหาโครงสร้างที่เหมาะสมหรือเสถียรที่สุดของโมเลกุลโดยใช้หลักเกณฑ์%
พลังงานรวมที่ต่ำที่สุด ซึ่งการที่เราทราบโครงสร้างที่เหมาะสมที่สุดนั้นมีประโยชน์อย่างมากเพราะเราสามารถนำผลการคำนวณไปเทียบกับผลจากการทดลอง%
ด้วยเทคนิค X-ray Crystallography, Electron Diffractiom, หรือ Microwave Spectroscopy เป็นต้น โดยการหาโครงสร้างที่สภาวะ%
เหมาะสมหรือสมดุล (Equilibrium Structure) นั้นสามารถทำได้โดยหาอนุพันธ์ของพลังงานศักย์ของโมเลกุลเทียบกับพิกัดนิวเคลียร์ ซึ่งวิธีการที่%
เราสามารถนำมาหาอนุพันธ์เพื่อให้ได้ผลลัพเชิงวิเคราะห์ (Analytical Method) เรียกว่า Gradient Method ซึ่งเร็วและให้ผลลัพธ์ที่แม่นยำกว่า%
ระเบียบวิธีเชิงตัวเลข (Numerical Method)

สำหรับอนุพันธ์ของพลังงานนั้นเราจะเริ่มต้นพิจารณากรณีที่ง่าย ๆ นั่นก็คือโมเลกุลที่มีอะตอมสองอะตอม ซึ่งพลังงานศักย์ของโมเลกุลซึ่งเขียนแทนด้วย $E$ 
นี้จะมีเทอมที่เป็นแรงผลักระหว่างนิวเคลียสด้วยซึ่งจะขึ้นกับระยะห่างระหว่างนิวเคลียส (Internuclear Distance, $R$) สำหรับโครงสร้างที่อยู่ในสมดุลนั้น 
แรง (Force) ที่กระทำต่อนิวเคลียสโดยอิเล็กตรอนนั้นจะเท่ากับศูนย์ ซึ่งแรงดังกล่าวเป็นแรงย่อยมีนิยามคืออนุพันธ์อันดับที่หนึ่งของพลังงานศักย์เทียบ%
กับพิกัดของนิวเคลียสที่ $i$

\begin{align}
    f_{i} &= - \pdv{E}{q_{i}} \\
    &= 0
\end{align}

โดยการคำนวณหาอนุพันธ์ข้างต้นด้วยวิธีการวิเคราะห์หรือ Analytical Method นั้นเราจะต้องทำการคำนวณหาอนุพันธ์ของอินทิกรัลของอิเล็กตรอน%
หนึ่งตัวและอิเล็กตรอนสองตัว (One-electron กับ Two-electron Integrals) เทียบกับพิกัดนิวเคลียร์ นั่นคือเราจะต้องทำการหาอนุพันธ์ของ 
Basis Function นั่นเอง\footnote{Basis Function ก็คือ Basis ที่เกิดขึ้นมาจาก Atomic Orbtials ที่ถูก centered หรือมีตำแหน่ง%
อยู่ที่จุดอ้างอิงของนิวเคลียสของอะตอมในโมเลกุล} ซึ่งเราสามารถทำได้ผ่านการใช้กฎลูกโซ่ (Chain Rule) โดยทำการหาอนุพันธ์ของพลังงานศักย์%
เทียบกับ Expansion Coefficient

ลำดับถัดมาคือการหาเมทริกซ์เฮสเซียน (Hessian Matrix) ซึ่งสามารถทำได้โดยการหาอนุพันธ์ย่อยอันดับที่สองของพลังงานศักย์เทียบกับนิวเคลียส%
ของอะตอมตัวที่ $i$ และ $j$ ($\pdv{E}{q_{i}}{q_{j}}$) ซึ่งช่วยให้เราสามารถระบุได้ว่าค่าพลังงานที่คำนวณออกมาได้นั้นสอดคล้องกับ%
จุดต่ำสุดหรือสูงสุดบนพื้นผิวพลังงานศักย์ (Potential Energy Surface หรือ PES) โดยจะสอดคล้องกับอนุพันธ์อันดับที่สองที่ได้ค่าออกมาเป็นบวก 
(สำหรับ Minimum Point) และลบ (สำหรับ Maximum Point) ตามลำดับ

%--------------------------
\subsection{จากอนุพันธ์ของพลังงานสู่คุณสมบัติเชิงโมเลกุล}
\label{ssec:ener_der_mol_prop}
\idxboth{อนุพันธ์ของพลังงาน}{Energy Derivative}
\idxboth{คุณสมบัติเชิงโมเลกุล}{Molecular Properties}
%--------------------------

คุณสมบัติเชิงโมเลกุลที่เกี่ยวข้องโครงสร้างเชิงอิเล็กทรอนิกส์นั้นเป็นสิ่งที่สำคัญและจำเป็นมากในการคำนวณทางด้านเคมีควอนตัม เพราะว่าคุณสมบัติหรือ%
ปริมาณเหล่านี้เป็นสิ่งที่เรานำไปใช้ในการศึกษาโมเลกุลและปฏิกิริยาเคมี แล้วเราสามารถนำผลการคำนวณไปเปรียบเทียบกับค่าที่วัดได้จากการทดลองเพื่อ%
ตรวจสอบและยืนยันความถูกต้องของทฤษฎีที่ใช้ในการคำนวณคุณสมบัตินั้น ๆ ด้วย ตามที่ได้อธิบายไปก่อนหน้านี้ว่าอนุพันธ์ของพลังงานนั้นเปรียบเสมือน%
เป็นกุญแจที่สามารถนำไปไขกล่องที่เก็บซ่อนความลับของโมเลกุลได้ โดยเราสามารถแบ่งความสำคัญของคุณบัติเชิงโมเลกุลออกได้เป็น 3 ประเภท ดังนี้

\begin{enumerate}[topsep=0pt]
    \item ความแตกต่างของพลังงาน (Energy Differences) เช่น พลังงานของปฏิกิริยา (Reaction Energies), พลังงานในการทำให้%
    กลายเป็นอะตอม (Atomization Energies), พลังงานที่ใช้ในการสลายโมเลกุล (Dissociation Energies), พลังงานที่แตกต่างกัน%
    ระหว่างคอนฟอร์เมอร์หรือไอโซเมอร์

    \item คุณสมบัติเชิงโมเลกุลสำหรับสถานะเชิงอิเล็กทรอนิกส์ เช่น โครงสร้าง ณ สภาวะสมดุล (Equilibrium Structure), ไดโพลโมเมนต์ 
    (Dipole Moment), ความสามารถในการมีสภาพขั้ว (Polarizability), ความถี่เชิงการสั่น (Vibrational Frequencies), 
    ความสามารถในการมีสภาพแม่เหล็ก (Magnetazibility), NMR Chemical Shifts

    \item คุณสมบัติที่บ่งบอกการทรานซิชั่นระหว่างสถานะเชิงอิเล็กทรอนิกส์ที่แตกต่างกันได้ เช่น พลังงานกระตุ้นเชิงอิเล็กทรอนิกส์ (Electronic 
    Excitation Energies), ความเข้มของการทรานซิชั่นของโฟตอน 1 ตัวและ 2 ตัว (One- and two-photon Transition Strengths), 
    ระยะเวลาชีวิตในการแผ่รังสี (Radiative Life Times), พลังงานศักย์ในการทำให้เกิดไอออน (Ionization Potentials) 
\end{enumerate}

โดยในหัวข้อนี้เราจะสนใจคุณสมบัติเชิงโมเลกุลประเภทที่ 2 ซึ่งเกี่ยวกับสถานะเชิงอิเล็กทรอนิกส์เป็นพิเศษ โดยต้องเกริ่นก่อนว่าคุณสมบัติเชิงโมเลกุลนั้น%
เกิดขึ้นจากการที่โมเลกุลมีการตอบสนอง (Response) ต่อสนาม (Feid) ที่กระทำต่อโมเลกุลซึ่งมองได้ในรูปของอนุพันธ์อันดับต่าง ๆ ของพลังงาน 
เช่น อนุพันธ์สามอันดับแรก ดังนี้

\begin{itemize}[topsep=0pt]
    \item อนุพันธ์อันดับหนึ่ง: แรง (Force), ความเครียด (Stress), Dipole Moment เป็นต้น
    \item อนุพันธ์อันดับสอง: Dielectric Susceptibility, Polarizability, Born Effective Charges เป็นต้น
    \item อนุพันธ์อันดับสาม: Nonlinear Dielectric Susceptibility, (First-order) Hyperpolarizability เป็นต้น  
\end{itemize}

โดยเราสามารถเขียนพลังงานที่อยู่ภายใต้สนามภายนอก (External Field) ในรูปของฟังก์ชันการกระจายของเทเลอร์ (Taylor Expansion) 
รอบ ๆ ตำแหน่งที่ไม่มีสนาม (Field-free) ได้ดังนี้

\begin{equation}\label{eq:ener_taylor}
    E(\epsilon) = E(\epsilon = 0) 
    + \underbrace{\evaluated{\dv{E}{\epsilon}}_{\epsilon = 0} \epsilon}_{%
    \text{First Response}}
    + \underbrace{\frac{1}{2} \evaluated{\dv[2]{E}{\epsilon}}_{\epsilon = 0} \epsilon^{2}}_{%
    \text{Second Response}}
    + \dots
\end{equation}

\noindent สำหรับเทอมที่สองที่เป็นอนุพันธ์อันดับสองของพลังงานนั้นคือ Gradient ซึ่งเป็นฟังก์ชันแบบเส้นตรงสำหรับ (Linear) เราจึงเรียกคุณสมบัติ%
ที่ได้จากเทอมนี้ว่า Linear Response Properties ส่วนเทอมอื่น ๆ เช่น เทอมที่สามนั้นเป็นอนุพันธ์อันดับสองซึ่งจะเกี่ยวข้องกับฟังก์ชัน Quadratic  
นอกจากนี้เรายังสามารถสรุปได้แบบนี้ว่า

\begin{framed}
    \begin{align*}
        &\text{Dipole Moment}\,(\mu) &&= &&-\evaluated{\dv{E}{\epsilon}}_{\epsilon = 0} \\
        &\text{Polarizability}\,(\alpha) &&= &&-\evaluated{\dv[2]{E}{\epsilon}}_{\epsilon = 0} \\
        &\text{First Hyperpolarizability}\,(\beta) &&= &&-\evaluated{\dv[3]{E}{\epsilon}}_{\epsilon = 0}
    \end{align*}
\end{framed}

โดยเราสามารถนำคำนวณคุณสมบัติเชิงโมเลกุลต่าง ๆ เหล่านี้ด้วยวิธีการคำนวณทั่วไป (เช่น ใช้วิธี HF หรือ DFT) เพื่อนำไปใช้เป็น Feature 
สำหรับการฝึกสอนโมเดล ML หรือนำมาใช้เป็นเอาต์พุตสำหรับการทำนายก็ได้

%--------------------------
\section{ทฤษฎีฟังก์ชันนอลความหนาแน่น}
\label{sec:dft}
\idxboth{ทฤษฎีฟังก์ชันนอลความหนาแน่น}{Density Functional Theory}
%--------------------------

ตามที่ผู้เขียนได้พูดถึง \textit{\enquote{ทฤษฎีฟังก์ชันนอลความหนาแน่น}} หรือ \textit{\enquote{Density Functional Theory 
(DFT)}} ในบทก่อนหน้านี้แล้วว่าเป็นทฤษฎีที่มีความสำคัญมากในวงการวิทยาศาสตร์ นั่นก็เพราะว่าเป็นทฤษฎีที่ได้พลิกโฉมงานวิจัยที่เกี่ยวกับกับการ%
ศึกษาโครงสร้างเชิงอิเล็กทรอนิกส์ของโมเลกุลไปอย่างสิ้นเชิง นั่นก็เพราะว่าแทนที่จะใช้ Wavefunction ในการอธิบายโมเลกุลตรง ๆ วิธี DFT นั้นจะ%
มองโมเลกุลว่าเป็นกลุ่มก้อนของอิเล็กตรอนที่อธิบายโดยใช้ความหนาแน่น ทำให้เราไม่จำเป็นที่จะต้องแก้สมการเพื่อหา Wavefunction (ซึ่งไม่มีใครรู้ว่า%
หน้าตาที่แท้จริงของระบบที่มีมากกว่าหนึ่งอิเล็กตรอนนั้นเป็นอย่างไร)

%--------------------------
\subsection{ฟังก์ชันและฟังก์ชันนอล}
\label{ssec:function_functional}
\idxboth{ฟังก์ชัน}{Function}
\idxboth{ฟังก์ชันนอล}{Functional}
%--------------------------

ก่อนที่ผู้อ่านจะได้ศึกษาในหัวข้อต่อไปซึ่งจะลงรายละเอียดมากกว่านี้ ผู้เขียนขอเริ่มด้วยการอธิบายความหมายและการใช้งานของสิ่งที่เรียกว่าฟังก์ชันและ%
ฟังก์ชันนอลก่อนครับ เพราะว่าการที่เราเข้าใจความหมายและความแตกต่างของคำศัพท์สองคำนี้จะเป็นพื้นฐานสำคัญในการเข้าใจทฤษฎี DFT ที่ว่าด้วย%
เรื่องของความหนาแน่นของอิเล็กตรอนที่ผู้อ่านจะได้ศึกษาในหัวข้อที่ \ref{ssec:elec_density} โดยฟังก์ชันกับฟังก์ชันนอลนั้นต่างกันตรงที่อินพุต 
ดังนี้

\begin{description}
    \item[$\bullet$ ฟังก์ชัน (Function)] รับอินพุตที่เป็นตัวเลขและให้เอาต์พุตที่เป็นตัวเลขเช่นเดียวกัน โดยสามารถเขียนการ Mapping 
    ได้เป็น $x_0 \mapsto f(x_0)$ โดยที่ $x_{0}$ คืออาร์กิวเมนต์หรืออินพุตของฟังก์ชัน $f$ 
    
    ตัวอย่างของฟังก์ชัน เช่น
    \begin{align*}
        f(x) &= x^{2} \\
        g(x,y) &= \cos(x) + e^{-3\sqrt{x^{2} + y^{2}}}
    \end{align*}

    \item[$\bullet$ ฟังก์ชันนอล (Functional)] เป็นฟังชันก์ชนิดหนึ่งซึ่งรับอินพุตที่เป็นฟังก์ชันและให้เอาต์พุตที่เป็นตัวเลข ซึ่งสรุปได้ง่าย ๆ 
    ว่า \enquote{ฟังก์ชันนอลนั้นก็คือฟังก์ชันของฟังก์ชัน} โดยสามารถเขียนการ Mapping ได้เป็น $f \mapsto f(x_0)$ โดยที่ $x_{0}$ 
    คือพารามิเตอร์ 
    
    ตัวอย่างของฟังก์ชันนอล เช่น 
    \begin{align*}
        F[f] &= \int^{\infty}_{-\infty} f^{3}(x) \, dx \\
        H[g] &= \int^{3}_{2} \int^{4}_{-10} \left ( \pdv[2]{g(x,y)}{x} - 2.3 g(x,y) \right ) \, dx dy
    \end{align*}
\end{description}

เมื่อทราบความแตกต่างแล้วผู้อ่านก็น่าจะพอเดาออกแล้วว่าคำว่า \enquote{Functional} ในชื่อของทฤษฎี Density Functional Theory 
นั้นหมายถึงว่าเป็นทฤษฎีที่ขึ้นอยู่กับฟังก์ชันที่สามารถอธิบายความหนาแน่นของอิเล็กตรอนได้

%--------------------------
\subsection{จากฟังก์ชันคลื่นสู่ความหนาแน่นเชิงอิเล็กทรอนิกส์}
\label{ssec:elec_density}
\idxboth{ความหนาแน่นเชิงอิเล็กทรอนิกส์}{Electronic Density}
%--------------------------

ในการพิจารณาฟังก์ชันคลื่นของอิเล็กตรอนนั้นเราไม่สามารถพิจารณาแค่พิกัดหรือตำแหน่งของอิเล็กตรอนแบบสปาเชียล (Spatial Coordinates) หรือ
($x, y, z$) แต่ยังต้องพิจารณาตำแหน่งของสปิน (Spin Coordinates) หรือ $\omega$ ด้วย ดังนั้นจำนวนตัวแปรที่ส่งผลต่อ Wavefunction 
มีทั้งหมด 4 ตัวแปรต่อหนึ่งอิเล็กตรอน ถ้าหากระบบของเรามี $N$ อิเล็กตรอน จำนวนตัวแปรของ Wavefunction ก็จะเท่ากับ $4 \times 
N_{\text{electrons}}$

เพื่อทำให้ชีวิตง่ายขึ้น แนวคิดในการใช้ความหนาแน่น (Density) สำหรับศึกษาโครงสร้างเชิงอิเล็กทรอนิกส์ของโมเลกุลแทนที่จะใช้ Wavefunction 
โดยตรงนั้นจึงได้รับความสนใจและถูกพัฒนาเรื่อยมาจนถึงปัจจุบัน ข้อดีของการอธิบายระบบ (โมเลกุล) ด้วยความหนาแน่นนั้นง่ายกว่า Wavefunction 
มาก เพราะความหนาแน่นนั้นสามารถถูกเขียนให้เป็นฟังก์ชันที่ขึ้นอยู่ตัวแปรเพียงแค่ 3 ตัวเท่านั้น กล่าวคือสำหรับ Wavefunction ที่เขียนด้วย 
Schr\"{o}dinger Equation นั้นจะเป็นฟังก์ชันที่มีจำนวนมิติคือ $3N_{\text{electron}}$ (สำหรับกรณีที่ไม่พิจารณาสปินของอิเล็กตรอน) 
แต่สำหรับความหนาแน่นนั้นเราจะได้สมการที่มีจำนวนมิติคือ 3 มิติด้วยกันทั้งหมดจำนวน $N$ สมการ (ตามจำนวนอิเล็กตรอน) ซึ่งจะเห็นได้ว่าความ%
ซับซ้อนในการคำนวณจะต่างกันอย่างมาก สรุปเป็นความสัมพันธ์ได้ดังนี้

\begin{framed}
    \centering
    $3N$-dimensional Schr\"{o}dinger Equation\\ $\downarrow$\\ $N$ $3$-dimensional Single Particle Equation
\end{framed}

\noindent โดยที่ Single Particle Equation ในที่นี้คือสมการที่ใช้ในการอธิบายอนุภาค 1 ตัวซึ่งก็คืออิเล็กตรอนนั่นเอง

ความหนาแน่นเชิงอิเล็กทรอนิกส์นี้จริง ๆ แล้วก็คือความหนาแน่นของอิเล็กตรอน (Electron Density, $n(\bm{r})$) ซึ่งเป็นตัวแปรที่เป็นหัวใจ%
ของทฤษฎี DFT นั้นเอง โดยที่ความหนาแน่นของอิเล็กตรอนในโมเลกุลสามารถคำนวณได้จากการใช้ปริพันธ์ (Integral) ในการรวมความหนาแน่น%
ของอิเล็กตรอนแต่ละตัวเข้าด้วยกัน (ทำการรวมทั้งหมด $N$ ครั้ง) ดังนี้
\idxboth{ความหนาแน่นของอิเล็กตรอน}{Electron Density}

\begin{equation}\label{eq:elec_density}
    n(\bm{r}) = N \int{d}^{3} \bm{r}_{2} \cdots \int{d}^{3} \bm{r}_{N} \, 
                \psi^*(\bm{r}, \bm{r}_2, \dots, \bm{r}_N) \psi(\bm{r}, \bm{r}_2, 
                \dots, \bm{r}_N)
\end{equation}

\noindent โดยที่ $\psi$ คือฟังก์ชันคลื่นที่ถูกทำให้เป็นปรกติ (Normalized Wavefunction) แล้ว

%--------------------------
\subsection{จากความหนาแน่นเชิงอิเล็กทรอนิกส์สู่พลังงานของระบบ}
\label{ssec:ener_density}
\idxen{Density Functional Theory!Electronic Density}
%--------------------------

เมื่อเราเข้าใจนิยามและไอเดียของความหนาแน่นเชิงอิเล็กทรอนิกส์หรือความหนาแน่นของอิเล็กตรอนแล้ว ลำดับต่อไปก็คือเราจะคำนวณพลังงานของระบบ
(โมเลกุล) โดยใช้ความหนาแน่นได้อย่างไร ซึ่งตามทฤษฎีนั้นเราสามารถคำนวณพลังงานได้แบบอ้อม ๆ ผ่าน Wavefunction 

\begin{figure}[htbp]
    \centering
    \includegraphics[width=0.9\linewidth]{fig/density_wavefunc_ener.png}
    \caption{ความเชื่อมโยงแบบตรงและแบบอ้อมระหว่างความหนาแน่นของอิเล็กตรอนและพลังงานเชิงอิเล็กทรอนิกส์ของระบบ}
    \label{fig:density_wavefunc_ener}
\end{figure}

จากไดอะแกรมที่แสดงในภาพที่ \ref{fig:density_wavefunc_ener} นั้นสามารถตีความได้ว่าเราสามารถคำนวณพลังงานของระบบโดยผ่าน 
Hamiltonian และ Wavefunction ได้ซึ่งก็จะมีความซับซ้อนในเชิงคำนวณ ดังนั้นคำถามสำคัญที่ตามมาก็คือจะเป็นไปได้ไหมที่เราจะคำนวณพลังงาน%
จากความหนาแน่นของอิเล็กตรอนตรง ๆ ซึ่งคำตอบก็คือจริง ๆ แล้วไม่สามารถหาได้ตรง ๆ แต่เรามีทริคที่สามารถทำได้ ดังนี้

เริ่มจากการกำหนดให้พลังงานเชิงอิเล็กทรอนิกส์ได้จากการคำนวณค่า Expectation Value (ค่าเฉลี่ย) ของ Hamiltonian Operator 

\begin{equation}\label{eq:ener_expect_value}
    E_{\text{el}} = \int \cdots \int \Psi^{\ast} \hat{H}_{\text{el}} \Psi \, d\bm{x}_{1} \dots \, 
    d\bm{x}_{N_{\text{el}}}
\end{equation}

\noindent โดยที่ Hamiltonian Operator สำหรับอิเล็กตรอนมีสมการดังต่อไปนี้

\begin{equation}\label{eq:hamil_one_elec}
    \hat{H}_{\text{el}} = \sum^{N_{\text{el}}}_{i=1} -\frac{1}{2} \nabla^{2}_{i} 
    + \sum^{N_{\text{el}}}_{i=1} \sum^{N_{\text{el}}}_{j=i+1} \frac{1}{|\bm{r}_{i}-\bm{r}_{j}|}
    + \underbrace{\sum^{N_{\text{el}}}_{i=1} \sum^{N_{\text{nu}}}_{A=1} \frac{-Z_{A}}{|\bm{r}_{i}-\bm{R}_{A}|}}%
    _{\text{Nuclear Attraction Energy}}
\end{equation}

\noindent ซึ่งเทอมที่ 3 ของสมการที่ \ref{eq:hamil_one_elec} นั้นคือพลังงานดึงดูดระหว่างอิเล็กตรอนกับนิวเคลียสซึ่งมีชื่อเรียกอีกชื่อว่า 
\enquote{ศักย์ภายนอก} (External Potential) โดยเป็นคำศัพท์ที่ใช้ในทฤษฎี DFT ซึ่งคำว่า External นี้มาจากการที่เราใช้การประมาณ%
ของ Born-Oppenheimer แล้วทำให้นิวเคลียสนั้นเป็นวัตถุที่ถูกตรึงอยู่กับที่ (Fixed) ซึ่งทำให้เกิดพลังงานศักย์คูลอมป์ (Coulomb Potential) 
ต่ออิเล็กตรอน ดังนั้นจากสมการที่ \ref{eq:hamil_one_elec} จะได้เป็น 
\idxboth{ศักย์ภายนอก}{External Potential}

\begin{align}\label{eq:hamil_ext_pot}
    \hat{H}_{\text{el}} &= \sum^{N_{\text{el}}}_{i=1} -\frac{1}{2} \nabla^{2}_{i} 
    + \sum^{N_{\text{el}}}_{i=1} \sum^{N_{\text{el}}}_{j=i+1} \frac{1}{|\bm{r}_{i}-\bm{r}_{j}|}
    + \sum^{N_{\text{el}}}_{i=1} 
    \underbrace{\left ( \sum^{N_{\text{nu}}}_{A=1} \frac{-Z_{A}}{|\bm{r}_{i}-\bm{R}_{A}|} \right )}%
    _{\text{Nuclear Attraction Energy}} \nonumber \\
    &= \sum^{N_{\text{el}}}_{i=1} -\frac{1}{2} \nabla^{2}_{i} 
    + \sum^{N_{\text{el}}}_{i=1} \sum^{N_{\text{el}}}_{j=i+1} \frac{1}{|\bm{r}_{i}-\bm{r}_{j}|}
    + \sum^{N_{\text{el}}}_{i=1} V_{\text{ext}}(\bm{r}_{i})
\end{align}

\noindent โดยที่มี External Potential ($V_{\text{ext}}(\bm{r}_{i}))$) กระทำต่ออิเล็กตรอนทุกตัวในโมเลกุล 

ลำดับต่อไปก็คือเราลองมาทำการกระจาย Expectation Value ของพลังงานเชิงอิเล็กทรอนิกส์โดยการแทนสมการที่ \ref{eq:hamil_ext_pot} 
เข้าไปในสมการที่ \ref{eq:ener_expect_value} ซึ่งเราจะได้พลังงานที่ประกอบไปด้วย 3 เทอม ดังนี้

\begin{align}\label{eq:ener_express_ext_pot}
    E_{\text{el}} =& \int \cdots \int \Psi^{\ast} 
    \left ( \sum^{N_{\text{el}}}_{i=1} -\frac{1}{2} \nabla^{2}_{i} \right ) 
    \Psi \, d\bm{x}_{1} \dots \, d\bm{x}_{N_{\text{el}}} \nonumber \\
    &+ \int \cdots \int \Psi^{\ast} 
    \left ( \sum^{N_{\text{el}}}_{i=1} \sum^{N_{\text{el}}}_{j=i+1} \frac{1}{|\bm{r}_{i}-\bm{r}_{j}|} \right ) 
    \Psi \, d\bm{x}_{1} \dots \, d\bm{x}_{N_{\text{el}}} \nonumber \\
    &+ \underbrace{\int \cdots \int \Psi^{\ast} 
    \left ( \sum^{N_{\text{el}}}_{i=1} V_{\text{ext}}(\bm{r}_{i}) \right ) 
    \Psi \, d\bm{x}_{1} \dots \, d\bm{x}_{N_{\text{el}}}%
    }_{\textstyle \int V_{\text{ext}}(\bm{r}) n(\bm{r}) \, d\bm{r}}
\end{align}

\noindent โดยสมการที่ \ref{eq:ener_express_ext_pot} มีเพียงแค่เทอมที่ 3 เท่านั้นที่สามารถเขียนให้อยู่ในรูปของฟังก์ชันนอลของ%
ความหนาแน่นได้ (Explicit Functional of Density)

%--------------------------
\subsection{ความสัมพันธ์ระหว่างความหนาแน่นของอิเล็กตรอนและศักย์ภายนอก}
\label{ssec:ener_density_ext_pot}
%--------------------------

\begin{figure}[H]
    \centering
    \frame{\includegraphics[width=\linewidth]{fig/hohenberg_kohn_abstract.png}}
    \caption{บทคัดย่อของบทความงานวิจัยที่นำเสนอทฤษฎีบท Hohenberg-Kohn ในปี ค.ศ. 1964}
    \label{fig:hohenberg_kohn_abs}
\end{figure}

สำหรับระบบที่มีจำนวนอิเล็กตรอน $N$ ตัวนั้น ศาสตราจารย์ Pierre Hohenberg (New york University) และศาสตราจารย์ Walter Kohn 
(University of California at Santa Barbara) ได้เสนอทฤษฎีบทที่เป็นรากฐานสำคัญของทฤษฎี DFT ในปี ค.ศ. 1964 นั่นก็คือทฤษฎี%
โฮเฮนเบิร์ค-โคห์น (Hohenberg-Kohn Theorem)\autocite{hohenberg1964} ซึ่งเป็นทฤษฎีที่ว่าด้วยการพิสูจน์ความสัมพันธ์ระหว่างความ%
หนาแน่นและศักย์ภายนอกว่าเป็นแบบหนึ่งต่อหนึ่ง (One-to-one) โดยใช้หลักการแปรค่า (Variational Principle) โดยบทความงานวิจัย%
ฉบับนี้ถือว่ามีความสำคัญอย่างมากต่อวงการวิทยาศาสตร์โดยเฉพาะสาขาฟิสิกส์และเคมีเชิงโมเลกุล

\begin{framed}
    \centering
    \begin{align*}
        &n^{(1)}(\bm{r}) &\underset{\text{H-K}}{\rightleftarrows} &&V_{\text{ext}^{(1)}}(\bm{r}) \\[0.5ex]
        &n^{(2)}(\bm{r}) &\underset{\text{H-K}}{\rightleftarrows} &&V_{\text{ext}^{(2)}}(\bm{r}) \\[0.5ex]
        &n^{(3)}(\bm{r}) &\underset{\text{H-K}}{\rightleftarrows} &&V_{\text{ext}^{(3)}}(\bm{r}) \\[0.5ex]
        &\cdots & &&\cdots 
    \end{align*}
\end{framed}

นอกจากนี้แล้วเรา Hohenberh และ Kohn ยังได้นำเสนอพลังงานเชิงอิเล็กทรอนิกส์ที่เขียนให้อยู่ในรูปของฟังก์ชันทั่วไป ดังนี้

\begin{align}\label{eq:ener_univer_ext_pot}
    E_{\text{el}} =& \underbrace{\int \Psi^{\ast} 
    \left ( \sum^{N_{\text{el}}}_{i=1} -\frac{1}{2} \nabla^{2}_{i} \right ) 
    \Psi \, d\bm{X} 
    + \int \Psi^{\ast} 
    \left ( \sum^{N_{\text{el}}}_{i=1} \sum^{N_{\text{el}}}_{j=i+1} \frac{1}{|\bm{r}_{i}-\bm{r}_{j}|} \right ) 
    \Psi \, d\bm{X}}_{\textstyle F_{\text{H-K}}[n]} \nonumber \\
    &+ \underbrace{\int \Psi^{\ast}  
    \left ( \sum^{N_{\text{el}}}_{i=1} V_{\text{ext}}(\bm{r}_{i}) \right ) 
    \Psi \, d\bm{X}%
    }_{\textstyle \int V_{\text{ext}}(\bm{r}) n(\bm{r}) \, d\bm{r}} \\
    =& E_{\text{el}}[n]
\end{align}

\noindent โดยผลรวมของสองเทอมแรกนั้นคือสิ่งที่เรียกว่าฟังก์ชันนอลสากล (Universal Functional, $F_{\text{H-K}}[n]$) ซึ่งไม่ขึ้น%
กับศักย์ภายนอก แต่อย่างไรก็ตาม ปัญหาก็คือเราไม่ทราบหน้าตาหรือผลเฉลยที่แน่นอนของฟังก์ชันนอลสากลแต่ว่าเรายังต้องการฟังก์ชันนอลนี้สำหรับ%
การคำนวณพลังงาน ซึ่งสิ่งที่เราทำได้ก็คือการหาฟังก์ชันสากลโดยใช้วิธีการประมาณ
\idxboth{ฟังก์ชันนอลสากล}{Universal Functional}

%--------------------------
\subsection{ฟังก์ชันนอลสากลและทฤษฎีฟังก์ชันนอลความหนาแน่นแบบไร้ออร์บิทัล}
\label{ssec:univer_functional}
%--------------------------

สำหรับพลังงานเชิงอิเล็กทรอนิกส์ที่สามารถเขียนได้จากองค์ประกอบ 3 ส่วนคือ

\begin{equation}\label{eq:ener_elec_simplified}
    E_{\text{el}}[n] = E_{\text{kin}}[n] + E_{\text{pot}}[n] + E_{\text{ext}}[n]
\end{equation}

\noindent เราสามารถเขียนเทอมที่ 2 ของฟังก์ชัน \ref{eq:ener_elec_simplified} ซึ่งเป็นผลรวมของพลังงานศักย์คูลอมป์และพลังงาน%
ของอินตรกิริยะระหว่างอิเล็กตรอนซึ่งก็คือพลังงานแลกเปลี่ยน (Exchange Energy) และพลังงานสหสัมพันธ์ (Coorelation Energy) ได้ดังนี้

\begin{equation}\label{eq:ener_elec_full}
    E_{\text{el}}[n] = E_{\text{kin}}[n] + E_{\text{Col}}[n] + E_{\text{X}}[n] + E_{\text{C}}[n] + 
    E_{\text{ext}}[n]
\end{equation}

\noindent ซึ่งเทอมที่ 2 กับเทอมที่ 5 ซึ่งเป็นพลังงานคูลอมป์และศักย์ภายนอกนั้นเรารู้สมการที่แน่อน ดังนี้

\begin{multline}\label{eq:ener_elec_full_exact}
    E_{\text{el}}[n] = E_{\text{kin}}[n] + \frac{1}{2} \int \int \frac{n(\bm{r})n(\bm{r'})}{|\bm{r}-\bm{r'}|} 
    \, d\bm{r} d\bm{r'} + E_{\text{X}}[n] + E_{\text{C}}[n] \\ 
    + \int V_{\text{ext}}(\bm{r}) n(\bm{r}) \, d\bm{r}
\end{multline}

\noindent ส่วนเทอมที่ 1, 3, และ 4 ซึ่งเป็นพลังงานจลน์ พลังงานแลกเปลี่ยนและพลังงานสหสัมพันธ์นั้นเราไม่รู้สมการที่แน่อน ซึ่งเป็นสิ่งที่ต้อง%
ประมาณค่าเองและการประมาณค่าเพื่อหาฟังก์ชันของพลังงานทั้ง 3 เทอมนี้ที่แม่นยำที่สุดเท่าที่จะเป็นไปได้ก็เป็นหนึ่งในงานวิจัยที่ได้รับความสนใจ%
จนถึงปัจจุบัน

วิธีข้างต้นที่คำนวณพลังงานเชิงอิเล็กทรอนิกส์โดยผ่านฟังก์ชันสากล (Universal Functional) นั้นจะเรียกว่าฟังก์ชันนอลความหนาแน่นแบบบริสุทธิ์ 
(Pure DFT) ก็ได้เพราะว่าไม่มีการพิจารณาออร์บิทัล (Orbital-free) ซึ่งเป็นการคำนวณพลังงานของระบบที่อิเล็กตรอนมีอันตรกิริยาต่อกัน 
(Interacting Electrons) ด้วยฟังก์ชันนอลของความหนาแน่น ข้อดีของวิธี Orbital-free DFT คือมีความเรียบง่ายและไม่ซับซ้อนมากนัก 
(Simplicity) แต่ข้อด้อยก็คือมีความแม่นยำในการคำนวณที่ต่ำมากนั่นก็เพราะว่าการประมาณค่าของเทอมพลังงานจลน์ (เทอมแรกของสมการที่ 
\ref{eq:ener_elec_full_exact}) นั้นทำได้ยากมากและขาดความแม่นยำในการประมาณ (เพราะว่าเทอม $\frac{1}{\bm{r}_{i} - 
\bm{r}_{j}}$ นั้นไม่สามารถถูกแยกแบ่งออกเป็นผลรวมของ $\bm{r}_{i}$ และ $\bm{r}_{j}$ ได้)  เมื่อเราไม่สามารถประมาณค่าพลังงาน%
จลน์ได้อย่างแม่นยำจึงทำให้พลังงานเชิงอิเล็กทรอนิกส์ที่คำนวณออกมานั้นมีความแม่นยำต่ำตามไปด้วย 

\begin{figure}[H]
    \centering
    \frame{\includegraphics[width=\linewidth]{fig/kohn_sham_abstract.png}}
    \caption{บทคัดย่อของบทความงานวิจัยที่นำเสนอทฤษฎีบท Kohn-Sham ในปี ค.ศ. 1965}
    \label{fig:kohn_sham_abs}
\end{figure}

สำหรับการแก้ปัญหาดังกล่าวนั้น ในปี ค.ศ. 1965 ศาสตราจารย์ Walter Kohn และศาสตราจารย์ Lu Jeu Sham (University of California, 
San Diego) ก็ได้นำเสนอบทความงานวิจัย (หนึ่งปีหลังจากนำเสนอทฤษฎี Pure (Orbital-free) DFT) โดยได้เสนอการคำนวณฟังก์ชัน%
ของพลังงานโดยใช้ระบบที่อิเล็กตรอนไม่มีอันตรกิริยาต่อกัน (Non-interacting Electrons) แทนการแก้ผ่านระบบที่อิเล็กตรอนมีอันตรกิริยาต่อกัน%
\autocite{kohn1965} ซึ่งมีข้อดีคือทำให้ DFT มีความแม่นยำมากขึ้นเพราะว่าพลังงานจลน์ของระบบที่อิเล็กตรอนแต่ละตัวไม่ขึ้นหรือมีความสัมพันธ์%
กับอิเล็กตรอนตัวอื่น ๆ นั้นมีสมการที่เรารู้หน้าตาแน่นอน จึงไม่มีความจำเป็นที่จะต้องประมาณค่าฟังก์ชันนอลของพลังงานจลน์ในรูปของความหนาแน่น%
อีกต่อไป โดยในเวลาต่อมาทฤษฎีนี้คือ Kohn-Sham DFT นั่นเอง
\idxen{Density Functional Theory!Kohn-Sham Theorem}

%--------------------------
\subsection{จาก Hohenberg-Kohn สู่ Kohn-Sham}
\label{ssec:from_hk_to_ks}
%--------------------------

ในหัวข้อนี้เราจะมารู้จักกับความแตกต่างระหว่างระบบที่อิเล็กตรอนมีอันตรกิริยาต่อกัน (Interacting Electrons) และไม่มีอันตรกิริยาต่อกัน 
(Non-interacting Electrons) กันให้มากขึ้น เพราะว่าเป็นระบบที่ถูกนำมาใช้ในการพิจารณาความหนาแน่นของโมเลกุล

ตามที่ Kohn กับ Sham ได้เสนอการแก้ปัญหาของ Pure DFT โดยการเปลี่ยนมาพิจารณาระบบที่อิเล็กตรอนไม่มีอันตรกิริยาต่อกันแทนนั้น จริง ๆ 
แล้ว Wavefunction และความหนาแน่นของทั้งสองระบบนั้นแตกต่างกันอย่างสิ้นเชิง แต่ว่าทริคของวิธี Kohn-Sham นั้นคือทำการจำลองหรือสร้าง%
ระบบอิเล็กตรอนที่ไม่มีอันตรกิริยาต่อกันแบบปลอม ๆ ขึ้นมา (Fictitious Non-interacting Electrons) โดยบังคับให้ความหนาแน่นของ%
ระบบนี้มีค่าเท่ากันกับความหนาแน่นของระบบที่อิเล็กตรอนมีอัตตรกิริยาต่อกัน ดังนั้นความท้าทายจึงเปลี่ยนจากการหาฟังก์ชันสากลสำหรับ Pure DFT 
มาเป็นการหาระบบที่อิเล็กตรอนไม่มีอันตรกิริยากันแบบปลอม ๆ ที่มีความหนาแน่นเท่ากัน ซึ่งการใช้ทริคนี้ทำให้เราไม่ต้องมาประมาณค่าพลังงานจลน์%
และทำให้การคำนวณ DFT นั้นมีความแม่นยำมากขึ้นเพราะว่าเรามีผลเฉลยที่แน่นอนของพลังงานตามที่ได้อธิบายไว้ก่อนหน้านี้ในย่อหน้าสุดท้ายของ%
หัวข้อที่ \ref{ssec:univer_functional}

\begin{figure}[H]
    \centering
    \includegraphics[width=\linewidth]{fig/electron_system.png}
    \caption{การเปรียบเทียบแบบจำลองของ (1) ระบบที่อิเล็กตรอนมีอันตรกิริยาต่อกัน, (2) ระบบที่อิเล็กตรอนไม่มีอันตรกิริยาต่อกัน, และ 
    (3) ระบบที่อิเล็กตรอนไม่มีอันตรกิริยาต่อกันของโมเลกุลปลอมตามทฤษฎี Kohn-Sham}
    \label{fig:electron_system}
\end{figure}

เพื่อให้ผู้อ่านเข้าใจได้ง่ายขึ้นว่าทำไมระบบที่อิเล็กตรอนไม่มีอันตรกิริยาต่อกันของ Kohn-Sham นั้นถึงมีความหนาแน่นเท่ากันกับระบบที่อิเล็กตรอนมี%
อันตรกิริยาต่อกัน ให้ผู้อ่านเริ่มด้วยการศึกษาภาพที่ \ref{fig:electron_system} ซึ่งเป็นการเปรียบเทียบระหว่างโมเดลของอิเล็กตรอนที่แตกต่างกัน

\begin{itemize}[topsep=0pt]
    \item ระบบที่ 1 คือระบบที่อิเล็กตรอนมีอันตรกิริยาต่อกัน ซึ่งเราสามารถหาผลเฉลยแบบแม่นตรงของ Wavefunction ของระบบนี้ได้ 
    และความหนาแน่นของโมเดลนี้จะเท่ากับความหนาแน่นของโมเลกุลจริงด้วย

    \item ระบบที่ 2 ระบบที่อิเล็กตรอนไม่มีอันตรกิริยาต่อกัน หมายความว่าอิเล็กตรอนแต่ละตัวตัวจะมี Hamiltonian Operator เป็นของตัวเอง 
    ซึงอิเล็กตรอนแต่ละตัวจะวิ่งอยู่ภายในสนามของศักย์เฉลี่ย (Average Potential) ที่เกิดจากอิเล็กตรอนตัวอื่นในระบบ สำหรับระบบนี้เราจะทำ%
    การรวม Hamiltonian ของอิเล็กตรอนแต่ละตัวเข้าด้วยกันเพื่อประมาณค่า Wavefunction สำหรับอิเล็กตรอนทุกตัว

    \item ระบบที่ 3 จะคล้ายกับระบบที่ 2 แต่จะมีความแตกต่างกันที่ศักย์เฉลี่ยที่กระทำต่ออิเล็กตรอน กล่าวคือในระบบนี้ (เรียกว่าระบบอิเล็กตรอน 
    ของ Kohn-Sham ก็ได้) ศักย์เฉลี่ยที่เกิดขึ้นจะมาจากระบบของอิเล็กตรอนแบบปลอม ๆ (Fictitious System of Electrons) โดยเรา%
    จะทำการรวม Wavefunction ของอิเล็กตรอน (Molecular Orbitals) เข้าด้วยกันเพื่อประมาณค่า Wavefunction สำหรับอิเล็กตรอนทุกตัว
    ซึ่งความหนาแน่นของระบบ Kohn-Sham นี้จะมีค่าเท่ากับความหนาแน่นของระบบที่เป็นโมเลกุลจริง ๆ นั่นคือเท่ากับความหนาแน่นของระบบที่ 1 
    ด้วย
\end{itemize}

%--------------------------
\subsection{แฮมิลโทเนียนสำหรับอิเล็กตรอนที่ไม่มีอันตรกิริยาต่อกัน}
\label{ssec:hamil_noninter_elec}
%--------------------------

การที่เราเปลี่ยนมาพิจารณาระบบที่อิเล็กตรอนไม่มีอันตรกิริยาต่อกันแทนนั้น เราสามารถเปลี่ยนเทอมที่เป็นพลังงานที่เกิดจากการผลักกันของอิเล็กตรอน 
(Direct Interaction) ให้เป็นโอเปอเรเตอร์สำหรับอิเล็กตรอน 1 ตัวได้ (เพราะว่าอิเล็กตรอนทุกตัวเป็นอิสระและไม่ขึ้นต่อกันอีกต่อไปแล้ว) 
ซึ่งโอเปอเรเตอร์ที่ว่านั้นคือพลังงานศักย์ที่อธิบายค่าเฉลี่ยของผลที่เกิดจากอันตรกิริยาระหว่างอิเล็กตรอน (Average Effect) อธิบายง่าย ๆ คือ%
เราใช้เทอมนี้เป็นตัวแทนของอันตรกิริยาระหว่างอิเล็กตรอนในระบบที่อิเล็กตรอนไม่มีอันตรกิริยาต่อกัน (เป็นเทอมที่เป็นส่วนเติมเต็ม) โดยเราสามารถ%
พิสูจน์แฮมิลโทเนียนของ Effective Interaction สำหรับอิเล็กตรอนที่ไม่มีอันตรกิริยาได้จากแฮมิลโทเนียนแบบดั้งเดิม ดังต่อไปนี้
\idxen{Density Functional Theory!Effective Interaction}

\begin{align}\label{eq:hamil_inter_elec}
    \hat{H}_{\text{el}} &= \sum^{N_{\text{el}}}_{i=1} -\frac{1}{2} \nabla^{2}_{i} 
    + \sum^{N_{\text{el}}}_{i=1} \sum^{N_{\text{el}}}_{j=i+1} \frac{1}{|\bm{r}_{i}-\bm{r}_{j}|}
    + \sum^{N_{\text{el}}}_{i=1} V_{\text{ext}}(\bm{r}_{i})
\end{align}

\noindent สมการนี้จะเปลี่ยนเป็นสมการสำหรับ Effective Interaction

\begin{align}\label{eq:hamil_noninter_eff_full}
    \hat{H}_{\text{eff}} &= \sum^{N_{\text{el}}}_{i=1} -\frac{1}{2} \nabla^{2}_{i} 
    + \sum^{N_{\text{el}}}_{i=1} V_{\text{aver}}(\bm{r}_{i})
    + \sum^{N_{\text{el}}}_{i=1} V_{\text{ext}}(\bm{r}_{i}) \nonumber \\
    &= \sum^{N_{\text{el}}}_{i=1} -\frac{1}{2} \nabla^{2}_{i} 
    + \sum^{N_{\text{el}}}_{i=1} \{ V_{\text{aver}}(\bm{r}_{i}) + V_{\text{ext}}(\bm{r}_{i}) \} \nonumber \\
    &= \sum^{N_{\text{el}}}_{i=1} -\frac{1}{2} \nabla^{2}_{i} 
    + \sum^{N_{\text{el}}}_{i=1} V_{\text{eff}}(\bm{r}_{i}) \nonumber \\
    &= \sum^{N_{\text{el}}}_{i=1} \{ -\frac{1}{2} \nabla^{2}_{i} + V_{\text{eff}}(\bm{r}_{i}) \} \nonumber \\
    &= \sum^{N_{\text{el}}}_{i=1} \hat{h}(\bm{r}_{i})
\end{align}

จากการพิสูจน์ข้างต้นจะได้ว่าสุดท้ายแล้วแฮมิลโทเนียนทั้งหมดของระบบที่อิเล็กตรอนไม่มีอันตรกิริยาต่อกันนั้นคือผลรวมของแฮมิลโทเนียนของ%
อิเล็กตรอนหนึ่งตัว (1 แฮมิลโทเนียนต่ออิเล็กตรอน 1 ตัว) นั่นคือ 

\begin{equation}\label{eq:hamil_noninter_eff}
    \hat{H}_{\text{eff}} = \sum^{N_{\text{el}}}_{i=1} \hat{h}(\bm{r}_{i})
\end{equation}

นอกจากนี้เราสามารถเขียนและแก้ Schr\"{o}dinger Equation สำหรับ Hamiltonian ของอิเล็กตรอนแต่ละตัวแยกกันได้ดังนี้

\begin{equation}
    \hat{h}(\bm{r}_{i}) \underbrace{\psi_{a}(\bm{r}_{i})}_{\text{MOs}} = 
    \underbrace{\epsilon_{a}}_{\text{Energy}} \psi_{a}(\bm{a}_{i})
\end{equation}

\noindent ซึ่งจากตรงนี้เราสามารถคำนวณหา Molecular Orbitals (MOs) และพลังงานที่สอดคล้องกันได้

รายละเอียดของทฤษฎี DFT นั้นมีอีกเยอะมาก ผู้อ่านที่สนใจศึกษาเพิ่มเติมสามารถศึกษาได้จากหนังสือดังต่อไปนี้

\begin{enumerate}[topsep=0pt]
    \item Introduction to Computational Chemistry \\ แต่งโดย Frank Jensen
    
    \item Electronic Structure - Basic Theory and Practical Methods \\ แต่งโดย Richard M. Martin
\end{enumerate}

%--------------------------
\section{ความหนาแน่นเชิงประจุและเมทริกซ์ความหนาแน่น}
\label{sec:charge_den}
\idxboth{ความหนาแน่นเชิงประจุ}{Charge Density}
%--------------------------

ความหนาแน่นเชิงประจุ (Charge Density) เป็นปริมาณที่บ่งบอกถึงประจุของอะตอมที่อยู่ในโมเลกุล ถ้าหากเราทำการอินทิเกรต Charge Density 
ทั่วทั้งปริมาตรเราจะได้ผลลัพธ์เป็นจำนวนของอิเล็กตรอนในระบบของเรา (โมเลกุล) ดังนี้\autocite{szabo1996}

\begin{equation}
    N = \int \rho(\bm{r}) dV
\end{equation}

โดยนิยามของ Charge Density จะเป็นผลรวมของโอกาสที่เราจะพบอิเล็กตรอนที่อยู่ภายใน Molecular Orbitals ของทั้งระบบ ดังนี้
\idxen{Charge Density}

\begin{equation}\label{eq:charge_density}
    \rho(\bm{r}) = 2 \sum^{N/2}_{i=1} \int |\varphi_{i}(\bm{r})|^{2}
\end{equation}

\noindent โดยเลข 2 ด้านหน้าเครื่องหมาย Summation ก็คือ Occupation Number สำหรับกรณีที่ Molecular Orbital ($i$) 
นั้นมีอิเล็กตรอนทั้งแบบ Spin Up และ Spin Down และ $\varphi_{i}(r)$ คือ Wavefunction ซึ่งเราสามารถเขียน Wavefunction 
ให้อยู่ในรูปผลรวมเชิงเส้น (LCAO) ของ Basis Function ($\phi_{i}$) ซึ่ง Basis Function นี้จะเป็นฟังก์ชันอะไรก็ได้ที่สามารถอธิบายการ%
มีอยู่ของ Molecular Orbital โดยในกรณีแบบที่ง่ายที่สุดคือเราจะมองว่า Molecular Orbital นั้นเกิดขึ้นจากการรวมกันของ Atomic Orbitals 
ดังนั้นเราจะกำหนดให้ Atomic Orbitals เป็น Basis Function\footnote{Basis Function ในที่นี้คือ Atomic Orbitals ที่ถูกกำหนดให้%
มีจุดศูนย์กลางอยู่ที่อะตอมนั้น ๆ} ดังนั้นเราสามารถเขียน LCAO ได้ดังต่อไปนี้ 
ตามสมการดังต่อไปนี้
\idxen{Basis Function}

\begin{equation}
    \rho(\bm{r}) = 2 \sum_{i} \left ( \sum_{\mu} c_{\mu i} \phi_{\mu}^{*} \right ) 
    \left ( \sum_{\nu} c^{*}_{\nu i}  \phi_{\nu} \right )
\end{equation}

\noindent โดยที่ $c$ คือสัมประสิทธิ์ของ LCAO ลำดับต่อมาคือเมื่อเราจัดรูปให้มีเทอมที่เป็นผลคูณของ Basis Function ($\phi_{\mu}^{*} 
\phi_{\nu}$) เราจะกำหนดให้เทอมนี้เป็นสิ่งที่เรียกว่าเมทริกซ์ซ้อนทับ (Overlap Matrix) ($S_{\mu\nu}$) โดยจะได้สมการที่จัดรูปแล้ว ดังนี้
\idxboth{เมทริกซ์ซ้อนทับ}{Overlap Matrix}

\begin{equation}
    \rho(\bm{r}) = 2 \sum_{i}\sum_{\mu\nu} c_{\mu i} c^{*}_{\nu i} S_{\mu\nu}
\end{equation}

หลังจากนั้นเราจะพบว่าจะมีเทอมที่เป็นผลคูณระหว่าง $c$ ซึ่งเราเรียกผลคูณแบบนี้ว่าเมทริกซ์ความหนาแน่น (Density Matrix) 
\idxboth{เมทริกซ์ความหนาแน่น}{Density Matrix}

\begin{equation}\label{eq:density_matrix}
    P_{\mu\nu} = c_{\mu i} c^{*}_{\nu i}
\end{equation}

\noindent ซึ่งเราจะได้สมการของความหนาแน่นเชิงประจุในรูปของเมทริกซ์ความหนาแน่นดังต่อไปนี้

\begin{equation}\label{eq:charge_density_matrix}
    \rho(\bm{r}) = 2 \sum_{i} \sum_{\mu\nu} P_{\mu\nu}S_{\mu\nu}
\end{equation}

%--------------------------
\section{ประจุย่อย}
\label{sec:partial_charge}
\idxboth{ประจุย่อย}{Partial Charge}
\idxboth{ประจุเชิงอะตอม}{Atomic Charge}
%--------------------------

การวิเคราะห์ Wavefunction หลังจากการคำนวณเป็นสิ่งที่สำคัญมากเพราะจะช่วยให้เราเข้าใจถึงพฤติกรรมเชิงอิเล็กทรอนิกส์ของอะตอมภายในโมเลกุล 
โดยสิ่งที่นักเคมีทฤษฎีมักจะทำการวิเคราะห์เป็นอันดับแรกเสมอนั่นก็คือประจุย่อยของแต่ละอะตอม (Partial Atomic Charge) คำถามคือ 
ทำไมประจุย่อยถึงมีความสำคัญ? คำตอบคือถ้าหากเราทราบถึงประจุย่อยของโมเลกุลแล้วนั้นจะช่วยทำให้สามารถเข้าใจว่าอะตอมแต่ละตัวส่งผลหรือมี 
Contribution มากน้อยเพียงใดเมื่อเทียบกับอะตอมอื่น ๆ ภายในโมเลกุลเดียวกัน

%--------------------------
\section{พลังงานของออร์บิทัล}
\label{sec:ener_orb}
\idxboth{พลังงานของออร์บิทัล}{Orbital Energy}
%--------------------------

\idxen{Frontier Orbitals}

\idxen{Ground State}

%--------------------------
\subsection{พลังงานของ HOMO และ LUMO}
\label{ssec:ener_homo_lumo}
\idxboth{พลังงานของ HOMO และ LUMO}{HOMO and LUMO Energy}
%--------------------------

\idxen{Frontier Orbitals!HOMO}

\idxen{Frontier Orbitals!LUMO}

%--------------------------
\subsection{ผลต่างของพลังงานของ HOMO และ LUMO}
\label{sec:ener_diff_orb}
%--------------------------

\idxen{Energy Gap}

%--------------------------
\section{พื้นผิวพลังงานศักย์}
\label{sec:pef}
\idxboth{พื้นผิวพลังงานศักย์}{Potential Energy Surface}
%--------------------------

หนึ่งในหัวข้อที่สำคัญของเคมีเชิงคำนวณก็คือพื้นผิวพลังงานศักย์ (Potential Energy Surface) ซึ่งเป็นสิ่งที่อธิบายความสัมพันธ์ระหว่างรูปร่างเชิง%
เรขาคณิตของโมเลกุล (Molecular Geometry) เช่น ตำแหน่งที่สัมพันธ์กันของอะตอมในโมเลกุลและพลังงานเชิงโมเลกุล พื้นผิวพลังงานศักย์ที่เรา%
จะมาศึกษากันในบทนี้จะเป็นพื้นผิวแบบง่ายสำหรับโมเลกุลเล็ก ๆ เช่น โมเลกุลอะตอมคู่ (Diatomic Molecular) และ โมเลกุลที่มีสามอะตอม 
เพื่อให้ง่ายต่อการอ่านและเพื่อความกระชับ ผู้เขียนจะขอใช้ตัวย่อ PES ซึ่งย่อมาจาก Potential Energy Surface แทนการเรียกพื้นผิวพลังงานศักย์%
ซึ่งจะยาวเกินไป

%--------------------------
\subsection{พื้นผิวพลังงานศักย์สำหรับโมเลกุลอะตอมคู่}
\label{ssec:pef_di_atomic}
%--------------------------

โดยทั่วไปแล้ว PES สำหรับระบบที่ประกอบไปด้วยอะตอมหลายอะตอมนั้นจริง ๆ แล้วก็เป็นฟังก์ชันหลายมิติเชิงซ้อนแบบหนึ่ง ซึ่งเรียกเป็นภาษาอังกฤษว่า
Complex Multidimensional Function ตัวอย่างเช่นเรามีระบบ (โมเลกุล) ที่มีอะตอม $N$ อะตอม ความสัมพันธ์ระหว่างอะตอมภายในระบบนี้%
สามารถถูกอธิบายได้ด้วย Degree of Freedom ซึ่งมีจำนวนเท่ากับ $3N-6$ สำหรับกรณีโมเลกุลที่ไม่เป็นเชิงเส้น เช่น โมเลกุลน้ำ (\ce{H2O}) 
และมีจำนวนเท่ากับ $3N-5$ สำหรับกรณีที่โมเลกุลเป็นแบบเชิงเส้น เช่น โมเลกุลแก๊สคาร์บอนไดออกไซด์ (\ce{CO2}) ซึ่งการที่ฟังก์ชัน Degree 
of Freedom มีจำนวนมิติที่มากเกินกว่า 3 มิตินี้ ทำให้ยากต่อมิงและวิเคราะห์ PES ดังนั้นวิธีที่ง่ายที่สุดคือเรามักจะทำการพิจารณาเฉพาะ Degree of 
Freedom ที่สำคัญและเกี่ยวข้องกับการเปลี่ยนเปลี่ยนของระบบและพลังงาน โดยที่เราเรียกพารามิเตอร์ที่เราทำการเปลี่ยนค่าไปเรื่อย ๆ เพื่อดูผลต่อการ%
เปลี่ยนแปลงพลังงานของโมเลกุลนี้ว่าพิกัดของปฏิกิริยา (Reaction Coordinates) 

เรามาเริ่มกันด้วยตัวอย่างแรกด้วย PES ของอะตอมคู่ ดังต่อไปนี้

\begin{figure}[htbp]
    \centering
    \begin{subfigure}{0.5\textwidth}
        \centering
        \includegraphics[width=0.9\linewidth]{fig/diatomic_molecule.png}
        \caption{กำหนดระยะห่างระหว่างอะตอม}
        \label{fig:diatomic_mol}
    \end{subfigure}%
    \begin{subfigure}{0.5\textwidth}
        \centering
        \includegraphics[width=0.9\linewidth]{fig/PES_diatomic_mol.png}
        \caption{พลังงานศักย์ของโมเลกุลคู่}
        \label{fig:PES_diatomic}
    \end{subfigure}
    \caption{โมเลกุลคู่}
    \label{fig:diatomic_mol_and_PES}
\end{figure}

สำหรับคู่อะตอม A และ B มี Degree of Freedom เพียงแค่ 1 Degree เท่านั้น และกำหนดให้ระยะห่างระหว่างอะตอมเป็น $r_{AB}$ ถ้าอะตอม 
A มีการสร้างพันธะกับอะตอม B สิ่งที่เกิดขึ้นคือเรา(อาจจะ)สามารถทำนายลักษณะของ PES ของโมเลกุลนี้ได้ ดังต่อไปนี้

\begin{figure}[htbp]
    \centering
    \begin{subfigure}{0.7\textwidth}
        \centering
        \includegraphics[width=0.9\linewidth]{fig/3-body_collinear.png}
        \caption{กำหนดระยะห่างระหว่างอะตอม}
        \label{fig:3_body_mol}
    \end{subfigure}%
    \\
    \begin{subfigure}{0.9\textwidth}
        \centering
        \includegraphics[width=0.9\linewidth]{fig/3-body_collinear_PES.png}
        \caption{พลังงานศักย์ของโมเลกุลสามอะตอมแบบเชิงเส้นตรงร่วม}
        \label{fig:PES_3_body_mol}
    \end{subfigure}
    \label{fig:3_body_mol_and_PES}
\end{figure}

\begin{figure}[htbp]
    \centering
    \includegraphics[width=0.9\linewidth]{fig/3-body_non-collinear.png}
    \caption{โมเลกุลสามอะตอมแบบไม่เป็นเชิงเส้นตรงร่วม}
    \label{fig:non_collinear}
\end{figure}

\begin{figure}[htbp]
    \centering
    \includegraphics[width=0.9\linewidth]{fig/PES_C2H4Cl2.png}
    \caption{พลังงานศักย์ของโมเลกุล \ce{C22H4Cl2}}
    \label{fig:pes_c2h4cl2}
\end{figure}

%--------------------------
\section{ไดโพลโมเมนต์}
\label{sec:dipole_moment}
\idxboth{ไดโพลโมเมนต์}{Dipole Moment}
%--------------------------

ไดโพลโมเมนต์ (Dipole Moment)

%--------------------------
\section{สภาพการเกิดขั้ว}
\label{sec:polariz}
\idxboth{สภาพการเกิดขั้ว}{Polarizability}
%--------------------------

สภาพการเกิดขั้ว (Polarizability)

%--------------------------
\section{เทคนิคสเปกโทรสโกปีแบบสั่น}
\label{sec:spectro}
\idxen{Spectroscopy}
\idxen{Spectroscopy!Vibrational Spectroscopy}
%--------------------------

สเปกโทรสโกปี (Spectroscopy) เป็นการศึกษาอันตรกิริยา (Interaction) ระหว่างสสารกับรังสีแม่เหล็กไฟฟ้า (Electromagnetic Radiation) 
ที่เกิดจากการเปลี่ยนระดับพลังงานของอิเล็คตรอน การเปลี่ยนระดับพลังงานการหมุน (Rotation) และการสั่นสะเทือน (Vibration) ของโมเลกุล 
ซึ่งการที่เราทราบจากสเปกตรัมของโมเลกุลจะทำให้เราทราบข้อมูลหลายอย่างเกี่ยวกับโครงสร้างของโมเลกุลของสสารและสมบัติทางเคมี เช่น

\begin{itemize}
    \item สมมาตรของโมเลกุล (Symmetry)
    
    \item ความยาวพันธะ (Bond Length)
    
    \item มุมพันธะ (Bond Angle)
    
    \item ความแข็งแรงของพันธะ (Bond Strength)
    
    \item การเปลี่ยนแปลงภายในโมเลกุล
    
    \item การเปลี่ยนแปลงระหว่างโมเลกุล
\end{itemize}

โดยในหัวข้อนี้เราจะมาดูรายละเอียดเกี่ยวกับการคำนวณความเข้มของการดูดกลืนสำหรับเทคนิค Infrared (IR) และรามาน (Raman) 
ซึ่งทั้งสองเทคนิคนี้ต่างก็เป็นเทคนิคสเปกโทรสโกปีแบบสั่น (Vibrational Spectroscopy) ซึ่งมีการนำมาใช้ในการทำงานวิจัยสำหรับการศึกษา%
คุณสมบัติของโมเลกุลอย่างแพร่หลาย

%--------------------------
\subsection{อินฟราเรดสเปกโทรสโกปี}
\label{ssec:ir_spectro}
\idxboth{สเปกโทรสโกปี!อินฟราเรด}{Spectroscopy!IR}
%--------------------------

อินฟราเรดสเปกโทรสโกปี (IR Spectroscopy) เป็นการวัดการดูดกลืนของการแผ่รังสีของโมเลกุลในช่วงอินฟราเรดซึ่งเกี่ยวข้องกับการเปลี่ยนแปลงของ%
อิเล็กทริกไดโพลโมเมนต์ (Electric Dipole Moment) ของโมเลกุลที่ศึกษา สำหรับการคำนวณความเข้มของการดูดกลืน IR ในรูปแบบของวิธีแบบ
Dynamic นั้นสามารถทำได้โดยใช้สมการ (ความสมพันธ์) ดังต่อไปนี้s\autocite{thomas2013}

\begin{equation}\label{eq:IR_corr}
    I_{IR} (\omega) \propto \int \braket{\bm{\dot{\mu}}(\tau) \bm{\dot{\mu}}(\tau+t)}_{\tau} e^{-i \omega t} dt
\end{equation}

\noindent โดยที่ $\bm{\dot{\mu}}$ คืออนุพันธ์ของไดโพลโมเมนต์เทียบกับเวลา, $\omega$ คือความถี่เชิงการสั่น (Vibrational Frequency),
$\tau$ คือเวลาที่เปลี่ยนแปลงไปอย่างช้า ๆ และ $t$ คือเวลาสำหรับการทำ Integration นอกจากนี้ยังจะสังเกตได้ว่าจะมีเทอม
$\braket{\bm{\dot{\mu}}(\tau) \bm{\dot{\mu}}(\tau+t)}_{\tau}$ ซึ่งจะเป็นตัวที่บ่งบอกถึงสหสัมพันธ์ของเวลา (Time Correlation) 
ของ $\bm{\dot{\mu}}$ 

สำหรับกรณีที่เป็นแบบ Static นั้น สเปกตรัทของ IR สามารถคำนวณได้ผ่านอนุพันธ์ของไดโพลโมเมนต์เทียบกับพิกัดหรือตำแหน่งของโหมดการสั่น%
แบบปกติ (Normal Coordinates) ซึ่งจะไม่ขึ้นกับเวลา ด้วยสมการดังต่อไปนี้

\begin{equation}\label{eq:mu_qm}
    \bm{\mu}= \sum_{\mu\nu} P_{\mu\nu} \braket{\phi_{\mu}|\bm{r}|\phi_{\nu}}
\end{equation}

\begin{equation}\label{eq:mu_classical}
    \bm{\mu}=\sum_{J} q_J \bm{R_J}
\end{equation}

โดยที่สมการ \ref{eq:mu_qm} จะเป็นสำหรับกรณีแบบควอนตัมซึ่งจะคำนวณผ่านเมทริกซ์ความหนาแน่นและ Basis Function แต่สมการ 
\ref{eq:mu_classical} จะเป็นสำหรับกรณีแบบดั้งเดิมซึ่งจะคำนวณผ่านจุดประจุ (Point Charge) และพิกัดคาร์ทีเซียนของอะตอม

%--------------------------
\subsection{รามานสเปกโทรสโกปี}
\label{ssec:raman_spectro}
\idxboth{สเปกโทรสโกปี!รามาน}{Spectroscopy!Raman}
%--------------------------

รามานสเปกโทรสโกปี (Raman Spectroscopy) เป็นเทคนิคหนึ่งที่เปรียบเสมือนเป็นพี่น้องกับเทคนิคอินฟราเรดสเปกโทรสโกปี โดยที่ Raman Spectroscopy 
จะเป็นผลมาจากการเกิดการกระเจิงของแสงแบบไม่ยืดหยุ่นในช่วงอินฟราเรด วิสิเบิล (Visible) และอัลตราไวโอเล็ต (Ultraviolet) ซึ่งเกี่ยวข้อง%
กับการเปลี่ยนแปลงสภาพการเกิดขั้ว (Polarizability) แบบอิเล็กทริกไดโพล-อิเล็กทริกไดโพล (Electric-dipole--electric-dipole) 
ของสสาร โดยความเข้มของการกระเจิงแบบรามาน ($I_{Raman}$) สามารถคำนวณได้ด้วยความสัมพันธ์ดังต่อไปนี้\autocite{thomas2013}

\begin{equation}\label{eq:Raman_corr}
    I_{Raman} (\omega) \propto \frac{(\omega_{in}-\omega)^4}{\omega} 
    \frac{1}{1-\exp(-\frac{\hbar\omega}{k_{B}T})}S(a^{2}, \gamma^{2})
\end{equation}

\noindent โดยที่ $S(a^{2}, \gamma^{2})$ คือตัวแปรที่เป็นผลจากการรวมกันของความคงที่ (ไม่เปลี่ยนแปลง) แบบไอโซโทรปิค (Isotropic) 
และแอนิโซโทรปิค (Anisotropic)\footnote{คำจำกัดความ: คุณสมบัติที่เท่ากันทุกทิศทาง (Isotropic) และคุณสมบัติที่ขึ้นอยู่กับทิศทาง 
(Anisotropic)} ของเทนเซอร์แบบ Placzek-type Polarizability ($\bm{\alpha}$)\autocite{jensen2005}, $\omega$ 
คือความถี่เชิงการสั่น, $\omega_{in}$ คือความถี่ของแสดง, $k_{B}$ คือค่าคงที่ของโบลทซ์มานน์ (Boltzmann Constant) และ $T$ 
คืออุณหภูมิของระบบในหน่วย Kelvin โดยสมการที่จะใช้ในการอธิบาย $S(a^{2}, \gamma^{2})$ จะขึ้นอยู่กับรูปแบบของการทดลองและสมการของ 
Time Correlation\autocite{mattiat2021}

%--------------------------
\section{การถ่ายโอนอิเล็กตรอน}
\label{sec:et}
\idxth{การถ่ายโอนอิเล็กตรอน}
\idxen{Electron Transfer}
%--------------------------

การถ่ายโอนอิเล็กตรอน (Electron Transfer) เป็นกระบวนการที่อิเล็กตรอนเปลี่ยนตำแหน่งหรือเคลื่อนย้ายจากอะตอมหนึ่งไปยังอีกอะตอมหนึ่ง 
(Transfering) โดยเราสามารถแบ่งการถ่ายโอนอิเล็กตรอนออกได้เป็นสองกรณีคือการถ่ายโอนระหว่างโมเลกุล (Intermolecular Electron 
Transfer) และการถ่ายโอนภายในโมเลกุล (Intramolecular Electron Transfer) สำหรับการถ่ายโอนกรณีแรกนั้นมีสิ่งเร้าภายนอกเป็นปัจจัยหลัก 
ตัวทำละลายหรือสิ่งแวดล้อมภายนอกเป็นตัวกระตุ้นหรือตัวขับเคลื่อน (Driving Force) ที่ทำให้เกิดการถ่ายโอนจากโมเลกุลหนึ่งไปยังโมเลกุลหนึ่ง
สำหรับการถ่ายโอนกรณีที่สองนั้นจริง ๆ แล้วมีปัจจัยหลายอย่างที่ทำให้เกิดกระบวนการนี้ เช่น ความเสถียรเชิงโครงสร้างของโมเลกุล (Stability) 
ซึ่งเกิดจากการรบกวนจากภายนอกที่ส่งผลให้โครงสร้างเชิงอิเล็กทรอนิกส์ของโมเลกุลเปลี่ยนไป 

ในการพิจารณาการถ่ายโอนอิเล็กตรอนทั้งสองกรณีนี้สามารถอธิบายได้ดังนี้ ให้ผู้อ่านลองจินตนาการมีกล่องอยู่สองกล่อง โดยกล่องซ้ายใส่ลูกบอลไว้ 
ส่วนกล่องขวานั้นว่างเปล่า หลังจากนั้นเราทำการหยิบลูกบอลจากกล่องซ้ายแล้วนำไปใส่ไว้ในกล่องขวา นี่คือเป็นการจำลองการถ่ายโอนอิเล็กตรอน 
จากเหตุการณ์ดังกล่าวเราแบ่งออกได้เป็นสองเหตุการณ์ย่อยคือ

\begin{enumerate}
    \item เหตุการณ์ที่เกิดขึ้นก่อนที่จะเกิดการถ่ายโอนอิเล็กตรอน
    
    \item เหตุการณ์ที่เกิดหลังจากถ่ายโอนอิเล็กตรอนแล้ว
\end{enumerate}

\begin{figure}[htbp]
    \centering
    \includegraphics[width=0.7\linewidth]{fig/et_diagram.png}
    \caption{แผนภาพแสดงพื้นผิวพลังงานศักย์ของกระบวนการถ่ายโอนอิเล็กตรอน (เครดิตภาพ: \url{https://chem.libretexts.org})}
    \label{fig:et_diagram}
\end{figure}

%--------------------------
\subsection{ค่าคู่ควบของการถ่ายโอนอิเล็กตรอน}
\label{ssec:et_coupling}
\idxth{การถ่ายโอนอิเล็กตรอน!ค่าคู่ควบ}
\idxen{Electron Transfer!Electron Transfer Coupling}
%--------------------------

ค่าคู่ควบของการถ่ายโอนอิเล็กตรอน (Electron Transfer Coupling) เป็นค่าคู่ควบที่เกิดขึ้นจากการถ่ายโอนอิเล็กตรอน

%--------------------------
\subsection{พลังงานการปรับเปลี่ยนโครงสร้าง}
\label{ssec:reor_ener}
\idxth{พลังงานการปรับเปลี่ยนโครงสร้าง}
\idxen{Reorganization Energy}
%--------------------------

พลังงานการปรับเปลี่ยนโครงสร้าง (Reorganization Energy) คือพลังงาน(ที่น้อยที่สุด)ที่ใช้ในการปรับเปลี่ยนโครงสร้างของโมเลกุลเพื่อทำให้เกิด%
การถ่ายโอนอิเล็กตรอนได้

%--------------------------
\section{คุณสมบัติของสถานะกระตุ้น}
\label{sec:ex_prop}
\idxth{สถานะกระตุ้น}
\idxen{Excited State}
%--------------------------

คุณสมบัติของอิเล็กตรอน ณ สถานะกระตุ้น (Excited State Properties)
\idxth{สถานะกระตุ้น!คุณสมบัติของอิเล็กตรอน ณ สถานะกระตุ้น}
\idxen{Excited State Properties}

%--------------------------
\subsection{พลังงานของสถานะกระตุ้น}
\label{ssec:ex_ener}
\idxth{สถานะกระตุ้น!พลังงานของสถานะกระตุ้น}
\idxen{Excited State!Excited State Energies}
%--------------------------

พลังงานของสถานะกระตุ้น (Excited State Energies)

%--------------------------
\subsection{ค่าคู่ควบของกระบวนการนอนอะเดียแบติก}
\label{ssec:nonadia_ener}
\idxth{สถานะกระตุ้น!ค่าคู่ควบแบบนอนอะเดียแบติก}
\idxen{Excited State!Nonadiabatic Coupling}
%--------------------------

ค่าคู่ควบแบบนอนอะเดียแบติก (Nonadiabatic Coupling)

%--------------------------
\section{การคำนวณโครงสร้างเชิงอิเล็กทรอนิกส์ของโมเลกุล}
\label{sec:comp_elec_strct}
\idxth{โครงสร้างเชิงอิเล็กทรอนิกส์!การคำนวณ}
\idxen{Electronic Structure!Calculation}
%--------------------------

ในหัวข้อนี้เราจะมาดูการคำนวณโครงสร้างเชิงอิเล็กทรอนิกส์ของโมเลกุลกันครับ ซึ่งสิ่งที่เราต้องการนั้นก็คือคุณสมบัติเชิงโมเลกุลนั่นเอง โดยโปรแกรม%
เคมีเชิงคำนวณที่ผู้เขียนเลือกมาให้ศึกษาเป็นตัวอย่างนั้นคือโปรแกรม PySCF ซึ่งเป็นโปรแกรมที่ติดตั้งและใช้งานได้ง่าย มีฟังก์ชันที่หลากหลาย รองรับ%
การคำนวณหลากหลายวิธี โดยผู้อ่านสามารถศึกษารายละเอียดเพิ่มเติมได้ในหัวข้อที่ 

\begin{lstlisting}[style=MyPython]
import pyscf

mol = pyscf.M(
    atom = 'H 0 0 0; F 0 0 1.1',  # in Angstrom
    basis = '631g(d)',
    symmetry = True,
)

mf = mol.KS()
mf.xc = 'pbe0'
mf.kernel()

# Orbital energies, Mulliken population etc.
mf.analyze()
\end{lstlisting}

\noindent ซึ่งจะได้เอาต์พุตดังต่อไปนี้

\begin{lstlisting}[style=plain]
converged SCF energy = -100.302481944224
Wave-function symmetry = Coov
occupancy for each irrep:     A1  E1x  E1y  E2x  E2y
                                3    1    1    0    0
**** MO energy ****
MO #1 (A1 #1), energy= -24.7448119170483 occ= 2
MO #2 (A1 #2), energy= -1.15590146781068 occ= 2
MO #3 (A1 #3), energy= -0.497762978336231 occ= 2
MO #4 (E1x #1), energy= -0.378844054318716 occ= 2
MO #5 (E1y #1), energy= -0.378844054318716 occ= 2
MO #6 (A1 #4), energy= 0.0180305141394873 occ= 0
MO #7 (A1 #5), energy= 0.718896484194941 occ= 0
MO #8 (E1x #2), energy= 1.21692188697545 occ= 0
MO #9 (E1y #2), energy= 1.21692188697545 occ= 0
MO #10 (A1 #6), energy= 1.31220491703922 occ= 0
MO #11 (A1 #7), energy= 1.62220484001697 occ= 0
MO #12 (E1x #3), energy= 1.84298258830569 occ= 0
MO #13 (E1y #3), energy= 1.84298258830569 occ= 0
MO #14 (E2x #1), energy= 1.89656974390515 occ= 0
MO #15 (E2y #1), energy= 1.8965699570922 occ= 0
MO #16 (A1 #8), energy= 2.33936741542906 occ= 0
    ** Mulliken atomic charges  **
charge of  0H =      0.37993
charge of  1F =     -0.37993
Dipole moment(X, Y, Z, Debye):  0.00000,  0.00000, -2.08373
\end{lstlisting}

โดยสรุปผลการคำนวณได้ดังนี้ โมเลกุล \ce{HF} มีพลังงานเชิงอิเล็กทรอนิกส์ ($E_{HF} + E_{Exchange} + E_{Correlation})$ เท่ากับ 
-100.302481944224 Hartree และมีพลังงานของออร์บิทัลเชิงโมเลกุล (MO) ตามที่แสดงทั้ง 16 ออร์บิทัล
