% LaTeX source for ``การเรียนรู้ของเครื่องสำหรับเคมีควอนตัม (Machine Learning for Quantum Chemistry)''
% Copyright (c) 2022 รังสิมันต์ เกษแก้ว (Rangsiman Ketkaew).

% License: Creative Commons Attribution-NonCommercial-NoDerivatives 4.0 International (CC BY-NC-ND 4.0)
% https://creativecommons.org/licenses/by-nc-nd/4.0/

\chapter{ไลบรารี่การเรียนรู้ของเครื่องสำหรับเคมีควอนตัม}
\label{ch:ml_lib}

%--------------------------
\section{ไลบรารี่สำหรับคำนวณลักษณะเฉพาะเชิงโครงสร้าง}
\label{sec:lib_feat}
\idxboth{ไลบรารี่!ลักษณะเฉพาะเชิงโครงสร้าง}{Library!Molecular Structural Feature}
%--------------------------

ลักษณะเฉพาะเชิงโครงสร้างของโมเลกุลไม่มีความซับซ้อนและสามารถคำนวณออกมาได้ง่าย จริง ๆ แล้วเราอาจจะเขียนสคริปต์สำหรับคำนวณลักษณเฉพาะได้โดยที่ไม่ต้องใช้ไลบรารี่เลยก็ได้ อย่างไรก็ตาม ในกรณีที่เราต้องการที่จะคำนวณลักษณะเฉพาะเชิงโครงสร้างหลาย ๆ ตัวสำหรับหลาย ๆ โมเลกุลพร้อมกัน การใช้ไลบรารี่ก็จะสะดวกกว่าในกรณีแบบนี้

%--------------------------
\subsection{RDKit}
\label{ssec:rdkit}
%--------------------------

ไลบรารี่ RDKit\autocite{rdkit} เป็นไลบรารี่สำหรับงานทางด้านเคมีสารสนเทศน์ (Cheminformatics)\footnote{Cheminformatics หรือเรียกอีกอย่างว่า Chemoinformatics เป็นทฤษฎีทางเคมีเชิงฟิสิกส์ที่ใช้คอมพิวเตอร์และข้อมูลสารสนเทศน์ (Informatics) หรือที่เรียกในภาษาละตินว่า \textit{in silico} มาใช้ในการแก้ปัญหาทางเคมี โดยมีการประยุกต์ใช้ Cheminformatics ทั้งในชีวิวทยาและเคมี นอกจากนี้ยังถูกใช้ในงานวิจัยที่ศึกษาการค้นหาตัวยา (Drug Discovery) อีกด้วย} ได้รับความนิยมเป็นอย่างมากเนื่องจากสามารถคำนวณลักษณะเฉพาะเชิงโครงสร้างได้หลากหลาย\footnote{\url{https://www.rdkit.org}} ตัวอย่างโค้ดสำหรับการใช้งาน RDKit สามารถดูได้ที่ \url{https://www.rdkit.org/docs/Cookbook.html} โดยจะมีโค้ดสำหรับการเปลี่ยนข้อมูลเชิงโมเลกุลจากโครงสร้างไปเป็น SMILES และโค้ดสำหรับการแสดงข้อมูลเชิงโครงสร้างของโมเลกุล เช่น จำนวณของอะตอมแต่ละชนิด, จำนวนพันธะคู่, และจำนวนวงเบนซีน 
\idxboth{เคมีสารสนเทศน์}{Cheminformatics}

%--------------------------
\section{ไลบรารี่สำหรับคำนวณลักษณะเฉพาะเชิงอิเล็กทรอนิกส์}
\label{sec:lib_elec_feat}
\idxboth{ไลบรารี่!ลักษณะเฉพาะเชิงอิเล็กทรอนิกส์}{Library!Molecular Electronic Feature}
%--------------------------

ลักษณะเฉพาะเชิงอิเล็กทรอนิกส์ (Molecular Electronic Feature) เช่น Electron Density, Electrostatic Map, และ Frontier Molecular Orbitals (FMOs) มักจะถูกนำมาใช้ในการสร้างชุดข้อมูลเพื่อฝึกสอนโมเดลเพราะว่าคุณสมบัติเหล่านี้เป็นคุณสมบัติเชิงอิเล็กทรอนิกส์ที่มักจะถูกคำนวณออกมาโดยปกติอยู่แล้วด้วยโปรแกรมทางเคมีควอนตัมเชิงคำนวณ (Computational Quantum Chemistry Software)\footnote{\url{https://en.wikipedia.org/wiki/List_of_quantum_chemistry_and_solid-state_physics_software}} อย่างไรก็ตามลักษณะเฉพาะเหล่านี้อาจจะยังไม่สามารถที่จะอธิบายคุณลักษณะเชิงอิเล็กทรอนิกส์บางอย่างที่ซับซ้อนของโมเลกุลได้ดีเท่าที่ควร เช่น คุณสมบัติที่เกี่ยวข้องกับควอนตัม ดังนั้นจึงมีลักษณะเฉพาะอื่น ๆ ที่ถูกพัฒนาขึ้นมาโดยรวมข้อมูลเชิงควอนตัมของเข้าไปด้วยเพื่อให้มีความถูกต้องมากขึ้น ซึ่งการที่จะคำนวณลักษณะเฉพาะเหล่านี้ก็มีความซับซ้อนอยู่มิใช่น้อย ดังนั้นจึงได้มีผู้พัฒนาไลบรารี่ที่สามารถคำนวณพารามิเตอร์เหล่านี้ให้เราได้

\begin{table}[H]
    \centering
    \caption{รายชื่อและเว็บไซต์ของไลบรารี่ Dataset และ Feature สำหรับเคมีควอนตัม}
    \label{tab:review_lib_data_qm}
    \begin{tabular}{cll}
    \toprule
    \textbf{ไลบรารี่} &\textbf{เว็บไซต์} &\textbf{อ้างอิง} \\
    \midrule
    AFLOWLIB &\url{https://aflow.org/aflow-ml} &Curtarolo และคณะ\autocite{curtarolo2012} \\
    Materials Project &\url{https://materialsproject.org} &Jain และคณะ\autocite{jain2013} \\
    OQMD &\url{https://oqmd.org} &Kirklin และคณะ\autocite{kirklin2015} \\
    libAtoms &\url{https://github.com/libAtoms} &Bart\'{o}k และคณะ\autocite{bartok2018} \\
    Matminer &\url{https://github.com/markovmodel/deeptime} &Ward และคณะ\autocite{ward2018} \\
    Chemical VAE &\url{https://ccs-psi.org/node/22} 
    &G\'{o}mez-Bombarelli และคณะ\autocite{gomez-bombarelli2018} \\
    JARVIS-DFT &\url{https://github.com/usnistgov/jarvis} &Choudhary และคณะ\autocite{choudhary2018} \\
    DScribe &\url{https://github.com/SINGROUP/dscribe} &Himanen และคณะ\autocite{himanen2020} \\
    NOMAD &\url{https://analytics-toolkit.nomad-coe.eu} &Draxl และ Scheffler\autocite{draxl2019} \\
    OMDB &\url{https://omdb.mathub.io/dataset} &Olsthoorn และคณะ\autocite{olsthoorn2019} \\
    Khazana &\url{https://khazana.gatech.edu} &Chapman และคณะ\autocite{chapman2020} \\
    MolDis &\url{https://moldis.tifrh.res.in/index.html} &Kayastha และ Ramakrishnan\autocite{kayastha2021} \\
    \bottomrule
    \end{tabular}
\end{table}

ตารางที่ \ref{tab:review_lib_data_qm} แสดงไลบรารี่หรือโปรแกรมแพคเกจที่มีชุดข้อมูลให้เรานำมาใช้ได้ รวมถึงไลบรารี่ที่สามารถคำนวณ Feature ของอะตอมและโมเลกุลได้ด้วย

%--------------------------
\subsection{DScribe}
\label{ssec:dscribe}
%--------------------------

DScribe เป็นไลบรารี่ที่สามารถคำนวณลักษณะเฉพาะเชิงอิเล็กทรอนิกส์ของโมเลกุลที่ซับซ้อนได้เยอะมาก\autocite{himanen2020} สำหรับเวอร์ชันปัจจุบันสามารถคำนวณลักษณะเฉพาะดังต่อไปนี้\footnote{$^{\ast}$ สามารถคำนวณ Derivative ได้}
%
\begin{itemize}[topsep=0pt,noitemsep]
    \setlength\itemsep{0.5em}
    \item Coulomb Matrix$^{\ast}$
    
    \item Sine Matrix
    
    \item Ewald Sum Matrix
    
    \item Atom-centered Symmetry Functions
    
    \item Smooth Overlap of Atomic Positions$^{\ast}$
    
    \item Many-body Tensor Representation
    
    \item Local Many-body Tensor Representation
    
    \item Valle-Oganov Descriptor
\end{itemize}

คู่มือการใช้งานอ่านได้ที่ \url{https://singroup.github.io/dscribe/latest/index.html}

\noindent ตัวอย่างโค้ดของการใช้ DScribe ในการสร้าง Coulomb Matrix (CM)

\begin{lstlisting}[style=MyPython]
from dscribe.descriptors import CoulombMatrix

atomic_numbers = [1, 8]
rcut = 6.0
nmax = 8
lmax = 6

# Setting up the CM descriptor
cm = CoulombMatrix(
    n_atoms_max=6,
)
\end{lstlisting}

\vspace{1em}
\noindent ตัวอย่างโค้ดของการใช้ DScribe ในการสร้าง SOAP kernel

\begin{lstlisting}[style=MyPython]
from ase.build import molecule

# Molecule created as an ASE.Atoms
water = molecule("H2O")

# Create SOAP output for the system
soap_water = soap.create(water, positions=[0])

# Create output for multiple system
samples = [molecule("H2O"), molecule("NO2"), molecule("CO2")]
positions = [[0], [1, 2], [1, 2]]
# Serial
coulomb_matrices = soap.create(samples, positions)
# Parallel     
coulomb_matrices = soap.create(samples, positions, n_jobs=2)
\end{lstlisting}

\vspace{1em}

จากโค้ดด้านบนนั้นสิ่งที่เราจะได้ออกมาก็คือเมทริกซ์ที่แต่ละแถวนั้นจะบ่งบอกถึง Local Environment ของแต่ละอะตอมในโมเลกุล โดยความยาวของ Feature Vector ของ SOAP นั้นจะขึ้นอยู่กับจำนวนของ Species ที่เรากำหนดและขึ้นอยู่กับค่า $n_{\text{max}}$ และ $l_{\text{max}}$ ด้วย โดยผู้อ่านสามารถลองเปลี่ยนค่าของ $n_{\text{max}}$ และ $l_{\text{max}}$ เพื่อดูว่ามีผลต่อการเปลี่ยนค่าของ Feature Vector อย่างไร

นอกจากนี้เรายังสามารถเรียกใช้ Method \pyinline{derivatives} ของ Attribute \pyinline{soap} เพื่อคำนวณ Derivative ของ SOAP ได้ด้วย ดังนี้

\begin{lstlisting}[style=MyPython]
derivatives, descriptors = soap.derivatives(
    traj,
    positions=[[[0, 0, 0]]] * len(r),
    method="analytical"
)
\end{lstlisting}

%--------------------------
\section{ไลบรารี่สำหรับสร้างโมเดล}
\label{sec:lib_ml_model}
%--------------------------

\begin{table}[H]
    \centering
    \caption{รายชื่อและเว็บไซต์ของไลบรารี่โมเดล ML สำหรับเคมีควอนตัม}
    \label{tab:review_lib_ml_qm}
    \begin{tabular}{cll}
    \toprule
    \textbf{ไลบรารี่/โมเดล} &\textbf{เว็บไซต์} &\textbf{อ้างอิง} \\
    \midrule
    \multirow{2}{*}{AMP} &\url{https://amp.readthedocs.io/en/latest} &\multirow{2}{*}{Khorshidi and 
    Peterson\autocite{khorshidi2016}} \\ &\url{https://singroup.github.io/dscribe} & \\
    GAP &\url{https://github.com/libAtoms/QUIP} &Bart\'{o}k และ Cs\'{a}nyi\autocite{bartok2010} \\
    \multirow{2}{*}{SNAP} &\url{https://lammps.sandia.gov/doc/pair_snap.html} 
    &\multirow{2}{*}{Thompson และคณะ\autocite{thompson2015}} \\
    &\url{https://github.com/materialsvirtuallab/snap} & \\
    AENET &\url{http://ann.atomistic.net} &Artrith และ Urban\autocite{artrith2016} \\
    AGNI &\url{https://lammps.sandia.gov/doc/pair_agni.html} &Huan และคณะ\autocite{huan2017} \\
    PROPhet &\url{https://github.com/biklooost/PROPhet} &Kolb และคณะ\autocite{kolb2017} \\
    TensorMol &\url{https://github.com/jparkhill/TensorMol} &Yao และคณะ\autocite{yao2018} \\
    ANI &\url{https://github.com/isayev/ASE_ANI} &Smith และคณะ\autocite{smith2017} \\
    COMP6 &\url{https://github.com/isayev/COMP6} &Smith และคณะ\autocite{smith2018} \\
    DeePMD-kit95 &\url{https://github.com/deepmodeling/deepmd-kit} &Wang และคณะ\autocite{wang2018} \\
    VAMPnet &\url{https://github.com/markovmodel/deeptime} &Mardt และคณะ\autocite{mardt2018} \\
    CGCNN &\url{https://github.com/txie-93/cgcnn} &Xie และ Grossman\autocite{xie2018} \\
    ElemNet &\url{https://github.com/dipendra009/ElemNet} &Jha และคณะ\autocite{jha2018} \\
    OQMD-SC &\url{https://github.com/dipendra009/ElemNet} &Jha และคณะ\autocite{jha2019} \\
    \bottomrule
    \end{tabular}
\end{table}

%--------------------------
\subsection{SchNetPack}
\label{ssec:lib_schnetpack}
%--------------------------

SchNetPack เป็นชุดโปรแกรมสำหรับการฝึกสอนโมเดล Neural Network สำหรับการทำนายคุณสมบัติเชิงอะตอม\autocite{schutt2019} ถูกเขียนโดยภาษา Python 100\% และใช้ไลบรารี PyTorch เป็น Backend สำหรับการสร้างโมเดล SchNet ซึ่งเป็นตัวโมเดลหลักของ SchNetPack ผู้อ่านสามารถดาวน์โหลดซอร์สโค้ดของ SchNetPack และศึกษาวิธีการติดตั้งและใช้งานได้ที่เว็บไซต์ \url{https://schnetpack.readthedocs.io}

\noindent Feature หลัก ๆ ของ SchNetPack มีดังนี้
%
\begin{itemize}[topsep=0pt,noitemsep]\setlength\itemsep{0.5em}
    \item รองรับการสร้างโมเดล SchNet ซึ่งถูกพัฒนาโดยใช้อัลกอริทึม Convolutional Neural Network (CNN) สำหรับโมเลกุลโดยเฉพาะ\autocite{schutt2017,schutt2017a}
    
    \item รองรับการสร้างโมเดล PaiNN ซึ่งเป็น Equivariant Message-Passing สำหรับโมเลกุลเช่นเดียวกัน\autocite{schutt2021}
    
    \item สามารถทำนายค่าเอาต์พุตได้หลากหลาย เช่น Dipole Moments, Polarizability, Stress และคุณสมบัติอื่น ๆ ของโมเลกุล
    
    \item มีโมดูลสำหรับ Electrostatics และ Ewald Summation
    
    \item รองรับการเพิ่มความเร็วการคำนวณและฝึกสอนโมเดลด้วย GPU
\end{itemize}

%--------------------------
\subsection{sGDML}
\label{ssec:lib_sgdml}
%--------------------------

sGDML เป็นไลบรารี่ที่รับความนิยมในการสร้างโมเดลของโมเลกุลโดยการใช้ค่าพลังงานและแรงในการฝึกสอนโมเดล\autocite{chmiela2019} คู่มือการใช้งานอ่านได้ที่ \url{http://quantum-machine.org/gdml/doc}

\noindent เราสามารถติดตั้ง sGDML โดยใช้ Python Package Manager เช่น PIP ได้ด้วยคำสั่งต่อไปนี้\footnote{เพื่อป้องกันปัญหา Conflict ระหว่างไลบรารี่ใน Python Environment ผู้เขียนได้สร้าง Environment แยกขึ้นมาสำหรับ sGDML โดยเฉพาะซึ่งสามารถใช้ \inlinehighlight{venv} หรือ \inlinehighlight{conda} ในการสร้าง Environment ได้ครับ}

\begin{lstlisting}[style=MyBash]
pip install sgdml
\end{lstlisting}

\vspace{1em}
\noindent ตรวจสอบว่า sGDML ถูกติดตั้งและพร้อมใช้งานหรือไม่

\begin{lstlisting}[style=MyBash]
(sgdml) rangsiman@linux:~$ which sgdml
/home/rangsiman/sgdml/bin/sgdml

(sgdml) rangsiman@linux:~$ sgdml
usage: sgdml [-h] [--version] {all,create,train,validate,select,test,show,reset} ...
sgdml: error: the following arguments are required: command
\end{lstlisting}

%--------------------------
\subsubsection{ตัวอย่างการใช้งาน Force Field ที่ฝึกสอนด้วย sGDML}
%--------------------------

เมื่อเราฝึกสอนโมเดลด้วย sGDML แล้วสิ่งที่เราจะได้ออกมาก็คือโมเดล Force Field ที่เราสามารถนำไปใช้ในในการจำลองโมเลกุล เช่น การจำลองพลวัตโมเลกุลหรือ Molecular Dynamics (MD) ได้ซึ่ง Force Field ก็เปรียบเสมือนเป็นสิ่งที่คำนวณพลังงานและแรงของโมเลกุลให้กับ MD เพื่อนำไปใช้ในการขยับหรือเปลี่ยนตำแหน่งของโมเลกุลในเฟรม (Frame) ต่อ ๆ ไปในการจำลอง  โดยผู้อ่านสามารถดาวน์โหลดไฟล์โมเดล Force Field ที่ผ่านการฝึกสอนมาแล้วหรือ Pre-trained Model (\inlinehighlight{m_ethanol.npz}) ได้โดยใช้คำสั่ง

\begin{lstlisting}[style=MyBash]
(sgdml) rangsiman@linux:~$ sgdml-get model
\end{lstlisting}

ตัวอย่างด้านล่างคือโค้ดที่ใช้ Force Field ที่ถูกฝึกสอนด้วย sGDML กับไลบรารี่ ASE ในการจำลองโมเลกุล Ethanol

\begin{lstlisting}[style=MyPython]
from sgdml.intf.ase_calc import SGDMLCalculator

from ase.io import read
from ase.optimize import QuasiNewton
from ase.md.velocitydistribution import (MaxwellBoltzmannDistribution, Stationary, ZeroRotation)
from ase.md.verlet import VelocityVerlet
from ase import units

model_path = 'm_ethanol.npz'
calc = SGDMLCalculator(model_path)

mol = read('ethanol.xyz')
mol.set_calculator(calc)

# do a quick geometry relaxation
qn = QuasiNewton(mol)
qn.run(1e-4, 100)

# set the momenta corresponding to T=300K
MaxwellBoltzmannDistribution(mol, 300 * units.kB)
Stationary(mol) # zero linear momentum
ZeroRotation(mol) # zero angular momentum

# run MD with constant energy using the velocity verlet algorithm
dyn = VelocityVerlet(mol, 0.2 * units.fs, trajectory='md.traj')  # 0.2 fs time step.

# function to print the potential, kinetic and total energy
def printenergy(a):
    epot = a.get_potential_energy() / len(a)
    ekin = a.get_kinetic_energy() / len(a)
    print('Energy per atom: Epot = %.3feV  Ekin = %.3feV (T=%3.0fK)  '
            'Etot = %.3feV' % (epot, ekin, ekin / (1.5 * units.kB), epot + ekin))

# now run the MD simulation
printenergy(mol)
for i in range(10000):
    dyn.run(10)
    printenergy(mol)
\end{lstlisting}

\vspace{1em}
\noindent โดยผู้อ่านสามารถใช้พิกัดตำแหน่งเริ่มต้นของโมเลกุล Ethanol ดังต่อไปนี้ได้ (\inlinehighlight{ethanol.xyz})

\begin{lstlisting}[style=MyBash]  
9

C 	  0.423139 	  0.365862 	  0.380202
C 	 -0.565742 	  0.883093 	 -0.769835
O 	  0.153111 	 -1.105280 	  0.316890
H 	  1.470913 	  0.558882 	  0.104617
H 	  0.067699 	  0.805186 	  1.280056
H 	 -0.194544 	  0.628689 	 -1.745578
H 	 -1.498634 	  0.279985 	 -0.631160
H 	 -0.685436 	  1.905989 	 -0.598842
H 	  0.107894 	 -1.518377 	  1.203705
\end{lstlisting}

%--------------------------
\subsection{PiNN}
\label{ssec:lib_pinn}
%--------------------------

PiNN\autocite{shao2020} เป็น Python Library ที่รวบรวมโมเดลและวิธีในการทำนายคุณสมบัติต่าง ๆ ของโมเลกุลเข้าไปด้วยกัน โดยในเวอร์ชันปัจจุบันนั้นมี Neural Network ที่ถูกติดตั้งไว้ใน PiNN แล้ว 2 โมเดลคือ PiNet และ Behler-Parrinello Neural Network

ผู้อ่านสามารถติดตั้งไลบรารี่ PiNN ได้โดยใช้คำสั่งต่อไปนี้

\begin{lstlisting}[style=MyBash]
pip install git+https://github.com/Teoroo-CMC/PiNN
\end{lstlisting}

\noindent หรือ

\begin{lstlisting}[style=MyBash]
git clone https://github.com/Teoroo-CMC/PiNN.git 
cd PiNN && pip install -e .
\end{lstlisting}

รายละเอียดเพิ่มเติมดูได้ที่เว็บไซต์ \url{https://github.com/Teoroo-CMC/PiNN} และศึกษาคู่มือการใช้งานได้ที่ \url{https://teoroo-cmc.github.io/PiNN}

%--------------------------
\subsection{TorchANI}
\label{ssec:lib_torchani}
%--------------------------

TorchANI\autocite{gao2020} เป็น Neural Network ที่ใช้สำหรับการทำนายศักย์ (Potential) ของโมเลกุลซึ่งถูกพัฒนาโดยใช้ PyTorch Framework เป็นหลัก 

การติดตั้ง TorchANI สามารถทำได้โดยต้องติดตั้ง PyTorch ก่อนแล้ว ดังนี้

\begin{lstlisting}[style=MyBash]
# PyTorch
pip install numpy
DL_PT="https://download.pytorch.org/whl/nightly/cu100/torch_nightly.html"
pip install --pre torch torchvision -f $DOWNLOAD_PYTORCH

# TorchANI
pip install torchani
\end{lstlisting}

รายละเอียดเพิ่มเติมดูได้ที่เว็บไซต์ \url{https://github.com/aiqm/torchani} และศึกษาคู่มือการใช้งานได้ที่ 
\url{https://aiqm.github.io/torchani}
