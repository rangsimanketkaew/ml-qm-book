% LaTeX source for ``การเรียนรู้ของเครื่องสำหรับเคมีควอนตัม (Machine Learning for Quantum Chemistry)''
% Copyright (c) 2022 รังสิมันต์ เกษแก้ว (Rangsiman Ketkaew).

% License: Creative Commons Attribution-NonCommercial-NoDerivatives 4.0 International (CC BY-NC-ND 4.0)
% https://creativecommons.org/licenses/by-nc-nd/4.0/

\chapter{ลักษณะเฉพาะของอะตอมและโมเลกุล}
\label{ch:feature}

\begin{figure}[H]
    \centering
    \includegraphics[width=\linewidth]{fig/workflow_chem_ml.png}
    \caption{ขั้นตอนแสดงการสร้างโมเดลการเรียนรู้ของเครื่องเพื่อใช้ในการทำนายคุณสมบัติของโมเลกุล เริ่มจากการเปลี่ยนข้อมูลทางเคมีจากจากโครงสร้างของโมเลกุลไปเป็นข้อมูลแบบดิจิทัลที่คอมพิวเตอร์สามารถเข้าใจและประมวลผลต่อได้ ตามด้วยขั้นตอนการสร้างโมเดลสำหรับการเรียนรู้ และขั้นตอนสุดท้ายคือการทำนายคุณสมบัติของโมเลกุล (เครดิตภาพ: https://chemintelligence.com)}
    \label{fig:workflow_chem_ml}
\end{figure}

ในบทนี้เราจะมาดูความสำคัญของลักษณะเฉพาะ (Feature) ของโมเลกุลต่อประสิทธิภาพของโมเดล ML ในการทำนายคุณสมบัติของโมเลกุล\autocite{yang2019a} ซึ่งการคำนวณ Feature เป็นหนึ่งในขั้นตอนที่สำคัญมากของ ML ดังแสดงในภาพที่ \ref{fig:workflow_chem_ml} ซึ่งแสดงขั้นตอน (Workflow) ในการสร้างโมเดลเพื่อเชื่อมโยงความสัมพันธ์ระหว่างโครงสร้างของโมเลกุลกับคุณสมบัติทางเคมีของโมเลกุลนั้น ๆ จะเห็นได้ว่าทั้งสองอย่างนี้จริง ๆ แล้วมีความเชื่อมโยงกันผ่านโครงสร้างเชิงอิเล็กทรอนิกส์แต่ว่าความสัมพันธ์นั้นมีความซับซ้อนและการที่จะหาสมการทางคณิตศาสตร์ที่อธิบายความเชื่อมโยงนี้ได้นั้นเป็นเรื่องอาจจะทำได้ไม่ง่ายนัก ซึ่งปัญหาตรงนี้ ML ก็ได้เข้ามาช่วยในฐานะที่เป็นเครื่องมือหนึ่งที่พยายามสร้างสมการทางคณิตศาสตร์ในรูปแบบของฟังก์ชันที่ขึ้นอยู่กับตัวแปรที่อ้างอิงกับรูปแบบที่เกิดจากข้อมูลภายในชุดข้อมูลขนาดใหญ่โดยเชื่อมโยงผ่าน Feature นั่นเอง
\idxboth{โครงสร้างเชิงอิเล็กทรอนิกส์}{Electronic Structure}

Feature หรือ Representation คือคุณลักษณะที่บ่งบอกความเฉพาะตัวของอะตอมหรือโมเลกุลนั้น ๆ ซึ่งอาจจะเรียกว่าเป็นคุณลักษณะแบบพิเศษ (Special Attributes) ก็ได้ นอกจากนี้เราสามารถตีความได้ว่า Feature นั้นจริง ๆ แล้วก็เปรียบเสมือนเป็นตัวแทนของสิ่งที่เราสนใจอีกด้วย ซึ่งในบริบททางเคมีนั้นเราจะเรียกสิ่งที่เป็นตัวแทนของโมเลกุลว่า Molecular Representation อย่างไรก็ตามผู้เขียนมีความเห็นว่าจริง ๆ แล้ว Feature กับ Representation ก็ไม่ได้มีความหมายเหมือนกันเสียทีเดียว ขึ้นอยู่กับประเภทของข้อมูลที่ใช้เป็น Feature แต่ผู้เขียนจะใช้คำว่า Representation ในการอ้างถึงลักษณะเฉพาะของระบบที่เรากำลังศึกษา (อะตอม, โมเลกุล, และสารประกอบ) เพราะว่าให้ความหมายที่สื่อการเป็นตัวแทนของระบบที่เราสนใจได้ดีกว่า\autocite{stepisnik2021}
\idxboth{ลักษณะเฉพาะ}{Feature}
\idxboth{ตัวแทนของโมเลกุล}{Molecular Representation}
\idxboth{ตัวแทน}{Representation}

การแบ่งประเภทของ Feature หรือ Representation ในทางเคมีนั้นสามารถแบ่งได้หลายประเภท ขึ้นอยู่กับเกณฑ์ที่จะใช้ในการแบ่ง ความคิดเห็นส่วนตัวของผู้เขียนก็คือเราสามารถแบ่ง Feature ได้อย่างง่ายที่สุดเลยก็คือแบ่งตามสเกลของความจำเพาะเจาะจงของ Feature ที่คำนวณมาจากโมเลกุล กล่าวคือการที่เราต้องการที่จะทำนายคุณสมบัติของโมเลกุลนั้น เราควรจะต้องทราบก่อนว่าคุณสมบัติของโมเลกุลชนิดนั้นอยู่ในสเกลไหน โดยสเกลในที่นี้แบ่งง่าย ๆ เป็นระดับเฉพาะที่ (Local) หรือแบบทั่วทั้งพื้นที่ (Global) ตัวอย่างเช่น พลังงานรวมของโมเลกุลกับความสามารถในการละลายในน้ำเป็นคุณสมบัติในระดับ Global เพราะว่าแต่ละโมเลกุลมีค่าเหล่านี้ได้เพียงแค่หนึ่งค่าเท่านั้น แต่ถ้าหากเป็นคุณสมบัติอย่างเช่นแรงเชิงอะตอมหรือประจุย่อย คุณสมบัติเหล่านี้จะถูกจัดให้เป็นคุณสมบัติระดับ Local เพราะว่าอยู่ในสเกลระดับอะตอม เมื่อเราทราบแล้วว่าคุณสมบัติของเราเป็นแบบ Local หรือ Global เราก็สามารถที่จะพิจารณาเลือก Feature ที่อยู่ในระดับเดียวกันเพื่อมาใช้ในการฝึกสอนโมเดลได้ เพราะการที่เราใช้ Feature ที่อยู่ในระดับเดียวกับเอาต์พุตที่ต้องการทำนายนั้นจะทำให้การหาความสัมพันธ์ระหว่างสองสิ่งนี้ทำได้ง่ายและสมเหตุสมผล
\idxboth{ลักษณะเฉพาะ!แบบเฉพาะที่}{Feature!Local}
\idxboth{ลักษณะเฉพาะ!แบบทั่วทั้งพื้นที่}{Feature!Global}

%--------------------------
\section{ความสำคัญของลักษณะเฉพาะ}
\label{sec:why_feature}
%--------------------------

คำถามที่ตามมาก็คือ \enquote{\textit{Molecular Representation มีความสำคัญมากไหม และมีความสำคัญอย่างไร}} แน่นอนว่าคำตอบคือต้องมีความสำคัญอยู่แล้ว (เพราะถ้าหากไม่สำคัญผู้เขียนก็คงไม่เขียนหัวข้อนี้ใช่ไหมครับ) และมีความสำคัญมากด้วย โดยส่วนตัวของผู้เขียนนั้นคิดว่า Molecular Representation มีความสำคัญมากที่สุดเลยก็ว่าได้ เพราะ Representation ก็คืออินพุตที่เรานำมาใช้ฝึกสอนโมเดล ML นั่นเอง ดังนั้น Molecular Representation จึงเป็นปัจจัยหลักที่กำหนดประสิทธิภาพในการทำนายของโมเดลด้วย

เนื่องจากว่ามนุษย์สามารถแยกแยะโมเลกุลแต่ละตัวออกจากกันได้ แต่ว่าคอมพิวเตอร์ไม่สามารถทำได้เพราะว่าคอมพิวเตอร์เข้าใจข้อมูลที่เป็นแบบดิจิทัลในภาษาเครื่องจักรเท่านั้น (Machine Language) ดังนั้นเราจึงต้องมีการใช้ Representation เพื่ออธิบายโมเลกุลในรูปแบบของค่าพารามิเตอร์ที่คอมพิวเตอร์สามารถเข้าใจได้ เช่น แปลงโมเลกุลเป็นข้อมูลเชิงตัวเลข (Numeric) ให้อยู่ในรูปของเวกเตอร์หรือเมทริกซ์ สำหรับการอธิบายโมเลกุลแบบง่าย ๆ นั้นสามารถทำได้โดยใช้ Representation เพื่อมาอธิบายข้อมูลเชิงโครงสร้าง (Structural Properties) ซึ่งสามารถใช้ข้อมูลทางเคมีทั่วไปได้ ยกตัวอย่างเช่น รูปร่างของโมเลกุล, จำนวนหมู่ฟังก์ชัน, ชนิดของพันธะระหว่างอะตอมคาร์บอน, และจำนวนวงเบนซีน ฯลฯ ซึ่งข้อมูลเหล่านี้เราสามารถคำนวณออกมาได้ง่าย ๆ ไม่มีความซับซ้อนอะไร แต่ปัญหาคือ Representation ที่เป็น Structure-based นั้นมีข้อมูลที่น้อยเกินไป จึงทำให้ไม่สามารถถูกนำมาใช้เป็นอินพุตสำหรับการสร้างโมเดลเพื่อทำนายคุณสมบัติหรือพารามิเตอร์ทางเคมีที่ซับซ้อนหรือละเอียดกว่าได้ เช่น พลังงานพันธะ (Bond Energy), พลังงานของออร์บิทัล (Orbital Energy), ความถี่เชิงการสั่น (Vibrational Frequency), ไดโพลโมเมนต์ (Dipole Moment), ฯลฯ นั่นก็เพราะว่าอินพุตของเราเป็น Representation ที่ไม่มีความสัมพันธ์กับเอาต์พุตที่เราต้องการทำนาย (จริง ๆ ก็อาจจะสอดคล้องแต่ว่าไม่ได้สอดคล้องกันแบบโดยตรง)

\begin{figure}[H]
    \centering
    \includegraphics[width=\linewidth]{fig/descriptor_classes.png}
    \caption{แผนภาพแสดงการแบ่งประเภทของ Descriptor หรือตัวคำนวณคุณลักษณะเฉพาะของโมเลกุลโดยแบ่งตามจำนวนมิติของโมเลกุล (เครดิตภาพ: https://chemintelligence.com)}
    \label{fig:descriptor_classes}
\end{figure}

ดังนั้นถ้าหากเราต้องการที่จะทำนายเอาต์พุตที่เป็นปริมาณที่มีความละเอียดอยู่ในระดับอะตอม (ปริมาณเชิงอิเล็กทรอนิกส์) เราควรจะใช้ Representation ที่อยู่ในระดับเดียวกันและ Representation ควรจะต้องให้ Feature Vectors ที่เก็บข้อมูลทางเคมีควอนตัมและฟิสิกส์ไว้ด้วย โดยการพัฒนา Representation โดยใช้องค์ความรู้ทางฟิสิกส์ (Physics-inspired Representation) ก็เป็นหนึ่งในหัวข้องานวิจัยที่กำลังมาแรงในขณะนี้ ข้อมูลทางฟิสิกส์ที่เราเพิ่มเข้าไปก็เปรียบเสมือนเป็นส่วนเติมเต็มที่เพิ่มความถูกต้อง (เรียกอีกอย่างว่าการทำ Correction) ให้กับ Representation มากขึ้น โดยเราสามารถใส่ความเป็นสมมาตร (Symmetricity) หรือคุณสมบัติจากปรากฎการณ์ทางฟิสิกส์เชิงควอนตัมของโมเลกุลเข้าไปได้ เป็นต้น
\idxboth{ความเป็นสมมาตร}{Symmetricity}

%--------------------------
\section{การแปลงข้อมูลเชิงโมเลกุล}
\label{sec:mol_transform}
\idxth{การแปลงข้อมูลเชิงโมเลกุล}
%--------------------------

\begin{figure}[H]
    \centering
    \includegraphics[width=\linewidth]{fig/qm_descriptors_map.jpeg}
    \caption{แผนผังแสดงความเชื่อมโยงของ Descriptor ทั้งเชิงโครงสร้างและอิเล็กทรอนิกส์โดยเริ่มจาก Cartesian Coordinates ของโมเลกุลแล้วพัฒนาต่อเป็น Descriptor แบบต่าง ๆ
        (เครดิตภาพ: \textit{Chem. Rev.} \textbf{2021}, 121, 16, 9759-9815\autocite{musil2021})}
    \label{fig:qm_descriptor_map}
\end{figure}

โมเลกุลประกอบไปด้วยอะตอมหลายอะตอมมารวมกัน เราจึงเปรียบเทียบโมเลกุลเป็นประโยคหรือข้อความและเปนียบเทียบอะตอมเป็นคำแต่ละคำได้ ดังที่บอกไปในข้างต้นว่าการทำให้คอมพิวเตอร์เข้าใจความเชื่อมโยงระหว่างอะตอมในโมเลกุลนั้นต้องมีการเลือกใช้ Representation ที่เหมาะสม โดยคุณสมบัติของ Representation ที่ดีนั้นไม่เพียงแต่จะต้องไม่ขึ้นกับการเคลื่อนที่เชิงการหมุน (Rotational Motion) และการเคลื่อนที่เชิงเส้น (Translational Motion) เท่านั้น แต่ควรจะต้องมีความเรียบง่ายและไม่ซับซ้อนหรือยุ่งยากเกินไปในการคำนวณเพื่อสร้าง Machine Code จากพิกัดตำแหน่งคาร์ทีเซียน (Cartesian Coordinates) ด้วย

\begin{figure}[H]
    \centering
    \includegraphics[width=0.85\linewidth]{fig/mol_rep_invar.jpg}
    \caption{Invariance ของการเคลื่อนที่เชิงเส้น (Translation), การเคลื่อนที่เชิงการหมุน (Rotation) และการเปลี่ยนลำดับ (Permutation) ของโมเลกุล Formaldehyde}
    \label{fig:mol_rep_invariance}
\end{figure}

ภาพที่ \ref{fig:mol_rep_invariance} แสดงตัวอย่างของการเคลื่อนที่แบบต่าง ๆ รวมถึงการเปลี่ยนลำดับซึ่งคุณสมบัติเหล่านี้จะสอดคล้องกับสมมาตรของโมเลกุลด้วย โดยเงื่อนไขในการพิจารณาว่า Molecular Representation ไหนมีความเหมาะสมในการนำมาคำนวณ Feature Vectors มีดังต่อไปนี้
%
\begin{enumerate}[topsep=0pt,noitemsep]\setlength\itemsep{0.5em}
    \item การไม่ขึ้นกับการหมุน (Rotational Invariance): Representation จะต้องไม่ขึ้นกับโอเปอเรเตอร์ของการหมุน

    \item การไม่ขึ้นกับการเลื่อนตำแหน่ง (Translational Invariance): Representation จะต้องไม่เปลี่ยนไปเมื่อมีการเลื่อนตำแหน่งแบบเชิงเส้นภายในปริภูมิ

    \item การไม่ขึ้นกับการเปลี่ยนลำดับ (Permutation Invariance): Representation จะต้องไม่เปลี่ยนไปเมื่อมีการเปลี่ยนลำดับหรือสลับอะตอม
\end{enumerate}

%--------------------------
\section{ลักษณะเฉพาะเชิงโครงสร้างแบบทั่วไป}
\label{sec:struc_feat}
\idxth{ลักษณะเฉพาะ!เชิงโครงสร้างแบบทั่วไป}
\idxen{Representation!Structural Representation}
%--------------------------

%--------------------------
\subsection{Internal Coordinates}
\label{ssec:internal_coord}
\idxen{Representation!Internal Coordinates}
%--------------------------

พิกัดภายในของโมเลกุล (Internal Coordinates) หรือเรียกอย่างว่า $Z$ Matrix เป็น Representation พื้นฐานที่สุดในการอธิบายโครงสร้างของโมเลกุล (ไม่นับการใช้พิกัดของอะตอมแต่ละอะตอมโดยตรง) อาจจะเรียกได้ว่าเป็น Representation ที่เรียบง่ายที่สุดเลยก็ว่าได้ โดยถูกใช้อย่างแพร่หลายในยุคแรก ๆ ที่มีการนำ ML มาใช้สำหรับเคมีและยังถูกใช้มาอย่างยาวนานจึงถึงปัจจุบัน ข้อดีอย่างหนึ่งของ Internal Coordinates คือสามารถอธิบายได้ทั้งโมเลกุล โดยองค์ประกอบของ Representation อันนี้มีความยาวพันธะระหว่างอะตอม (Bond Distance) หรือ $d$, มุมพันธะ (Bond Angle) หรือ $\alpha$, และมุมบิดเบี้ยว (Dihedral Angle) หรือ $\theta$ ซึ่งอะตอมที่ถูกเลือกมาคำนวณ Internal Coordinates นั้นมักจะเป็นอะตอมที่เรียงติดกัน (มีพันธะเคมีระหว่างกัน) หรืออยู่ใกล้กัน อย่างไรก็ตามเราสามารถคำนวณหา Internal Coordinates ของโมเลกุลได้โดยพิจารณาอะตอมทุก ๆ คู่หรือทุกความเป็นไปได้ทั้งหมดภายในโมเลกุล โดยเซตของ Internal Coordinates สามารถเขียนได้ดังนี้
%
\begin{equation}\label{eq:internal_coord}
    Z = \{ d, \alpha, \theta \}
\end{equation}

%--------------------------
\subsection{Geometric Descriptors}
\label{ssec:geom_descriptor}
\idxth{ลักษณะเฉพาะ!เชิงเรขาคณิต}
\idxen{Representation!Geometric Descriptors}
%--------------------------

Geometric Descriptors (ลักษณะเฉพาะเชิงเรขาคณิต) เป็น Descriptor (จะเรียกแทนด้วย Representation ก็ได้) ที่อ้างอิงกับข้อมูลตำแหน่งของอะตอมในโมเลกุล โดยมักจะเชื่อมโยงกับ Representation แบบที่เป็นสัญลักษณ์ (Symbolic Representation) เช่น SMILES ซึ่ง Geometric Descriptors ก็สามารถแบ่งออกเป็นได้หลาย Descriptor ซึ่งก็รวมไปถึง $Z$ Matrix, Coordination Number, Adjacency Matrix อย่างไรก็ตาม Representation ในกลุ่มนี้มักจะให้ผลการทำนายด้วย ML ไม่ค่อยดีนัก นั่นก็เพราะว่าความสามารถในการกักเก็บข้อมูลเชิงอิเล็กทรอนิกส์นั้นน้อยมากเมื่อเทียบกับ Representation ประเภทที่เป็นแบบเชิงอะตอม (Atom-wise Descriptor) และยังมีโมเลกุลบางประเภทที่ Geometric Descriptors ไม่สามารถนำไปใช้ได้ อย่างเช่นโมเลกุลไอโซเมอร์ เช่น cis/trans สเตอริโอไอโซเมอร์ ดังนั้น Representation ประเภทนี้จึงไม่เป็นที่นิยมในการนำมาฝึกสอนโมเดล ML โดยเฉพาะโมเดลที่ใช้ในงานวิจัยทางด้านเคมีควอนตัม\autocite{keith2021,musil2021}

%--------------------------
\section{ลักษณะเฉพาะเชิงโครงสร้างสำหรับโมเลกุล}
\label{sec:struc_feat_mol}
\idxth{ลักษณะเฉพาะ!เชิงโครงสร้างสำหรับโมเลกุล}
\idxen{Representation!Molecular Representation}
%--------------------------

Representation ประเภทนี้จะเป็นการอธิบายสภาพแวดล้อม (Environment) ของอันตรกิริยาระหว่างอะตอมทุกอะตอมในโมเลกุล โดยมักจะอยู่ในรูปของเมทริกซ์ เช่น เมทริกซ์ของส่วนกลับของระยะห่างระหว่างอะตอม (Inverse Distance Matrix) และเมทริกซ์คูลอมบ์ (Coulomb Matrix)

%--------------------------
\subsection{Inverse Distance Matrix}
\label{ssec:inv_dist_mat}
\idxth{ลักษณะเฉพาะ!เมทริกซ์ของส่วนกลับของระยะห่างระหว่างอะตอม}
\idxen{Representation!Inverse Distance Matrix}
%--------------------------

Inverse Distance Matrix (เมทริกซ์ของส่วนกลับของระยะห่างระหว่างอะตอม) เป็น Representation Matrix แบบที่ง่ายที่สุดและมีความหมายทางเคมีที่ชัดเจน นั่นก็คือการใช้ส่วนกลับของระยะห่างระหว่างนิวเคลียสของอะตอมนั้นเป็นการจำลองเทอมของอันตรกิริยาระหว่างนิวเคลียสที่อยู่ใน Hamiltonian ของพลังงาน นิยามทางคณิตศาสตร์ของ Inverse Distance Matrix $(D)$ อธิบายได้ตามสมการต่อไปนี้
%
\begin{equation}
    D_{ij} = \frac{1}{||r_{i} - r_{j}||}
\end{equation}
%
เมื่อเราคำนวณ Inverse Distance ออกมาเป็นเมทริกซ์ขนาด $i \times j$ แล้ว เราจะพบว่าสมาชิกของเมทริกซ์ในแนวทแยง (Diagonal Elements) นั้นจะไม่มีความหมาย ดังนั้นเราจึงสนใจเฉพาะสมาชิกนอกแนวทแยง (Off-diagonal Elements)

%--------------------------
\subsection{Coulomb Matrix}
\label{ssec:coulomb_mat}
\idxth{ลักษณะเฉพาะ!เมทริกซ์คูลอมบ์}
\idxen{Representation!Coulomb Matrix}
%--------------------------

Coulomb Matrix (เมทริกซ์คูลอมบ์) เป็น Molecular Representation ที่ถูกเสนอครั้งแรกในปี ค.ศ. 2012 โดย Matthias Rupp และทีมวิจัย\autocite{rupp2012} โดยได้ถูกนำมาใช้อย่างแพร่หลายในงานวิจัยทางด้าน ML นั่นก็เพราะว่าไม่มีความสิ้นเปลืองในการคำนวณและให้ความแม่นยำในการทำนายค่าพลังงานของโมเลกุลสูง ซึ่ง Coulomb Matrix นั้นถูกพัฒนาขึ้นมาโดยมีพื้นฐานมาจาก Inverse Distance Matrix ซึ่งเป็นการแก้ปัญหาที่พบใน Distance Matrix สองส่วนดังนี้
%
\begin{enumerate}[topsep=0pt]
    \item มีการกำหนดเงื่อนไขในการคำนวณสมาชิกโดยกำหนดค่าของสมาชิกในแนวทแยง

    \item รวมประจุของอะตอมเข้าไปด้วย ซึ่งเป็นพารามิเตอร์สำคัญในการพัฒนา Force Field สำหรับการจำลองพลวัตเชิงโมเลกุล (Molecular Dynamics หรือ MD)
\end{enumerate}

สมการสำหรับการคำนวณสมาชิกของ Coulomb Matrix คือ
%
\begin{equation}\label{eq:cm}
    C_{ij} =
    \begin{cases}
        \frac{1}{2} Z_i^{2.4}  & \text{ if } i = j    \\
        \frac{Z_i Z_j}{R_{ij}} & \text{ if } i \neq j
    \end{cases}
\end{equation}

จากสมการที่ \eqref{eq:cm} จะเห็นได้ว่าเราได้แบ่งเงื่อนไขในการคำนวณสมาชิกของ Coulomb Matrix ออกเป็นสองเงื่อนไข สำหรับกรณีคู่อะตอมเหมือนกันและต่างกัน โดยที่พลังงานศักย์เชิงไฟฟ้าสถิตย์ของอะตอมแต่ละคู่นั้นจะถูกเข้ามาด้วย (ผ่านเทอมของประจุ) นั่นก็คือสมาชิกนอกแนวทแยงจะแสดงถึงแรงผลักคูลอมบ์ระหว่างอะตอม ในขณะที่สมาชิกในแนวทแยงจะเป็นการเทียบเคียงค่าเลขอะตอมกับพลังงานของอะตอมในกรณีที่ไม่ได้มีอันตรกิริยากับอะตอมตัวอื่น

ถึงแม้ว่า Coulomb Matrix จะเป็น Representation ที่ไม่ขึ้นกับการเลื่อนตำแหน่งและการหมุนของโมเลกุล แต่ว่าก็ยังขึ้นอยู่กับการเปลี่ยนตำแหน่งหรือการสลับที่กันของอะตอม เพื่อแก้ปัญหาดังกล่าว ได้มีนักวิจัยได้นำเสนอ Representation ตัวใหม่อีกหลายตัวที่เปรียบเสมือนเป็น Coulomb Matrix ที่ถูกปรับปรุงให้ดีขึ้น เช่น Sine Matrix\autocite{faber2015}, Ewald Sum Matrix\autocite{faber2015}, Permutation-Invariant Polynomials (PIP)\autocite{braams2009}, Randomly Sorted Coulomb Matrices (RSCM)\autocite{hansen2013}, Bag of Bonds (BoB)\autocite{hansen2013}, Permutation Invariant Vectors (PIV)\autocite{gallet2013} โดย Representation เหล่านี้ได้แก้ปัญหาเกี่ยวลำดับของอะตอม ทำให้ Representation ประเภทนี้มีประสิทธิภาพมากขึ้นและลด Bias ที่อาจจะเกิดขึ้นด้วย ซึ่งจะผู้อ่านจะได้ศึกษาต่อไป

ตัวอย่างโค้ดของการคำนวณ Coulomb Matrix โดยใช้ไลบรารี่ molml\autocite{collins2018}

\begin{lstlisting}[style=MyPython]
>>> from molml.features import CoulombMatrix
>>> feat = CoulombMatrix()
>>> H2 = (
...         ['H', 'H'],
...         [
...             [0.0, 0.0, 0.0],
...             [1.0, 0.0, 0.0],
...         ]
... )
>>> HCN = (
...         ['H', 'C', 'N'],
...         [
...             [-1.0, 0.0, 0.0],
...             [ 0.0, 0.0, 0.0],
...             [ 1.0, 0.0, 0.0],
...         ]
... )
>>> feat.fit([H2, HCN])
CoulombMatrix(input_type='list', n_jobs=1, sort=False, eigen=False, drop_values=False, only_lower_triangle=False)
>>> feat.transform([H2])
array([[ 0.5,  1. ,  0. ,  1. ,  0.5,  0. ,  0. ,  0. ,  0. ]])
>>> feat.transform([H2, HCN])
array([[0.5, 1. , 0. , 1. , 0.5,
        0. , 0. , 0. , 0. ],
       [0.5, 6. , 3.5, 6. , 36.8581052,
        42., 3.5, 42., 53.3587074]])
\end{lstlisting}

%--------------------------
\subsection{Sine Matrix}
\label{ssec:sine_matrix}
\idxen{Representation!Sine Matrix}
%--------------------------

Sine Matrix เป็น Representation ที่คำนวณอันตรกิริยาระหว่างอะตอมของระบบที่เป็นแบบ Perodic System และรวมผลของ Coulomb เข้าไปด้วย\autocite{faber2015} มีสมการดังนี้
%
\begin{equation}\label{eq:sine_matrix}
    M_{ij}^\text{sine} = \left \{
    \begin{matrix}
        \frac{1}{2} Z_i^{2.4} & \text{for } i = j    \\
        \\
        \frac{\textstyle Z_i Z_j}{\textstyle \lvert \bm{B} \cdot \sum_{k=\{x,y,z\}}
            \hat{\bm{e}}_k \sin^2
            \left ( \pi \bm{B}^{-1} \cdot \left ( \bm{R}_{i} - \bm{R}_{j} \right ) \right ) \rvert}
                              & \text{for } i \neq j
    \end{matrix}
    \right .
\end{equation}
%
\noindent โดยที่ $\bm{B}$ คือพารามิเตอร์ที่ถูกคำนวณมาจาก Lattice Vectors และ $\hat{\bm{e}}_k$ คือ Cartesian Unit Vectors

ตัวอย่างของโค้ดสำหรับการคำนวณ Sine Matrix Feature

\begin{lstlisting}[style=MyPython]
from dscribe.descriptors import SineMatrix

sm = SineMatrix(
    n_atoms_max=6,
    permutation="sorted_l2",
    sparse=False,
    flatten=True
)
\end{lstlisting}

\begin{figure}[H]
    \centering
    \includegraphics[width=0.75\linewidth]{fig/sine_matrix.png}
    \caption{อันตรกิริยาแบบ Periodic ที่คำนวณด้วย Sine Matrix}
    \label{fig:sine_matrix}
\end{figure}

%--------------------------
\subsection{Ewald Sum Matrix}
\label{ssec:ewald_sum_matrix}
\idxen{Representation!Ewald Sum Matrix}
%--------------------------

Ewald Sum Matrix เป็น Representation ที่พัฒนาต่อมาจาก Coulomb Matrix ซึ่งได้รวมผลของอันตรกิริยาระหว่างอะตอมแบบ Periodic โดยใช้อันตรกิริยาไฟฟ้าสถิตย์เข้าไปด้วย\autocite{faber2015}
%
\begin{equation}\label{eq:ewald_sum}
    \phi^{\text{self}}_{ij} + \phi^{\text{bg}}_{ij} 
    = 
    -\frac{\alpha}{\sqrt{\pi}}(Z_i^2 + Z_j^2)
    -\frac{\pi}{2V\alpha^2}(Z_i + Z_j)^2~\forall~i \neq j
\end{equation}
%
\noindent โดยที่ $\phi^{\text{bg}}_{ij}$ คือ Background Charge

ตัวอย่างของโค้ดสำหรับการคำนวณ Ewald Sum Matrix

\begin{lstlisting}[style=MyPython]
from dscribe.descriptors import EwaldSumMatrix

atomic_numbers = [1, 8]
rcut = 6.0
nmax = 8
lmax = 6

# Setting up the Ewald sum matrix descriptor
esm = EwaldSumMatrix(
    n_atoms_max=6,
)
\end{lstlisting}

%--------------------------
\subsection{Bag of Bonds}
\label{ssec:bob}
\idxen{Representation!Bag of Bonds}
%--------------------------

นอกจากนี้ยังมี Bag of Bonds (BoB)\autocite{hansen2015} ซึ่งเป็น Representation ที่พัฒนาต่อจาก Coulomb Matrix โดยจะมีการจัดกลุ่มประเภทของพันธะเข้าด้วยกัน ซึ่งได้แรงบันดาลใจในการพัฒนามาจาก Bag of Words ที่ใช้ใน Natural Language Processing (NLP) โดยคำว่า \textit{Bag} ในที่นี้คือประเภทของพันธะเคมีที่แตกต่างกันในโมเลกุลนั้น ๆ เช่น \ce{C-C}, \ce{C-O}, และ \ce{C-H} ส่วน \textit{Bond} นั้นจะแยกตามชนิดของพันธะ คือพันธะเดี่ยว พันธะคู่ และพันธะสาม

ตัวอย่างโค้ดของการคำนวณ Bag of Bonds ของโมเลกุล Methane\footnote{พิกัดคาร์ทีเซียนดูได้ที่ \url{https://en.wikipedia.org/wiki/Z-matrix_(chemistry)}} โดยใช้ไลบรารี่ qml\autocite{zotero-1799} โดยสามารถดูรายละเอียดเพิ่มเติมได้ที่\footnote{https://www.qmlcode.org}

\begin{lstlisting}[style=MyPython]
>>> import qml
>>> CH4 = qml.Compound(xyz="methane.xyz")
>>> CH4.generate_bob(asize={"C":4, "H":8})
>>> print(CH4.representation)
[36.8581052   0.          0.          0.          0.          0.          0.          0.
  0.          0.          5.50964209  5.50964187  5.50963981  5.50963981  0.          0.
  0.          0.          0.          0.          0.          0.          0.          0.
  0.          0.          0.          0.          0.          0.          0.          0.
  0.          0.          0.          0.          0.          0.          0.          0.
  0.          0.          0.5         0.5         0.5         0.5         0.          0.
  0.          0.          0.56232548  0.56232539  0.56232539  0.56232533  0.56232532  0.56232532
  0.          0.          0.          0.          0.          0.          0.          0.
  0.          0.          0.          0.          0.          0.          0.          0.
  0.          0.          0.          0.          0.          0.        ]
\end{lstlisting}

%--------------------------
\section{ลักษณะเฉพาะเชิงอิเล็กทรอนิกส์สำหรับอะตอม}
\label{sec:elec_feat}
\idxth{ลักษณะเฉพาะ!เชิงอิเล็กทรอนิกส์สำหรับอะตอม}
\idxen{Representation!Electronic Descriptor}
%--------------------------

นอกเหนือจาก Representation ที่เรานำมาอธิบายโมเลกุลแบบทั้งโมเลกุล ยังมี Representation แบบอื่นซึ่งมีความซับซ้อนมากกว่าที่ถูกพัฒนาขึ้นมาเพื่ออธิบาย Environment ของโมเลกุลในระดับอะตอมแบบเฉพาะเจาะจง ซึ่ง Representation แบบนี้จะเป็นการรวมความสำคัญของกฎทางฟิสิกส์และเคมีแบบต่าง ๆ เข้ามาไว้ด้วยกัน จึงทำให้ Representation ประเภทนี้มีความสามารถในการที่จะอธิบายโครงสร้างเชิงอิเล็กทรอนิกส์ของโมเลกุลได้ดีมาก ๆ โดยเฉพาะการทำนายคุณสมบัติในระดับอะตอม โดยในปัจจุบันก็ได้มีการพัฒนา Representation ที่สามารถอธิบายอะตอมแบบทีละอะตอม ซึ่งเราเรียก Representation ประเภทนี้ว่า Atom-wise ดังนั้นผู้เขียนจะขออธิบายเฉพาะ Representation ที่มีความโดดเด่นน่าสนใจและเป็นที่นิยมในการนำมาใช้ในงานวิจัยเท่านั้น

การพัฒนา Representation สำหรับอธิบายโครงสร้างเชิงอิเล็กทรอนิกส์ของโมเลกุลนั้นสามารถใช้องค์ความรู้ที่เกี่ยวข้องกับความหนาแน่น (Density) ของโมเลกุล\autocite{yang1991}, เทคนิค Linear Scaling ในเชิงคำนวณ\autocite{galli1992,goedecker1999}, หลักการมองเห็นระยะสั้น (Nearsightedness Principle)\autocite{prodan2005} รวมไปถึงการพิจารณาสมมาตร (Symmetry) ของโมเลกุล เพื่อมาใช้ในการออกแบบและปรับปรุงประสิทธิภาพของ Representation เพื่อให้ครอบคลุมและอธิบายปรากฏการณ์ทางควอนตัมให้ได้มากที่สุด\autocite{ceriotti2018}

%--------------------------
\subsection{Smooth Overlap of Atomic Positions}
\label{ssec:soap}
\idxen{Representation!Smooth Overlap of Atomic Positions}
%--------------------------

Representation ที่เป็นประเภทเชิงอิเล็กทรอนิกส์สำหรับอะตอมอันแรกที่จะพูดถึงก็คือ Smooth Overlap of Atomic Positions (SOAP) ซึ่งเป็นตัวที่โด่งดังมากแล้วก็มีการนำมาใช้อย่างแพ่หลาย เรียกได้ว่านักวิจัยเคมีควอนตัมและปัญญาประดิษฐ์ต่างก็รู้จัก SOAP กันเป็นอย่างดี โดยบทความงานวิจัยของ SOAP ได้ถูกตีพิมพ์ครั้งแรกในปี ค.ศ. 2013 ซึ่งไอเดียของ SOAP ก็คือจะเป็นการนำข้อมูลโครงสร้างของอะตอมมาเข้ารหัส (Encoding) ไว้ได้โดยได้รวมสภาพแวดล้อมทางเคมีโดยการใช้ความหนาแน่นเชิงอะตอมแบบเกาส์เซียน (Gaussian Smeared Atomic Density) ซึ่ง Atomic Density ที่ว่านี้ก็ถูกคำนวณมาจากฟังก์ชันพื้นฐานเชิงรัศมี (Radial Basis Function) หรือ $g_{n}(r)$ และฟังก์ชันฮาร์โมนิคเชิงทรงกลม (Real Spherical Harmonic Functions) หรือ $Y_{lm}(\theta, \phi)$\autocite{bartok2013,de2016} โดย SOAP นั้นเหมาะสำหรับนำมาใช้ทำนายคุณสมบัติของโมเลกุลแบบเฉพาะที่ (Local Properties) อย่างเช่นแรงเชิงอะตอม (Atomic Force) หรือ Chemical Shift

การคำนวณ SOAP นั้นจริง ๆ แล้วสามารถทำได้ผ่านการคำนวณ Kernel ของ Atomic Environment 2 อันเข้าด้วยกัน $(\mathcal{X}$ และ $\mathcal{X}')$ ซึ่งเขียนออกมาได้ในรูปของ Polynomial Kernel $(K^\text{SOAP}$ ของพารามิเตอร์ที่ชื่อว่า Partial Power Spectra $(\bm{p}$ และ $\bm{p}')$\footnote{ผู้เขียนไม่แน่ใจว่าจะแปลเป็นภาษาไทยอย่างไรดี ต้องขออภัยผู้อ่านด้วยครับ}

โดยสมการที่เรานำมาใช้ในการคำนวณ SOAP นั้นคือ
%
\begin{equation}\label{eq:soap}
    K^\text{SOAP}(\bm{p}, \bm{p'}) 
    = 
    \left( \frac{\bm{p} \cdot \bm{p'}}{\sqrt{\bm{p} \cdot \bm{p} \;\;\; \bm{p'} \cdot \bm{p'}}}\right)^{\xi}
\end{equation}
%
\noindent โดยที่กำหนดให้ $\xi$ เป็นจำนวนเต็มบวกและสมาชิกของเวกเตอร์ $\bm{p}$ มีนิยามคือ
%
\begin{equation}\label{eq:soap_power_spec}
    p^{Z_1 Z_2}_{n n' l} 
    = 
    \pi \sqrt{\frac{8}{2l+1}}\sum_m {c^{Z_1}_{n l m}}^{\dagger} c^{Z_2}_{n' l m}
\end{equation}
%
\noindent โดยที่ $n$ และ $n'$ เป็นดัชนีสำหรับ Radial Basis Function ซึ่งมีค่าได้สูงสุดถึง $n_{\max}$, $l$ เป็นดีกรีเชิงมุม (Angular Degree) ของ Spherical Harmonics ซึ่งมีค่าได้สูงสุดถึง $l_{\max}$, $m$ เป็นจำนวนเต็มที่สอดคล้องกับเงื่อนไขคือ $\abs{m} \leq l$, และ $Z_{1}$ และ $Z_{2}$ เป็นสปีชีส์เชิงอะตอม (Atomic Species) นอกจากนี้ค่าสัมประสิทธิ์ $c^{Z}_{n'lm}$ และ ${c^{Z}_{nlm}}^{\dagger}$ ถูกนิยามให้เป็นผลคูณภายในของ Spherical Harmonic Functions กับความหนาแน่นเชิงอะตอมแบบเกาส์เซียน $(\rho^Z)$ และคอนจูเกตเชิงซ้อน (Complex Conjugate) ตามลำดับ\autocite{de2016} ดังนี้
%
\begin{equation}
    c^{Z}_{nlm}(\bm{r}) 
    = 
    \iiint_{\mathcal{R}^{3}} dV g_{n}(r) Y_{lm}(\theta, \phi) \rho^{Z}(\bm{r})
\end{equation}
%
\noindent โดยที่ $\bm{r}$ คือตำแหน่งในปริภูมิและ $\rho^Z(\bm{r})$ คือความหนาแน่นเชิงอะตอมแบบเกาส์เซียน (Gaussian Smoothed Atomic Density) สำหรับอะตอม $Z$

ในการคำนวณ $c^{Z}_{nlm}(\bm{r})$ นั้นเราจะต้องนั้นเราจะต้องมีการกำหนด Radial Basis Function (RBF) ที่เราต้องการใช้ด้วย โดยเราสามารถใช้ฟังก์ชัน Spherical Gaussian-type Orbitals (สมการที่ \eqref{eq:rbf_gto}) หรือฟังก์ชัน Polynomial Basis (ตามสมการที่ \eqref{eq:rbf_poly}) ก็ได้
%
\begin{equation}\label{eq:rbf_gto}
    g_{nl}(r) = \sum_{n'=1}^{n_{\max}}\,\beta_{nn'l} r^l e^{-\alpha_{n'l}r^2}
\end{equation}
%
\begin{equation}\label{eq:rbf_poly}
    g_{n}(r) = \sum_{n'=1}^{n_{\max}}\,\beta_{nn'} (r-r_\text{cut})^{n'+2}
\end{equation}
%
\noindent และในส่วนของ Spherical Harmonics นั้นเราสามารถใช้เฉพาะส่วนจริง (Real Part) ได้ (เพราะว่าง่ายต่อการ Implement)
โดยรูปแบบหรือสมการของ Spherical Harmonics มีดังนี้
%
\begin{align}\label{eq:spher_harmo}
    Y_{\ell m} & = \begin{cases}
                       \dfrac{i}{\sqrt{2}} \left(Y_\ell^{m} - (-1)^m\, Y_\ell^{-m}\right) & \text{if}\ m < 0 \\
                       Y_\ell^0                                                           & \text{if}\ m=0   \\
                       \dfrac{1}{\sqrt{2}} \left(Y_\ell^{-m} + (-1)^m\, Y_\ell^{m}\right) & \text{if}\ m > 0
                   \end{cases}       \\
               & = \begin{cases}
                       \dfrac{i}{\sqrt{2}} \left(Y_\ell^{-|m|} - (-1)^{m}\, Y_\ell^{|m|}\right) & \text{if}\ m < 0 \\
                       Y_\ell^0                                                                 & \text{if}\ m=0   \\
                       \dfrac{1}{\sqrt{2}} \left(Y_\ell^{-|m|} + (-1)^{m}\, Y_\ell^{|m|}\right) & \text{if}\ m>0
                   \end{cases} \\
               & = \begin{cases}
                       \sqrt{2} \, (-1)^m \, \Im [{Y_\ell^{|m|}}] & \text{if}\ m<0 \\
                       Y_\ell^0                                   & \text{if}\ m=0 \\
                       \sqrt{2} \, (-1)^m \, \Re [{Y_\ell^m}]     & \text{if}\ m>0
                   \end{cases}
\end{align}

เนื่องจากว่ารายละเอียดในการพิสูจน์สมการ Kernel ของ SOAP นั้นมีความซับซ้อนพอสมควร ผู้เขียนจึงขอแนะนำให้ผู้อ่านที่สนใจศึกษาเพิ่มเติมทฤษฎีของ SOAP ได้ที่บทความต้นฉบับ\autocite{bartok2013}

ตัวอย่างโค้ดสำหรับการตั้งค่า SOAP Representation และคำนวณ SOAP ของโมเลกุลน้ำโดยใช้ไลบรารี่ DScribe และ ASE

\begin{lstlisting}[style=MyPython]
from dscribe.descriptors import SOAP
from ase.build import molecule

species = ["H", "C", "O", "N"]
r_cut = 6.0
n_max = 8
l_max = 6

# Setting up the SOAP descriptor
soap = SOAP(
    species=species,
    periodic=False,
    r_cut=r_cut,
    n_max=n_max,
    l_max=l_max,
)

# Molecule created as an ASE.Atoms
water = molecule("H2O")

# Create SOAP output for the system
soap_water = soap.create(water, positions=[0])

print(soap_water)
print(soap_water.shape)
\end{lstlisting}

\vspace{1em}

\begin{figure}[H]
    \centering
    \includegraphics[width=\linewidth]{fig/soap_matrix.jpg}
    \caption{ตัวอย่างของ SOAP Matrix ของระบบ Cu/fcc Lattices สำหรับ (A) Cubic และ (B) Orthorhombic}
    \label{fig:soap_matrix}
\end{figure}

\vspace{1em}

รายละเอียดเพิ่มเติมสำหรับการคำนวณและการเขียนโปรแกรมสำหรับคำนวณ SOAP นั้นสามารถดูได้ที่
\url{https://singroup.github.io/dscribe/latest/tutorials/descriptors/soap.html}

%--------------------------
\subsection{Atom-centered Symmetry Functions}
\label{ssec:acsf}
\idxen{Representation!Atom-centered Symmetry Functions}
%--------------------------

Representation ลำดับถัดมาคือฟังก์ชันสมมาตรแบบมีศูนย์กลางบนอะตอม (Atom-centered Symmetry Functions หรือ ACSF) เป็นวิธีที่ทำการสร้างผลลัพธ์หรือคำตอบจากฟังก์ชันหลายวัตถุ (Many-body Functions) อีกทีหนึ่งเพื่อทำการประมาณค่า Electronic Environment รอบ ๆ อะตอมที่เราสนใจในโมเลกุล ซึ่ง ACSF นี้ได้ถูกพัฒนาโดยศาสตราจารย์ J\"{o}rg Behler ซึ่งตีพิมพ์ในปี ค.ศ. 2011 โดยถูกออกแบบเพื่อนำมาใช้กับโมเดลประเภท Neural Network \autocite{behler2011a}

ไอเดียของ ACSF ก็คือจะเป็นการแปลง (Transformation) พิกัดคาร์ทีเซียนให้เป็นฟังก์ชันสมมาตร (Symmetry Function) โดยจะมีการกำหนดฟังก์ชันสำหรับ Cutoff ขึ้นมาก่อน ดังนี้

\begin{equation}\label{eq:acsf_cutoff}
    f_{c}(R_{ij}) =
    \begin{cases}
        \frac{1}{2} \left[\cos(\frac{\textstyle \pi R_{ij}}{\textstyle R_{c}}) + 1 \right], & R_{ij} \le R_{c} \\
        0,                                                                                  & R_{ij} \ge R_{c}
    \end{cases}
\end{equation}

\noindent โดยที่ $R_{ij}$ คือระยะห่างระหว่างอะตอม $i$ กับ $j$ และ $R_{c}$ คือค่า Cutoff

Symmetry Function นั้นเป็นฟังก์ชันที่สามารถอธิบายกรณีแบบหลายวัตถุ (Many-body) แต่ว่าจริง ๆ แล้วมีชื่อเรียก Symmetry Function เฉพาะ นั่นคือ Symmetry Function ที่เป็นฟังก์ชันแบบสองวัตถุ (Two-body Function) นั้นเราจะเรียกว่าฟังก์ชันเชิงรัศมี (Radial Function) และเรียกที่เป็นฟังก์ชันแบบสามวัตถุ (Three-body Function) ว่าฟังก์ชันเชิงมุม (Angular Function) ตามลำดับ ซึ่งความแตกต่างหลัก ๆ ก็คือ Radial Function จะเป็นผลรวมเชิงเส้นของเทอมที่เป็น Two-body ส่วน Angular Function นั้นจะมีการเพิ่มเทอมที่เป็น Three-body เข้ามาด้วยนั่นเอง
\idxboth{ฟังก์ชันสมมาตร}{Symmetry Function}
\idxboth{ฟังก์ชันสมมาตร!ฟังก์ชันเชิงรัศมี}{Symmetry Function!Radial Function}
\idxboth{ฟังก์ชันสมมาตร!ฟังก์ชันเชิงมุม}{Symmetry Function!Angular Function}

โดยเรามาดูกันที่ Radial Function ก่อนซึ่งจะมีด้วยกัน 3 ฟังก์ชันย่อย ดังนี้
%
\begin{equation}\label{eq:rf_g1}
    G^{1}_{i} = \sum_{j} f_{c}(R_{ij})
\end{equation}
%
\begin{equation}\label{eq:rf_g2}
    G^{2}_{i} = \sum_{j} e^{-\eta(R_{ij} - R_{s})^{2}} f_{c}(R_{ij})
\end{equation}
%
\begin{equation}\label{eq:rf_g3}
    G^{3}_{i} = \sum_{j} \cos(\kappa R_{ij}) f_{c}(R_{ij})
\end{equation}
%
\noindent ซึ่ง $G^{1}_{i}$ ก็คือฟังก์ชันผลรวมของ Cutoff Function (สมการที่ \eqref{eq:acsf_cutoff}), $G^{2}_{i}$ คือผลรวม Gaussian Function คูณด้วย Cutoff Function และ $G^{3}_{i}$ คือฟังก์ชัน Cosine ที่ถูกปรับการหน่วงโดยค่าพารามิเตอร์ $\kappa$

นอกจากนี้ยังมี Angular Function อีกสองฟังก์ชันย่อยด้วยกัน นั่นคือฟังก์ชัน $G^{4}_{i}$
%
\begin{align}\label{eq:rf_g4}
    G^{4}_{i} =~ & 2^{1 - \xi}\sum^{\text{all}}_{j,k \neq i} (1+\lambda \cos \theta_{ijk})^{\xi}
    \cdot e^{-\eta(R^{2}_{ij} + R^{2}_{ik} + R^{2}_{jk})} \nonumber                              \\
                 & \cdot f_{c}(R_{ij}) \cdot f_{c}(R_{ik}) \cdot f_{c}(R_{jk})
\end{align}
%
\noindent และฟังก์ชัน $G^{5}_{i}$
%
\begin{align}\label{eq:rf_g5}
    G^{5}_{i} =~ & 2^{1 - \xi}\sum^{\text{all}}_{j,k \neq i} (1+\lambda \cos \theta_{ijk})^{\xi}
    \cdot e^{-\eta(R^{2}_{ij} + R^{2}_{ik})} \nonumber                                           \\
                 & \cdot f_{c}(R_{ij}) \cdot f_{c}(R_{ik}))
\end{align}

สำหรับการคำนวณหาแรง (Force) และเทนเซอร์แรงผลัก (Stress Tensor) สำหรับกรณีของ ACSF นั้นสามารถทำได้โดยใช้กฎลูกโซ่ (Chain Rule) ซึ่งต้องการทำพิสูจน์จากสมการของพลังงานที่เกิดขึ้นจากการคำนวณใน Neural Network โดยสำหรับค่าพลังงานย่อยต่อหนึ่งหน่วย Neuron นั้นมีนิยามตามนี้
%
\begin{equation}\label{eq:acsf_sub_ener}
    E_{\text{Neuron}} = f(w G + b)
\end{equation}
%
\noindent โดยที่ $f$ คือฟังก์ชันกระตุ้น, $w$ คือค่าน้ำหนัก (Weight Parameter), $b$ คือค่าความอคติ (Bias Parameter) และ $G$ คืออินพุตของ Neuron นั้น ๆ ซึ่งก็คือ Symmetry Function ตามด้านบนที่ได้อธิบายไป โดยที่พลังงานรวมทั้งหมดของโมเลกุล (Total Energy) หรือ $E$ สามารถคำนวณได้จากผลรวมของพลังงานย่อย ๆ ที่เกิดขึ้นจากแต่ละอะตอม (สมการที่ \eqref{eq:acsf_sub_ener}) ซึ่งมีสมการดังนี้
%
\begin{equation}
    E = \sum_{i} E_{i}
\end{equation}
%
นอกจากนี้ยังมี Representation อื่น ๆ ที่คล้ายกับ ACSF นั่นคือถูกพัฒนาขึ้นมาด้วยไอเดีย Symmetry Function เหมือนกัน เช่น Spectrum of London and Axilrod-Teller-Muto (SLATM) ซึ่งถูกใช้อย่างแพร่หลายเหมือนกันแต่จะนิยมใช้กับ KRR มากกว่า\autocite{faber2018,huang2020} แล้วก็ยังมีการใช้ฟังก์ชันพหุนาม (Polynomial functions) ของส่วนกลับของระยะห่างระหว่างอะตอม (Inverse Bond Distance) ด้วย\autocite{kwac2019,musil2021}

ตัวอย่างโค้ดสำหรับการตั้งค่า ACSF Representation และคำนวณ ACSF ของโมเลกุลน้ำโดยใช้ไลบรารี่ DScribe และ ASE

\begin{lstlisting}[style=MyPython]
from dscribe.descriptors import SOAP
from ase.build import molecule

# Setting up the ACSF descriptor
acsf = ACSF(
    species=["H", "O"],
    rcut=6.0,
    g2_params=[[1, 1], [1, 2], [1, 3]],
    g4_params=[[1, 1, 1], [1, 2, 1], [1, 1, -1], [1, 2, -1]],
)

# Molecule created as an ASE.Atoms
water = molecule("H2O")

# Create ACSF output for the hydrogen atom at index 1
acsf_water = acsf.create(water, positions=[1])

print(acsf_water)
print(acsf_water.shape)
\end{lstlisting}

\vspace{1em}

นอกจากนี้แล้ว ACSF ยังได้ถูกนำมาพัฒนาและปรับปรุงให้มีประสิทธิภาพในการทำนายคุณสมบัติเชิงอิเล็กทรอนิกส์แบบเฉพาะ เช่น Weighted ACSF\autocite{gastegger2018}, การใช้ Polynomial Function มาเป็นฟังก์ชันเสริม\autocite{bircher2021}, Physically-inspired ACSF\autocite{zhang2021} ซึ่งเป็นการช่วยลดตัวแปรที่ต้องกำหนดสำหรับ Symmetry Function, และ Spin-dependent ACSF สำหรับทำนายโมเลกุลที่มีคุณสมบัติเชิงแม่เหล็ก\autocite{eckhoff2021}

%--------------------------
\subsection{Gaussian-type Orbital-based Density Vectors}
\label{ssec:gauss_orb_den}
\idxen{Representation!Gaussian-type Orbital-based Density Vectors}
%--------------------------

เวกเตอร์ของความหนาแน่นเชิงออร์บิทัลประเภทเกาส์เซียน (Gaussian-type Orbital-based Density Vector หรือเรียกสั้น ๆ ว่า GTB-based Density Vector) เป็น Representation อีกอันหนึ่งที่ใช้หลักการของ Molecular/Atomic Orbitals ซึ่งถูกพัฒนาขึ้นมาเพื่อให้เป็นอีกทางเลือกหนึ่งนอกเหนือจาก ACSF และ Symmetric Polynomial Function\autocite{kwac2021} โดยสมการที่ใช้คำนวณ GTO-based Density Vector คือ
%
\begin{equation}\label{eq:gto_density}
    \rho^{i}_{L,\alpha,r_{s}} = \sum^{l_{x}+l_{y}+l_{z} = L}_{l_{x},l_{y},l_{z}}
    \frac{L!}{l_{x}!l_{y}!l_{z}!} \left ( \sum^{n_{type}}_{t=1} c_{t} \sum^{N^{t}_{atom}}_{j=1}
    \phi^{\alpha,r_{s}}_{l_{x}l_{y}l_{z}} (\bm{r}_{ij}) \right )
\end{equation}
%
\noindent โดยกำหนดให้ $l_{x}+l_{y}+l_{z} = L$ เป็นเลข Angular Momentum ของออร์บิทัล, $n_{type}$ เป็นชนิดของอะตอมในโมเลกุล, $c_{t}$ เป็นค่าถ่วงน้ำหนักที่ขึ้นกับชนิดของอะตอม (Type-dependent Weight), $N^{t}_{atom}$ คือจำนวนของอะตอมชนิดนั้น ๆ และ $\phi^{\alpha,r_{s}}_{l_{x}l_{y}l_{z}} (r_{ij})$ เป็นออร์บิทัลแบบเกาส์เซียนของแต่ละอะตอม นอกจากนี้ยังมีการกำหนดพารามิเตอร์เพิ่มเติมคือ $\alpha$ และ $r_{s}$ ซึ่งเป็นตัวกำหนดฟังก์ชันออร์บิทัลของ Radial Distribution\autocite{kwac2021} โดยออร์บิทัลแบบเกาส์เซียนมีสมการดังต่อไปนี้
%
\begin{equation}\label{eq:gaussian_orb}
    \phi^{\alpha,r_{s}}_{l_{x}l_{y}l_{z}} (\bm{r}_{ij}) = x^{l_{x}}y^{l_{y}}z^{l_{z}} e^{-\alpha
    |r-r_{s}|^{2}}
\end{equation}
%
\noindent โดยกำหนดให้ $\bm{r}_{ij} = (x,y,z)$ เป็นเวกเตอร์จากอะตอม $i$ ไปยังอะตอม $j$ และ $r$ เป็นขนาด (Magnitude) ของ $\bm{r}_{ij}$ โดยทั่วไปแล้วเรามักจะกำหนดให้ค่า $L=0,1$ สำหรับการสร้าง Density Vector ด้วย GTO สำหรับโมเลกุลขนาดเล็ก เช่น โมเลกุลเคมีอินทรีย์ที่ประกอบไปด้วยอะตอมขนาดเล็ก เช่น คาร์บอน (C), ไฮโดรเจน (H), ออกซิเจน (O), และไนโตรเจน (N)

%--------------------------
\subsection{ลักษณะเฉพาะเชิงอิเล็กทรอนิกส์อื่น ๆ}
\label{ssec:other_feat_elec}
%--------------------------

นอกจากนี้ยังมีลักษณะเฉพาะเชิงอิเล็กทรอนิกส์อีกมากมายที่ถูกพัฒนาขึ้นมา\autocite{faber2018} ดังนี้
%
\begin{itemize}[topsep=0pt,noitemsep]\setlength\itemsep{0.5em}
    \item Bonds and Angles based Machine Learning (BAML)\autocite{huang2016}

    \item Histogram of Distances, Angles, and Dihedrals (HDAD)\autocite{faber2017}

    \item Many-body Tensor Representation (MBTR)\autocite{huo2022,langer2022}

    \item Faber-Christensen-Huang-Lilienfeld (FCHL)\autocite{faber2018}

    \item Atomic Cluster Expansion (ACE)\autocite{drautz2019,kovacs2021}

    \item N-body iterative contraction of equivariants (NICE)\autocite{nigam2020}
\end{itemize}

เมื่ออ่านบทความวิจารณ์ (Review) ต่าง ๆ แล้วผู้อ่านจะพบว่าไม่มี Descriptor ไหนที่ให้ผลการทำนายคุณสมบัติเชิงโมเลกุลทุกคุณสมบัติได้แม่นยำที่สุด กล่าวคือ Descriptor แต่ละอันนั้นมีความสามารถในการทำนายคุณสมบัติที่แตกต่างกันไป ถ้าเปรียบเทียบง่าย ๆ ก็เหมือนกับ Functional ของวิธี DFT ซึ่งแต่ละ Functional ก็มีข้อดีและข้อเสียแตกต่างกันไป ดังนั้นผู้เขียนควรศึกษาความแตกต่างของ Descriptor แต่ละอันว่ามีทฤษฎีที่อยู่เบื้องหลังที่ใช้ในการพัฒนา Descriptor นั้น ๆ เป็นอย่างไร จะช่วยทำให้เราเข้าใจและสามารถเลือกใช้ Descriptor ที่เหมาะกับคุณสมบัติเชิงโมเลกุลได้อย่างเหมาะสมและให้ผลการทำนายที่ถูกต้องและแม่นยำ ซึ่งจะช่วยประหยัดเวลาโดยไม่ต้องทำการสุ่มหรือทดลองเปลี่ยน Descriptor ไปเรื่อย ๆ นั่นเอง

นอกจากนี้ผู้เขียนขอแนะนำผู้อ่านที่สนใจการศึกษา Descriptor แบบเชิงลึกได้อ่านบทความงานวิจัย \enquote{Physics-Inspired Structural Representations for Molecules and Materials}\autocite{musil2021} ที่ได้สรุป Descriptor (ในบทความต้นฉบับผู้วิจัยใช้คำว่า Representation ซึ่งเป็นสิ่งเดียวกัน) ที่เกี่ยวข้องกับ Atomic Composition สำหรับศึกษาคุณสมบัติเชิงเคมีควอนตัมของโมเลกุลไว้อย่างครบถ้วน รวมไปถึงสรุปคุณสมบัติของ Descriptor ที่ดีไว้ด้วย
