% LaTeX source for ``การเรียนรู้ของเครื่องสำหรับเคมีควอนตัม (Machine Learning for Quantum Chemistry)''
% Copyright (c) 2022 รังสิมันต์ เกษแก้ว (Rangsiman Ketkaew).

% License: Creative Commons Attribution-NonCommercial-NoDerivatives 4.0 International (CC BY-NC-ND 4.0)
% https://creativecommons.org/licenses/by-nc-nd/4.0/

{
% \pagenumbering{gobble}

\chapter*{\centering กิตติกรรมประกาศ}
\addcontentsline{toc}{chapter}{กิตติกรรมประกาศ}

ความรู้และแรงบัลดาลใจในการเขียนหนังสือเล่มนี้ของผู้เขียนมาจากแรงผลักดันและการสนับสนุนของบุคคลหลายท่าน 
การเขียนหนังสือเล่มนี้จะไม่เกิดขึ้นหรือสำเร็จไม่ได้ถ้าหากขาดบุคคลดังต่อไปนี้

ครอบครัวของผู้เขียนที่สนับสนุนให้ผู้เขียนได้ทำตามความฝันในการเรียนต่อระดับอุดมศึกษา ทั้งในระดับปริญญาโทและปริญญาเอก 
โดยเฉพาะการเห็นคุณค่าของการเรียนและการทำงานวิจัยทางด้านวิทยาศาสตร์พื้นฐาน

รศ.ดร. ยุทธนา ตันติรุ่งโรจน์ชัย บุคคลผู้เป็นต้นแบบด้านการเรียนและเป็นผู้สร้างแรงบันดาลใจให้ผู้เขียนเรียนต่อต่างประเทศและทำงานวิจัยทางด้าน%
เคมีทฤษฎีและเคมีคอมพิวเตอร์

อาจารย์และเพื่อน ๆ ในช่วงมัธยมศึกษา (ตอนต้น-ปลาย) ที่โรงเรียนพนัสพิทยาคาร และช่วงปริญญาตรี-โท ที่มหาวิทยาลัยธรรมศาสตร์ 
สำหรับความทรงจำอันดีงามและความเป็นกัลยาณมิตรที่ดีเสมอมา

เพื่อน ๆ ที่เมืองซูริค ประเทศสวิตเซอร์แลนด์ สำหรับมิตรภาพอันดีงาม รอยยิ้มและเสียงหัวเราะที่เกิดขึ้น รวมไปถึงกิจกรรมที่ได้ทำร่วมกันในระหว่าง%
ที่ผู้เขียนกำลังศึกษาปริญญาเอกซึ่งเป็นช่วงเวลาเดียวกันกับที่ผู้เขียนกำลังเขียนหนังสือเล่มนี้

เพื่อนร่วมงานทั้งนักศึกษาปริญญาโท ปริญญาเอก และนักวิจัยหลังปริญญาเอกของกลุ่มวิจัยของผู้เขียนที่ภาควิชาเคมี
มหาวิทยาลัยแห่งซูริค สำหรับการแลกเปลี่ยนความรู้ ไอเดียใหม่ ๆ และการช่วยเหลือเกี่ยวกับงานวิจัยทางด้านเคมีทฤษฎี

\medskip

\begin{flushright}
รังสิมันต์ เกษแก้ว
\end{flushright}
}
