% LaTeX source for ``การเรียนรู้ของเครื่องสำหรับเคมีควอนตัม (Machine Learning for Quantum Chemistry)''
% Copyright (c) 2022 รังสิมันต์ เกษแก้ว (Rangsiman Ketkaew).

% License: Creative Commons Attribution-NonCommercial-NoDerivatives 4.0 International (CC BY-NC-ND 4.0)
% https://creativecommons.org/licenses/by-nc-nd/4.0/

{
\pagenumbering{gobble}

\chapter*{\centering กิตติกรรมประกาศ}

ความรู้และแรงบัลดาลใจในการเขียนหนังสือเล่มนี้ของผู้เขียนมาจากแรงผลักดันและการสนับสนุนของบุคคลหลายท่าน 
การเขียนหนังสือเล่มนี้จะไม่เกิดขึ้นหรือสำเร็จไม่ได้ถ้าหากขาดบุคคลดังต่อไปนี้

ขอขอบคุณครอบครัวของผู้เขียนที่สนับสนุนให้ผู้เขียนได้ทำตามความฝันในการเรียนต่อระดับอุดมศึกษา ทั้งในระดับปริญญาโทและปริญญาเอก 
โดยเฉพาะการเห็นคุณค่าของการเรียนและการทำงานวิจัยทางด้านวิทยาศาสตร์พื้นฐานซึ่งเป็นสิ่งสำคัญและมีประโยชน์ต่อสังคม

ขอขอบคุณ รศ.ดร. ยุทธนา ตันติรุ่งโรจน์ชัย บุคคลผู้เป็นต้นแบบด้านการเรียนและสนับสนุนผู้เขียนทางด้านการศึกษาโดยไม่เรียกร้องหรือหวังสิ่งตอบแทนใด ๆ 
อาจารย์ยุทธนาเป็นผู้สร้างแรงบันดาลใจให้ผู้เขียนเรียนต่อต่างประเทศและทำงานวิจัยทางด้านเคมีทฤษฎีและเคมีคอมพิวเตอร์

ขอขอบคุณอาจารย์และเพื่อน ๆ ในช่วงมัธยมศึกษาตอนต้น-ปลาย ที่โรงเรียนพนัสพิทยาคาร และช่วงปริญญาตรี-โท ที่มหาวิทยาลัยธรรมศาสตร์ 
สำหรับความทรงจำอันดีงามและความเป็นกัลยาณมิตรที่ดีเสมอมา

ขอขอบคุณเพื่อน ๆ ที่เมืองซูริค ประเทศสวิตเซอร์แลนด์ สำหรับมิตรภาพอันดีงาม รอยยิ้มและเสียงหัวเราะที่เกิดขึ้น 
รวมไปถึงกิจกรรมที่ได้ทำร่วมกันในระหว่างที่ผู้เขียนกำลังศึกษาปริญญาเอกซึ่งเป็นช่วงเวลาเดียวกันกับที่ผู้เขียนกำลังเขียนหนังสือเล่มนี้

\medskip

\begin{flushright}
รังสิมันต์ เกษแก้ว
\end{flushright}
}
