% LaTeX source for ``การเรียนรู้ของเครื่องสำหรับเคมีควอนตัม (Machine Learning for Quantum Chemistry)''
% Copyright (c) 2022 รังสิมันต์ เกษแก้ว (Rangsiman Ketkaew).

% License: Creative Commons Attribution-NonCommercial-NoDerivatives 4.0 International (CC BY-NC-ND 4.0)
% https://creativecommons.org/licenses/by-nc-nd/4.0/

\chapter{การเรียนรู้แบบมีผู้สอน}
\label{ch:sup_ml}

การเรียนรู้แบบมีผู้สอนหรือ Supervised learning เป็นเทคนิคแรก ๆ ที่ถูกพัฒนาขึ้นมาในช่วงยุคเริ่มต้นของ ML ซึ่งเป็นแนวคิดที่ใช้ Input และ 
Output ในการเทรนโมเดล ซึ่งผู้เขียนมีความคิดเห็นส่วนตัวว่าการสร้าง Model ประเภทนี้จะง่าย ๆ ที่สุดทั้งในแง่ทฤษฎี การเรียนรู้ และการนำไปใช้ 
โดยเทคนิคนี้ได้รับความนิยมมากที่สุดนั่นก็เพราะว่าสามารถนำไปประยุกต์ใช้งานกับโจทย์ที่หลากหลาย

%--------------------------
\section{Partial Least Squares (PLS)}
%--------------------------

วิธีกำลังสองน้อยที่สุดบางส่วน (partial least squares: PLS) 
เป็นวิธีเชิงสถิติที่ใช้สำหรับการวิเคราะห์หลาย ตัวแปรเพื่อสร้างตัวแบบความสัมพันธ์ระหว่างกลุ่มของ ตัวแปรทำนาย (Predictor variable) 
โดยอาศัยตัวแปรแฝง (Latent variable) ซึ่งเทคนิคนี้มีความคล้ายกับ Principle Component Analysis (PCA) 
ซึ่งจะเป็นการลดจำนวนมิติของข้อมูล.\cite{wold1984} ในช่วงยุคเริ่มต้นของ AI ในด้านเคมี เทคนิคนี้ได้ถูกนำมาใช้อย่างแพร่หลาย เช่น 
นำมาใช้สำหรับการระบุ vibrational bands สำหรับ vibrational spectra และนำผลที่ได้มาเปรียบเทียบกับค่าการทำนายที่ได้จากวิธีอื่น เช่น
ANN และ PCA-ANN.

%--------------------------
\section{Gaussian Process Regression (GPR)}
%--------------------------

การถดถอยของกระบวนการเกาส์เซี่ยน เป็นวิธีการถดถอยของเบส์แบบหนึ่งโดยใช้ Kernel function ที่สามารถบ่งบอกหรือแสดงค่าความแปรปรวน (covariance) 
ในขั้นตอน Gaussian process ได้\cite{rasmussen2005} โดย GPR จะทำการสร้างโมเดลแบบ non-parametric และสามารถคำนวณค่าความเชื่อมั่น 
(Confidence intervals) ไปพร้อม ๆ กับการทำนาย

%--------------------------
\section{Random Forest}
%--------------------------

Random Forest (RF) เป็นวิธีหนึ่งในกลุ่มของโมเดลที่เรียกว่า Ensemble learning (การเรียนรู้แบบกลุ่มก้อน) 
ที่มีหลักการคือการเทรนโมเดลที่เหมือนกันหลาย ๆ ครั้ง (Multitude) บนข้อมูลชุดเดียวกัน โดยแต่ละครั้งของการเทรนจะเลือกส่วนของข้อมูลที่เทรนไม่เหมือนกัน 
แล้วเอาการตัดสินใจของโมเดลเหล่านั้นมาโหวตกันว่า Class ไหนถูกเลือกมากที่สุด\cite{breiman2001,quinlan1986}

%--------------------------
\section{Artificial Neural Network}
%--------------------------

Artificial Neural Network (ANN) โครงสร้างประสาทเทียมเป็นอัลกิริธึมรูปแบบหนึ่งที่เลียนแบบการทำงานของสมองมนุษย์
โดยทำการสร้างโมเดลเรียนรู้ที่ประกอบไปด้วยชั้นเรียนรู้ระหว่างกลาง (Hidden Layer) และหน่วยย่อยที่เกิดการเรียนรู้ (Node หรือ Artiticial neuron หรือ Unit)

โดยโมเดล ANN ที่มีการนำไปใช้มากที่สุดคือข่ายงานประสาทแบบป้อนไปหน้า (Feedforward Network) และ ANN ยังสามารถแบ่งออกเป็น
หลายประเภท ดังนี้

\begin{itemize}
    \item เพอร์เซ็ปตรอนชั้นเดียว (Single-layer Perceptron)
    \item เพอร์เซ็ปตรอนหลายชั้น (Multi-layer Perceptron
    \item โครงข่ายแบบวนซ้ำ (Recurrent Network)
    \item แผนผังจัดระเบียบเองได้ (Self-organizing Map)
    \item เครื่องจักรโบลทซ์แมน (Boltzmann Machine)
    \item กลไกแบบคณะกรรมการ (Committee of Machines)
    \item โครงข่ายความสัมพันธ์ (Associative Neural Network)
    \item โครงข่ายกึ่งสำเร็จรูป (Instantaneously Trained Networks)
    \item โครงข่ายแบบยิงกระตุ้น (Spiking Neural Networks) 
\end{itemize}
