% LaTeX source for ``การเรียนรู้ของเครื่องสำหรับเคมีควอนตัม (Machine Learning for Quantum Chemistry)''
% Copyright (c) 2022 รังสิมันต์ เกษแก้ว (Rangsiman Ketkaew).

% License: Creative Commons Attribution-NonCommercial-NoDerivatives 4.0 International (CC BY-NC-ND 4.0)
% https://creativecommons.org/licenses/by-nc-nd/4.0/

% LaTeX source for ``ปัญญาประดิษฐ์สำหรับเคมีควอนตัม (Machine Learning for Quantum Chemistry)''
% Copyright (c) 2022 รังสิมันต์ เกษแก้ว (Rangsiman Ketkaew).

% License: Creative Commons Attribution-NonCommercial-NoDerivatives 4.0 International (CC BY-NC-ND 4.0)
% https://creativecommons.org/licenses/by-nc-nd/4.0/

\documentclass[a4paper,12pt,twoside,openany]{book}
\usepackage[
    width=5.5in,
    height=8.5in,
    hmarginratio=3:2,
    vmarginratio=1:1
]{geometry}

% Adjust size of the body text
\AtBeginDocument{\fontsize{14}{16.2}\selectfont}

\usepackage[T1]{fontenc}
\usepackage{textcomp}
\usepackage{url}
\usepackage{graphicx}
\usepackage{float}
\usepackage{subcaption}
\usepackage{booktabs}  % For nicely typeset tabular material
\usepackage{multirow}  % For multple rows table
\usepackage{changepage,threeparttable}  % For wide tables

\usepackage{amsmath}
\usepackage{amsthm}
\usepackage{amssymb}
\usepackage{mathtools}
\usepackage{physics}
\usepackage{bm}
\usepackage{algorithm}
\usepackage{algpseudocode}
% \usepackage{exercise}
\usepackage{setspace}
\usepackage{csquotes}
\usepackage{enumitem}
\usepackage{hyphenat}
\usepackage[bookmarks]{hyperref}
% import lstlisting setting
% LaTeX source for ``ปัญญาประดิษฐ์สำหรับเคมีควอนตัม (Machine Learning for Quantum Chemistry)''
% Copyright (c) 2022 รังสิมันต์ เกษแก้ว (Rangsiman Ketkaew).

% License: Creative Commons Attribution-NonCommercial-NoDerivatives 4.0 International (CC BY-NC-ND 4.0)
% https://creativecommons.org/licenses/by-nc-nd/4.0/

%%%%%%% List of commands: %%%%%%%
% Code block:
% \begin{lstlisting}[style=MyBash]{}
% \begin{lstlisting}[style=MyPython]{}
% \begin{lstlisting}[style=MyC++]{}
% \begin{lstlisting}[style=MyJSON]{}
% ----------------------------------
% Code in line:
% \bashinline{}
% \pyinline{}
% \cppinline{}
% \inlinehighlight{}
%%%%%%%%%%%%%%%%%%%%%%%%%%%%%%%%%

\usepackage{listings} % for code listing
\usepackage{xcolor} % for color

% define color
\colorlet{shadecolor}{gray!10} % increase the value will produce darker color
\colorlet{punct}{red!60!black}
\colorlet{numb}{magenta!60!black}
\definecolor{mymauve}{rgb}{0.58,0,0.82}
\definecolor{deepblue}{rgb}{0,0,0.5}
\definecolor{deepred}{rgb}{0.6,0,0}
\definecolor{deepgreen}{rgb}{0,0.5,0}
\definecolor{pythoncolor}{RGB}{102,102,255}

\lstset{
  backgroundcolor=\color{shadecolor},   % choose the background color; you must add \usepackage{color} or \usepackage{xcolor}; should come as last argument
  basicstyle=\ttfamily\footnotesize\linespread{0.5},        % the size of the fonts that are used for the code
  keywordstyle=\color{blue}\ttfamily,       % keyword style
  commentstyle=\color{pink}\ttfamily,    	   % comment style
  breaklines=true,                 % sets automatic line breaking
  breakatwhitespace=true,         % sets if automatic breaks should only happen at whitespace
  captionpos=b,                    % sets the caption-position to bottom
  deletekeywords={...},            % if you want to delete keywords from the given language
  escapeinside={\%*}{*)},          % if you want to add LaTeX within your code
  extendedchars=true,              % lets you use non-ASCII characters; for 8-bits encodings only, does not work with UTF-8
%   firstnumber=1000,                % start line enumeration with line 1000
  frame=tlrb,	                   % adds a frame around the code, use a combination of t l r and b
  frameshape={RYR}{Y}{Y}{RYR},     % rounded corner
  keepspaces=true,                 % keeps spaces in text, useful for keeping indentation of code (possibly needs columns=flexible)
  columns=flexible,
%   basewidth={.88em},
  numbers=left,                    % where to put the line-numbers; possible values are (none, left, right)
  numbersep=10pt,                   % how far the line-numbers are from the code
  numberstyle=\normalsize\color{gray}, % the style that is used for the line-numbers
  rulecolor=\color{lightgray},         % if not set, the frame-color may be changed on line-breaks within not-black text (e.g. comments (green here))
  showspaces=false,                % show spaces everywhere adding particular underscores; it overrides 'showstringspaces'
  showstringspaces=false,          % underline spaces within strings only
  showtabs=false,                  % show tabs within strings adding particular underscores
  stepnumber=1,                    % the step between two line-numbers. If it's 1, each line will be numbered
  stringstyle=\color{mymauve},     % string literal style
  tabsize=4,	                   % sets default tabsize to 2 spaces
  title=\lstname,                   % show the filename of files included with \lstinputlisting; also try caption instead of title
  xleftmargin = 0.8cm,             % left margin for the code
  xrightmargin = -0.5cm,           % right margin for the code
  framexleftmargin = 2em,          % left margin for the whole frame
%   aboveskip=3mm,
  belowskip=-1.5 \baselineskip,
}

\lstdefinestyle{plain}{
    language=Bash,
    keywordstyle=\color{black},
    stringstyle=\color{black},
    commentstyle=\color{black},
}

\lstdefinestyle{MyBash}{
    language=Bash,
    keywordstyle=\color{blue},
    stringstyle=\color{black},
    commentstyle=\color{pink},
    morekeywords={}, % for letter 
    otherkeywords={}, % for non-letter
    deletekeywords={cd,echo,enable,export,jobs,local,source,test},
    morecomment=[l][\color{magenta}]{\#},
}

\lstdefinestyle{MyPython}{
    language=Python,
    keywordstyle=\color{deepblue},
    emph={MyClass,__init__},          % Custom highlighting
    emphstyle=\color{deepred},    % Custom highlighting style
    stringstyle=\color{deepgreen},
    commentstyle=\color{pink},
    morekeywords={self},              % Add keywords here
    morecomment=[l][\color{magenta}]{\#},
}

\lstdefinestyle{MyC++}{
    language=C++,
    keywordstyle=\color{blue},
    stringstyle=\color{red},
    commentstyle=\color{pink},
    morekeywords={},
    morecomment=[l][\color{magenta}]{\/\/},
}

\lstdefinestyle{MyFortran}{
    language=Fortran,
    keywordstyle=\color{blue},
    stringstyle=\color{red},
    commentstyle=\color{green},
    morecomment=[l][\color{magenta}]{!\ } % Comment only with space after !
}

\lstdefinelanguage{MyJSON}{
    stringstyle=\color{black},
    morestring=[b]",
    literate=
     *{0}{{{\color{numb}0}}}{1}
      {1}{{{\color{numb}1}}}{1}
      {2}{{{\color{numb}2}}}{1}
      {3}{{{\color{numb}3}}}{1}
      {4}{{{\color{numb}4}}}{1}
      {5}{{{\color{numb}5}}}{1}
      {6}{{{\color{numb}6}}}{1}
      {7}{{{\color{numb}7}}}{1}
      {8}{{{\color{numb}8}}}{1}
      {9}{{{\color{numb}9}}}{1}
      {:}{{{\color{punct}{:}}}}{1}
      {,}{{{\color{punct}{,}}}}{1}
      {\{}{{{\color{mymauve}{\{}}}}{1}
      {\}}{{{\color{mymauve}{\}}}}}{1}
      {[}{{{\color{mymauve}{[}}}}{1}
      {]}{{{\color{mymauve}{]}}}}{1},
}

\lstdefinestyle{nonumber}{
    number=none,
}

% for skipping lines with dots
% https://tex.stackexchange.com/questions/476658/how-to-skip-lines-in-lstlisting-with-dots
%------------------------
\let\origthelstnumber\thelstnumber
\makeatletter
\newcommand*\Suppressnumber{%
  \lst@AddToHook{OnNewLine}{%
    \let\thelstnumber\relax%
  }%
}

\newcommand*\Reactivatenumber{%
  \lst@AddToHook{OnNewLine}{%
   \let\thelstnumber\origthelstnumber%
  }%
}
\makeatother
%------------------------

% define code in line with highlighting
\newcommand{\bashinline}[1]{\colorbox{shadecolor}{\lstinline[style=MyBash]{#1}}}
\newcommand{\pyinline}[1]{\colorbox{shadecolor}{\lstinline[style=MyPython]{#1}}}
% inline highlight using a special color
\newcommand{\inlinehighlight}[1]{\colorbox{shadecolor}{\lstinline[
    style=MyPython,
    basicstyle=\ttfamily\small\linespread{0.5}\color{pythoncolor},
]{#1}}}
\newcommand{\cppinline}[1]{\colorbox{shadecolor}{\lstinline[style=MyC++]{#1}}}


% Table of Contents
\usepackage{tocloft}
\renewcommand{\contentsname}{\hspace*{\fill}\bfseries\huge สารบัญ\hspace*{\fill}}
\setcounter{tocdepth}{1} % Level of TOC
% https://tex.stackexchange.com/a/397486/117615
\renewcommand\cftchappresnum{บทที่ }
\renewcommand\cftchapafterpnum{\vskip10pt}
\newlength\mylen
\settowidth\mylen{\bfseries บทที่ :\hspace{3em}}
\cftsetindents{chap}{0pt}{\mylen}

\renewcommand\cftsecafterpnum{\vskip10pt}
\renewcommand\cftsubsecafterpnum{\vskip10pt}

% Header
\usepackage{fancyhdr}
\fancyhf{}
\fancyhead[LE,RO]{\textbf{\thepage}}
\fancyhead[RE,LO]{\textbf{\leftmark}}

% Part
\renewcommand\thepart{\arabic{part}}

% Chapter
\renewcommand{\partname}{ส่วนที่}
\renewcommand{\chaptername}{บทที่}

% Section & Subsection
\usepackage{float} % for [H] option

% Appendix
\usepackage[toc,page]{appendix}
\renewcommand{\appendixname}{ภาคผนวก}
\renewcommand{\appendixtocname}{ภาคผนวก}
\renewcommand{\appendixpagename}{ภาคผนวก}

% Figure & Table Captions
\usepackage{caption}
\captionsetup{labelsep=space,justification=centering,singlelinecheck=off}
\captionsetup[figure]{name={ภาพ}}
\captionsetup[table]{name={ตาราง}}

% Paragraph
\usepackage{indentfirst}
\setlength{\parindent}{2em}
\setlength{\parskip}{1em}

% Indices
\usepackage{imakeidx}
\makeindex
\makeindex[name=th,title={ดรรชนีภาษาไทย}]
\makeindex[name=en,title={ดรรชนีภาษาอังกฤษ}]
\newcommand{\idxth}[1]{\index[th]{#1}}
\newcommand{\idxen}[1]{\index[en]{#1}}
\newcommand{\idxboth}[2]{\index[th]{#1}\index[en]{#2}}

% URL font
\urlstyle{same} % use the same font as main font

%----------------------------------------------------

%%% Option 1. Font for Thai %%%

\usepackage[utf8]{inputenc}
\usepackage{xltxtra} % this will load the fontspec, metalogo, and realscripts packages
\XeTeXlinebreaklocale "th_TH"
\XeTeXlinebreakskip = 0pt plus 1pt  
\defaultfontfeatures{Scale=1.23}
% \setmainfont{TH Sarabun New}
\setmainfont[
  Renderer=Basic,
  BoldFont={THSarabunNew_Bold.ttf},
  ItalicFont={THSarabunNew_Italic.ttf},
  BoldItalicFont={THSarabunNew_BoldItalic.ttf},
]{THSarabunNew.ttf} 

%%% Option 2. Fonts for different languages %%%
% Fonts need to be installed on the system

% \usepackage{fontspec,xunicode}
% \defaultfontfeatures{Mapping=tex-text}
% \setmainfont[Renderer=Basic, Numbers=OldStyle, Scale = 1.0]{TeX Gyre Pagella}
% \setmathrm[Renderer=Basic, Scale=0.90]{TeX Gyre Pagella}
% \setsansfont[Renderer=Basic, Scale=0.90]{TeX Gyre Heros} 
% \setmonofont[Renderer=Basic]{TeX Gyre Cursor}
% \newfontfamily{\thaifont}{TH Sarabun New}[Scale=MatchLowercase] 

% \usepackage[Latin, Thai]{ucharclasses}
% \setTransitionTo{Thai}{\thaifont}
% \setTransitionFrom{Thai}{\normalfont}

% \XeTeXlinebreaklocale "th_TH"
% \XeTeXlinebreakskip = 0pt plus 1pt  

% \defaultfontfeatures{Scale=1.23}

%----------------------------------------------------

\usepackage{tikz}
%\usepackage{etoolbox} % for \ifthen
\usepackage{listofitems} % for \readlist to create arrays
\usetikzlibrary{arrows.meta} % for arrow size
\usepackage[outline]{contour} % glow around text
\contourlength{1.4pt}

\tikzset{>=latex} % for LaTeX arrow head
\usepackage{xcolor}
\colorlet{myred}{red!80!black}
\colorlet{myblue}{blue!80!black}
\colorlet{mygreen}{green!60!black}
\colorlet{myorange}{orange!70!red!60!black}
\colorlet{mydarkred}{red!30!black}
\colorlet{mydarkblue}{blue!40!black}
\colorlet{mydarkgreen}{green!30!black}
\tikzstyle{node}=[thick,circle,draw=myblue,minimum size=22,inner sep=0.5,outer sep=0.6]
\tikzstyle{node in}=[node,green!20!black,draw=mygreen!30!black,fill=mygreen!25]
\tikzstyle{node hidden}=[node,blue!20!black,draw=myblue!30!black,fill=myblue!20]
\tikzstyle{node convol}=[node,orange!20!black,draw=myorange!30!black,fill=myorange!20]
\tikzstyle{node out}=[node,red!20!black,draw=myred!30!black,fill=myred!20]
\tikzstyle{connect}=[thick,mydarkblue] %,line cap=round
\tikzstyle{connect arrow}=[-{Latex[length=4,width=3.5]},thick,mydarkblue,shorten <=0.5,shorten >=1]
\tikzset{ % node styles, numbered for easy mapping with \nstyle
  node 1/.style={node in},
  node 2/.style={node hidden},
  node 3/.style={node out},
}
\def\nstyle{int(\lay<\Nnodlen?min(2,\lay):3)} % map layer number onto 1, 2, or 3

%----------------------------------------------------

% Bibliography
\usepackage[
  backend=bibtex,
  autocite=superscript,
  sorting=none,
  style=numeric, 
  natbib=true,
  mincitenames=1,
  doi=false,
  url=false, 
  isbn=false,
  backref=true
]{biblatex}
\addbibresource{references.bib}


%% Avoid create page after chapter
\let\cleardoublepage\clearpage
% \let\clearpage\relax

% Book metadata
\title{การเรียนรู้ของเครื่องสำหรับเคมีควอนตัม}
\author{รังสิมันต์ เกษแก้ว}

\begin{document}

% Book front cover
\includepdf[pages=-]{cover_front.pdf}

\newpage
\ % empty page
\pagenumbering{gobble}
\newpage

\frontmatter
% LaTeX source for ``การเรียนรู้ของเครื่องสำหรับเคมีควอนตัม (Machine Learning for Quantum Chemistry)''
% Copyright (c) 2022 รังสิมันต์ เกษแก้ว (Rangsiman Ketkaew).

% License: Creative Commons Attribution-NonCommercial-NoDerivatives 4.0 International (CC BY-NC-ND 4.0)
% https://creativecommons.org/licenses/by-nc-nd/4.0/

{
\thispagestyle{empty}

\begin{flushright}
    \vspace*{2.0in}
    
    \begin{spacing}{3}
    {\Huge การเรียนรู้ของเครื่องสำหรับเคมีควอนตัม}\\
    {\LARGE Machine Learning for Quantum Chemistry}
    \end{spacing}
    
    \vspace{0.25in}
    
    {\Large รังสิมันต์ เกษแก้ว}
    
    % \date{}
    \vspace{1in}
    
    {ฉบับพิมพ์ครั้งที่ 1}
    \vspace{0.5in}
    
    %\includegraphics[width=1in]{figs/logo.pdf}
    \vfill
\end{flushright}
}

% LaTeX source for ``การเรียนรู้ของเครื่องสำหรับเคมีควอนตัม (Machine Learning for Quantum Chemistry)''
% Copyright (c) 2022 รังสิมันต์ เกษแก้ว (Rangsiman Ketkaew).

% License: Creative Commons Attribution-NonCommercial-NoDerivatives 4.0 International (CC BY-NC-ND 4.0)
% https://creativecommons.org/licenses/by-nc-nd/4.0/

{
~\vfill
\thispagestyle{empty}
\setlength{\parindent}{0em}

รังสิมันต์ เกษแก้ว

การเรียนรู้ของเครื่องสำหรับเคมีควอนตัม\\
Machine Learning for Quantum Chemistry

\bigskip

\par{ฉบับพิมพ์ครั้งที่ 1 พ.ศ. 2565}

สงวนลิขสิทธิ์ตาม พ.ร.บ. ลิขสิทธิ์ พ.ศ. 2537/2540

อนุญาตให้ผู้อื่นเผยแพร่ผลงานชิ้นนี้ได้ ตราบใดที่ให้เครดิตแก่ผู้เขียนในฐานะผู้สร้างต้นฉบับและลิงก์กลับไปที่สัญญาอนุญาตของเจ้าของผลงาน 
ไม่อนุญาตให้นำไปใช้เพื่อการค้าและดัดแปลงแก้ไขไม่ว่าด้วยวิธีใด เว้นแต่จะได้รับอนุญาตเป็นลายลักษณ์อักษรจากผู้เขียน

หนังสือเล่มนี้อยู่ภายใต้ลิขสิทธิ์สัญญาอนุญาตแบบเปิด A Creative Commons Attribution-NonCommercial-NoDerivatives 4.0 International 
(CC BY-NC-ND 4.0), https://creativecommons.org/licenses/by-nc-nd/4.0/.

\includegraphics[scale=1.2]{by-nc-nd.pdf}

\bigskip

ซอร์สโค้ดของหนังสือเล่มนี้ถูกเขียนขึ้นโดยใช้ภาษา {\fontfamily{lmr}\selectfont \LaTeX}
และไฟล์ PDF ถูกสร้างขึ้นโดยใช้ {\fontfamily{lmr}\selectfont \XeLaTeX} 
เผยแพร่ที่ https://rangsimanketkaew.github.io/ml-qm-book.pdf

หากต้องการติดต่อผู้เขียน กรุณาส่งอีเมลมาที่ rangsiman1993@gmail.com
}
\include{toc}
% LaTeX source for ``การเรียนรู้ของเครื่องสำหรับเคมีควอนตัม (Machine Learning for Quantum Chemistry)''
% Copyright (c) 2022 รังสิมันต์ เกษแก้ว (Rangsiman Ketkaew).

% License: Creative Commons Attribution-NonCommercial-NoDerivatives 4.0 International (CC BY-NC-ND 4.0)
% https://creativecommons.org/licenses/by-nc-nd/4.0/

{
\pagenumbering{gobble}

\chapter*{\centering คำนำ}

ในปัจจุบันได้มีการนำการเรียนรู้ของเครื่องไปใช้ประโยชน์ในหลากหลายด้าน เช่น คอมพิวเตอร์วิทัศน์, คอมพิวเตอร์กราฟฟิก, ภาษาศาสตร์, เศรษฐศาสตร์, 
อุตสาหกรรม, เกษตรกรรม รวมไปถึงวิทยาศาสตร์และวิศวกรรม โดยสาขาเคมีนั้นก็เป็นอีกหนึ่งศาสตร์ที่ได้มีการนำการเรียนรู้ของเครื่องเข้ามาประยุกต์ใช้%
มาเป็นระยะเวลานานนับตั้งแต่ช่วงปี ค.ศ. 1990 โดยนักวิจัยได้ใช้การเรียนรู้ของเครื่องในการพัฒนายาสำหรับรักษาโรค การศึกษาโครงสร้างโปรตีน 
การศึกษาคุณสมบัติของโมเลกุลและสารประกอบ การออกแบบโมเลกุลและวัสดุชนิดใหม่ ๆ รวมไปถึงการศึกษาปฏิริยาเคมีและตัวเร่งปฏิกิริยา

โดยผู้เขียนได้เล็งเห็นว่าการประยุกต์ใช้การเรียนรู้ของเครื่องกับเคมีควอนตัมนั้นเป็นสิ่งใหม่ที่หลาย ๆ คนต่างก็ให้ความสนใจ เช่น นักเรียน นักศึกษา 
อาจารย์ และนักวิจัย ดังนั้นผู้เขียนจึงได้เรียบเรียงหนังสือเล่มนี้ขึ้นมาเพื่อเป็นแนวทางให้แก่ผู้ที่ต้องการศึกษาทางด้านนี้ โดยหนังสือเล่มนี้จะครอบคลุม%
เนื้อหาทฤษฎีของเคมีควอนตัม ซึ่งเป็นสาขาหนึ่งของเคมีเชิงฟิสิกส์ที่ช่วยให้เราสามารถเข้าใจองค์ความรู้พื้นฐานของโมเลกุลได้เป็นอย่างดี 
หนังสือเล่มนี้จะมีการอธิบายตั้งแต่ทฤษฎีและเทคนิคการเรียนรู้ของเครื่องแบบต่าง ๆ ความรู้ทางเคมีควอนตัม หัวข้อที่สามารถนำการเรียนรู้ของเครื่อง%
ไปประยุกต์ใช้ได้ รวมไปถึงรายละเอียดเชิงเทคนิคในการพัฒนาวิธีแบบใหม่เพื่อปรับปรุงประสิทธิภาพการพยากรณ์ของแบบจำลอง

สุดท้ายนี้ ผู้เขียนหวังเป็นอย่างยิ่งว่าหนังสือเล่มนี้จะช่วยให้ผู้อ่านทุกท่านได้รับความรู้และความเข้าใจที่ครบถ้วนและชัดเจนเกี่ยวกับการเรียนรู้ของเครื่อง%
สำหรับเคมีควอนตัม

\medskip

\begin{flushright}
รังสิมันต์ เกษแก้ว
\end{flushright}
}

% LaTeX source for ``การเรียนรู้ของเครื่องสำหรับเคมีควอนตัม (Machine Learning for Quantum Chemistry)''
% Copyright (c) 2022 รังสิมันต์ เกษแก้ว (Rangsiman Ketkaew).

% License: Creative Commons Attribution-NonCommercial-NoDerivatives 4.0 International (CC BY-NC-ND 4.0)
% https://creativecommons.org/licenses/by-nc-nd/4.0/

{
\pagenumbering{gobble}

\chapter*{\centering กิตติกรรมประกาศ}

ความรู้และแรงบัลดาลใจในการเขียนหนังสือเล่มนี้ของผู้เขียนมาจากแรงผลักดันและการสนับสนุนของบุคคลหลายท่าน 
การเขียนหนังสือเล่มนี้จะไม่เกิดขึ้นหรือสำเร็จไม่ได้ถ้าหากขาดบุคคลดังต่อไปนี้

ขอขอบคุณครอบครัวของผู้เขียนที่สนับสนุนให้ผู้เขียนได้ทำตามความฝันในการเรียนต่อระดับอุดมศึกษา ทั้งในระดับปริญญาโทและปริญญาเอก 
โดยเฉพาะการเห็นคุณค่าของการเรียนและการทำงานวิจัยทางด้านวิทยาศาสตร์พื้นฐาน

ขอขอบคุณเพื่อนร่วมงานทั้งนักศึกษาปริญญาโท ปริญญาเอก และนักวิจัยหลังปริญญาเอกของกลุ่มวิจัย ศ.ดร. แซนดรา ลูเบอร์ ภาควิชาเคมี
มหาวิทยาลัยแห่งซูริค สำหรับการแลกเปลี่ยนความรู้ ไอเดียใหม่ ๆ และการช่วยเหลือเกี่ยวกับงานวิจัยทางด้านเคมีทฤษฎี

ขอขอบคุณ รศ.ดร. ยุทธนา ตันติรุ่งโรจน์ชัย บุคคลผู้เป็นต้นแบบด้านการเรียนและเป็นผู้สร้างแรงบันดาลใจให้ผู้เขียนเรียนต่อต่างประเทศและ%
ทำงานวิจัยทางด้านเคมีทฤษฎีและเคมีคอมพิวเตอร์

ขอขอบคุณอาจารย์และเพื่อน ๆ ในช่วงมัธยมศึกษาตอนต้น-ปลาย ที่โรงเรียนพนัสพิทยาคาร และช่วงปริญญาตรี-โท ที่มหาวิทยาลัยธรรมศาสตร์ 
สำหรับความทรงจำอันดีงามและความเป็นกัลยาณมิตรที่ดีเสมอมา

ขอขอบคุณเพื่อน ๆ ที่เมืองซูริค ประเทศสวิตเซอร์แลนด์ สำหรับมิตรภาพอันดีงาม รอยยิ้มและเสียงหัวเราะที่เกิดขึ้น 
รวมไปถึงกิจกรรมที่ได้ทำร่วมกันในระหว่างที่ผู้เขียนกำลังศึกษาปริญญาเอกซึ่งเป็นช่วงเวลาเดียวกันกับที่ผู้เขียนกำลังเขียนหนังสือเล่มนี้

\medskip

\begin{flushright}
รังสิมันต์ เกษแก้ว
\end{flushright}
}

\include{quote_1}

\pagestyle{fancy} % Add header from this page on

\mainmatter
\part{การเรียนรู้ของเครื่อง}
% LaTeX source for ``การเรียนรู้ของเครื่องสำหรับเคมีควอนตัม (Machine Learning for Quantum Chemistry)''
% Copyright (c) 2022 รังสิมันต์ เกษแก้ว (Rangsiman Ketkaew).

% License: Creative Commons Attribution-NonCommercial-NoDerivatives 4.0 International (CC BY-NC-ND 4.0)
% https://creativecommons.org/licenses/by-nc-nd/4.0/

\chapter{การเรียนรู้ของเครื่อง}
\label{ch:ml}

%--------------------------
\section{ความสำคัญของ ML}
%--------------------------
\index{การเรียนรู้ของเครื่อง}

การเรียนรู้ของเครื่องหรือ Machine Learning (ML) เป็นวิทยาการคอมพิวเตอร์ประเภทหนึ่งที่เราทำให้เครื่องจักรสมองกลเกิด \enquote{สติปัญญา} 
ซึ่งกระบวนการดังกล่าวนั้นเรียกว่าการเรียนรู้ของเครื่องจักร ซึ่งในบริบทนี้เครื่องจักรสมองกลที่เรารู้จักกันดีก็คือโปรแกรมที่ถูกติดตั้งอยู่ในคอมพิวเตอร์นั่นเอง
โดยวิธีการเรียนรู้ก็คือเราป้อนข้อมูลและคำตอบเข้าไปให้กับโปรแกรม โปรแกรมจะทำการสร้างโมเดลที่สามารถอธิบายความสัมพันธ์ระหว่างข้อมูลที่เราป้อนเข้าไปได้ 
ซึ่งขั้นตอนที่เกิดขึ้นระหว่างการเรียนรู้ก็คือการฝึกสอนโมเดล (Model training) ซึ่งโปรแกรมสามารถแปลข้อมูลทั้งหมดเป็นโมเดลที่ปรับปรุงได้ 
นั่นหมายความว่าเทคโนโลยีการเรียนรู้ของเครื่องสามารถทำให้คอมพิวเตอร์เรียนรู้วิธีทำงานเลียนเบียบคล้ายมนุษย์ได้นั่นเอง 

%--------------------------
\section{บทบาทของ ML ในควอนตัมเคมี}
%--------------------------
\index{เคมีควอนตัม}

เคมีควอนตัม (Quantum Chemistry) เป็นแขนงหนึ่งของวิชาเคมีเชิงฟิสิกส์ (Physical Chemistry) ซึ่งเป็นการผสมผสานระหว่างกลศาสตร์ควอนตัม 
(Quantum Mechanics) กับการศึกษาอะตอมและโมเลกุลเข้าด้วยกัน กล่าวคือเรานำมาความทางด้านกลศาสตร์มาศึกษาอะตอมและโมเลกุล
โดยนักวิทยาศาสตร์ได้ศึกษาและค้นคว้างานวิจัยศาสตร์ด้านนี้มากว่าหนึ่งศควรรษ นับตั้งแต่ช่วงต้นปี ค.ศ. 1920 โดยได้มีการพัฒนาทฤษฎีต่าง ๆ มากมาย 
แต่สิ่งที่น่าสนใจก็คือจุดเปลี่ยนที่สำคัญของเคมีควอนตัมยุคใหม่ก็คือทฤษฎีฟังก์ชันความหนาแน่น หรือ Density Functional Theory (DFT) 
ซึ่งถูกคิดค้นมากว่าครึ่งศตวรรษ ถ้าหากใครที่เคยเรียนวิชาเคมีเชิงฟิสิกส์ หรือฟิสิกส์เชิงโมเลกุล หรือฟังการนำเสนอผลงานวิชาการตามงานประชุมวิชาการเคมีก็น่าจะเคยได้ยินชื่อทฤษฎีนี้กันมาบ้าง 

DFT เป็นทฤษฎีที่เรานำมาใช้ในการศึกษาคุณสมบัติของโมเลกุล ไม่ว่าจะเป็นขนาดเล็กอย่างเช่นโมเลกุลของสารประกอบอินทรีย์ และอนินทรีย์ 
หรือจะเป็นโมเลกุลขนาดใหญ่ เช่น โปรตีน, วัสดุโลหะ, และพอลิเมอร์ นั่นก็เพราะว่า DFT เป็นวิธีการคำนวณที่ให้ผลแม่นยำและไม่เปลืองพลังในการคำนวณมากนัก 
นั่นจึงทำให้ทฤษฎี DFT ได้รับการเชิดชูเกียรติด้วยรางวัลโนเบลสาขาเคมีในปี ค.ศ. 1998 และถูกนำมาใช้อย่างแพร่หลายในงานวิจัยไม่เพียงแต่ในสาขาเคมีเท่านั้น 
แต่ยังรวมไปถึงสาขาฟิสิกส์และชีววิทยาอีกด้วย แต่ในความเป็นจริงนั้น DFT ไม่ได้ให้ผลการคำนวณที่แม่นยำสูงมากนัก และยังไม่สามารถคำนวณคุณสมบัติของระบบยางระบบได้ 
จึงทำให้ในปัจจุบันนั้นได้มีการพัฒนาระเบียบวิธีใหม่ ๆ ขึ้นมาเพื่อปรับปรุงประสิทธิภาพหรือความสามารถของ DFT ให้เทียบเท่ากับวิธีที่อ้างอิงด้วยวิธีฟังก์ชันคลื่น หรือ Wavefunction Theory (WFT)

ในขณะเดียวกันนั้น ML ก็ถูกนำมาใช้ประโยชน์ในงานวิจัยเคมีมานานกว่า 30 ปีแล้ว แต่ในปัจจุบันนั้น เทคโนโลยีต่าง ๆ เช่น Supercomputing Cloud และ Graphical Processing Unit (GPU)
ได้เข้ามามีบทบาทอย่างมากในวิทยาศาสตร์เชิงคำนวณ (Computaitonal Science) โดยเฉพาะเคมีเชิงคำนวณ (Computational Chemistry) 
จึงทำให้มีจุดเปลี่ยนที่ทำให้ความสนใจของนักวิจัยในช่วง 10 ปีที่ผ่านมานี้ในหันมาทำงานวิจัยโดยใช้ ML กันมากขึ้น นั่นก็เพราะว่าในปัจจุบันนั้น ML สามารถศึกษาได้ง่ายขึ้นเมื่อเทียบกับในอดีต
ทุกวันนี้เราไม่จำเป็นต้องมานั่งเขียนโค้ดเพื่อสร้างโมเดล ML แบบเริ่มจากศูนย์กันแล้ว ตอนนี้เรามี Library แบบ Open-source ต่าง ๆ มากมายให้เลือกใช้ 
เช่น TensorFlow, PyTorch, Scikit-learn, หรือแม้แต่ Matlab ที่ก็มีฟังก์แบบสำเร็จรูปมาให้เราใช้งานได้เลย ซึ่งทำให้เราสามารถเลือกใช้โมเดล ML ต่าง ๆ ได้ตามต้องการ

ขอยกตัวอย่างงานวิจัยหนึ่งที่ตอนนี้กำลังเป็นหัวข้อที่มาแรง (อย่างน้อย ๆ ก็ ณ วันที่ผู้เขียนกำลังเขียนหนังสือเล่มนี้) นั่นคือการใช้ ML สร้างโมเดลที่ใช้ออกแบบ 
Exchange-Correlation (XC) Functional ที่มันมีความ Universal ให้กับ DFT ซึ่งถ้าหากเราทำสำเร็จหรือใกล้เคียง เราจะมีโมเดล XC ที่จะนำไปใช้ในการคำนวณอะไรก็ได้
เรียกได้ว่าเป็น XC สารพัดประโยชน์ (General-purpose) เลยก็ว่าได้ แต่ในความเป็นจริงนั้น XC นั้นก็เปรียบเสมือนเป็นกล่องดำ (Black Box) 
ซึ่งไม่มีใครที่รู้หน้าตาสมการหรือผลเฉลยทั่วไปของมันที่แน่นอน นั่นก็เพราะมันเป็นเทอมที่อธิบายอันตรกิริยาระหว่างอิเล็กตรอน ดังนั้นเราจึงทำได้เพียงหารูปแบบที่เป็นการประมาณเท่านั้น
ตรงจุดนี้เองที่ ML ก็เข้ามามีบทบาท เพราะมันก็คือเป็นการประมาณค่าแบบหนึ่งที่ใช้หลักการทางสถิติเข้ามาช่วยในการหาความสัมพันธ์ระหว่างของสองสิ่ง 
ถึงแม้ว่าตอนนี้มันจะยังอยู่ในขั้นของการพัฒนา แต่สิ่งหนึ่งที่เราเห็นได้เลยก็คือ ML มันช่วยลดระยะเวลาในคำนวณคุณสมบัติเชิงอิเล็กทรอนิกส์ (Electronic Properties) 
ของโมเลกุลอย่างเห็นได้ชัด

\begin{table}[!htp]
    \centering
    \caption{ตารางเปรียบเทียบความซับซ้อนเชิงคำนวณของวิธีทางเคมีควอนตัม\cite{rupp2015}}
    \begin{tabular}{lll}\toprule
    ตัวย่อ &วิธี &Runtime \\\midrule
    FCI &Full Configuration Interaction (CISDTQ) &$\mathcal(O)(N^{10})$ \\
    CC &Coupled Cluster (CCSD(T)) &$\mathcal(O)(N^{7})$ \\
    FCI &Full Configuration Interaction (CISD) &$\mathcal(O)(N^{6})$ \\
    MP2 &M$\o$llor-Plesset second order perturbation theory &$\mathcal(O)(N^{5})$ \\
    QMC &Quantum Monte Carlo &$\mathcal(O)(N^{3}) - \mathcal(O)(N^{4})$ \\
    HF &Hartree-Fock &$\mathcal(O)(N^{3}) - \mathcal(O)(N^{4})$ \\
    DFT &Density Functional Theory (Kohn-Sham) &$\mathcal(O)(N^{3})$ \\
    TB &Tight Binding &$\mathcal(O)(N^{3})$ \\
    MM &Molecular Mechanics &$\mathcal(O)(N^{2})$ \\
    \bottomrule
    \end{tabular}
\end{table}

%--------------------------
\section{เริ่มต้นศึกษา ML}
%--------------------------

การมีความรู้พื้นฐานก่อนเริ่มศึกษา ML อย่างจริงจังนั้นมันเป็นสิ่งสำคัญ ผู้เขียนได้สรุป 5 สิ่งสำคัญที่ควรจะต้องรู้

\begin{enumerate}
    \item \textbf{พีชคณิตเชิงเส้นและแคลคูลัสแบบหลายตัวแปร} : ทั้งสองวิชานี้ถือว่าเป็นรากฐานของ ML เลยก็ว่าได้ 
    เพราะว่าโมเดลทุกรูปแบบของ ML นั้นต่างก็ล้วนแต่เป็นคณิตศาสตร์ ถ้าหากเราต้องการที่จะพัฒนาอังกอริธึมใหม่ ๆ 
    หรือปรับปรุงอัลกอริทึมที่มีอยู่แล้ว เราจะต้องอาศัยความรู้พีชคณิตเชิงเส้น (เวกเตอร์และเมทริกซ์) และแคลคูลัส (การหาอนุพันธ์) 
    แต่ถ้าหากว่าเราเน้นไปทางสายแอพพลิเคชัน เราก็อาจจะไม่จำเป็นต้องรู้แบบลึกหรือละเอียดมากก็ได้ เพราะว่าปัจจุบันนี้มี Library สำเร็จรูปให้เราเลือกใช้มากมาย
    \item \textbf{สถิติ} : เนื่องจากว่าในขั้นตอนก่อนที่จะเริ่มสร้างและเทรนโมเดล ML นั้น เราจะต้องใช้เวลาส่วนใหญ่ 
    (อาจจะมากถึง 80\%) ไปกับการรวบรวมข้อมูล ทำความสะอาดข้อมูล การศึกษาการกระจายตัวของข้อมูล การตั้งและทดสอบสมมติฐาน 
    การทำการถดถอย (Regression) การแยกประเภท (Classification) เราจึงจำเป็นจะต้องใช้สถิติเข้ามาช่วยเพื่อให้เข้าใจถึงรายละเอียด
    ของชุดข้อมูลที่เรากำลังจะเล่นกับมัน ยิ่งเข้าใจข้อมูลมากเท่าไหร่ ยิ่งช่วยให้เราสามารถเลือกใช้โมเดล ML ได้เหมาะสมเท่านั้น 
    \item \textbf{โปรแกรมมิ่ง} : สิ่งสำคัญลำดับถัดมาคือทักษะในการเขียนโปรแกรมหรือเขียนโค้ด ถึงแม้ว่าเราจะมีความรู้ด้านทฤษฎีที่แม่นยำ 
    แต่ถ้าหากเราไม่สามารถเขียนโปรแกรมได้ แล้วก็ไม่สามารถสร้างโมเดลหรือนำ ML มาใช้งานจริงได้เลย ดังนั้นเราควรจะต้องเรียนรู้การเขียนโปรแกรม
    ให้ได้อย่างน้อยสัก 1 ภาษา ซึ่งภาษาที่ได้รับความนิยมมากที่สุดสำหรับงานทางด้านวิทยาศาสตร์ข้อมูล ณ ตอนนี้คือภาษา Python 
    นั่นก็เพราะตัวภาษาเองมี Syntax ที่ง่าย มี Library ให้เลือกใช้เยอะ มี Community ที่ใหญ่มาก ไม่ต้องกลัวเลยว่าถ้าหากมีปัญหา
    เกี่ยวกับการเขียน Python แล้วจะไม่มีคนช่วยหรือหาวิธีแก้ปัญหาไม่ได้
    \item \textbf{แนวคิดของ ML} : แนวคิดหรือ Concept ทางด้าน ML (วิทยาศาสตร์ข้อมูล) เป็นสิ่งที่สำคัญมากเช่นเดียวกัน
    เราควรจะทราบคำศัพท์เฉพาะทางและความหมาย (Terminology) ประเภทของ ML แนวทางการนำ ML มาใช้ (Best Practice)
    \item \textbf{ฝึกทำโจทย์จริง} : ตัวช่วยที่ดีที่สุดให้เราเรียนรู้ ML ได้ง่ายและเร็วนั่นก็คือการฝึกฝน ลองหาโจทย์จริง ๆ มาฝึกทำ
    หรืออาจจะลองเก็บเกี่ยวประสบการณ์โดยเข้าร่วมการแข่งขันวิทยาศาสตร์ข้อมูล ซึ่ง ณ ปัจจุบันก็มีการจัดแข่งขันบ่อยมาก ๆ เรียกได้ว่ามีสนาม
    ให้ได้ฝึกฝนวิทยายุทธเป็นร้อย ๆ พัน ๆ เลย
\end{enumerate}

%--------------------------
\section{คำศัพท์ ML}
%--------------------------

\begin{description}[style=nextline]
    \item[Algorithm] วิธีหรือขั้นตอนกระบวนการคิดคำนวณทางคณิตศาสตร์เพื่อให้ได้ผลลัพธ์ออกมา
    \item[Classification] การจำแนกข้อมูลหรือการทำนายค่าที่มีความไม่ต่อเนื่อง เช่น ประเภทของยานพาหนะ ชนิดของผลไม้
    \item[Data set หรือ Dataset] ชุดข้อมูลที่ได้เตรียมไว้ ประกอบไปด้วยข้อมูล input และ/หรือ output
    \item[Descriptor] Vector ของข้อมูล เช่น Feature vector
    \item[Features/Attribute/Representation] คุณลักษณะเด่นของข้อมูล
    \item[Model]  ชุดคำสั่งหรือโปรแกรมที่ถูกสร้างขึ้นมาโดยมีความสามารถในการคำนวณ ประมวลผลและตัดสินใจ
    \item[Target/Class/Label/Output] คำตอบหรือเป้าหมายที่ต้องการคำนวณ ประมาณค่า หรือทำนาย
    \item[Training] กระบวนการสร้างและฝึกสอน Model โดยใช้ Training set 
    \item[Prediction] กระบวนการทำนายค่าของ Model โดยจะทำนายค่า Output ของข้อมูลใหม่ที่ถูกป้อนเข้าไป
    \item[Regression]  การทำนายค่าที่มีความต่อเนื่อง เช่น ราคาสินค้า ปริมาณน้ำมัน
    \item[Reinforment learning] การเรียนรู้แบบเสริมแรง 
    \item[Supervised learning] การเรียนรู้ของ Model แบบมีผู้สอน (Output)
    \item[Test set] ชุดข้อมูลที่ใช้ทดสอบความถูกต้องและแม่นยำของ Model
    \item[Training set] ชุดข้อมูลที่นำมาใช้ในการสอนคอมพิวเตอร์เพื่อสร้าง Model
    \item[Unsupervised learning] การเรียนรู้ของ Model แบบไม่มีผู้สอน (Output-free)
    ซึ่งยังสามารถแบ่งออกได้เป็นสองประเภทคือ 1. Binary classification กับ 2. Multi-class classification
    \item[Validation set] ชุดข้อมูลสำหรับประเมินประสิทธิภาพของ Model ก่อนที่จะนำไปทดสอบกับ Test set จริง 
    โดย data set ประเภทนี้มักจะถูกนำมาใช้ในการทำ Cross-validation
\end{description}
 % การเรียนรู้ของเครื่อง
% LaTeX source for ``การเรียนรู้ของเครื่องสำหรับเคมีควอนตัม (Machine Learning for Quantum Chemistry)''
% Copyright (c) 2022 รังสิมันต์ เกษแก้ว (Rangsiman Ketkaew).

% License: Creative Commons Attribution-NonCommercial-NoDerivatives 4.0 International (CC BY-NC-ND 4.0)
% https://creativecommons.org/licenses/by-nc-nd/4.0/

\chapter{การเรียนรู้แบบมีผู้สอน}
\label{ch:sup_ml}

\begin{figure}[htbp]
    \centering
    \includegraphics[width=0.9\linewidth]{fig/supervised_ml.png}
    \caption{หุ่นยนต์กำลังเรียนรู้หาความเชื่อมโยงระหว่างโครงสร้างกับคุณสมบัติของโมเลกุล (เครดิตภาพ: https://puentesdigitales.com)}
    \label{fig:supervised_ml}
\end{figure}

การเรียนรู้แบบมีผู้สอนหรือ Supervised Learning เป็นเทคนิคแรก ๆ ที่ถูกพัฒนาขึ้นมาในช่วงยุคเริ่มต้นของ ML ซึ่งเป็นแนวคิดที่ใช้อินพุตและ%
เอาต์พุตในการฝึกสอนโมเดล ซึ่งโมเดลที่ได้ออกมานั้นจะเก็บข้อมูลที่อธิบายความสัมพันธ์ระหว่างอินพุตและเอาต์พุตนั่นเอง $($เปรียบเสมือนฟังก์ชัน%
ทางคณิคศาสตร์ $f(x))$ ซึ่งผู้เขียนมีความคิดเห็นส่วนตัวว่าการสร้างโมเดลประเภทนี้ง่ายกว่าประเภทอื่นทั้งในแง่ทฤษฎีของอัลกอริทึม การเรียนรู้ของ%
ผู้เริ่มต้นศึกษาและการนำไปใช้จริง โดยเทคนิคนี้ได้รับความนิยมมากที่สุดนั่นก็เพราะว่าสามารถนำไปประยุกต์ใช้งานกับโจทย์ที่หลากหลาย

Supervised ML เป็นเทคนิคที่เข้าใจได้ง่ายที่สุดเพราะว่าเป็นการฝึกให้โมเดลมีความสามารถในการเรียนรู้ Target อย่างตรงไปตรงมา จริง ๆ แล้ว%
นิยามของ Supervised ML นั้นมีเยอะมากขึ้นอยู่กับว่าเราต้องการนิยามในเชิงปรัชญา เชิงคณิตศาสตร์ หรือเชิงปฏิบัติ ทุกครั้งที่มีคนถามผู้เขียนว่า
Supervised ML คืออะไร ผู้เขียนก็มักจะตอบไปสั้น ๆ แบบไม่จริงจังว่า \enquote{\textit{Supervised ML คือการ Fit Curve}} (จริง ๆ
แล้ว ML ทุกอัลกอริทึมเลยก็ว่าได้) ซึ่งการให้นิยามแบบนี้เป็นการอธิบายในเชิงปฏิบัติ ตัวอย่างเช่น กำหนดให้มีข้อมูลในตารางที่
\ref{tab:simple_x_y} เป็นความสัมพันธ์ระหว่างโดเมน (Domain) และเรนจ์ (Range) ของฟังก์ชันเลขชี้กำลังง่าย ๆ ดังต่อไปนี้

\begin{table}[htbp]
    \centering
    \caption{ตัวอย่างข้อมูลอินพุตกับเอาต์พุตของฟังก์ชันเลขชี้กำลัง (Exponential Function)}
    \label{tab:simple_x_y}
    \begin{tabular}{cc}
        \toprule
        \textbf{Input $(x)$} & \textbf{Output $(y)$} \\
        \midrule
        1                    & 5                     \\
        2                    & 25                    \\
        3                    & 125                   \\
        4                    & 625                   \\
        5                    & ?                     \\
        \bottomrule
    \end{tabular}
\end{table}

ถ้าหากถามว่ากรณีที่อินพุต $(x)$ เท่ากับ 5 แล้วเอาต์พุต $(y)$ มีค่าเท่ากับเท่าไร ผู้อ่านก็คงตอบได้ทันทีเลยว่าเท่ากับ 3125 เพราะว่ามนุษย์นั้น%
มองเห็นรูปแบบที่เกิดขึ้นระหว่าง $x$ กับ $y$ ซึ่งการเปลี่ยนของ $y$ นั้นก็คือเพิ่มขึ้นครั้งละ 5 เท่า โดยสัมพันธ์กับการเปลี่ยนแปลงของ $x$ ที่%
เพิ่มขึ้นครั้งละ 1 ดังนั้นจากกรณีที่ $x$ เปลี่ยนจาก 4 เป็น 5 ค่าของ $y$ นั้นก็จะต้องเพิ่มขึ้นจาก 625 เป็น 625 $\times$ 5 = 3125 นั่นเอง
สำหรับความสัมพันธ์นี้เราสามารถสรุปฟังก์ชันคณิตศาสตร์ได้เป็น $y = 5^{x}$

ประเด็นที่น่าสนใจก็คือถ้าหากเราถามคำถามเดียวกันนี้กับคอมพิวเตอร์หรือเครื่องจักรว่าคำตอบของ $y$ จะมีค่าเป็นเท่าไรเมื่อ $x = 5$ แน่นอนว่า%
การรับรู้ของเครื่องจักรนั้นไม่สามารถเทียบเท่ากับการรับรู้ของมนุษย์ถึงแม้ว่าเครื่องจักรจะประมวลผลได้เร็วกว่ามากก็ตาม (ความสามารถในการรับรู้กับ%
ความสามารถในการประมวลนั้นต่างกันนะครับ) ดังนั้นสิ่งที่เราต้องการให้เครื่องจักรมีเหมือนมนุษย์ก็คือความสามารถในการเรียนรู้รูปแบบของข้อมูลโดย%
คำนวณออกมาเป็นฟังก์ชันคณิตศาสตร์ แต่แน่นอนว่าถ้าหากมนุษย์เจอโจทย์หรือข้อมูลที่มีความซับซ้อน เช่น ความสัมพันธ์ที่ไม่เป็นเชิงเส้น ก็ยากที่จะ%
หาคำตอบหรือฟังก์ชันออกมาได้เช่นเดียวกัน ดังนั้นข้อดีของเครื่องจักรก็คือนำความสามารถหรือความเร็วในการคำนวณมาใช้ในการปรับปรุงความสามารถ%
ในการเรียนรู้หรือที่เราเรียกว่าการฝึกสอนหรือเทรน (Train) โมเดลนั่นเอง
\idxboth{การฝึกสอนโมเดล}{Model Training}

%--------------------------
\section{การถดถอยเชิงเส้น}
\label{sec:lin_res}
\idxth{การเรียนรู้แบบมีผู้สอน!การถดถอยแบบเชิงเส้น}
\idxen{Supervised Learning!Linear Regression}
%--------------------------

เทคนิคของการเรียนรู้แบบมีผู้สอนที่พื้นฐานที่สุดและได้รับความนิยมอย่างมากในช่วงยุคแรกของปัญญาประดิษฐ์ก็คือ การถดถอยแบบเชิงเส้น
(Linear Regression) สมมติว่าเราพิจารณาชุดข้อมูลที่มีตัวแปรต้น 2 ตัว $(x_{1}$ และ $x_{2})$ และมีตัวแปรตาม 1 ตัว $(y)$
ซึ่งตัวแปรตามในที่นี้ก็คือคำตอบหรือเป้าหมายที่เราต้องการทำนายนั่นเอง โดยยกตัวอย่างเช่น กำหนดให้ $x_{1}$ เป็นจำนวนพันธะเดี่ยวในโมเลกุล
$x_{2}$ เป็นจำนวนวงอะโรมาติก (Aromatic) ในโมเลกุล และ $y$ เป็นค่าพลังงานรวมของโมเลกุล เราพบว่าเราสามารถสร้างหรือกำหนดสมการ%
ที่อธิบายความสัมพันธ์ระหว่างตัวแปรทั้งสามตัวนี้ได้แบบง่าย ๆ ดังนี้

\begin{equation}
    h_\theta(x) = \theta_0 + \theta_1 x_1 + \theta_2 x_2
\end{equation}

\noindent โดยที่ $x$ ในที่นี้คือเวกเตอร์แบบสองมิติในปริภูมิ $\mathbb{R}^{2}$ และ $\theta_{i}$ คือพารามิเตอร์หรือเรียกว่าน้ำหนัก
(Weights) ก็ได้ ซึ่งจะเป็นตัวแปรที่ปรับความเชื่อมโยง (Mapping) ระหว่าง $x_{i}$ และ $y$ ซึ่งเราสามารถเขียนให้อยู่ในรูปทั่วไปได้ดังนี้

\begin{align}
    h(x) & = \sum_{i=0}^{d} \theta_{i} x_{i} \\
         & = \theta^{\top} x
\end{align}

\noindent โดยสมการด้านบนนั้นจะเขียนในรูปของผลคูณระหว่างเวกเตอร์ของพารามิเตอร์ $(\theta^{\top})$ และเวกเตอร์ $x$

ลำดับถัดมาคือเราจะทำการปรับพารามิเตอร์ $\theta$ อย่างไรเพื่อให้ได้ชุดพารามิเตอร์ที่ทำการ Mapping ได้ดีที่สุด คำตอบก็คือเราสามารถทำได้โดย%
การกำหนดฟังก์ชันที่จะเป็นตัววัดพารามิเตอร์ $\theta_{i}$ ทีละตัว ซึ่งเรากำหนดและเรียกฟังก์ชันที่จะมาช่วยเราว่า Cost Function (Loss Function)
โดยมีรูปสมการทั่วไปดังต่อไปนี้

\begin{equation}
    J(\theta) = \frac 1 2 \sum_{i=1}^n \left( h_\theta(x^{(i)}) - y^{(i)} \right)^2
\end{equation}

\noindent ซึ่งจะมีความคล้ายกันกับ Ordinary Least Square นั่นเอง โดยในหัวข้อต่อไปเราจะมาดูรายละเอียดของเทคนิคที่เราสามารถนำมาใช้%
ในการแก้ปัญหาของ Cost Function

ขออธิบายเสริมครับ: สำหรับฟังก์ชันที่มีความเป็นเชิงเส้นนั้นจะต้องสอดคล้องกับเงื่อนไขดังต่อไปนี้

\begin{equation}
    f(\vec{x} + \vec{y}) = f(\vec{x}) + f(\vec{y})
\end{equation}

\noindent สำหรับ $\vec{x}$ และ $\vec{y}$ ทุกค่า และเงื่อนไขที่สองคือ

\begin{equation}
    f(s\vec{x}) = sf(x)
\end{equation}

ถ้าหากฟังก์ชันไม่สอดคล้องกับเงื่อนไขทั้งสองข้อด้านบน ฟังก์ชันนั้นจะมีความไม่เป็นเชิงเส้น (Nonlinearity)

%--------------------------
\subsection{การถดถอยแบบง่าย}
\label{ssec:simple_lin_res}
\idxth{การเรียนรู้แบบมีผู้สอน!การถดถอยแบบง่าย}
\idxen{Supervised Learning!Simple Regression}
%--------------------------

เรามาดูตัวอย่างของกรณีแรกของการถดถอย นั่นก็คือการถดถอยแบบง่าย (Simple Regression) โดยพิจารณาข้อมูลในตาราง
\ref{tab:simple_reg_data} ดังต่อไปนี้

\begin{table}[htbp]
    \centering
    \caption{แสดงเงินที่ใช้ในการลงทุนการโฆษณาของบริษัทต่าง ๆ กับยอดขายรายปี}
    \label{tab:simple_reg_data}
    \begin{tabular}{lcc}
        \toprule
        \textbf{บริษัท} & \textbf{วิทยุ} & \textbf{ยอดขาย (ต่อหน่วย)} \\
        \midrule
        Amazon        & 37.8         & 22.1                     \\
        Google        & 39.3         & 10.4                     \\
        Facebook      & 45.9         & 18.3                     \\
        Apple         & 41.3         & 18.5                     \\
        \bottomrule
    \end{tabular}
\end{table}

ตารางที่ \ref{tab:simple_reg_data} แสดงความสัมพันธ์ระหว่างเงินที่ใช้ในการลงทุนในสื่อวิทยุของบริษัทต่าง ๆ กับยอดขายรายปีต่อหน่วยการ%
ลงทุน โดยเราสามารถกำหนดตัวแปรได้เป็นตัวแปร $x$ กับ $y$ ซึ่งเป็นอินพุตและเอาต์พุตตามลำดับ ดังสมการต่อไปนี้

\begin{equation}
    y = mx + b
\end{equation}

\noindent โดยที่ $m$ คือความชันหรือน้ำหนัก (Weight) และ $b$ คือจุดตัดแกน $y$ หรือความโน้มเอียง (Bias) นั่นเอง สำหรับ Loss
Function ที่เราจะเลือกใช้นั้น คือ Mean Square Error (MSE)

\begin{equation}
    \text{MSE} = \frac{1}{N} \sum_{i=1}^{n} (y_i - (m x_i + b))^2
\end{equation}

ในการ Optimize ฟังก์ชัน MSE ด้านบนนั้นเราจะใช้เทคนิค Gradient Descent ซึ่งเป็นเทคนิคที่เราจะคำนวณหา Gradient ซึ่งสามารถใช้สมการที่
\eqref{eq:grad_simple_reg} ในการคำนวณได้

\begin{align}\label{eq:grad_simple_reg}
    f'(m,b) & =
    \begin{bmatrix}
        \dv{f}{m} \\
        \dv{f}{b} \\
    \end{bmatrix}                                  \\
            & =
    \begin{bmatrix}
        \frac{1}{N} \sum -x_i \cdot 2(y_i - (mx_i + b)) \\
        \frac{1}{N} \sum -1 \cdot 2(y_i - (mx_i + b))   \\
    \end{bmatrix} \\
            & =
    \begin{bmatrix}
        \frac{1}{N} \sum -2x_i(y_i - (mx_i + b)) \\
        \frac{1}{N} \sum -2(y_i - (mx_i + b))    \\
    \end{bmatrix}
\end{align}

หลังจากนั้นเราจะทำการฝึกสอนโมเดลโดยการใช้วิธีวนซ้ำเพื่อปรับค่าพารามิเตอร์ต่าง ๆ ทั้ง Weight และ Bias โดยภาพด้านล่างแสดงการเปลี่ยนแปลง%
ของการทาบเส้นตรงกับข้อมูล (Fitting) ระหว่างการฝึกสอน เราจะพบว่าเส้นตรง (Linear Line) ของเรานั้นจะลากผ่านข้อมูลที่อยู่ในช่วงบริเวณ%
ตรงกลางได้ดีขึ้นเรื่อย ๆ

\begin{figure}[htbp]
    \centering
    \begin{subfigure}{0.8\textwidth}
        \centering
        \includegraphics[width=0.9\linewidth]{fig/plot_simple_reg_1.png}
        \caption{ครั้งที่ 1}
        \label{fig:plot_simple_reg_1}
    \end{subfigure}
    \\
    \begin{subfigure}{0.8\textwidth}
        \centering
        \includegraphics[width=0.9\linewidth]{fig/plot_simple_reg_2.png}
        \caption{ครั้งที่ 2}
        \label{fig:plot_simple_reg_2}
    \end{subfigure}
\end{figure}%
\begin{figure}[htbp]\ContinuedFloat
    \centering
    \begin{subfigure}{0.8\textwidth}
        \centering
        \includegraphics[width=0.9\linewidth]{fig/plot_simple_reg_3.png}
        \caption{ครั้งที่ 3}
        \label{fig:plot_simple_reg_3}
    \end{subfigure}
    \\
    \begin{subfigure}{0.8\textwidth}
        \centering
        \includegraphics[width=0.9\linewidth]{fig/plot_simple_reg_4.png}
        \caption{ครั้งที่ 4}
        \label{fig:plot_simple_reg_4}
    \end{subfigure}
    \caption{การเปลี่ยนแปลงของเส้นตรงที่ถูกทาบ (Fitting) เข้ากับชุดข้อมูลอย่างง่าย}
    \label{fig:simple_reg_change}
\end{figure}

%--------------------------
\subsection{การถดถอยแบบหลายตัวแปร}
\label{ssec:multi_lin_res}
\idxth{การเรียนรู้แบบมีผู้สอน!การถดถอยแบบหลายตัวแปร}
\idxen{Supervised Learning!Multivariate Regression}
%--------------------------

สำหรับกรณีที่เรามีอินพุตหรือ Feature มากกว่าหนึ่งตัว เช่น ข้อมูลในตาราง \ref{tab:multi_reg_data} ด้านล่างที่เป็นการนำข้อมูลในตาราง
\ref{tab:simple_reg_data} มาเพิ่มข้อมูลเงินที่ใช้ในการลงทุนสำหรับการโฆษณาทางสื่อโทรทัศน์และหนังสือพิมพ์เข้าไป (คอลัมน์ที่ 3 กับ 4)

\begin{table}[htbp]
    \centering
    \caption{แสดงเงินที่ใช้ในการลงทุนการโฆษณาของบริษัทต่าง ๆ กับยอดขายรายปี}
    \label{tab:multi_reg_data}
    \begin{tabular}{lcccc}\toprule
        \textbf{บริษัท} & \textbf{วิทยุ} & \textbf{โทรทัศน์} & \textbf{หนังสือพิมพ์} & \textbf{ยอดขาย (ต่อหน่วย)} \\\midrule
        Amazon        & 37.8         & 230.1           & 69.1              & 22.1                     \\
        Google        & 39.3         & 44.5            & 23.1              & 10.4                     \\
        Facebook      & 45.9         & 17.2            & 34.7              & 18.3                     \\
        Apple         & 41.3         & 151.5           & 13.2              & 18.5                     \\
        \bottomrule
    \end{tabular}
\end{table}

\begin{figure}[htbp]
    \centering
    \includegraphics[width=0.8\linewidth]{fig/plot_multivar_reg.png}
    \caption{ความสัมพันธ์ของข้อมูลหลายตัวแปร (Multivariables Data)}
    \label{fig:multi_var_reg}
\end{figure}

โดยในกรณีที่ข้อมูลมีความซับซ้อนมากขึ้นแบบนี้ เราไม่สามารถใช้สมการเส้นตรงแบบง่าย ๆ ที่เราใช้ไปก่อนหน้านี้มาอธิบายความสัมพันธ์ระหว่าง
Feature ได้ ดังนั้นเราจะต้องมีการกำหนด Loss Function ขึ้นมาใหม่ โดยตอนนี้เราจะต้องมีการกำหนดค่า Weight ขึ้นมา 3 ค่า นั่นคือจากที่%
เราเคยมีฟังก์ชัน $mx + b$ ก็จะกลายเป็นฟังก์ชัน $W_1 x_1 + W_2 x_2 + W_3 x_3$ โดยจะได้สมการ Loss Function ใหม่ดังนี้

\begin{equation}
    \text{MSE} = \frac{1}{2N} \sum_{i=1}^{n} (y_i - (W_1 x_1 + W_2 x_2 + W_3 x_3))^2
\end{equation}

สำหรับสมการที่เราจะมาใช้ในการหา Gradient ของกรณีนี้สามารถพิสูจน์ได้โดยใช้กฎลูกโซ่ (Chain Rule) เช่นเดียวกับกรณีก่อนหน้านี้

\begin{align}
    f'(W_1) = -x_1(y - (W_1 x_1 + W_2 x_2 + W_3 x_3)) \\
    f'(W_2) = -x_2(y - (W_1 x_1 + W_2 x_2 + W_3 x_3)) \\
    f'(W_3) = -x_3(y - (W_1 x_1 + W_2 x_2 + W_3 x_3))
\end{align}

%--------------------------
\section{การจำแนกประเภท}
\label{sec:classification}
\idxth{การเรียนรู้แบบมีผู้สอน!การจำแนกประเภท}
\idxen{Supervised Learning!Classification}
%--------------------------

ในหัวข้อนี้จะเป็นการศึกษาโจทย์ปัญหาแบบการจำแนกประเภท (Classification) ซึ่งก็คล้าย ๆ กับโจทย์แบบ Regression แต่ว่าจะต่างกันตรงที่ค่า
$y$ ที่เราต้องการทำนายนั้นจะมีความไม่ต่อเนื่อง (Discrete Data) ซึ่งจะตรงข้ามกับ Regression ที่ค่า $y$ จะมีความต่อเนื่อง (Continuous
Data) โดยเริ่มต้นเราจะสนใจกรณี Classification แบบง่ายก่อน นั่นก็คือมีประเภทของข้อมูลที่เราจะจำแนกเพียงแค่ 2 ประเภท เรียกว่าโจทย์ปัญหา
Binary Classification ซึ่งค่า $y$ จะมีค่าได้แค่ 0 กับ 1 เท่านั้น ซึ่งในภายหลังเราจึงค่อยมาพิจารณากรณีที่มีประเภทมากกว่า 2 ประเภท
(Multiple-class Case)

สำหรับการระบุชื่อของประเภทหรือคลาส (Class) นั้น เราจะเรียกคลาส 0 ว่าเป็น Negative Class และเรียกคลาส 1 ว่า Positive Class
ซึ่งบ่อยครั้งเรามักจะเจอการใช้เครื่องหมาย - และ + แทนการเขียน 0 กับ 1 โดยที่เราจะกำหนดให้ $y^{i}$ คือ Label ของข้อมูลลำดับที่ $i$
สำหรับตัวอย่างการฝึกสอน

%--------------------------
\section{การถดถอยแบบโลจิสติค}
\label{sec:logis_regress}
\idxth{การเรียนรู้แบบมีผู้สอน!การถดถอยแบบโลจิสติค}
\idxen{Supervised Learning!Logistic Regression}
%--------------------------

การวิเคราะห์การถดถอยโลจิสติค (Logistic Regression) เป็นการวิเคราะห์ที่มีเป้าหมายเพื่อประมาณค่าหรือทํานายเหตุการณ์ที่สนใจว่าจะเกิดหรือ%
ไม่เกิดเหตุการณ์นั้นภายใต้อิทธิพลของตัวปัจจัยโดยอาศัยฟังก์ชันโลจิสติค (Logistic Function) ที่สร้างขึ้นจากชุดตัวแปรทำนายที่เป็นตัวแปรที่มีข้อมูล%
อยู่ในระดับช่วงเป็นอย่างน้อย โดยที่ระหว่างตัวแปรทำนายจะต้องมีความสัมพันธ์กันต่ำและในการวิเคราะห์จะต้องใช้ขนาดตัวแปรทำนายไม่ต่ำกว่า 30
ตัวแปร Logistic Regression จัดเป็นเครื่องมือวิเคราะห์ข้อมูลในการศึกษาวิจัยที่มีวัตถุประสงค์เพื่อทํานายเหตุการณ์หรือประเมินความเสี่ยง (เช่น
\enquote{เสี่ยง} หรือ \enquote{ไม่เสี่ยง}) จึงมีการประยุกต์ใช้ในงานวิจัยหลากหลายสาขา ทั้งสาขาทางการแพทย์ วิศวกรรมศาสตร์ นิเวศวิทยา
เศรษฐศาสตร์ และสังคมศาสตร์
\idxboth{การถดถอยโลจิสติค}{Logistic Regression}
\idxboth{ฟังก์ชันโลจิสติค}{Logistic Function}

นอกจากการทำนายการเกิดเหตุการณ์ที่สนใจว่าเกิด (0) หรือไม่เกิด (1) ได้แล้ว Logistic Regression ยังสามารถทำนายค่าความน่าจะเป็นของ%
เหตุการณ์ได้ด้วย (ค่าระหว่าง 0 กับ 1) จริง ๆ แล้วเทคนิค Logistic Regression นั้นคล้ายกับ Linear Regression มาก โดยทั้งสองเทคนิค%
นี้ต่างกันตรงที่การนำไปใช้งาน โดยเราใช้ Linear Regression สำหรับการแก้ปัญหาการถดถอยแต่ Logistic Regression สำหรับการแก้ปัญหา%
การแยกคลาสหรือจัดกลุ่มของข้อมูล

\begin{figure}[htbp]
    \centering
    \includegraphics[width=0.8\linewidth]{fig/s_curve_logistic_func.png}
    \caption{Logistic Function หรือ Sigmoid Function}
    \label{fig:s_curve_logistic}
\end{figure}

การทำ Logistic Regression นั้นเราไม่ได้ทำการ Fit เส้น Regression กับข้อมูลแต่จะเป็นการ Fit กับ Logistic Function (ภาพที่
\ref{fig:s_curve_logistic}) แทนซึ่งจะเป็นตัวที่ทำนายค่าออกมา โดย Logistic Function ที่เราใช้นั้นจริง ๆ แล้วก็คือเส้นโค้งตัว S หรือ
Sigmoid Function นั่นเอง โดยมีนิยามทางคณิตศาสตร์ดังต่อไปนี้
\idxen{Supervised Learning!Logistic Regression!Sigmoid Function}

\begin{equation}\label{eq:logistic_func}
    f(x) = \frac{L}{1 + e^{-k(x-x_0)}}
\end{equation}

\noindent สำหรับกรณีที่กำหนดให้ $k = 1$, $x_{0} = 0$, และ $L = 1$ เราจะได้สมการดังต่อไปนี้

\begin{align}\label{eq:std_logis_func}
    f(x) & = \frac{1}{1 + e^{-x}} \nonumber                  \\
         & = \frac{e^x}{e^x + 1} \nonumber                   \\
         & = \frac12 + \frac12 \tanh\left(\frac{x}{2}\right)
\end{align}

\noindent ซึ่งเราเรียกสมการที่ \eqref{eq:std_logis_func} นี้ว่าฟังก์ชันโลจิสติคมาตรฐาน (Standard Logistic Function)
\idxth{การเรียนรู้แบบมีผู้สอน!การถดถอยแบบโลจิสติค!ฟังก์ชันโลจิสติคมาตรฐาน}
\idxen{Supervised Learning!Logistic Regression!Standard Logistic Function}

โดยเทคนิคนี้ถูกนำมาใช้เยอะมากใน ML เพราะว่ามีความสามารถในการทำนายค่าความน่าจะเป็นและแยกข้อมูลโดยใช้ชุดข้อมูลที่มีความต่อเนื่องหรือ%
แบบไม่ต่อเนื่องก็ได้ หนังสือบางเล่มหรือบทความวิจัยบางฉบับเรียก Logistic Regression ว่า Maximum-entropy Classification (MaxEnt)
หรือ Log-linear Classifier เพราะว่าถูกนำมาใช้กับโจทย์ปัญหา Classification มากกว่า Regression ตามที่ได้อธิบายไว้

โค้ดด้านล่างคือตัวอย่างการเรียกใช้ฟังก์ชัน \inlinehighlight{LogisticRegression} ของไลบรารี่ Scikit-Learn สำหรับการฝึกสอนโมเดล%
ด้วย Logistic Regression โดยใช้ข้อมูลตัวอย่างที่สมมติขึ้นมา

\begin{lstlisting}[style=MyPython]
import numpy as np
from sklearn.linear_model import LogisticRegression

# Create dataset
x = np.arange(10).reshape(-1, 1)
y = np.array([0, 0, 0, 0, 1, 1, 1, 1, 1, 1])

# Create a logistic regression model
model = LogisticRegression(solver='liblinear', random_state=0)

# Train the model
model.fit(x, y)

# Get results
model.classes_
# Output
array([0, 1])
model.intercept_
# Output
array([-1.04608067])
model.coef_
# Output
array([[0.51491375]])
\end{lstlisting}

\vspace{1em}

เราสามารถแสดงเมทริกซ์ของค่าความน่าจะเป็น (Probability Matrix) ของข้อมูลแต่ละตัวได้ด้วย ดังนี้

\begin{lstlisting}[style=MyPython]
model.predict_proba(x)
# Output
array([[0.74002157, 0.25997843],
       [0.62975524, 0.37024476],
       [0.5040632 , 0.4959368 ],
       [0.37785549, 0.62214451],
       [0.26628093, 0.73371907],
       [0.17821501, 0.82178499],
       [0.11472079, 0.88527921],
       [0.07186982, 0.92813018],
       [0.04422513, 0.95577487],
       [0.02690569, 0.97309431]])
\end{lstlisting}

%--------------------------
\section{เครื่องเวกเตอร์ค้ำยัน}
\label{sec:svm}
\idxth{การเรียนรู้แบบมีผู้สอน!เครื่องเวกเตอร์ค้ำยัน}
\idxen{Supervised Learning!Support Vector Machine}
%--------------------------

เครื่องเวกเตอร์ค้ำยัน (Support Vector Machine หรือ SVM) เป็นวิธีเคอร์เนลแบบหนึ่งที่มีความคล้ายกับ GPR หรือ KRR เป็นอย่างมาก โดย SVM
จะทำการทำนายค่าโดยทำการเปรียบเทียบข้อมูลใหม่กับข้อมูลอ้างอิงด้วยฟังก์ชัน $k(x_{i},x_{j})$ และคำนวณค่าความเหมือน (Similarity)
ระหว่างจุดสองจุด ซึ่งเราเรียกสิ่งนี้ว่าเคอร์เนล (Kernel) โดยความซับซ้อนของวิธีนี้นั้นไม่มีกฎเกณฑ์ที่แน่นอนในการกำหนด (Arbitrarily)
ดังนั้นเราจะต้องทำการปรับ Hyperparameters เพื่อให้มีความเหมาะสมและสามารถควบคุมความซับซ้อนของวิธี SVM ซึ่งเราเรียกวิธีการปรับนี้ว่า
Regularization เพื่อทำการหลีกเลี่ยงปัญหา Overfit นั่นเอง ผู้อ่านสามารถศึกษาเคอร์เนลเพิ่มเติมได้ในหัวข้อที่ \ref{sec:kernel}
\idxboth{การทำให้ถูกต้อง}{Regularization}

\begin{figure}[htbp]
    \centering
    \includegraphics[width=0.65\linewidth]{fig/svm.png}
    \caption{Maximum Margin Hyperplan และ Margi สำหรับการฝึกสอนโมเดลของชุดข้อมูลตัวอย่างที่มี 2 คลาสด้วย Support Vector
        Machine (เครดิตภาพ: \url{https://en.wikipedia.org/wiki/Support_vector_machine})}
    \label{fig:svm_margin}
\end{figure}

ภาพที่ \ref{fig:svm_margin} แสดงชุดข้อมูลตัวอย่างที่มี 2 คลาส (สีน้ำเงินกับสีเขียว) โดยมีระนาบระยะห่างที่มากที่สุด (Maximum Margin
Hyperplan หรือ MMH) เป็นตัวแบ่งข้อมูลซึ่งอ้างอิงโดยจุดข้อมูลที่อยู่ใกล้กับ Hyperplan ซึ่งจุดข้อมูลเหล่านี้มีชื่อเรียกว่าเวกเตอร์ค้ำยัน (Support
Vector) โดยเราคำนวณ Support Vector จากช่องว่าง (Margin) ระหว่างคลาสทั้ง 2 คลาสโดยใช้ระยะห่างที่น้อยที่สุด ดังนั้นเป้าหมายของการ%
ฝึกสอนโมเดลด้วย SVM ก็คือการหา Hyperplan ที่สามารถแบ่งข้อมูลทั้ง 2 คลาสออกจากกันได้ดีที่สุด
\idxen{Supervised Learning!Support Vector Machine!Hyperplane}
\idxen{Supervised Learning!Support Vector Machine!Margin}

โค้ดด้านล่างคือตัวอย่างการเรียกใช้ฟังก์ชัน \inlinehighlight{svm} ของไลบรารี่ Scikit-Learn สำหรับการฝึกสอนโมเดลด้วย SVM โดยใช้%
ข้อมูลตัวอย่างที่สมมติขึ้นมา

\begin{lstlisting}[style=MyPython]
import numpy as np
from sklearn import svm

# Create dataset
x = np.arange(10).reshape(-1, 1)
y = np.array([0, 0, 0, 0, 1, 1, 1, 1, 1, 1])
x_test = x + 1.2

# Create a SVM classifier using linear kernel
clf = svm.SVC(kernel='linear')

# Train the model
clf.fit(x, y)

# Predict the response for test dataset
clf.predict(x_test)
# Output
array([0, 0, 0, 1, 1, 1, 1, 1, 1, 1])
\end{lstlisting}

%--------------------------
\section{เทคนิคการเรียนรู้แบบมีผู้สอนแบบอื่น ๆ}
\label{sec:other_ml}
\idxth{เทคนิคการเรียนรู้แบบผู้มีสอน}
\idxth{เทคนิคการเรียนรู้ของเครื่อง}
\idxen{Machine Learning Techniques}
\idxen{Unsupervised Machine Learning Techniques}
%--------------------------

%--------------------------
\subsection{Partial Least Squares (PLS)}
\label{ssec:pls}
\idxth{เทคนิคการเรียนรู้ของเครื่อง!วิธีกำลังสองน้อยที่สุดบางส่วน}
\idxen{Machine Learning Techniques!Partial Least Squares}
%--------------------------

วิธีกำลังสองน้อยที่สุดบางส่วน (Partial Least Squares หรือ PLS) เป็นวิธีเชิงสถิติที่ใช้สำหรับการวิเคราะห์หลายตัวแปรเพื่อสร้างตัวแบบ%
ความสัมพันธ์ระหว่างกลุ่มของตัวแปรทำนาย (Predictor Variable) โดยอาศัยตัวแปรแฝง (Latent variable) ซึ่งเทคนิคนี้มีความคล้ายกับ
Principle Component Analysis (PCA) ซึ่งจะเป็นการลดจำนวนมิติของข้อมูล\autocite{wold1984} ในช่วงยุคเริ่มต้นที่มีการใช้ปัญญาประดิษฐ์%
ในงานด้านเคมีนั้น เทคนิคนี้ได้ถูกนำมาใช้อย่างแพร่หลาย เช่น นำมาใช้สำหรับการระบุ Vibrational Bands สำหรับ Vibrational Spectra
และนำผลที่ได้มาเปรียบเทียบกับค่าการทำนายที่ได้จากวิธีอื่น เช่น ANN และ PCA-ANN

%--------------------------
\subsection{Gaussian Process Regression (GPR)}
\label{ssec:gpr}
\idxth{เทคนิคการเรียนรู้ของเครื่อง!การถดถอยของกระบวนการเกาส์เซียน}
\idxen{Machine Learning Techniques!Gaussian Process Regression}
%--------------------------

การถดถอยของกระบวนการเกาส์เซียน (Gaussian Process Regression หรือ GPR) เป็นวิธีการถดถอยของเบส์แบบหนึ่งโดยใช้ Kernel Function
ที่สามารถบ่งบอกหรือแสดงค่าความแปรปรวน (Covariance) ในขั้นตอน Gaussian Process ได้\autocite{rasmussen2005} โดย GPR
จะทำการสร้างโมเดลแบบ Non-parametric และสามารถคำนวณค่าความเชื่อมั่น (Confidence Intervals) ไปพร้อม ๆ กับการทำนาย
รายละเอียดเพิ่มเติมของ GPR สามารถศึกษาได้ในหัวข้อ \ref{sec:gaussian_process}

%--------------------------
\subsection{Random Forest}
\label{ssec:rs}
\idxth{เทคนิคการเรียนรู้ของเครื่อง!เครื่องเวกเตอร์ค้ำยัน}
\idxen{Machine Learning Techniques!Random Forest}
%--------------------------

การสุ่มป่าไม้ (Random Forest หรือ RF) เป็นวิธีหนึ่งในกลุ่มของโมเดลที่เรียกว่าการเรียนรู้แบบกลุ่มก้อน (Ensemble Learning) ที่มีหลักการคือ%
การฝึกสอนโมเดลที่เหมือนกันหลาย ๆ ครั้ง (Multitude) บนข้อมูลชุดเดียวกัน โดยแต่ละครั้งของการเทรนจะเลือกส่วนของข้อมูลที่ฝึกสอนไม่เหมือนกัน
แล้วนำการตัดสินใจของโมเดลเหล่านั้นมาโหวตเลือกกันว่า Class ไหนถูกเลือกมากที่สุด\autocite{breiman2001,quinlan1986}

ตัวอย่างการเขียนโค้ดโมเดล Random Forest สำหรับการทำ Regression

\begin{lstlisting}[style=MyPython]
from sklearn.ensemble import RandomForestRegressor
from sklearn.datasets import make_regression

X, y = make_regression(n_features=4, n_informative=2,
                       random_state=0, shuffle=False)
regr = RandomForestRegressor(max_depth=2, random_state=0)
regr.fit(X, y)

print(regr.predict([[0, 0, 0, 0]]))
# Output
[-8.32987858]
\end{lstlisting}

\vspace{1em}

ตัวอย่างการเขียนโค้ดโมเดล Random Forest สำหรับการทำ Classification

\begin{lstlisting}[style=MyPython]
from sklearn.ensemble import RandomForestClassifier
from sklearn.datasets import make_classification

X, y = make_classification(n_samples=1000, n_features=4,
                           n_informative=2, n_redundant=0,
                           random_state=0, shuffle=False)
clf = RandomForestClassifier(max_depth=2, random_state=0)
clf.fit(X, y)

print(clf.predict([[0, 0, 0, 0]]))
# Output
[1]
\end{lstlisting}

%--------------------------
\subsection{Artificial Neural Network}
\label{ssec:ann}
\idxth{เทคนิคการเรียนรู้ของเครื่อง!โครงข่ายประสาทเทียมประดิษฐ์}
\idxen{Machine Learning Techniques!Artificial Neural Network}
%--------------------------

โครงข่ายประสาทเทียมประดิษฐ์ (Artificial Neural Network หรือ ANN) หรือเรียกว่าโครงข่ายประสาทเทียม (Neural Network หรือ
Neural Net) เป็นอัลกอริทึมรูปแบบหนึ่งที่เลียนแบบการทำงานของสมองมนุษย์ โดยทำการสร้างโมเดลเรียนรู้ที่ประกอบไปด้วยชั้นเรียนรู้ระหว่างกลาง
(Hidden Layer) และหน่วยย่อยที่เกิดการเรียนรู้ (Node หรือ Artificial Neuron หรือ Unit)

\begin{figure}[htbp]
    \centering
    \includegraphics[width=0.9\linewidth]{fig/neuron.png}
    \caption{การรับส่งข้อมูลภายในเซลล์ประสาท}
    \label{fig:neuron}
\end{figure}

จริง ๆ แล้ว Neural Network ก็คือการจำลองสมองมนุษย์โดยพยายามสร้างองค์ประกอบต่าง ๆ ให้มีความคล้ายกันให้มากที่สุด เช่น ในสมองมี%
เซลล์ประสาท (Neurons) และจุดประสานประสาท (Synapses) แต่ละเซลล์ประสาทประกอบด้วยปลายในการรับกระแสประสาทเรียกว่า
\enquote{เดนไดรท์} (Dendrite) ซึ่งเป็นอินพุตและปลายในการส่งกระแสประสาทเรียกว่าแอคซอน (Axon) ซึ่งเปรียบเหมือนเป็นเอาต์พุตของเซลล์

โดยโมเดล Neural Network ที่มีการนำไปใช้มากที่สุดคือเครือข่ายประสาทแบบป้อนไปหน้า (Feed-forward Network) และโมเดล Neural Network
ยังสามารถแบ่งออกได้เป็นหลายประเภท ดังนี้

\begin{itemize}
    \item เพอร์เซ็ปตรอนชั้นเดียว (Single-layer Perceptron)

    \item เพอร์เซ็ปตรอนหลายชั้น (Multi-layer Perceptron)

    \item โครงข่ายแบบวนซ้ำ (Recurrent Neural Network)

    \item แผนผังจัดระเบียบเองได้ (Self-organizing Map)

    \item เครื่องจักรโบลทซ์แมน (Boltzmann Machine)

    \item กลไกแบบคณะกรรมการ (Committee of Machines)

    \item โครงข่ายความสัมพันธ์ (Associative Neural Network)

    \item โครงข่ายกึ่งสำเร็จรูป (Instantaneously Trained Networks)

    \item โครงข่ายแบบยิงกระตุ้น (Spiking Neural Networks)
\end{itemize}

โดยในหนังสือเล่มนี้จะอธิบายเฉพาะ Neural Network แบบเพอร์เซ็ปตรอนชั้นเดียวและเพอร์เซ็ปตรอนหลายชั้น (บทที่ \ref{ch:deep_learning})
สำหรับผู้อ่านที่สนใจศึกษารายละเอียดของ Neural Network ประเภทอื่น ๆ นั้นสามารถศึกษาได้จากหนังสือเฉพาะทางด้าน Neural Network เช่น
\enquote{Deep Learning} เขียนโดย Ian Goodfellow, Yoshua Bengio และ Aaron Courville\autocite{Goodfellow-et-al-2016}
รายละเอียดเพิ่มเติมดูได้ที่เว็บไซต์ \url{https://www.deeplearningbook.org}
 % การเรียนรู้แบบมีผู้สอน
% LaTeX source for ``การเรียนรู้ของเครื่องสำหรับเคมีควอนตัม (Machine Learning for Quantum Chemistry)''
% Copyright (c) 2022 รังสิมันต์ เกษแก้ว (Rangsiman Ketkaew).

% License: Creative Commons Attribution-NonCommercial-NoDerivatives 4.0 International (CC BY-NC-ND 4.0)
% https://creativecommons.org/licenses/by-nc-nd/4.0/

\chapter{วิธีเคอร์เนล}
\label{ch:kernel}

%--------------------------
\section{เคอร์แนลคืออะไร}
%--------------------------

%--------------------------
\section{ฟังก์ชันเคอร์แนล}
%--------------------------

%--------------------------
\subsection{Linear Regression}
%--------------------------

%--------------------------
\subsection{Ridge Regression}
%--------------------------

%--------------------------
\section{Kernel Ridge Regression}
%--------------------------

Kernel Ridge Regression (KRR) เป็นการต่อยอดจาก Ridge Regression หรืออธิบายง่าย ๆ ว่า KRR คือ RR ในเวอร์ชันที่เป็น Nonlinear problem

%--------------------------
\section{Gaussian Process Regression}
%--------------------------

%--------------------------
\section{Support Vector Machine}
%--------------------------

 % วิธีเคอร์เนล
% LaTeX source for ``การเรียนรู้ของเครื่องสำหรับเคมีควอนตัม (Machine Learning for Quantum Chemistry)''
% Copyright (c) 2022 รังสิมันต์ เกษแก้ว (Rangsiman Ketkaew).

% License: Creative Commons Attribution-NonCommercial-NoDerivatives 4.0 International (CC BY-NC-ND 4.0)
% https://creativecommons.org/licenses/by-nc-nd/4.0/

\chapter{การเรียนรู้เชิงลึก}
\label{ch:dl}

%--------------------------
\section{โครงข่ายประสาทเทียม}
%--------------------------

โครงข่ายประสาทเทียม (Neural Network)

การเรียนรู้เชิงลึกรูปแบบที่มาตรฐานที่สุดคือการเรียนรู้แบบมีผู้สอนด้วยโมเดลแบบไม่เป็นเชิงเส้น (Supervised learning with nonlinear model)

%--------------------------
\subsection{Forwardpropagation}
%--------------------------

%--------------------------
\subsection{Backpropagation}
%--------------------------

%--------------------------
\section{Activation Function}
%--------------------------

ฟังก์ชันกระตุ้น

%--------------------------
\section{Learning Layers}
%--------------------------

ชั้นการเรียนรู้

%--------------------------
\section{Loss Function}
%--------------------------

Loss Function หรือ Cost Function หรือ Error Function คือฟังก์ชันความคลาดเคลื่อน

%--------------------------
\section{Optimizer}
%--------------------------

ตัวปรับประสิทธิภาพการเรียนรู้

%--------------------------
\section{Architectures}
%--------------------------

สถาปัตยกรรมของโครงข่ายประสาทเทียม

 % การรเรียนรู้เชิงลึก
% LaTeX source for ``การเรียนรู้ของเครื่องสำหรับเคมีควอนตัม (Machine Learning for Quantum Chemistry)''
% Copyright (c) 2022 รังสิมันต์ เกษแก้ว (Rangsiman Ketkaew).

% License: Creative Commons Attribution-NonCommercial-NoDerivatives 4.0 International (CC BY-NC-ND 4.0)
% https://creativecommons.org/licenses/by-nc-nd/4.0/

\chapter{การเลือกและปรับแต่งโมเดล}
\label{ch:reg_sel_model}

\begin{figure}[htbp]
    \centering
    \includegraphics[width=0.9\linewidth]{fig/ml-prediction.jpg}
    \caption{การเลือกโมเดล ML (เครดิตภาพ: https://pythonnumericalmethods.berkeley.edu)}
    \label{fig:ml_prediction}
\end{figure}

การทำให้โมเดลมีความสม่ำเสมอ (Regularization) เพื่อเพิ่มความถูกต้องและการเลือกโมเดล (Model Selection) เป็นสิ่งที่จำเป็นมากใน%
ขั้นตอนของการฝึกสอนโมเดล ในบทนี้เราจะมาดูรายละเอียดและแนวทางในการเลือกอัลกอริทึมสำหรับสร้างโมเดล ML รวมไปถึงเทคนิคการปรับแต่ง%
โมเดลเพื่อให้มีประสิทธิภาพในการทำนายมากที่สุด ตัวอย่างของขั้นตอนการพิจารณาเลือกอัลกอริทึม ML นั้นแสดงตามภาพที่ \ref{fig:ml_prediction}
โดยเริ่มต้นจากปัญหาที่เราต้องการศึกษาก่อน แล้วก็พิจารณาว่าชุดข้อมูลของเรานั้นมี Label หรือคำตอบของแต่ละข้อมูลหรือไม่ ถ้าหากว่ามี Label 
เราก็สามารถใช้อัลกอริทึมแบบ Supervised ML ได้ แต่ถ้าหากไม่มี Label เราก็ไม่มีทางเลือกอื่นนอกจากจะต้องใช้อัลกอริทึมแบบ Unsupervised 
ML เท่านั้น (สำหรับกรณีที่มี Label นั้นเราอาจจะใช้ Unsupervised ML ด้วยก็ได้ โดยแกล้งทำเป็นไม่สนใจ Label) เมื่อเราแบ่งประเภทของ%
โจทย์ปัญหาได้แล้ว ขั้นตอนต่อมาคือการเลือกวิธีในการแก้ปัญหา ซึ่งขั้นตอนนี้ก็จะนำไปสู่การเลือกอัลกอริทึม ML แบบต่าง ๆ นั่นเอง โดยโจทย์ปัญหา%
หลัก ๆ ที่เรามักจะเจอนั้นมีด้วยกัน 4 แบบดังนี้

\begin{enumerate}
    \item Classification (การแบ่งประเภท)
    
    \item Regression (การถดถอย)
    
    \item Clustering (การจัดกลุ่ม)
    
    \item Dimensionality Reduction (การลดมิติของข้อมูล)
\end{enumerate}

โดยโจทย์ปัญหาแต่ละแบบนั้นก็จะมีอัลกอริทึมที่เหมาะสมสำหรับการแก้ปัญหานั้น ๆ เช่น Ridge Regression ก็จะเหมาะสำหรับโจทย์ Regression

%--------------------------
\section{การเลือกโมเดล}
\label{sec:choose_model}
\idxth{การเลือกโมเดลการเรียนรู้ของเครื่อง}
\idxen{Model Selection}
%--------------------------

อัลกอริทึมหรือโมเดล ML แต่ละอันนั้นก็มีข้อดีข้อเสียแตกต่างกันไป ผู้เขียนขอสรุปง่าย ๆ ดังนี้ (เน้นเฉพาะโมเดลที่ได้รับความนิยมในการใช้งาน)

%--------------------------
\subsection{Linear Regression}
\label{ssec:pros_cons_lin_reg}
\idxth{Model Selection!Linear Regression}
%--------------------------

\noindent \textbf{ข้อดี}
\begin{itemize}[topsep=0pt]
    \item สามารถเขียนโค้ดได้ง่าย และฝึกสอนโมเดลได้อย่างมีประสิทธิภาพ
    
    \item ปัญหา Overfitting ของ Linear Regression สามารถแก้ได้ด้วยการทำ Regularization
    
    \item มีประสิทธิภาพมาก ๆ เมื่อชุดข้อมูลสามารถแยกได้เชิงเส้น (Linearly Separable)
\end{itemize}

\noindent \textbf{ข้อด้อย}
\begin{itemize}[topsep=0pt]
    \item ข้อมูลที่อยู่ในชุดข้อมูลที่ใช่สำหรับการฝึกสอนโมเดล Linear Regression นั้นควรจะต้องไม่ขึ้นต่อกัน แต่ทว่าในชีวิตจริงนั้นข้อมูลก็%
    มักจะขึ้นต่อกันเสมอ

    \item สามารถเกิด Noise และ Overfitting ได้ง่าย
    
    \item การที่มี Outlier ในชุดข้อมูลนั้นจะส่งผลให้โมเดลมีประสิทธิภาพที่ต่ำลงมาก ๆ
\end{itemize}

%--------------------------
\subsection{Logistic Regression}
\label{ssec:pros_cons_log_reg}
\idxth{Model Selection!Logistic Regression}
%--------------------------

\noindent \textbf{ข้อดี}
\begin{itemize}[topsep=0pt]
    \item โอกาสเกิด Overfitting น้อย แต่ว่าสามารถเกิด Overfitting ได้ในชุดข้อมูลที่มีจำนวนมิติสูง ๆ
    
    \item มีประสิทธิภาพมาก ๆ เมื่อชุดข้อมูลมี Features ที่สามารถแยกกันได้แบบเชิงเส้น
    
    \item สามารถเขียนโค้ดได้ง่าย และฝึกสอนโมเดลได้อย่างมีประสิทธิภาพ
\end{itemize}

\noindent \textbf{ข้อด้อย}
\begin{itemize}[topsep=0pt]
    \item ไม่ควรใช้อัลกอริทึมนี้สำหรับกรณีที่จำนวน Observation นั้นมีน้อยกว่าจำนวนของ Feature

    \item เหมาะสำหรับชุดข้อมูลที่มีความเป็นเชิงเส้น ซึ่งหาได้ยากในชีวิต (ปกติเรามักจะมีเจอชุดข้อมูลแบบที่ไม่เป็นเชิงเส้น)
    
    \item ใช้ทำนายได้แค่ฟังก์ชันที่ไม่ต่อเนื่อง (Discrete Function)
\end{itemize}

%--------------------------
\subsection{Support Vector Machine}
\label{ssec:pros_cons_svm}
\idxth{Model Selection!Support Vector Machine}
%--------------------------

\noindent \textbf{ข้อดี}
\begin{itemize}[topsep=0pt]
    \item เหมาะสำหรับข้อมูลที่มีจำนวนมิติเยอะ ๆ (High-dimensional Data)
    
    \item สามารถใช้กับชุดข้อมูลที่มีขนาดเล็กได้ (จำนวนข้อมูลไม่เยอะ)
    
    \item สามารถแก้ปัญหาแบบไม่เป็นเชิงเส้นได้ (Non-linear Problem)
\end{itemize}

\noindent \textbf{ข้อด้อย}
\begin{itemize}[topsep=0pt]
    \item ไม่ค่อยมีประสิทธิภาพเมื่อใช้กับชุดข้อมูลที่มีขนาดใหญ่
    
    \item ต้องเลือก Kernel ที่เหมาะสม ถ้าเลือก Kernel ไม่ดีก็จะได้โมเดลที่ไม่มีประสิทธิภาพ
\end{itemize}

%--------------------------
\subsection{Neural Network}
\label{ssec:pros_cons_nn}
\idxth{Model Selection!Neural Network}
%--------------------------

\noindent \textbf{ข้อดี}
\begin{itemize}[topsep=0pt]
    \item มีคุณสมบัติที่ทำให้โมเดลสามารถทำงานต่อไปได้แม้จะเกิด Failure ขึ้น เรียกง่าย ๆ ว่า ทนทานต่อความเสียหายมี Fault Tolerance
    
    \item มีความสามารถในการเรียนรู้โมเดลที่เป็นแบบไม่เชิงเส้นและความสัมพันธ์ระหว่างตัวแปรที่มีความซับซ้อน
    
    \item สามารถ Generalize บนชุดข้อมูลที่ไม่เคยเห็นมาก่อนได้ (Unseen Data)
\end{itemize}

\noindent \textbf{ข้อด้อย}
\begin{itemize}[topsep=0pt]
    \item ใช้ระยะเวลาในการฝึกสอนโมเดลนาน
    
    \item ไม่การันตีว่าการฝึกสอนโมเดลจะลู่เข้า (Non-gauranteed Convergence)
    
    \item ตีความโมเดลได้ยาก เช่น เราไม่สามารถบอกความสัมพันธ์ระหว่างโหนดใน Hidden Layer ได้ ซึ่งเราเรียกว่าโมเดลแบบนี้ว่า Black 
    Box 

    \item ประสิทธิภาพในการฝึกสอนโมเดลขึ้นอยู่กับประสิทธิภาพของของเครื่องที่ใช้รันด้วย (Hardware)
    
    \item ไม่เหมาะสำหรับผู้เริ่มต้นศึกษา ML เพราะว่าต้องใช้ประสิทธิภาพและความสามารถในการนำไปจัดการปัญหาและปรับแก้โมเดลเพื่อทำให้%
    เรียนรู้ได้ดียิ่งขึ้น
\end{itemize}

%--------------------------
\subsection{Printipal Component Analysis}
\label{ssec:pros_cons_pca}
\idxth{Model Selection!Printipal Component Analysis}
%--------------------------

\noindent \textbf{ข้อดี}
\begin{itemize}[topsep=0pt]
    \item สามารถลดความซับซ้อนของความสัมพันธ์ระหว่าง Feature ได้
    
    \item สามารถลดปัญหา Overfitting
\end{itemize}

\noindent \textbf{ข้อด้อย}
\begin{itemize}[topsep=0pt]
    \item องค์ประกอบหลัก (Principal Component) นั้นวิเคราะห์และตีความได้ยาก
    
    \item การใช้วิธีนี้ทำให้เราสูญเสียข้อมูล (ความสัมพันธ์ระหว่าง Feature) หรือ Information Loss
    
    \item จำเป็นจะต้องทำการ Standardize ชุดข้อมูลก่อน
\end{itemize}

%--------------------------
\subsection{K Means Clustering}
\label{ssec:pros_cons_kmeans}
\idxth{Model Selection!K Means Clustering}
%--------------------------

\noindent \textbf{ข้อดี}
\begin{itemize}[topsep=0pt]
    \item สามารถเขียนโค้ดได้ง่าย ไม่ซับซ้อน
    
    \item สามารถนำไปใช้กับชุดข้อมูลที่มีขนาดใหญ่มาก ๆ ได้
    
    \item การันตีว่าการฝึกสอนโมเดลนั้นลู่เข้าแน่นอน
    
    \item สามารถปรับให้เข้ากับชุดข้อมูลใหม่ได้ง่าย ๆ
\end{itemize}

\noindent \textbf{ข้อด้อย}
\begin{itemize}[topsep=0pt]
    \item ไม่เหมาะสำหรับชุดข้อมูลที่มี Outlier
    
    \item การเลือกค่า $K$ สำหรับ Clustering นั้นค่อนข้างยุ่งยาก
    
    \item ประสิทธิภาพของโมเดลขึ้นอยู่กับพารามิเตอร์เริ่มต้น
    
    \item ความสามารถในการ Scale นั้นจะลดลงเมื่อจำนวนมิติเพิ่มขึ้น
\end{itemize}

%--------------------------
\subsection{K Nearest Neighbor}
\label{ssec:pros_cons_knn}
\idxth{Model Selection!K Nearest Neighbor}
%--------------------------

\noindent \textbf{ข้อดี}
\begin{itemize}[topsep=0pt]
    \item สามารถทำนายได้โดยไม่ต้องฝึกสอนโมเดล
    
    \item เป็นวิธีที่สิ้นเปลืองน้อยมาก โดยมี Time Complexity เท่ากับ $\mathcal{O}(n)$
    
    \item สามารถนำไปใช้ได้กับโจทย์ Regression และ Classification
\end{itemize}

\noindent \textbf{ข้อด้อย}
\begin{itemize}[topsep=0pt]
    \item ไม่เหมาะสำหรับชุดข้อมูลขนาดใหญ่
    
    \item ไม่เหมาะสำหรับชุดข้อมูลที่มี Noise เยอะมากเกินไป และข้อมูลไม่ครบ รวมไปถึง Outlier ด้วย 
    
    \item จำเป็นต้องมีการทำ Feature Scaling
    
    \item การเลือกค่า $K$ นั้นค่อนข้างยุ่งยาก
\end{itemize}

%--------------------------
\section{Cross Validation}
\label{sec:cross_val}
\idxboth{การทดสอบแบบข้าม}{Cross Validation}
%--------------------------

วิธีการตรวจสอบโมเดลวิธีแรกนี้เป็นวิธีที่ได้รับความนิยมเป็นอย่างมากเพราะว่าสามารถทำได้ง่ายและให้ผลลัพธ์ที่น่าเชื่อถือ นั่นก็คือ 
\enquote{K-Fold Cross Validation} หรือเรียกสั้น ๆ ว่า Cross Validation วิธีนี้เริ่มด้วยการแบ่งข้อมูล $k$ ให้มีขนาดของแต่ละส่วนเท่า ๆ กัน 
หลังจากนั้นทำเก็บข้อมูลหนึ่งส่วนไว้ใช้สำหรับเป็นตัวทดสอบโมเดลนั่นก็คือการทำ Validation แล้วทําวนไปเช่นนี้จนครบจํานวนที่แบ่งไว้ เช่น 
การทดสอบด้วยวิธี 5-fold Cross Validation ในรอบแรกเราจะทำการเทรนโมเดลด้วยชุดข้อมูลที่เกิดจากการวมส่วนที่ 2, 3, 4, และ 5 
และทำการทดสอบด้วยข้อมูลส่วนที่ 1 และในรอบที่สองเราจะเปลี่ยนมาเทรนโมเดลด้วยข้อมูลของส่วนที่ 1, 3, 4, และ 5 แล้วนำโมเดลมาทดสอบ%
ด้วยข้อมูลส่วนที่ 2

จริง ๆ แล้วมีวิธีการทำ Cross Validation หลากหลายวิธีมาก โดยมีภาพประกอบโค้ดที่เขียนด้วยภาษา Python และใช้ไลบรารี่ Scikit-learn 
สำหรับ Cross Validation แต่ละแบบดังนี้

%--------------------------
\subsection{Train Test Split}
\label{ssec:train_test_split}
\idxen{Cross Validation!Train Test Split}
%--------------------------

ฟังก์ชัน \pyinline{train_test_split} จะทำการแบ่ง Array หรือ Matrices ออกเป็น Train กับ Test Set แบบสุ่ม

\begin{figure}[H]
    \centering
    \includegraphics[width=0.9\linewidth,page=1]{fig/cross_validation.pdf}
    \caption{การทำ Cross Validation ด้วย \pyinline{train_test_split}}
    \label{fig:train_test_split}
\end{figure}

\begin{lstlisting}[style=MyPython]
import numpy as np
from sklearn.model_selection import train_test_split

X = np.array([[0, 1], [2, 3], [4, 5], [6, 7], [8, 9]])
y = np.array([0, 1, 2, 3, 4])
X_train, X_test, y_train, y_test = train_test_split(
    X, y, test_size=0.33, random_state=42)
\end{lstlisting}

%--------------------------
\subsection{Cross Val Score}
\label{ssec:cross_val_score}
\idxen{Cross Validation!Cross Val Score}
%--------------------------

ฟังก์ชัน \pyinline{cross_val_score} ทำการคำนวณคะแนน (Score) โดยการทำ Cross Validation และแสดงค่า Score ของแต่ละส่วน

\begin{figure}[H]
    \centering
    \includegraphics[width=0.9\linewidth,page=2]{fig/cross_validation.pdf}
    \caption{การทำ Cross Validation ด้วย \pyinline{cross_val_score}}
    \label{fig:cross_val_score}
\end{figure}

\begin{lstlisting}[style=MyPython]
from sklearn.model_selection import cross_val_score
from sklearn.datasets import load_iris

iris = load.iris()
clf = svm.SVC(kernel="linear", C=1)
scores = cross_val_score(clf, iris.data, iris.target, cv=5)
\end{lstlisting}

%--------------------------
\subsection{Cross Val Predict}
\label{ssec:cross_val_predict}
\idxen{Cross Validation!Cross Val Score}
%--------------------------

ฟังก์ชัน \pyinline{cross_val_predict} ทำการทำนายสมาชิกหรือข้อมูลแต่ละตัวที่อยู่ใน Test Set

\begin{figure}[H]
    \centering
    \includegraphics[width=0.9\linewidth,page=3]{fig/cross_validation.pdf}
    \caption{การทำ Cross Validation ด้วย \pyinline{cross_val_predict}}
    \label{fig:cross_val_predict}
\end{figure}

\begin{lstlisting}[style=MyPython]
from sklearn.model_selection import cross_val_predict
from sklearn.datasets import load_iris

iris = load.iris()
clf = svm.SVC(kernel="linear", C=1)
predicted = cross_val_predict(clf, iris.data, iris.target, cv=10)
\end{lstlisting}

%--------------------------
\subsection{K Fold}
\label{ssec:f_fold}
\idxen{Cross Validation!K Fold}
%--------------------------

ฟังก์ชัน \pyinline{KFold} จะทำการแบ่ง Dataset ออกเป็น K Fold (โดยที่ K คือจำนวนของการแบ่ง เช่น 3) โดยไม่มีการสลับข้อมูล 
โดยที่แต่ละ Fold จะถูกนำมาใช้เป็น Validation

\begin{figure}[H]
    \centering
    \includegraphics[width=0.9\linewidth,page=4]{fig/cross_validation.pdf}
    \caption{การทำ Cross Validation ด้วย \pyinline{KFold}}
    \label{fig:f_fold}
\end{figure}

\begin{lstlisting}[style=MyPython]
import numpy as np
from sklearn.model_selection import KFold

X = np.array([[1, 2], [3, 4], [1, 2], [3, 4]])
y = np.array([1, 2, 3, 4])
kf = KFold(n_splits=2)
kf.split(X)
kf.get_n_splits(X) # OUTPUT = 2

for i, (train_index, test_index) in enumerate(kf.split(X)):
    print(f"Fold {i}:")
    print(f"  Train: index={train_index}")
    print(f"  Test:  index={test_index}")
\end{lstlisting}

%--------------------------
\subsection{Leave One Out}
\label{ssec:leave_one_out}
\idxen{Cross Validation!Leave One Out}
%--------------------------

ฟังก์ชัน \pyinline{LeaveOneOut} เป็นการนำข้อมูลแต่ละตัวมาใช้เป็น Test Set 1 ครั้งหรือเรียกว่า Singleton ซึ่งจริง ๆ แล้ว 
\pyinline{LeaveOneOut()} นั้นจะเหมือนกับการใช้ \pyinline{KFold(n_splits=n)} และ \pyinline{LeavePOut(p=1)} โดยที่ 
\pyinline{n} คือจำนวนของข้อมูลหรือ Sample

\begin{figure}[H]
    \centering
    \includegraphics[width=0.9\linewidth,page=5]{fig/cross_validation.pdf}
    \caption{การทำ Cross Validation ด้วย \pyinline{LeaveOneOut}}
    \label{fig:leave_one_out}
\end{figure}

\begin{lstlisting}[style=MyPython]
import numpy as np
from sklearn.model_selection import LeaveOneOut

X = np.array([[1, 2], [3, 4]])
y = np.array([1, 2])
loo = LeaveOneOut()
loo.get_n_splits(X) # OUTPUT = 2 

for i, (train_index, test_index) in enumerate(loo.split(X)):
    print(f"Fold {i}:")
    print(f"  Train: index={train_index}")
    print(f"  Test:  index={test_index}")
\end{lstlisting}

%--------------------------
\subsection{Leave P Out}
\label{ssec:leave_p_out}
\idxen{Cross Validation!Leave P Out}
%--------------------------

ฟังก์ชัน \pyinline{LeavePOut} นั้นคล้ายกับ \pyinline{LeaveOneOut()} มาก แต่จะมีความแตกต่างกันตรงที่วิธีนี้นั้นสามารถกำหนดจำนวน%
ของข้อมูลหรือ Sample ที่เรานำไปไปใช้เป็น Test Set ได้

\begin{figure}[H]
    \centering
    \includegraphics[width=0.9\linewidth,page=6]{fig/cross_validation.pdf}
    \caption{การทำ Cross Validation ด้วย \pyinline{LeavePOut}}
    \label{fig:leave_p_out}
\end{figure}

\begin{lstlisting}[style=MyPython]
import numpy as np
from sklearn.model_selection import LeavePOut

X = np.array([[1, 2], [3, 4], [5, 6], [7, 8]])
y = np.array([1, 2, 3, 4])
lpo = LeavePOut(2)
lpo.get_n_splits(X) # OUTPUT = 6

for i, (train_index, test_index) in enumerate(lpo.split(X)):
    print(f"Fold {i}:")
    print(f"  Train: index={train_index}")
    print(f"  Test:  index={test_index}")
\end{lstlisting}

%--------------------------
\section{การคัดเลือกลักษณะเฉพาะ}
\label{sec:select_feat}
\idxen{Feature Selection}
%--------------------------

\begin{algorithm}[ht]
    \caption{อัลกอริทึม Forward Search สำหรับการทำ Feature Selection}
    \label{alg:forward_search}
    \begin{algorithmic}
    \State Initialize $\mathcal{F} = \emptyset$.
    \Repeat
        \For{$i=1,\ldots,d$}
            \If{$i \not\in \mathcal{F}$}
                \State $\mathcal{F}_i = \mathcal{F} \cup \{i\}$
                \State Use some version of cross validation to evaluate features $\mathcal{F}_i$.
                \State (i.e., train your learning algorithm using only the features in $\mathcal{F}_i$, and 
                estimate its generalization error.)
            \EndIf
        \EndFor
        \State Set $\mathcal{F}$ to be the best feature subset found in the previous step. % DIFF.
    \Until{convergence}
    \State Select and output the best feature subset that was evaluated during the entire search procedure.
    \end{algorithmic}
\end{algorithm}

การคัดเลือกลักษณะเฉพาะ (Feature Selection) เป็นการหา Feature ที่เหมาะสมที่สุดสำหรับการใช้อธิบายข้อมูลของโมเลกุล โดยเราจะทำการ%
เรียงลำดับความสำคัญของ Feature แล้วทำการคัดเลือกเฉพาะ Feature ที่คิดว่าสอดคล้องกับเอาต์พุตที่ต้องการทำนายและคัด Feature ที่มีความ%
สำคัญน้อยออกไปเพื่อหลีกเลี่ยง Bias ที่อาจจะเกิดขึ้น อธิบายง่าย ๆ คือเป็นเทคนิคที่เรานำมาใช้เพื่อลดจำนวณของ Feature นั่นเอง 

อัลกอริทึมของ Feature Selection แบบที่ง่ายที่สุดนั้นชื่อว่า Forward Search ซึ่งดูได้ตามอัลกอริทึมที่ \ref{alg:forward_search} โดยเริ่ม%
ต้นนั้นกำหนดให้ $\mathcal{F}$ เป็นเซตของจำนวน Feature ทั้งหมดซึ่งยังเป็นเซตว่างอยู่ แล้วเราก็ทำการ Cross Validation ไปทีละ Feature
โดยในลูปด้านในนั้นจะเพิ่ม Feature เข้าไปใน $\mathcal{F}$ ทีละอันจนกระทั้งครบทุก Feature $\mathcal{F} = \{1,\ldots ,d\}$ 
ซึ่งจะเป็นการสิ้นสุดกระบวนการทำ Feature Search 

นอกจากนี้ยังมีอัลกอริทึมที่ตรงข้ามกับ Forward Search เรียกว่า Backward Search โดยแทนที่เราจะกำหนด $\mathcal{F}$ ให้เป็นเซตว่างนั้น%
เราจะเริ่มด้วย $\mathcal{F}$ ที่เป็นเซตที่มี Feature อยู่ครบทั้งหมดแล้วทำการลบ Feature ออกทีละอันจนกระทั่ง $\mathcal{F}$ เป็นเซตว่าง

%--------------------------
\section{ปัญหา Bias-Variance}
\label{sec:bias_var_prob}
\idxen{Bias-Variance}
%--------------------------

หนึ่งในปัญหาที่เราทุกคนจะต้องเจอในการสร้างโมเดลนั่นก็คือ Bias-Variance Problem ซึ่งนำไปสู่ปัญหาเรื่อง Overfitting ต่อไป
เราลองมาดูรายละเอียดกันครับ กำหนดให้โมเดลของเราแทนด้วย $\hat{f}(\vec{x})$ และค่าอ้างอิงหรือคำตอบที่เราจะมาเทียบกับการทำนายเป็น 
$y$ และความคลาดเคลื่อนที่เกิดขึ้นเป็น

\begin{equation}
    E\left[\left(y - \hat{f}(\vec{x})\right)^2\right]
\end{equation}

ซึ่งจริง ๆ แล้ว เป็นฟังก์ชันที่สมบูรณ์แบบมาก แต่ทว่าในความเป็นจริงแล้วในชุดข้อมูลของเรานั้นย่อมมี Noise $(\epsilon)$ ซึ่งค่าความแตกต่าง%
ระหว่างโมเดลของเรากับคำตอบก็จะมีการปนเปื้อนหรือ Contaminate โดย $\epsilon$ ดังนี้

\begin{equation}
    y = f(\vec{x}) + \epsilon
\end{equation}

\noindent จึงทำให้ค่าความคลาดเคลื่อนที่เกิดขึ้นจริง ๆ นั้นมีสมการดังต่อไปนี้ 

\begin{align}
    E\left[\left(y - \hat{f}(\vec{x})\right)^2\right] &= 
    E\left[y^2\right] + E\left[\hat{f}(\vec{x})^2\right] - 2 E\left[y\hat{f}(\vec{x})\right] \\
    &= E\left[\left(f(\vec{x}) - \epsilon\right)^2\right] + \hat{f}(\vec{x})^2 - 2 E\left[\left(f(\vec{x}) - 
    \epsilon\right)\right]\hat{f}(\vec{x})
\end{align}

\noindent ซึ่งถ้าหากเราทำการพิสูจน์สมการด้านบนโดยพยายามจัดรูปให้อยู่ในเทอมที่มี Bias และ Variance จากชุดข้อมูล เราจะได้สมการดังต่อไปนี้

\begin{align}\label{eq:bias_variance}
E\left[\left(y - \hat{f}(\vec{x})\right)^2\right] = 
    & \underbrace{E\left[f(\vec{x}) - \hat{f}\left(\vec{x}; \bm{D}\right)\right]^2}_{\text{Bias}} \nonumber \\
    & + \underbrace{E\left[\left(E\left[\hat{f}\left(\vec{x}; \bm{D}\right)\right] - 
    \hat{f}\left(\vec{x}; \bm{D}\right)\right)^2\right]}_{\text{Variance}} \nonumber \\
    & + \sigma^2
\end{align}

โดยพจน์แรกนั้นเป็น Bias, พจน์ที่สองเป็น Variance และพจน์ที่สามเป็นค่าความแปรปรวนที่คำนวณจาก Standard Deviation ของ Noise 
$(\epsilon)$ ประเด็นก็คือว่าเราสามารถควบคุม Bias กับ Variance ได้ แต่เราไม่สามารถควบคุม Noise ได้เพราะมันเป็นสิ่งที่ผูกติดมากับชุด%
ข้อมูล ซึ่งการที่เรามี Bias และ Variance ที่ไม่สมดุลกันนั้นจะทำให้เกิดผลลัพธ์ที่ตามมาในระหว่างการฝึกสอนโมเดล นั่นคือ Overfitting และ 
Underfitting

%--------------------------
\section{การเพิ่มประสิทธิภาพการเรียนรู้และแก้ปัญหา Overfitting}
\label{sec:fix_overfit}
%--------------------------

\begin{figure}[htbp]
    \centering
    \includegraphics[width=\linewidth]{fig/overfitting.png}
    \caption{โมเดลที่มีความ Overfitting และ Underfitting กับชุดข้อมูลมากเกินไป}
    \label{fig:overfitting}
\end{figure}

\begin{description}[style=nextline]
    \item[Overfitting] โมเดลตอบสนองต่อ Noise ที่มากเกินไป ทำให้เกิดการเรียนรู้และจดจำ Noise และไม่สามารถที่จะเรียนรู้รายละเอียด%
    จริง ๆ ของข้อมูลได้ ซึ่งส่งผลให้ทำนายข้อมูลไม่ได้หรือผิดพลาดมากกว่าที่คาดไว้หรือยอมรับได้ โดยกรณีนี้โมเดลจะมีค่าความแปรปรวนของข้อมูลสูง 
    (High Variance)
    
    \item[Underfitting] โมเดลของเราไม่สามารถหาความสัมพันธ์ระหว่างอินพุต $(x)$ กับเอาต์พุต $(y)$ ได้เพราะว่ามีข้อมูลที่ใช้ใน%
    การเทรนน้อยเกินไปหรือดึงข้อมูลออกมาจาก Training Set ได้ไม่เพียงพอที่จะเรียนรู้ โดยในกรณีนี้โมเดลจะมีค่าความเอนเอียงสูง (High Bias)

    \item[Noisy] โมเดลไม่มี Overfitting และ Underfitting แต่ยังมีค่า Error ของการฝึกสอนที่ยังสูงอยู่มาก ซึ่งสาเหตุก็อาจจะมาจาก%
    การที่ชุดข้อมูลมี Noise มากเกินไปนั่นเอง
\end{description}

ภาพที่ \ref{fig:overfitting} แสดงการเปรียบเทียบระหว่างกรณีของ Underfitting และ Overfitting ซึ่งเป็นหนึ่งในปัญหาหลักที่มักจะพบ%
เจอได้ทั่วไปใน ML โดยเราสามารถสรุปความสัมพันธ์จากกรณีได้กล่าวได้ดังนี้

\fbox{%
\begin{minipage}{0.9\linewidth}
    \begin{align*}
        \text{Bias สูง} &\;\longleftrightarrow\; \text{Underfitting}\\
        \text{Variance สูง} &\;\longleftrightarrow\; \text{Overfitting}\\
        \text{$\sigma^2$ มาก} &\;\longleftrightarrow\; \text{Noisy Data}\\
    \end{align*}
\end{minipage}}

วิธีการจัดการกับ Overfitting แบบที่ง่ายที่สุดคือการเพิ่มจำนวนข้อมูลในการฝึกสอนโมเดล นอกจากนี้ยังมีวิธีอื่น ๆ ที่เราสามารถใช้ในการจัดการกับ%
ปัญหาข้างต้นได้เช่นเดียวกัน มีดังต่อไปนี้

%--------------------------
\subsection{Data Augmentation}
\label{ssec:data_aug}
\idxen{Data Augmentation}
%--------------------------

\begin{figure}[H]
    \centering
    \includegraphics[width=0.9\linewidth]{fig/data_aug_butterfly.jpg}
    \caption{ตัวอย่างการทำ Data Augmentation สำหรับข้อมูลที่เป็นรูปภาพ เช่น การเปลี่ยนสี การเพิ่มความคมชัด การเพิ่ม Noise 
    (เครดิตภาพ: \textit{PLoS ONE} 12(8): e0183838)}
    \label{fig:data_aug_butterfly}
\end{figure}

วิธีการทำ Data Augmentation นั้นจะตรงข้ามกับการทำความสะอาดข้อมูล (Data Cleaning) นั่นก็คือจะเป็นวิธีที่เราจะใส่ Noise หรือสิ่งที่%
ไม่ได้เกี่ยวข้องกับข้อมูลโดยตรงเข้าไปในชุดการฝึกสอน รวมไปถึงการแก้ไขข้อมูลให้แตกต่างไปจากเดิม แต่ยังคงไว้ซึ่งลักษณะของข้อมูลนั้น 
ซึ่งการทำ Data Augmentation จะเป็นการช่วยไม่ใช่ให้เกิดการเรียนรู้ที่มันยึดติดกับชุดข้อมูลฝึกสอนมากเกินไป ในปัจจุบันวิธีการนี้ได้รับความนิยม%
เพราะสามารถทำได้ง่าย สะดวก และไม่มีความซับซ้อนในการทำ โดยมีความจำเป็นอย่างยิ่งกรณีที่ชุดข้อมูลมีขนาดเล็ก (จำนวนข้อมูลไม่เยอะ) 
แต่ต้องการนำมาใช้ในการฝึกสอนด้วยเทคนิค ML ที่ต้องการข้อมูลในปริมาณที่เยอะในการฝึกสอน เช่น Deep Learning\autocite{bengio2021}

%--------------------------
\subsection{Early Stopping}
\label{ssec:early_stop}
\idxen{Early Stopping}
%--------------------------

\begin{figure}[H]
    \centering
    \includegraphics[width=0.9\linewidth]{fig/early_stopping.png}
    \caption{การทำ Regularization ด้วยวิธี Early Stopping สำหรับการฝึกสอนโมเดล High-degree Polynomial Regression 
    โดยใช้ Batch Gradient Descent และใช้ RMSE ในการวัดค่าความคลาดเคลื่อน (เครดิตภาพ: https://www.oreilly.com)}
    \label{fig:early_stopping}
\end{figure}

วิธี Early Stopping มีความหมายวิธีการทำงานตามชื่อเลยนั่นก็คือหยุดให้เร็วขึ้น เป็นวิธีการที่เราจะกำหนด (บังคับ) ให้การฝึกสอนหรือ Training 
นั้นหยุดก่อนที่โมเดลของเราจะเริ่มเรียนรู้ Noise ที่อยู่ภายในชุดข้อมูล แทนที่จะเรียนรู้เฉพาะชุดข้อมูลอย่างเดียว ซึ่งวิธีการนี้จะเป็นการป้องกันการเปิด 
Bias แบบตรงไปตรงมา อย่างไรก็ตามเราควรจะต้องระมัดระวังในการใช้เทคนิค Early Stopping เพราะว่าถ้าเราบังคับให้โมเดลหยุดเรียนรู้เร็วเกินไป
ปัญหาที่อาจจะเกิดขึ้นแทนการ Overfitting นั่นก็คือการ Underfitting ของโมเดล ซึ่งการเลือกจุดที่จะให้โมเดลนั้นหยุดการเรียนรู้ก็ถือว่ามีความ%
เป็น Art อย่างหนึ่ง ซึ่งจุดที่เราเลือกต้องมีความเหมาะสมระหว่าง Overfitting และ Underfitting

\noindent โค้ดของการทำ Early Stopping โดยใช้ไลบรารี่ Scikit-Learn

\begin{lstlisting}[style=MyPython]
from copy import deepcopy
from sklearn.metrics import mean_squared_error
from sklearn.preprocessing import StandardScaler

X_train, y_train, X_valid, y_valid = [...]  # split the quadratic dataset

preprocessing = make_pipeline(PolynomialFeatures(degree=90, include_bias=False),
                              StandardScaler())
X_train_prep = preprocessing.fit_transform(X_train)
X_valid_prep = preprocessing.transform(X_valid)
sgd_reg = SGDRegressor(penalty=None, eta0=0.002, random_state=42)
n_epochs = 500
best_valid_rmse = float('inf')

# Training with applying early stopping
for epoch in range(n_epochs):
    sgd_reg.partial_fit(X_train_prep, y_train)
    y_valid_predict = sgd_reg.predict(X_valid_prep)
    val_error = mean_squared_error(y_valid, y_valid_predict, squared=False)
    if val_error < best_valid_rmse:
        best_valid_rmse = val_error
        best_model = deepcopy(sgd_reg)
\end{lstlisting}

\noindent โค้ดของการสร้าง Callback ของ Early Stopping โดยใช้ไลบรารี่ TensorFlow 

\begin{lstlisting}[style=MyPython]
import tensorflow as tf

callback = tf.keras.callbacks.EarlyStopping(monitor='loss', patience=3)

model = tf.keras.models.Sequential([tf.keras.layers.Dense(10)])
model.compile(tf.keras.optimizers.SGD(), loss='mse')

history = model.fit(np.arange(100).reshape(5, 20), np.zeros(5),
                    epochs=10, batch_size=1, callbacks=[callback],
                    verbose=0)

len(history.history['loss'])  # Only 4 epochs are run.
4
\end{lstlisting}

\noindent โดย Callback จะทำการหยุดการฝึกสอน (Training) เมื่อค่า Loss ไม่มีการลดลงภายใน 3 Epochs ที่ต่อเนื่องกัน

%--------------------------
\subsection{Ensemble Method}
\label{ssec:ensemble_model}
\idxen{Ensemble Method}
%--------------------------

\begin{figure}[htbp]
    \centering
    \includegraphics[width=0.9\linewidth]{fig/ensemble_method.png}
    \caption{การทำงานร่วมกันของโมเดลหลาย ๆ โมเดลโดยใช้วิธี Ensemble (เครดิตภาพ: https://www.manning.com)}
    \label{fig:ensemble_method}
\end{figure}

เทคนิคนี้เป็นการนำโมเดลหลาย ๆ โมเดลมารวมกันเพื่อที่จะทำให้ผลลัพธ์ของการทำนายคำตอบมีค่าที่ดีที่สุด โดยโมเดล ML ที่เราจะมานำผสมกันนั้น%
จะเป็นอะไรก็ได้ เช่น Linear Regression, Logistic Regression, Gaussian Process Regression ผู้อ่านสามารถดูภาพที่ 
\ref{fig:ensemble_method} ประกอบได้ โดยจะเห็นว่าเรามีโมเดลที่มีประสิทธิภาพไม่ค่อยดีนักหลาย ๆ โมเดล เราสามารถนำโมเดลเหล่านี้มา%
รวมกันเพื่อให้ได้โมเดลที่มีประสิทธิมากขึ้นได้

โดยเทคนิคย่อยของEnsemble Method ที่นิยมใช้กันนั้นมีอยู่ด้วยกัน 3 วิธี ดังนี้

\begin{itemize}[topsep=0pt]
    \item \textbf{Bagging} เราจะทำการสร้างข้อมูลประเภทเดียวกันแบบหลาย ๆ ชุด แล้วทำการทดสอบกับข้อมูลเพียงแค่บางส่วน (Subset) 
    ของชุดข้อมูล จากนั้นนำผลการทำนายของโมเดลต่าง ๆ มารวมกัน ตัวอย่างของอัลกอริทึมที่ใช้ในการเรียนรู้สำหรับเทคนิค Bagging นี้ เช่น 
    Decision Tree, Random Forest และ Extra Tree
    \idxen{Ensemble Method!Bagging}

    \item \textbf{Boosting} จะทำคล้ายกับ Bagging เลยก็คือเริ่มต้นด้วยการสร้างข้อมูลประเภทเดียวกันแบบหลาย ๆ ชุด แล้วทำการทดสอบ%
    กับข้อมูลชุดเดียวกันโดยทำการทดสอบแบบวนซ้ำ (Iteration) แล้วปรับค่าน้ำหนักเพื่อทำให้ผลการทำนายของโมเดลนั้นดีขึ้นเรื่อย ๆ ซึ่งวิธีนี้ค่อน%
    ข้างเป็นที่นิยมเพราะมีความยืดหยุ่นและใช้ได้กับทุกอัลกอริทึม นอกจากนี้ยังสามารถปรับลดค่าความคลาดเคลื่อนของ Bias ของโมเดลได้ดีอีกด้งบ 
    ตัวอย่างของอัลกอริทึมที่ใช้ในการเรียนรู้สำหรับเทคนิค Boosting นี้ เช่น AdaBoost และ Stochastic Gradient Boosting
    \idxen{Ensemble Method!Boosting}

    \item \textbf{Voting} เราจะเริ่มด้วยการสร้างโมเดลที่แตกต่างกันหลาย ๆ โมเดล เช่น Decision Tree, Support Vector Machine, 
    K-Nearest Neighbors จากนั้นทำการฝึกสอนโมเดลด้วยชุดข้อมูลชุดเดียวกันเพื่อดูผลการทำนายที่ดีที่สุดของแต่ละโมเดล แล้วใช้การโหวตผลที่%
    เหมือนกันหรือคล้ายกันเพื่อเป็นคำตอบสุดท้าย
    \idxen{Ensemble Method!Voting} 
\end{itemize}

%--------------------------
\subsection{Dropout}
\label{ssec:dropout}
\idxen{Dropout}
%--------------------------

วิธีการ Dropout เป็นเทคนิคพิเศษที่ถูกคิดค้นขึ้นมาเพื่อแก้ปัญหา Overfitting ใน Deep Learning โดยเฉพาะ ซึ่งไอเดียของเทคนิคนี้ก็คือ%
เราจะทำการตัด (Drop out หรือเอาออกไป) หน่วยการเรียนรู้ (Learning Unit หรือ Neuron) ใน Neural Network ออกไป ซึ่งจะเป็นการช่วย%
ให้โมเดลของเราลด Bias ที่เกิดจากการเรียนรู้ของข้อมูลที่มากเกินไป โดยจำนวนของ Neuron ที่จะตัดออกไปนั้นส่วนใหญ้แล้วจะคิดเป็นเปอร์เซนต์ของ
Neuron ทั้งหมด เช่น ตัดออกไป 5 เปอร์เซนต์
\idxen{Overfitting}

โค้ดของการทำ Dropout โดยใช้ไลบรารี่ TensorFlow

\begin{lstlisting}[style=MyPython]
>>> tf.random.set_seed(0)
>>> layer = tf.keras.layers.Dropout(.2, input_shape=(2,))
>>> data = np.arange(10).reshape(5, 2).astype(np.float32)
>>> print(data)
[[0. 1.]
 [2. 3.]
 [4. 5.]
 [6. 7.]
 [8. 9.]]
>>> outputs = layer(data, training=True)
>>> print(outputs)
tf.Tensor(
[[ 0.    1.25]
 [ 2.5   3.75]
 [ 5.    6.25]
 [ 7.5   8.75]
 [10.    0.  ]], shape=(5, 2), dtype=float32)
\end{lstlisting}

%--------------------------
\subsection{L1 Regularization}
\label{ssec:l1_reg}
\idxen{Regularization!L1 (LASSO)}
%--------------------------

ตามที่เราได้ศึกษาเรื่อง L1 กันไปแล้วในบทที่ \ref{ch:kernel} เราสามารถทำการปรับปรุง Loss Function ของเราได้ด้วยการเพิ่มพารามิเตอร์%
แบบพิเศษเข้าไป นั่นก็คือการใส่การลงโทษหรือ Penalty ให้กับการเรียนรู้ของโมเดล โดยการปรับพารามิเตอร์ $\lambda$ (ในบทที่ \ref{ch:kernel}
จะใช้ตัวแปร $\alpha$ ซึ่งมีความหมายเหมือนกัน) ให้เพิ่มขึ้นนั้นจะเป็นการลด Variance แต่ในขณะเดียวกันก็จะเป็นการเพิ่ม Bias โดยใน 
Linear Regression นั้นเราจะเรียก Regularization แบบ L1 ว่า LASSO

\begin{equation}
    L = \frac{1}{N}\sum_i^N \left[y_i - \hat{f}(\vec{x}_i, \vec{w}, b)\right]^2 + \lambda \sum_k \left|w_k\right|
\end{equation}

%--------------------------
\subsection{L2 Regularization}
\label{ssec:l2_reg}
\idxen{Regularization!L2 (Ridge)}
%--------------------------

สำหรับ Regularization แบบ L2 นั้นก็จะมีความคล้ายกับ L1 มาก ซึ่งวิธีนี้ในการทำ Linear Regression จะมีชื่อเรียกว่า Ridge Regression

\begin{equation}
    L = \frac{1}{N}\sum_i^N \left[y_i - \hat{f}(\vec{x}_i, \vec{w}, b)\right]^2 + \lambda \sum_k w_k^2
\end{equation}

สำหรับการเลือก Regularization นั้น ผู้เขียนขอยกประโยคของศาสตราจารย์ Frank Harrell ที่ได้แนะนำการเลือก L1 และ L2 ไว้ดังนี้%
\footnote{อ้างอิง \url{https://stats.stackexchange.com/a/184022/283188}}

\begin{framed}
    \enquote{Generally speaking if you want optimum prediction use L2. 
    If you want parsimony at some sacrifice of predictive discrimination use L1. 
    But note that the parsimony can be illusory, e.g., repeating the \textit{lasso} 
    process using the bootstrap will often reveal significant instability 
    in the list of features \enquote{selected} especially when predictors are 
    correlated with each other.}
\end{framed}

ซึ่งตีความได้คร่าว ๆ ว่าถ้าหากต้องการการทำนายที่เหมาะสมที่สุดให้ใช้ L2 หรือ Ridge Regression แต่ถ้าหากต้องการทำให้การจำแนกเชิงพยากรณ์
(Predictive Discrimination) มีความสม่ำเสมอกันสำหรับทุก ๆ Feature ให้ใช้ L1 หรือ Lasso Regression แต่ควรเข้าใจไว้ด้วยว่าการ%
ใช้ L1 สามารถทำให้เกิดปัญหาได้ เช่น การทำ Lasso Regression โดยใช้เทคนิค Bootstrap (การ Sample ตัวอย่างจากชุดข้อมูล)
 % การเลือกและปรับแต่งโมเดล
% LaTeX source for ``การเรียนรู้ของเครื่องสำหรับเคมีควอนตัม (Machine Learning for Quantum Chemistry)''
% Copyright (c) 2022 รังสิมันต์ เกษแก้ว (Rangsiman Ketkaew).

% License: Creative Commons Attribution-NonCommercial-NoDerivatives 4.0 International (CC BY-NC-ND 4.0)
% https://creativecommons.org/licenses/by-nc-nd/4.0/

\chapter{การเรียนรู้แบบไม่มีผู้สอน}
\label{ch:unsup_ml}

การเรียนรู้แบบไม่มีผู้สอนหรือ Unsupervised Learning เป็นเทคนิคที่อาจจะเรียกว่าได้ตรงข้ามกับ supervised learning ก็ได้
เพราะว่าเทคนิคประเภทนี้จะเป็นการเทรนโมเดลแบบไม่มีการบอกคำตอบหรือเอาต์พุตให้โมเดลได้รับรู้ ดังนั้นสิ่งที่โมเดลจะต้องพยายามทำออกมา
ให้ได้คือเป็นการเรียนรู้หาความสัมพันธ์ (Relation) หรือ สหสัมพันธ์ (Correlation) ระหว่างข้อมูลแต่ละตัวภายในชุดข้อมูล (Input) 
ที่เราสนใจได้ได้ป้อนเข้าไป

%--------------------------
\section{Weighted Pair Group Method with Arithmetic Mean}
\idxen{Weighted Pair Group Method with Arithmetic Mean}
%--------------------------

Weighted Pair Group Method with Arithmetic Mean (WPGMA) เป็นเทคนิคการจัดกลุ่ม (Clustering) แบบ Hierarchical 
โดยถูกพัฒนาขึ้นมาโดยใช้ Pairwise Similarity Matrix\cite{sokal1958} 

%--------------------------
\section{Principal Component Analysis}
\idxen{Principal Component Analysis}
%--------------------------

\begin{figure}[H]
    \centering
    \includegraphics[width=0.8\linewidth]{fig/pca.png}
    \caption{Principal Component Analysis ของข้อมูลที่มีการกระจายตัวแบบเกาส์เซียนหลายตัวแปร (Multivariate Gaussian
    Distribution) เวกเตอร์ที่แสดงนั้นเป็น Eigenvector ของ Covariance Matrix ที่มีการปรับขนาด (Scaled) โดยใช้ค่ายกกำลังสองของ 
    Eigenvalue และมีการปรับตำแหน่งโดยใช้ค่าเฉลี่ย}
    \label{fig:pca}
\end{figure}

การวิเคราะห์องค์ประกอบหลักหรือ Principal Component Analysis (PCA) เป็นวิธีทางสถิติที่ถูกเอามาใช้เพื่อรับมือกับข้อมูลที่มี
จำวนหลายมิติหรือมีหลายตัวแปร โดย PCA สามารถหาความสัมพันธ์ของตัวแปรเหล่านั้นโดยทำการลดขนาดของมิติโดยสร้างชุดข้อมูลใหม่ที่อาศัย
แกนอ้างอิงจากชุดข้อมูลเดิม ซึ่งจำนวนมิติที่ถูกลดลงนั้นก็มีจำนวนมิติเพียง 2 หรือ 3 มิติเท่านั้น ซึ่งทำให้ง่ายต่อการตีความและวิเคราะห์ข้อมูล เช่น
การจัดกลุ่มชุดข้อมูลโดยจำแนกตาม Feature ซึ่ง Feature แต่ละคู่จะมีคุณสมบัติ Orthogonality หรือตั้งฉากกันนั่นเอง ทำให้เราสามารถแสดง
ผลลัพธ์ของ PCA ออกมาได้ในปริภูมิทั่วไป

%--------------------------
\section{Autoencoder}
\idxboth{การเข้ารหัสแบบอัตโนมัติ}{Autoencoder}
%--------------------------

\begin{center}
\begin{tikzpicture}[x=2.1cm,y=1.2cm]
    % \large
    \message{^^JNeural network without arrows}
    \readlist\Nnod{6,5,4,3,4,5,6} % array of number of nodes per layer
    
    % TRAPEZIA
    \node[above,align=center,myorange!60!black] at (3,2.4) {Encoder};
    \node[above,align=center,myblue!60!black] at (5,2.4) {Decoder};
    \draw[myorange!40,fill=myorange,fill opacity=0.02,rounded corners=2]
    (1.6,-2.7) --++ (0,5.4) --++ (2.8,-1.2) --++ (0,-3) -- cycle;
    \draw[myblue!40,fill=myblue,fill opacity=0.02,rounded corners=2]
    (6.4,-2.7) --++ (0,5.4) --++ (-2.8,-1.2) --++ (0,-3) -- cycle;
    
    \message{^^J  Layer}
    \foreachitem \N \in \Nnod{ % loop over layers
    \def\lay{\Ncnt} % alias of index of current layer
    \pgfmathsetmacro\prev{int(\Ncnt-1)} % number of previous layer
    \message{\lay,}
    \foreach \i [evaluate={\y=\N/2-\i+0.5; \x=\lay; \n=\nstyle;}] in {1,...,\N}{ % loop over nodes
    
    % NODES
    \node[node \n,outer sep=0.6] (N\lay-\i) at (\x,\y) {};
    
    % CONNECTIONS
    \ifnum\lay>1 % connect to previous layer
    \foreach \j in {1,...,\Nnod[\prev]}{ % loop over nodes in previous layer
    \draw[connect,white,line width=1.2] (N\prev-\j) -- (N\lay-\i);
    \draw[connect] (N\prev-\j) -- (N\lay-\i);
    %\draw[connect] (N\prev-\j.0) -- (N\lay-\i.180); % connect to left
    }
    \fi % else: nothing to connect first layer
    
    }
    }
    
    % LABELS
    \node[above=2,align=center,mygreen!60!black] at (N1-1.90) {Input};
    \node[above=2,align=center,myred!60!black] at (N\Nnodlen-1.90) {Output};
    
\end{tikzpicture}
\end{center}

ตัวเข้ารหัสแบบอัตโนมัติหรือ Autoencoder (AE) เป็นอัลกอริทึม Unsupervised Learning แบบหนึ่งที่สร้างโมเดล ANN โดยมีรูปแบบของ 
Network ที่เฉพาะตัวนั่นก็คือจะทำการลดหรือบีบอัดข้อมูล (Encoder) และทำการถอดรหัส (Decoder) ออกมาเป็นข้อมูลเดิม\cite{ballard1987} 
ตัวโมเดล AE มีความพิเศษคือจะมีลักษณะของความสมมาตร นอกจากนี้ยังมีความแตกต่างจาก PCA นั่นก็คือสามารถบีบอัดหรือลดจำนวนมิติของข้อมูล%
แบบไม่เป็นเส้นตรงได้ (Nonlinear) ได้ด้วยการใช้ Nonlinear Activation Function 

%--------------------------
\section{K-means Clustering Algorithm}
\idxen{K-means Clustering}
%--------------------------

 % การเรียนรู้แบบไม่มีผู้สอน
% LaTeX source for ``การเรียนรู้ของเครื่องสำหรับเคมีควอนตัม (Machine Learning for Quantum Chemistry)''
% Copyright (c) 2022 รังสิมันต์ เกษแก้ว (Rangsiman Ketkaew).

% License: Creative Commons Attribution-NonCommercial-NoDerivatives 4.0 International (CC BY-NC-ND 4.0)
% https://creativecommons.org/licenses/by-nc-nd/4.0/

{
\thispagestyle{empty}
~\vfill

\begin{doublespace}
\noindent\fontsize{16}{22}\selectfont\itshape
\nohyphenation
Science never solves a problem without creating ten more.\\
\noindent - \mbox{George Bernard Shaw} (1856 - 1950)
\end{doublespace}

\vfill
\vfill
}

\part{การทำนายคุณสมบัติเชิงเคมีควอนตัม}
% LaTeX source for ``การเรียนรู้ของเครื่องสำหรับเคมีควอนตัม (Machine Learning for Quantum Chemistry)''
% Copyright (c) 2022 รังสิมันต์ เกษแก้ว (Rangsiman Ketkaew).

% License: Creative Commons Attribution-NonCommercial-NoDerivatives 4.0 International (CC BY-NC-ND 4.0)
% https://creativecommons.org/licenses/by-nc-nd/4.0/

\chapter{คุณสมบัติเชิงอิเล็กทรอนิกส์ของโมเลกุล}
\label{ch:el_prop}

เนื้อหาในบทนี้จะเกี่ยวกับคุณสมบัติเชิงอิเล็กทรอนิกส์ (Electronic Properties) ของโมเลกุล ซึ่งเป็นวัตถุประสงค์ของการใช้ ML เข้ามาศึกษาเคมีควอนตัม
โมเลกุลเป็นหน่วยพื้นฐานของสิ่งต่าง ๆ รอบตัวเรา ซึ่งโมเลกุลก็คือเป็นกลุ่มของอะตอมหลาย ๆ อะตอมมารวมกัน และในอะตอมนั้นเราสนใจอิเล็กตรอนเป็นพิเศษ
ในวิชาควอนตัมนั้นเราจะอธิบายพฤติกรรมของโมเลกุลโดยมุ่งเน้นไปที่อิเล็กตรอนซึ่งสามารถที่จะอธิบายได้ด้วยฟังก์ชันทางคณิตศาสตร์ที่เรียกว่าฟังก์ชันคลื่น 
(Wavefunction) โดยหน้าตาของ Wavefunction นั้นจริง ๆ แล้วไม่มีการนิยามแบบตายตัว โดยหนึ่งในสมการที่โด่งดังที่สุดสมการหนึ่งของวงการ%
วิทยาศาสตร์นั่นคือสมการชโรดิงเงอร์ (Schrödinger Equation) โดยถูกนำมาใช้ในการศึกษาระบบทางกลศาสตร์ควอนตัม ซึ่งการแก้สมการชโรดิงเงอร์%
ได้นั้นจะทำให้ได้มาซึ่ง Wavefunction 

สมการชโรดิงเงอร์สามารถแบ่งออกได้เป็นสองแบบ คือ แบบที่ไม่ขึ้นกับเวลาและแบบที่ขึ้นกับเวลา ดังนี้\cite{szabo1996,cramer2004,jensen2017}

\noindent Time-dependent Schrödinger Equation

\begin{equation}
    i \hbar \frac{d}{d t} \ket{\Psi(t)} = \hat{H} \ket{\Psi(t)}
\end{equation}

\noindent Time-independent Schrödinger Equation

\begin{equation}
    \hat{H}\ket{\Psi} = E \ket{\Psi}
\end{equation}

โดย Wavefunction ($\Psi(t)$) ที่เป็น Eigenfunction นั้นจะบรรจุข้อมูลเชิงอิเล็กทรอนิกส์ทุกอย่างเกี่ยวกับระบบของเราเอาไว้ ซึ่งในที่นี้ก็คือโมเลกุล
โดยสมการข้างต้นเป็นการคำนวณหาพลังงานของระบบโดยใช้ Hamiltonian Operator ($\hat{H}$) ซึ่งเป็น Operator ที่สอดคล้องกับพลังงาน
ซึ่งจริง ๆ แล้ว Eigenvalue ของสมการข้างต้นจะเป็นคุณสมบัติของโมเลกุลอะไรก็ได้ ตราบใดที่เราใช้ Operator ที่สอดคล้องกับคุณสมบัตินั้น ๆ 

หนึ่งในเป้าหมายสำคัญของกลศาสตร์ควอนตัมเชิงโมเลกุลก็คือการแก้สมการ Time-independent Schrödinger Equation และคำนวณหาโครงสร้าง%
เชิงอิเล็กทรอนิกส์ (Electronic Structures) ของอะตอมและโมเลกุล โดยหัวข้อแรกของบทนี้ที่เราจะมาดูกันแบบละเอียดก็คือการใช้เทคนิคเชิงคำนวณ%
และอาศัยการประมาณค่าในการแก้สมการดังกล่าว โดยทั่วไปนั้นจะมีวิธีการหลัก ๆ 2 วิธีที่สามารถช่วยให้เราหาคำตอบของสมการชโรดิงเงอร์ ได้นั่นคือ 
\textbf{\textit{ab initio} method} ซึ่งเป็นวิธีที่ความถูกแม่นยำของผลลัพธ์ที่ได้จากการแก้สมการนั้นจะขึ้นอยู่กับโมเดลที่เรานำมาใช้ในการ%
อธิบาย Wavefunction ของโมเลกุล และเป็นที่ทราบกันดีว่าสำหรับโมเลกุลที่มีขนาดใหญ่นั้น วิธีการ \textit{ab initio} นี้จะมีความสิ้นเปลืองสูงมาก 
ดังนั้นจึงเป็นที่มาของวิธีการที่สอง นั่นคือ \textbf{Semiempirical method} ซึ่งจะใช้เทคนิคการมอง Hamiltonian ในรูปแบบที่ง่ายกว่า 
และอาศัยค่า Parameter ที่ได้จากการทดลองเพื่อเพิ่มความแม่นยำ อย่างไรก็ตาม วิธี Density Functional Theory (DFT) ก็ถูกพัฒนาขึ้นมา%
เพื่อแก้ปัญหาที่เราจะต้องมาแก้หรือประมาณค่า Wavefunction ตรง ๆ จึงทำให้ความสิ้นเปลืองของการคำนวณนั้นต่ำมากเมื่อเทียบกับสองวิธีข้างต้นทีได้กล่าวไป

%--------------------------
\section{การแก้สมการฟังก์ชันคลื่นเพื่อคำนวณพลังงาน}
%--------------------------

%--------------------------
\subsection{วิธี Self-Consistent Field}
%--------------------------

ในหัวข้อนี้เราจะมาพูดถึงการแก้สมการชโรดิงเงอร์โดยใช้วิธีที่ชื่อว่า Self-Consistent Field (SCF) ซึ่งเป็นการประมาณค่า Hamiltonian แบบวนซ้ำ
เริ่มต้นเราจะต้องมาดูกันก่อนว่าการมอง Wavefunction ของระบบหลายอิเล็กตรอนสำหรับวิธี SCF นั้นจะมีการตัดสิ่งที่ซับซ้อนออกไปนั่นก็คือ
Electron-Electron Repulsion หรืออันตรกิริยาระหว่างอิเล็กตรอน โดย Wavefunction สามารถถูกอธิบายได้ด้วยสมการต่อไปนี้\cite{cramer2004}

\begin{equation}
    H^{\circ} \Psi^{\circ} = E^{\circ} \Psi^{\circ}
\end{equation}

โดยกำหนดให้ $H^{\circ} = \sum^{N}_{i=1} h_{i}$ เมื่อ $h$ คือ Hamiltonian สำหรับหนึ่งอิเล็กตรอน (อิเล็กตรอนตัวที่ $i$) 
ในระบบที่มีอิเล็กตรอน $N$ ตัว นั่นคือสมการสำหรับระบบที่มีอิเล็กตรอน $N$ ตัวนั้น จะสามารถถูกแยกออกมาได้เป็นสมการของระบบหนึ่งอิเล็กตรอนได้ 
$N$ สมการ และ Wavefunction ของอิเล็กตรอนหนึ่งตัวนั้นจริง ๆ แล้วก็คือออร์บิทัล (Orbital) เราจึงสามารถเขียนสมการของอิเล็กตรอนหนึ่งตัวได้เป็น

\begin{equation}
    h_{i} \Psi^{\circ}(i) = E^{\circ}_{m} \Psi^{\circ}(i)
\end{equation}

\noindent โดยที่ $E^{\circ}_{m}$ คือพลังงานของอิเล็กตรอนหนึ่งตัวใน Molecular Orbital (MO) ซึ่งเขียนแทนด้วย $m$ นั่นเอง สำหรับระบบที่%
อิเล็กตรอนไม่ขึ้นต่อกันและกัน

ด้วยเหตุนี้ Wavefunction รวมของระบบ ($\Psi^{\circ}$) จึงสามารถเขียนให้อยู่ในรูปของ Wavefunction ของอิเล็กตรอนหนึ่งตัวได้ดังนี้

\begin{equation}
    \Psi^{\circ} = \psi^{\circ}_{a}(1) \psi^{\circ}_{b}(1) \dots \psi^{\circ}_{z}(N)
\end{equation}

\noindent ซึ่ง Wavefunction ด้านบนนี้จะขึ้นอยู่กับพิกัดของอิเล็กตรอนทุกตัวและขึ้นกับตำแหน่งของนิวเคลียสหรืออะตอมด้วย%
\footnote{ตอนนี้เราจะยังไม่พิจารณาสปินของอิเล็กตรอนที่จะต้องสอดคล้องและไม่ขัดกับหลักกีดกันของเพาลีหรือ Pauli Exclusion 
ซึ่งจะมีการรวม Spinorbital สำหรับ Molecular Orbital $m$ ($\varphi_{m}$) เข้าไปด้วย}

สำหรับกระบวนการหรือขั้นตอนที่เราจะนำมาใช้ในการแก้สมการของระบบอิเล็กตรอนหลายตัวนั้น เราจะพิจารณาสมการรูทฮาน (Roothaan Equation) 
เป็นหลัก ซึ่งเป็นวิธีหนึ่งในการแก้สมการ Hartree-Fock (HF) ซึ่งมีการกำหนดตัวดำเนินการใหม่ขึ้นมาใช้แทน Hamiltonian นั่นก็คือ Fock Operator 
โดยที่ Fock Operator ($f_{1}$) ถูกนิยามในเทอมของ Coulomb Operator และ Exchange Operator ขึ้นมา นั่นก็คือ Fock Operator 
ซึ่งเขียนสมการสำหรับอิเล็กตรอน 1 ตัวได้เป็น

\begin{equation}
    \label{eq:fock}
    f_{1} \psi_{m}(1) = \varepsilon_{n} \psi_{m}(1)
\end{equation}

%--------------------------
\subsection{สมการ Roothaan}
%--------------------------

สำหรับการแก้สมการ HF ตรง ๆ โดยใช้ SCF นั้นสามารถทำได้ตรง ๆ ด้วยวิธีการเชิงตัวเลข (Numerical Method) แต่ว่าผลเฉลยที่ได้มานั้นมีความ%
ซับซ้อนมาก โดยในเวลาต่อมานักฟิสิกส์และนักเคมีชาวดัตช์ที่ชื่อว่า Clemens C.J. Roothaan จึงได้เสนอวิธีการใหม่สำหรับการอธิบาย MO โดยเรียก%
วิธีนั้นว่า Linear Combination of Atomic Orbitals (LCAO) เรามาดูกันเลยว่าวิธีการนี้มีรายละเอียดอย่างไร\cite{atkins2010}

เริ่มต้นเราจะนิยามฟังก์ชันพื้นฐาน (Basis Function) สำหรับระบบที่มีอิเล็กตรอน $N$ ตัวขึ้นมาก่อน ซึ่งเขียนแทนด้วย $\chi_{o}$
ซึ่งไอเดียตอนนี้ก็คือเราจะมองว่า Basis Function แบบที่ง่ายที่สุดที่เราสามารถนำมาใช้ได้นั่นก็คือ Atomic Orbital (O) ซึ่งสามารถที่จะเขียน 
Spatial Wavefunction (ฟังก์ชันคลื่นที่ขึ้นกับตำแหน่ง ของ AO) ให้อยู่ในผลรวมเชิงเส้นของการคูณระหว่างสัมประสิทธิ์เชิงเส้นที่เรายังไม่ทราบค่า 
(Unknown Coefficients, $c_{om}$) กับ $\chi_{o}$ ดังนี้

\begin{equation}
    \label{eq:lcao}
    \psi_{m} = \sum^{N_{o}}_{o=1} c_{om} \chi_{o} 
\end{equation}

\noindent เมื่อเราแทนสมการ \ref{eq:lcao} เข้าไปในสมการ \ref{eq:fock} เราจะได้

\begin{equation}
    \label{eq:lcao_in_fock}
    f_{1} \sum^{N_{o}}_{o=1} \chi_{o}(1) = \varepsilon \sum^{N_{o}}_{o=1} c_{om} \chi_{o}(1)
\end{equation}

\noindent แล้วทำการคูณสมการ \ref{eq:lcao_in_fock} ทั้งสองข้างด้วย $\chi^{*}_{o}(1)$ และทำการอินทิเกรตทั่วทั้ง Space ซึ่งจะทำให้เราได้%
ความสัมพันธ์ต่อไปนี้

\begin{equation}
    \label{eq:lcao_in_fock_int}
    \sum^{N_{o}}_{o=1} c_{om} \int \chi^{*}_{o}(1) f_{1} \chi_{o}(1) d\tau_{1} =
    \varepsilon_{m} \sum^{N_{o}}_{o=1} c_{om} \int \chi^{*}_{o}(1) \chi_{o}(1) d\tau_{1}
\end{equation}

\noindent จากสมการข้างต้นเราจะพบว่าจะมีผลคูณของ Basis Function ทั้งสองฝั่ง โดยทางฝั่งซ้ายนั้นเราสามารถนิยาม Fock Matrix (F) ได้

\begin{equation}
    \label{eq:matrix_fock}
    F_{o'o} = \int \chi^{*}_{o'}(1) f_{1} \chi_{o}(1) d\tau_{1}
\end{equation}

\noindent และทางฝั่งขวา เรานิยามสิ่งที่เรียกว่า Overlap Matrix (S) ซึ่งเป็น Matrix ที่อธิบายถึงการซ้อนทับกันระหว่างสถานะ 2 สถานะ

\begin{equation}
    \label{eq:matrix_overlap}
    S_{o'o} = \int \chi^{*}_{o'}(1) \chi_{o}(1) d\tau_{1}
\end{equation}

\noindent ซึ่งเราสามารถเขียนสมการ \ref{eq:lcao_in_fock_int} ให้อยู่ในรูปของสมการที่เรียกว่า Roothaan Equation ได้กระชับ ๆ ดังนี้

\begin{equation}
    \label{eq:roothaan}
    F c = \varepsilon S c
\end{equation}

\noindent โดยที่ $c$ คือเมทริกซ์ขนาด $N_{o} \times N_{o}$ ซึ่งประกอบไปด้วยสมาชิกของ Coefficient $c_{om}$ และ $\varepsilon$ 
คือเมทริกซ์ที่มีขนาด $N_{o} \times N_{o}$ เช่นเดียวกันซึ่งเป็นเมทริกซ์แบบ Diagonal Matrix (สมาชิกที่ไม่ใช่แนวทแยงมีค่าเป็น 0 ทั้งหมด) 
ซึ่งก็คือพลังงานของ Orbital นั่นเอง ซึ่งตรงจุดนี้เราต้องไม่ลืมว่า Fock Operator ($f_{1}$) นั้นถูกกำหนดให้อยู่ในรูปของ Integral บน MO 
และขึ้นอยู่กับค่าของ Coefficient $c_{om}$ ด้วย

สำหรับการแก้สมการ \ref{eq:roothaan} นั้นสามารถทำได้ผ่าน Determinant ดังนี้

\begin{equation}
    \label{eq:scf_secular}
    det|F - \varepsilon S| = 0
\end{equation}

\noindent ซึ่งสมการด้านบนไม่สามารถแก้ได้แบบตรงไปตรงมาเพราะว่าสมาชิกของเมทริกซ์ $F_{o'o}$ นั้นเกี่ยวเนื่องโดยตรงกับ Integral ของ 
Coulomb Operator และ Exchange Operators ซึ่งขึ้นอยู่กับ Spatial Wavefunction นั่นจึงทำให้เป็นปัญหาแบบงูกินหาง ดังนั้นเราจึงต้อง%
ใช้กระบวนการวนซ้ำ (Iterative Method) ในการแก้ปัญหาจนกว่าคำตอบหรือผลลัพธ์ที่เราต้องการจากสมการ (พลังงาน) จะลู่เข้านั่นเอง

%--------------------------
\subsection{การแก้สมการ Roothaan ด้วย Self-Consistent Field}
%--------------------------

\begin{figure}[H]
    \centering
    \includegraphics[width=0.8\linewidth]{fig/scf.png}
    \caption{แผนผังขั้นตอนของ SCF ในการประมาณค่าหาพลังงานของออร์บิทัล}
    \label{fig:scf}
\end{figure}

ภาพที่ \ref{fig:scf} แสดงแผนผงอัลกอริทึมของวิธี SCF โดยเริ่มจากการเลือก Atomic Basis Function ซึ่งถือว่าเป็นองค์ประกอบหลักของ%
การนำไปสร้าง (Formulate) $S$ โดยใช้สมการ \ref{eq:matrix_overlap} กับ $c_{om}$ ซึ่งเราจะใช้วิธีการสร้างค่าเริ่มต้นด้วยวิธี Guess 
ซึ่งมีด้วยกันหลายวิธี เช่น

\begin{enumerate}
    \item \textbf{H{\"u}ckel guess} : ใช้ H{\"u}ckel Orbital\cite{jensen2017}
    \item \textbf{Superposition of Atomic Densities (SAD)} : ใช้ผลรวมของ Atomic Density ในการสร้าง Density Matrix
    \item \textbf{Generalized Wolfsberg-Helmholtz (GWH)} : เป็นวิธีการที่อาศัย H{\"u}ckel Theory โดยการใช้ Overlap 
    Matrix และ Core Hamiltonian\cite{wolfsberg1952}
    \item \textbf{CORE} : ทำการทำ Core Hamiltonian ให้เกิดเมทริกซ์รูปทแยง (Diagonalization)
    \item \textbf{Harris} : ใช้ Harris Functional ซึ่งเป็น Non-self-consistent Approximation สำหรับ Kohn-Sham 
    Orbital\cite{harris1985}
\end{enumerate}

ซึ่งโปรแกรมเคมีเชิงคำนวณต่างก็มีการเลือกใช้ Guess Method สำหรับการเดา Coefficient หรือ Wavefunction เริ่มต้นในการแก้ SCF แตกต่างกันไป
โปรแกรม Gaussian ใช้วิธี Harris สำหรับการคำนวณ HF และ DFT และใช้ H{\"u}ckel หรือ CORE สำหรับ Semiempirical Methods, 
โปรแกรม Q-Chem และ Psi4 ใช้วิธี SAD กับ GWH เป็นวิธีเริ่มต้นโดยอัตโนมัติ เป็นต้น

หลังจากสร้าง Coefficient Matrix ขั้นตอนต่อไปคือการสร้าง Fock Matrix $F$ โดยใช้สมการ \ref{eq:matrix_fock} 
หลังจากนั้นเราจะทำการแก้สมการลักษณะเฉพาะ (Secular Equation) สมการที่ \ref{eq:scf_secular} เพื่อหา Energy Matrix 
แล้วก็ทำการวนซ้ำขั้นตอนการสร้าง $S$ กับ $F$ ไปปรับหาค่าพลังงานไปเรื่อย ๆ จนกว่าค่าความคลาดเคลื่อนหรือ Error จะมีค่าน้อยกว่าค่าที่กำหนดไว้ 
(Threshold) แล้วจึงสิ้นสุดกระบวนการ SCF เมื่อค่าพลังงานนั้นลู่เข้า

%--------------------------
\subsection{การคำนวณอนุพันธ์ของพลังงานและเมทริกซ์เฮสเซียน}
\idxen{Energy Derivative}
%--------------------------

หลังจากที่เราสามารถหาพลังงานเชิงอิเล็กทรอนิกส์ (Electronic Energy) ได้แล้ว ลำดับถัดไปที่เราสามารถคำนวณได้ก็คือคุณสมบัติต่าง ๆ ของโมเลกุล
สิ่งแรกที่เราทำได้และถือว่าสำคัญมาก ๆ ในงานวิจัยทางด้านเคมีควอนตัมก็คือการหาโครงสร้างที่เหมาะสมหรือเสถียรที่สุดของโมเลกุลโดยใช้หลักเกณฑ์%
พลังงานรวมที่ต่ำที่สุด ซึ่งการที่เราทราบโครงสร้างที่เหมาะสมที่สุดนั้นมีประโยชน์อย่างมากเพราะเราสามารถนำผลการคำนวณไปเทียบกับผลจากการทดลอง%
ด้วยเทคนิค X-ray Crystallography, Electron Diffractiom, หรือ Microwave Spectroscopy เป็นต้น โดยการหาโครงสร้างที่สภาวะ%
เหมาะสมหรือสมดุล (Equilibrium Structure) นั้นสามารถทำได้โดยหาอนุพันธ์ของพลังงานศักย์ของโมเลกุลเทียบกับพิกัดนิวเคลียร์ ซึ่งวิธีการที่%
เราสามารถนำมาหาอนุพันธ์เพื่อให้ได้ผลลัพเชิงวิเคราะห์ (Analytical Method) เรียกว่า Gradient Method ซึ่งเร็วและให้ผลลัพธ์ที่แม่นยำกว่า%
ระเบียบวิธีเชิงตัวเลข (Numerical Method)

สำหรับอนุพันธ์ของพลังงานนั้นเราจะเริ่มต้นพิจารณากรณีที่ง่าย ๆ นั่นก็คือโมเลกุลที่มีอะตอมสองอะตอม ซึ่งพลังงานศักย์ของโมเลกุลซึ่งเขียนแทนด้วย $E$ 
นี้จะมีเทอมที่เป็นแรงผลักระหว่างนิวเคลียสด้วยซึ่งจะขึ้นกับระยะห่างระหว่างนิวเคลียส (Internuclear Distance, $R$) สำหรับโครงสร้างที่อยู่ในสมดุลนั้น 
แรง (Force) ที่กระทำต่อนิวเคลียสโดยอิเล็กตรอนนั้นจะเท่ากับศูนย์ ซึ่งแรงดังกล่าวเป็นแรงย่อยมีนิยามคืออนุพันธ์อันดับที่หนึ่งของพลังงานศักย์เทียบ%
กับพิกัดของนิวเคลียสที่ $i$

\begin{align}
    f_{i} &= - \pdv{E}{q_{i}} \\
    &= 0
\end{align}

โดยการคำนวณหาอนุพันธ์ข้างต้นด้วยวิธีการวิเคราะห์หรือ Analytical Method นั้นเราจะต้องทำการคำนวณหาอนุพันธ์ของอินทิกรัลของอิเล็กตรอน%
หนึ่งตัวและอิเล็กตรอนสองตัว (One-electron กับ Two-electron Integrals) เทียบกับพิกัดนิวเคลียร์ นั่นคือเราจะต้องทำการหาอนุพันธ์ของ 
Basis Function นั่นเอง\footnote{Basis Function ก็คือ Basis ที่เกิดขึ้นมาจาก Atomic Orbtials ที่ถูก centered หรือมีตำแหน่ง%
อยู่ที่จุดอ้างอิงของนิวเคลียสของอะตอมในโมเลกุล} ซึ่งเราสามารถทำได้ผ่านการใช้กฎลูกโซ่ (Chain Rule) โดยทำการหาอนุพันธ์ของพลังงานศักย์%
เทียบกับ Expansion Coefficient

ลำดับถัดมาคือการหาเมทริกซ์เฮสเซียน (Hessian Matrix) ซึ่งสามารถทำได้โดยการหาอนุพันธ์ย่อยอันดับที่สองของพลังงานศักย์เทียบกับนิวเคลียส%
ของอะตอมตัวที่ $i$ และ $j$ ($\pdv{E}{q_{i}}{q_{j}}$) ซึ่งช่วยให้เราสามารถระบุได้ว่าค่าพลังงานที่คำนวณออกมาได้นั้นสอดคล้องกับ%
จุดต่ำสุดหรือสูงสุดบนพื้นผิวพลังงานศักย์ (Potential Energy Surface หรือ PES) โดยจะสอดคล้องกับอนุพันธ์อันดับที่สองที่ได้ค่าออกมาเป็นบวก 
(สำหรับ Minimum Point) และลบ (สำหรับ Maximum Point) ตามลำดับ

%--------------------------
\section{ความหนาแน่นเชิงประจุและเมทริกซ์ความหนาแน่น}
%--------------------------

ความหนาแน่นเชิงประจุ (Charge Density) เป็นปริมาณที่บ่งบอกถึงประจุของอะตอมที่อยู่ในโมเลกุล ถ้าหากเราทำการอินทิเกรต Charge Density 
ทั่วทั้งปริมาตรเราจะได้ผลลัพธ์เป็นจำนวนของอิเล็กตรอนในระบบของเรา (โมเลกุล) ดังนี้\cite{szabo1996}

\begin{equation}
    N = \int \rho (\mathbf{r}) dV
\end{equation}

โดยนิยามของ Charge Density จะเป็นผลรวมของโอกาสที่เราจะพบอิเล็กตรอนที่อยู่ภายใน Molecular Orbitals ของทั้งระบบ ดังนี้
\idxen{Charge Density}

\begin{equation}
    \rho (\mathbf{r}) = 2 \sum^{N/2}_{i=1} \int |\varphi_{i}(\mathbf{r})|^{2}
\end{equation}

\noindent โดยเลข 2 ด้านหน้าเครื่องหมาย Summation ก็คือ Occupation Number สำหรับกรณีที่ Molecular Orbital ($i$) 
นั้นมีอิเล็กตรอนทั้งแบบ Spin Up และ Spin Down และ $\varphi_{i}(\mathbf{r})$ คือ Wavefunction ซึ่งเราสามารถเขียน Wavefunction 
ให้อยู่ในรูปผลรวมเชิงเส้นของ Basis Function ($\phi_{i}$) ซึ่ง Basis Function นี้จะเป็นฟังก์ชันอะไรก็ได้ที่สามารถอธิบายการมีอยู่ของ 
Molecular Orbitals โดยในกรณีแบบที่ง่ายที่สุดคือเราจะมองว่า Molecular Orbitals นั้นเกิดขึ้นมาจากการรวมกันของ Atomic Orbitals 
ดังนั้นเราจะกำหนดให้ Atomic Orbitals เป็น Basis Function\footnote{Basis Function ในที่นี้คือ Atomic Orbitals ที่ถูกกำหนดให้%
มีจุดศูนย์กลางอยู่ที่อะตอมนั้น ๆ} ดังนั้นผลรวมเชิงเส้นดังกล่าวจึงมีเรียกว่า Linear Combination of Atomic Orbitals หรือ LCAO 
ตามสมการดังต่อไปนี้

\begin{equation}
    \rho (\mathbf{r}) = 2 \sum_{i} \big( \sum_{\mu} c_{\mu i} \phi_{\mu}^{*} \big) 
    \big( \sum_{\nu} c^{*}_{\nu i}  \phi_{\nu} \big)
\end{equation}

\noindent โดยที่ $c$ คือสัมประสิทธิ์ของ Linear Combination ของ Atomic Orbitals (LCAO) ลำดับต่อมาคือเมื่อเราจัดรูปให้มีเทอมที่%
เป็นผลคูณของ Basis Function ($\phi_{\mu}^{*} \phi_{\nu}$) เราจะกำหนดให้เทอมนี้เป็นสิ่งที่เรียกว่า Overlap Matrix 
($S_{\mu\nu}$) โดยจะได้สมการที่จัดรูปแล้ว ดังนี้

\begin{equation}
    \rho (\mathbf{r}) = 2 \sum_{i}\sum_{\mu\nu} c_{\mu i} c^{*}_{\nu i} S_{\mu\nu}
\end{equation}

หลังจากนั้นเราจะพบว่าจะมีเทอมที่เป็นผลคูณระหว่าง $c$ ซึ่งเราจะกำหนดให้ผลคูณแบบนี้เรียกว่า Density Matrix 
($P_{\mu\nu} = c_{\mu i} c^{*}_{\nu i}$) ซึ่งเราจะได้สมการดังต่อไปนี้
\idxen{Density Matrix}

\begin{equation}     
    \rho (\mathbf{r}) = 2 \sum_{i} \sum_{\mu\nu} P_{\mu\nu}S_{\mu\nu}
\end{equation}

%--------------------------
\section{พลังงานของ Frontier Orbitals}
%--------------------------

\idxen{Frontier Orbitals}

\idxen{Ground State}

%--------------------------
\subsection{พลังงานของ HOMO และ LUMO}
\idxth{พลังงานของ HOMO และ LUMO}
%--------------------------

\idxen{Frontier Orbitals!HOMO}

\idxen{Frontier Orbitals!LUMO}

%--------------------------
\subsection{ผลต่างของพลังงานของ HOMO และ LUMO}
%--------------------------

\idxen{Energy Gap}

%--------------------------
\section{พื้นผิวพลังงานศักย์}
%--------------------------

\begin{figure}[H]
    \centering
    \subfloat[โมเลกุลคู่\label{fig:diatomic_mol}]{
        \includegraphics[width=0.3\linewidth]{fig/diatomic_molecule.png}}
    \hspace{1em}
    \subfloat[พลังงานศักย์ของโมเลกุลคู่\label{fig:PES_diatomic}]{
        \includegraphics[width=0.4\linewidth]{fig/PES_diatomic_mol.png}}
    \label{fig:diatomic_mol_and_PES}
\end{figure}


\begin{figure}[H]
    \centering
    \subfloat[โมเลกุลสามอะตอมแบบเชิงเส้นตรงร่วม\label{fig:3_body_mol}]{
        \includegraphics[width=0.4\linewidth]{fig/3-body_collinear.png}}
    \hspace{1em}
    \subfloat[พลังงานศักย์ของโมเลกุลสามอะตอมแบบเชิงเส้นตรงร่วม\label{fig:PES_3_body_mol}]{
        \includegraphics[width=0.5\linewidth]{fig/3-body_collinear_PES.png}}
    \label{fig:3_body_mol_and_PES}
\end{figure}

\begin{figure}[H]
    \centering
    \includegraphics[width=0.7\linewidth]{fig/3-body_non-collinear.png}
    \caption{โมเลกุลสามอะตอมแบบไม่เป็นเชิงเส้นตรงร่วม}
    \label{fig:non_collinear}
\end{figure}

\begin{figure}[H]
    \centering
    \includegraphics[width=0.7\linewidth]{fig/PES_C2H4Cl2.png}
    \caption{พลังงานศักย์ของโมเลกุล \ce{C22H4Cl2}}
    \label{fig:non_collinear}
\end{figure}

%--------------------------
\section{ไดโพลโมเมนต์}
%--------------------------

ไดโพลโมเมนต์ (Dipole Moment)

\idxen{Dipole Moment}

%--------------------------
\section{สภาพการเกิดขั้ว}
%--------------------------

สภาพการเกิดขั้ว (Polarizability)
\idxen{Polarizability}

%--------------------------
\section{เทคนิคสเปกโทรสโกปีแบบสั่น}
\idxen{Spectroscopy}
\idxen{Spectroscopy!Vibrational Spectroscopy}
%--------------------------

สเปกโทรสโคปี (Spectroscopy) เป็นการศึกษาอันตรกิริยา (Interaction) ระหว่างสสารกับรังสีแม่เหล็กไฟฟ้า (Electromagnetic Radiation) 
ที่เกิดจากการเปลี่ยนระดับพลังงานของอิเล็คตรอน การเปลี่ยนระดับพลังงานการหมุน (Rotation) และการสั่นสะเทือน (Vibration) ของโมเลกุล 
ซึ่งการที่เราทราบจากสเปกตรัมของโมเลกุลจะทำให้เราทราบข้อมูลหลายอย่างเกี่ยวกับโครงสร้างของโมเลกุลของสสารและสมบัติทางเคมี เช่น

\begin{itemize}
    \item สมมาตรของโมเลกุล (Symmetry)
    \item ความยาวพันธะ (Bond Length)
    \item มุมพันธะ (Bond Angle)
    \item ความแข็งแรงของพันธะ (Bond Strength)
    \item การเปลี่ยนแปลงภายในและระหว่างโมเลกุล
\end{itemize}

\noindent โดยในหัวข้อนี้เราจะมาดูรายละเอียดเกี่ยวกับการคำนวณความเข้มของการดูดกลืนสำหรับเทคนิค Infrared (IR) และรามาน (Raman) 
ซึ่งทั้งสองเทคนิคนี้ต่างก็เป็นเทคนิคสเปกโทรสโกปีแบบสั่น (Vibrational Spectroscopy) ซึ่งมีการนำมาใช้ในการทำงานวิจัยสำหรับการศึกษา%
คุณสมบัติของโมเลกุลอย่างแพร่หลาย

%--------------------------
\subsection{อินฟาเรดสเปกโทรสโกปี}
\idxboth{สเปกโทรสโกปี!อินฟาเรด}{Spectroscopy!IR}
%--------------------------

อินฟาเรดสเปกโทรสโกปี (IR Spectroscopy) เป็นการวัดการดูดกลืนของการแผ่รังสีของโมเลกุลในช่วงอินฟราเรดซึ่งเกี่ยวข้องกับการเปลี่ยนแปลงของ%
อิเล็กทริกไดโพลโมเมนต์ (Electric Dipole Moment) ของโมเลกุลที่ศึกษา สำหรับการคำนวณความเข้มของการดูดกลืน IR ในรูปแบบของวิธีแบบ
Dynamic นั้นสามารถทำได้โดยใช้สมการ (ความสมพันธ์) ดังต่อไปนี้s\cite{thomas2013}

\begin{equation}
    I_{IR} (\omega) \propto \int \braket{\bm{\dot{\mu}}(\tau) \bm{\dot{\mu}}(\tau+t)}_{\tau} e^{-i \omega t} dt
\label{eq:IR_corr}
\end{equation}

\noindent โดยที่ $\bm{\dot{\mu}}$ คืออนุพันธ์ของไดโพลโมเมนต์เทียบกับเวลา, $\omega$ คือความถี่เชิงการสั่น (Vibrational Frequency),
$\tau$ คือเวลาที่เปลี่ยนแปลงไปอย่างช้า ๆ และ $t$ คือเวลาสำหรับการทำ Integration นอกจากนี้ยังจะสังเกตได้ว่าจะมีเทอม
$\braket{\bm{\dot{\mu}}(\tau) \bm{\dot{\mu}}(\tau+t)}_{\tau}$ ซึ่งจะเป็นตัวที่บ่งบอกถึงสหสัมพันธ์ของเวลา (Time Correlation) 
ของ $\bm{\dot{\mu}}$ 

สำหรับกรณีที่เป็นแบบ Static นั้น สเปกตรัทของ IR สามารถคำนวณได้ผ่านอนุพันธ์ของไดโพลโมเมนต์เทียบกับพิกัดหรือตำแหน่งของโหมดการสั่น%
แบบปกติ (Normal Coordinates) ซึ่งจะไม่ขึ้นกับเวลา ด้วยสมการดังต่อไปนี้

\begin{equation}
    \bm{\mu}= \sum_{\mu\nu} P_{\mu\nu} \braket{\phi_{\mu}|\bm{r}|\phi_{\nu}}
    \label{eq:mu_qm}
\end{equation}

\begin{equation}
    \bm{\mu}=\sum_{J} q_J \bm{R_J}
    \label{eq:mu_classical}
\end{equation}

โดยที่สมการ \ref{eq:mu_qm} จะเป็นสำหรับกรณีแบบควอนตัมซึ่งจะคำนวณผ่านเมทริกซ์ความหนาแน่นและ Basis Function แต่สมการ 
\ref{eq:mu_classical} จะเป็นสำหรับกรณีแบบดั้งเดิมซึ่งจะคำนวณผ่านจุดประจุ (Point Charge) และพิกัดคาร์ทีเซียนของอะตอม

%--------------------------
\subsection{รามานสเปกโทรสโกปี}
\idxboth{สเปกโทรสโกปี!รามาน}{Spectroscopy!Raman}
%--------------------------

รามานสเปกโทรสโกปี (Raman Spectroscopy) เป็นเทคนิคหนึ่งที่เปรียบเมือนเป็นพี่น้องกับเทคนิคอินฟาเรดสเปกโทรสโกปี โดยที่ Raman Spectroscopy 
จะเป็นผลมาจากการเกิดการกระเจิงของแสงแบบไม่ยืดหยุ่นในช่วงอินฟราเรด วิสิเบิล (Visible) และอัลตราไวโอเล็ต (Ultraviolet) ซึ่งเกี่ยวข้องกับ%
การเปลี่ยนแปลงสภาพการเกิดขั้ว (Polarizability) แบบอิเล็กทริกไดโพล-อิเล็กทริกไดโพล (Electric-dipole--electric-dipole) ของสสาร
โดยความเข้มของการกระเจิงแบบรามาน ($I_{Raman}$) สามารถคำนวณได้ด้วยความสัมพันธ์ดังต่อไปนี้\cite{thomas2013}

\begin{equation}
    I_{Raman} (\omega) \propto \frac{(\omega_{in}-\omega)^4}{\omega} 
    \frac{1}{1-exp(-\frac{\hbar\omega}{k_{B}T})}S(a^{2}, \gamma^{2})
\label{eq:Raman_corr}
\end{equation}

\noindent โดยที่ $S(a^{2}, \gamma^{2})$ คือตัวแปรที่เป็นผลจากการรวมกันของความคงที่ (ไม่เปลี่ยนแปลง) แบบ Isotropic และ 
Anisotropic ของเทนเซอร์แบบ Placzek-type Polarizability ($\bm{\alpha}$)\cite{jensen2005}, $\omega$ คือความถี่เชิงการสั่น, 
$\omega_{in}$ คือความถี่ของแสดง, $k_{B}$ คือค่าคงที่ของโบลทซ์มานน์ (Boltzmann Constant) และ $T$ คืออุณหภูมิ 
โดยสมการที่จะใช้ในการอธิบาย $S(a^{2}, \gamma^{2})$ จะขึ้นอยู่กับรูปแบบของการทดลองและสมการของ Time Correlation
\cite{mattiat2021}.

%--------------------------
\section{การถ่ายโอนอิเล็กตรอน}
\idxth{การถ่ายโอนอิเล็กตรอน}
\idxen{Electron Transfer}
%--------------------------

การถ่ายโอนอิเล็กตรอน (Electron Transfer)

%--------------------------
\subsection{ค่าคู่ควบของการถ่ายโอนอิเล็กตรอน}
\idxth{การถ่ายโอนอิเล็กตรอน!ค่าคู่ควบ}
\idxen{Electron Transfer!Electron Transfer Coupling}
%--------------------------

ค่าคู่ควบของการถ่ายโอนอิเล็กตรอน (Electron Transfer Coupling)

%--------------------------
\subsection{พลังงานการปรับเปลี่ยนโครงสร้าง}
\idxth{พลังงานการปรับเปลี่ยนโครงสร้าง}
\idxen{Reorganization Energy}
%--------------------------

พลังงานการปรับเปลี่ยนโครงสร้าง (Reorganization Energy)

%--------------------------
\section{คุณสมบัติของสถานะกระตุ้น}
\idxth{สถานะกระตุ้น}
\idxen{Excited State}
%--------------------------

คุณสมบัติของสถานะกระตุ้น (Excited State Properties)
\idxth{สถานะกระตุ้น!คุณสมบัติของสถานะกระตุ้น}
\idxen{Excited State Properties}

%--------------------------
\subsection{พลังงานของสถานะกระตุ้น}
\idxth{สถานะกระตุ้น!พลังงานของสถานะกระตุ้น}
\idxen{Excited State!Excited State Energies}
%--------------------------

พลังงานของสถานะกระตุ้น (Excited State Energies)

%--------------------------
\subsection{ค่าคู่ควบของกระบวนการ Nonadiabatic}
\idxth{สถานะกระตุ้น!ค่าคู่ควบของกระบวนการ Nonadiabatic}
\idxen{Excited State!Nonadiabatic Coupling}
%--------------------------
 % คุณสมบัติอิเล็กทรอนิกส์ของโมเลกุล
% LaTeX source for ``การเรียนรู้ของเครื่องสำหรับเคมีควอนตัม (Machine Learning for Quantum Chemistry)''
% Copyright (c) 2022 รังสิมันต์ เกษแก้ว (Rangsiman Ketkaew).

% License: Creative Commons Attribution-NonCommercial-NoDerivatives 4.0 International (CC BY-NC-ND 4.0)
% https://creativecommons.org/licenses/by-nc-nd/4.0/

\chapter{ลักษณะเฉพาะของอะตอมและโมเลกุล}
\label{ch:feature}

ลักษณะเฉพาะ (Feature) คือคุณลักษณะเด่นที่มีความพิเศษและบ่งบอกความเฉพาะตัวของอะตอมหรือโมเลกุลนั้น ๆ ซึ่งอาจจะเรียกว่าเป็นคุณลักษณะแบบพิเศษก็ได้ 
(Special Attributes) นอกจากนี้เรายังเรียกสิ่งที่เป็น Feature ว่ามันคือการแสดงความเป็นตัวแทนของสิ่งที่เราสนใจอีกด้วยหรือ Representation
คำถามที่ตามมาก็คือแล้ว Feature หรือ Representation มีความสำคัญอย่างไร จริง ๆ แล้วมีความสำคัญมาก ๆ อาจจะกล่าวได้ว่าสำคัญที่สุดเลยก็ว่าได้ 
เพราะมันเป็น Input ที่เรานำมาใช้สร้าง Model ดังนั้นมันจึงเป็นปัจจัยที่กำหนดประสิทธิภาพของ Model นั่นเอง 

เนื่องจากว่ามนุษย์สามารถแยกแยะโมเลกุลแต่ละตัวออกจากกันได้ แต่ว่าคอมพิวเตอร์ไม่สามารถทำได้ เพราะว่ามันเข้าใจข้อมูลที่เป็นแบบดิจิตอลในภาษาเครื่องจักรเท่านั้น (Machine Language)
ดังนั้นเราจึงต้องมีการใช้ Representation เพื่ออธิบายโมเลกุลในรูปแบบของค่า Parameter ที่คอมพิวเตอร์สามารถเข้าใจได้ เช่น แปลงโมเลกุลเป็นเวกเตอร์ของตัวเลข 
สำหรับการอธิบายโมเลกุลแบบง่าย ๆ นั้นสามารถทำได้โดยใช้ Representation เพื่อมาอธิบายข้อมูลเชิงโครงสร้าง (Structural Properties) 
ซึ่งสามารถใช้ข้อมูลทางเคมีทั่วไปได้ ยกตัวอย่างเช่น รูปร่างของโมเลกุล, จำนวนหมู่ฟังก์ชัน, ชนิดของพันธะระหว่างอะตอมคาร์บอน, และจำนวนวงเบนซีน 
ซึ่งข้อมูลเหล่านี้เราสามารถคำนวณออกมาได้ง่าย ๆ ไม่มีความซับซ้อนอะไร แต่ปัญหาคือ Representation ที่เป็น Structural-based นั้นมีข้อมูลที่น้อยเกินไป
จึงทำให้ไม่สามารถถูกนำมาใช้เป็น Input สำหรับการสร้าง Model เพื่อทำนายคุณสมบัติหรือ Parameter ทางเคมีที่ซับซ้อนหรือละเอียดกว่าได้ เช่น 
พลังงานพันธะ (Bond Energy), ค่าความถี่การสั่น (Vibration), ไดโพลโมเมนต์ (Dipole Moment), ฯลฯ 
นั่นก็เพราะว่า Input ของเรามันเป็น Representation ที่ไม่มี Correlation กับ Output ที่เราต้องการทำนายโดยตรง

ดังนั้นถ้าหากเราต้องการที่จะทำนาย Output ที่มีความละเอียดอยู่ในระดับอะตอมหรือเชิงอิเล็กทรอนิกส์ เราควรจะใช้ Representation ที่อยู่ในระดับเดียวกัน 
และ Representation Input เหล่านี้ควรจะต้องเก็บข้อมูลทางควอนตัมเคมีและฟิสิกส์ไว้ด้วย โดยการพัฒนา Representation โดยใช้องค์ความรู้ทางฟิสิกส์
(Physics-inspired Representation) ก็เป็นหนึ่งในหัวข้องานวิจัยที่กำลังมาแรงในขณะนี้ ข้อมูลทางฟิสิกส์ที่เราเพิ่มเข้าไปก็เปรียบเสมือนเป็นองค์ประกอบที่เพิ่มความถูกต้อง
(Correction) ให้กับ Representation ให้มากขึ้น เราใส่ระบุความเป็นสมมาตร (Symmetricity) ของโมเลกุลเข้าไปได้ เป็นต้น 

%--------------------------
\section{การแปลงข้อมูลเชิงโมเลกุล}
%--------------------------

โมเลกุลประกอบไปด้วยอะตอมหลายอะตอมมารวมกัน เราจึงเปรียบเทียบโมเลกุลเป็นประโยคหรือข้อความและเปนียบเทียบอะตอมเป็นคำแต่ละคำได้
ดังที่บอกไปในข้างต้นว่าการทำให้คอมพิวเตอร์เข้าใจความเชื่อมโยงระหว่างอะตอมในโมเลกุลนั้นต้องมีการกำหนดหรือเลือกใช้ Representation ที่เหมาะสม
โดยคุณสมบัติของ Representation ที่ดีนั้นไม่เพียงแต่จะต้องไม่ขึ้นกับการเคลื่อนที่เชิงการหมุน (Rotational Motion) และ การเคลื่อนที่เชิงเส้นด้วย (Transitional Motion) เท่านั้น 
แต่ควรจะต้องมีความเรียบง่ายและไม่ซับซ้อนหรือยุ่งยากเกินไปในการคำนวณเพื่อสร้าง Machine Code จากพิกัดตำแหน่งคาร์ทคีเชียน (Cartesian Coordinates)

%--------------------------
\section{ลักษณะเฉพาะเชิงโครงสร้าง}
%--------------------------

%--------------------------
\subsection{Internal Coordinates}
%--------------------------

$Z$ matrix หรือเรียกอย่างว่า Internal Coordinates (แปลตรงว่าพิกัดภายในของโมเลกุล) เป็น Representation แบบที่ง่ายมาก 
อาจจะเรียกได้ว่าง่ายที่สุดเลยก็ว่าได้ ถูกใช้เป็น Descriptor มาอย่างยาวนาน โดยถูกใช้อย่างแพร่หลายในยุคแรก ๆ ของการเรียนรู้ของเครื่องสำหรับเคมี
Internal Coordinates สามารถอธิบายได้ทั้งโมเลกุล โดยองค์ประกอบของ Representation อันนี้มีความยาวพันธะระหว่างอะตอม (Bond Distance) 
มุมพันธะ (Bond Angle) และมุมบิดเบี้ยว (Dihedral Angle หรือ Torsion) ซึ่งจำนวนของอะตอมที่ถูกเลือกมาคำนวณ Internal Coordinates 
นั้นมักจะเป็นอะตอมที่เรียงกันอยู่หรืออยู่ใกล้กัน อย่างไรก็ตาม เราสามารถคำนวณหา Internal Coordinates ของโมเลกุลได้โดยพิจารณาอะตอมทุก ๆ คู่
หรือทุกความเป็นไปได้ทั้งหมดภายในโมเลกุล

%--------------------------
\subsection{Geometric Descriptors}
%--------------------------

Geometric Descriptors คือลักษณะเฉพาะเชิงเรขาคณิต

All geometric descriptors that provide information about the spatial coordinates of atoms in 
a molecule relate to the symbolic representation of the molecule. There are several geometric descriptors, 
including the molecular $Z$ matrix, standard and effective coordination numbers. 
The descriptor of the molecular matrix represents each atom's coordinates $(x, y, z)$ in Cartesian space. 
In contrast to most descriptors, these descriptors are able to distinguish isomeric molecules 
(e.g. cis/trans stereoisomers). However, geometric-based representations can still be problematic 
since they omit electronic structure. More sophisticated representations developed in 
the last decade leverage and include electronic descriptions such as atomic force, electron configuration, 
and correlation between orbitals.\cite{musil2021} It can also be difficult to calculate 
the geometric descriptors due to their complexity.\cite{keith2021}

%--------------------------
\subsection{Inverse Distance Matrix}
%--------------------------

%--------------------------
\section{ลักษณะเฉพาะเชิงอิเล็กทรอนิกส์}
%--------------------------

%--------------------------
\subsection{Coulomb Matrix}
%--------------------------

Coulomb Matrix หรือ เมทริกซ์คูลอมบ์

The Coulomb matrix (CM) introduced by \citeauthor{rupp2012} is widely employed because of its simplicity 
and relatively less requirement of \textit{a priori} knowledge of chemical properties of a molecule.\cite{rupp2012}
It stores information about how atoms interact with each other. Each pair of atoms in a molecule carries 
the pairwise electrostatic potential energy. In addition, every atom has a set of Cartesian coordinates 
representing its location in space and has a charge attached to it. The CM between atoms $i$ and $j$ 
is simply given by

\begin{equation}
C_{ij} =
\begin{cases}
 & 0.5 Z_i^{2.4} \text{ if } i = j \\ 
 & \frac{Z_i Z_j}{R_{ij}} \text{ if } i \neq j
\end{cases}
\end{equation}

\noindent where $Z$ is the atomic number and $R_{ij}$ is the interatomic separation. The off-diagonal entries 
of the CM reflect Coulomb's repulsions between the nuclei, and the exponents in the diagonal entries 
correspond to a polynomial fit linking the atomic number to the overall energies of the unbound atoms. 

Despite being invariant to molecular translation and rotation, the CM is not geometrically invariant to 
atom permutations. To solve this puzzle, a variety of alternative representation methods, such as 
Permutation-Invariant Polynomials (PIP)\cite{braams2009}, Randomly Sorted Coulomb Matrices (RSCM)\cite{hansen2013}, 
Bag of Bonds (BoB)\cite{hansen2013}, and Permutation Invariant Vectors (PIV)\cite{gallet2013} have been 
proposed for producing a representation of the molecule that is independent of the ordering of the atoms.

%--------------------------
\subsection{Smooth Overlap of Atomic Positions}
%--------------------------

Smooth Overlap of Atomic Positions (SOAP) can encode atomic geometries in their chemical environment 
using a local expansion of a Gaussian smeared atomic density based on the radial basis ($g_{n}(r)$) and 
real spherical harmonic functions ($Y_{lm}(\theta, \phi)$).\cite{bartok2013,de2016} SOAP is suitable for predicting 
local properties such as atomic forces or chemical shifts but requires partitioning of global properties 
such as total energies. The SOAP kernel between two atomic environments ($\mathcal{X}$ and $\mathcal{X}'$) 
can be retrieved as a normalized polynomial kernel of the partial power spectra. The working equation of 
the SOAP kernel ($K^\mathrm{SOAP}$) retrieved as a normalized polynomial kernel of the two neighbor partial 
power spectra $\mathbf{p}$ and $\mathbf{p}'$ is given by

\begin{equation}
    K^\mathrm{SOAP}(\mathbf{p}, \mathbf{p'}) = \left( \frac{\mathbf{p} \cdot \mathbf{p'}}{\sqrt{\mathbf{p} 
    \cdot \mathbf{p}~\mathbf{p'} \cdot \mathbf{p'}}}\right)^{\xi},
\end{equation}

\noindent where $\xi$ is a positive integer. The elements of the $\mathbf{p}$ vector are defined as 

\begin{equation}
    p^{Z_1 Z_2}_{n n' l} = \pi \sqrt{\frac{8}{2l+1}}\sum_m {c^{Z_1}_{n l m}}^{\dagger} c^{Z_2}_{n' l m}
\end{equation}

\noindent where $n$ and $n'$ are indices for radial basis functions up to $n_{max}$, $l$ is 
the angular degree of the spherical harmonics up to $l_{max}$, $m$ is an integer such that 
$\abs{m} \leq l$, and $Z_{1}$ and $Z_{2}$ are atomic species. The coefficients $c^{Z}_{n'lm}$ and 
${c^{Z}_{nlm}}^{\dagger}$ are defined as the inner products of spherical harmonic functions with 
the Gaussian smoothed atomic density for atoms with the atomic number $Z$ ($\rho^Z$), and its complex conjugate, 
respectively.\cite{de2016}

%--------------------------
\subsection{Atom-centered Symmetry Functions}
%--------------------------

Atom-centered Symmetry Functions (ACSF) generalize the output of multiple two- and three-body functions 
to estimate the local electronic environment near atoms using a fingerprint method that can be customized 
to detect specific structural features for symmetric functions.\cite{behler2011} Radial symmetric functions 
for the central atom $i$ with neighboring atom $j$ are given as follows:

\begin{equation}
    G^{radial}_{i} = \sum^{N_{atom}}_{j \neq i} e^{\eta (\boldsymbol{r}_{ij} - \mu)^{2}} f_{c}(\boldsymbol{r}_{ij})
\end{equation}

\noindent where $\eta$ and $\mu$ are the parameters controlling the width and the position of 
the Gaussian function. $f_{c}$ is a cutoff function that selects the relevant regions close to 
the central nucleus to be encoded into the ACSF and $\boldsymbol{r}_{ij}$ is the displacement between 
the atoms $i$ and $j$. Angular symmetric functions have been defined.\cite{behler2011} 
Moreover, an input required to ACSFs is fine-tuning of internal parameters that can be properly used 
to define the Gaussian function. A similar representation to ACSFs is the Spectrum of London and 
Axilrod-Teller-Muto (SLATM), which has been recently used in the literature, but mostly for 
kernel ridge regression (KRR) models.\cite{faber2018} Polynomial functions of the inverse of 
the interatomic distances have also been suggested but are not discussed in this review.\cite{musil2021}

%--------------------------
\subsection{Gaussian-type Orbital-based Density Vectors}
%--------------------------

Gaussian-type Orbital (GTO)-based Density Vector is a descriptor function alternative to 
the ACSF and symmetric polynomial function.\cite{kwac2021} The GTO-based density vector is given by

\begin{equation}
    \rho^{i}_{L,\alpha,r_{s}} = \sum^{l_{x}+l_{y}+l_{z} = L}_{l_{x},l_{y},l_{z}} 
    \frac{L!}{l_{x}!l_{y}!l_{z}!} \left ( \sum^{n_{type}}_{t=1} c_{t} \sum^{N^{t}_{atom}}_{j=1} 
    \phi^{\alpha,r_{s}}_{l_{x}l_{y}l_{z}} (\boldsymbol{r}_{ij}) \right )
\end{equation}

\noindent where $l_{x}+l_{y}+l_{z} = L$ specifies the orbital angular momentum, $n_{type}$ is the atomic type 
in the molecule under study, $c_{t}$ is the type-dependent weight, $N^{t}_{atom}$ is the number of atoms of 
the type $t$, and $\phi^{\alpha,r_{s}}_{l_{x}l_{y}l_{z}} (r_{ij})$ is the Gaussian orbital centered at 
each atom with the parameters $\alpha$ and $r_{s}$ which determine the radial distributions of orbital functions, 
given as

\begin{equation}
    \phi^{\alpha,r_{s}}_{l_{x}l_{y}l_{z}} (\boldsymbol{r}_{ij}) = x^{l_{x}}y^{l_{y}}z^{l_{z}} e^{-\alpha 
    |r-r_{s}|^{2}}
\end{equation}

\noindent where $\boldsymbol{r}_{ij} = (x,y,z)$ indicates the vector from nuclei $i$ to nuclei $j$ and $r$ 
is the magnitude of $\boldsymbol{r}_{ij}$. GTOs with $L=0,1$ are usually considered for constructing 
the density vectors for small organic molecules containing light atoms such as C, H, O, and N.\cite{kwac2021}
 % ลักษณะเฉพาะของอะตอมและโมเลกุล
% LaTeX source for ``การเรียนรู้ของเครื่องสำหรับเคมีควอนตัม (Machine Learning for Quantum Chemistry)''
% Copyright (c) 2022 รังสิมันต์ เกษแก้ว (Rangsiman Ketkaew).

% License: Creative Commons Attribution-NonCommercial-NoDerivatives 4.0 International (CC BY-NC-ND 4.0)
% https://creativecommons.org/licenses/by-nc-nd/4.0/

\chapter{ชุดข้อมูลทางเคมี}
\label{ch:chem_dataset}

ชุดข้อมูลคือการเก็บข้อมูลให้อยู่ในรูปแบบที่สามารถแบ่งประเภทของข้อมูลในชุดข้อมูลได้ โดยส่วนใหญ่แล้วเรามักจะเก็บข้อมูลในรูปแบบของตาราง
โดยตารางของชุดข้อมูลนั้นอาจจะมีหลายคอลัมน์ก็ได้ โดยแต่ละคอลัมน์จะแสดงถึงตัวแปรเฉพาะของข้อมูล ข้อมูลนั้นเป็นสิ่งที่สำคัญและเป็นองค์ประกอบ%
ที่ขาดไม่ได้เลยในการสร้างโมเดลปัญญาประดิษฐ์ ซึ่งการที่เรามีข้อมูลมหาศาลในทุกวันนี้ก็อาจจะเรียกได้ว่าเป็นสาเหตุหลักที่ทำให้เกิด ML ขึ้นมาได้ 
ชุดข้อมูลถือได้ว่าเป็นหัวใจสำคัญของ ML เลยก็ว่าได้ ถ้าหากเรามีชุดข้อมูลดี เราก็สามารถสร้าง Feature Input Vector ให้กับโมเดลต้องการฝึกสอนได้ 
แต่ถ้าหากชุดข้อมูลของเราไม่ได้ สิ่งที่ตามมาก็คือโมเดลที่ถูกฝึกสอนออกมานั้นก็จะมีประสิทธิภาพในการทำนายที่ต่ำมาก (Garbage In, Garbage Out) 

%--------------------------
\section{ชุดข้อมูล}
\label{sec:dataset}
%--------------------------

\begin{figure}[H]
    \centering
    \includegraphics[width=0.7\linewidth]{fig/dataset.png}
    \caption{ตัวอย่างชุดข้อมูลแบบ 2 มิติ โดยมี Feature คือ Score, Attempts, Qualify (เครดิตภาพ: w3resource.com)}
    \label{fig:dataset}
\end{figure}

เรามาทำความรู้จักกับชุดข้อมูลกันมากกว่านี้ดีกว่าครับ เริ่มจากเราต้องทำความเข้าใจมิติของชุดข้อมูล (Dimensionality of Dataset) กันก่อน
ซึ่งมิติของชุดข้อมูลก็จะมีตั้งแต่ 1 มิติ, 2 มิติ, 3 มิติ, 4 มิติ หรือสูงมากกว่านั้นก็ได้ ให้ลองนึกถึง Tensor ซึ่งเราสามารถเพิ่มจำนวนมิติของข้อมูลได้
สำหรับกรณีที่ชุดข้อมูลมี 1 มิตินั้นจะง่ายที่สุดเพราะเราจะมองว่าชุดข้อมูลแบบนี้เป็นเวกเตอร์ก็ได้ ชุดข้อมูล 1 มิติก็คือข้อมูลที่มีหลายแถวแต่มีแค่ 1 หลัก 
หรือจะเป็นชุดข้อมูลที่มีเพียงแค่ 1 แถวแต่มีหลายหลักก็ได้ สำหรับชุดข้อมูลแบบ 2 มิตินั้นให้เปรียบเทียบกับตาราง ซึ่งตารางประกอบไปด้วยแถวและหลัก
โดยเราจะมองว่าตารางนั้นจริง ๆ แล้วก็คือเมทริกซ์ก็ได้ ซึ่งชุดข้อมูล 2D ประกอบด้วยมิติของแถว (Row) และมิติของหลักหรือคอลัมน์ (Column) 
เมื่อนำจำนวนของแถวคูณกับจำนวนของหลัก (Row x Column) จะได้ขนาดของชุดข้อมูล (Size) ซึ่งสอดคล้องกับขนาดของเมทริกซ์ เช่น 
ชุดข้อมูลขนาด 125 x 50 หมายความว่าชุดข้อมูลนี้คือชุดข้อมูลขนาด 2 มิติ ที่มีจำนวนข้อมูลทั้งหมด 125 แถว แต่ละแถวมี 50 หลัก 
สำหรับชุดข้อมูล 3 มิติ ก็จะมีอีก 1 มิติเพิ่มเข้ามานอกเหนือจากแถวกับหลักซึ่งจะถูกเรียกว่าอะไรนั้นก็ขึ้นอยู่กับว่าข้อมูลนั้นเป็นข้อมูลประเภทไหน 
เพราะว่าบางครั้งเราก็ไม่ได้ใช้คำว่าแถวกับหลัก เช่น ถ้าเป็นชุดข้อมูลรูปภาพ ก็จะใช้ ความสูง x ความกว้าง x ความลึก (Height x Width x Depth) 
ซึ่งก็สามารถเรียงสลับได้
\idxth{ชุดข้อมูล!มิติของชุดข้อมูล}
\idxen{Dataset!Dimensionality of Dataset}

%--------------------------
\section{ประเภทและการแบ่งชุดข้อมูล}
\label{sec:split_dataset}
\idxth{ชุดข้อมูล!ประเภทของชุดข้อมูล}
\idxen{Dataset!Type of Dataset}
\idxth{ชุดข้อมูล!การแบ่งชุดข้อมูล}
\idxen{Dataset!Dataset Splitting}
%--------------------------

ชุดข้อมูลที่ใช้ใน ML นั้นโดยทั่วไปแล้วมักจะมีอยู่ 2 ประเภทคือชุดข้อมูลสำหรับการฝึกสอน (Training Set) และชุดข้อมูลสำหรับการทดสอบ 
(Test Set) ซึ่งวัตถุประสงค์ของชุดข้อมูลทั้งสองประเภทนี้ก็ตรงตัวเลยก็คือ Training Set จะถูกนำมาใช้ในการฝึกสอนโมเดล ส่วน Test Set 
จะถูกเก็บไว้ใช้ในการทดสอบโมเดลหรือการทำนายคำตอบที่โมเดลถูกสอนมา (Prediction) อย่างไรก็ตาม การฝึกสอนโมเดลโดยการใช้ Train Set 
ทั้งหมดนั้นมักจะทำให้เกิดความโน้มเอียง (Bias) ที่เกิดขึ้นจากชุดข้อมูลและส่งผลให้เกิด Biased ในขั้นตอน Prediction ด้วย เพราะป้องกันเหตุการณ์%
ดังกล่าวและทำให้เกิด Bias น้อยที่สุด เรามักจะทำการแบ่ง (Split) ชุดข้อมูลฝึกสอนให้เป็นชุดข้อมูลสำหรับการฝึกสอนจริง ๆ (Actual Training 
Set) และชุดข้อมูลสำหรับการตรวจสอบและยืนยันความถูกต้องซึ่งเรียกอีกอย่างว่า Validation Set

\begin{figure}[H]
    \centering
    \includegraphics[width=0.8\linewidth]{fig/dataset_splitting.png}
    \caption{การแบ่งชุดข้อมูลทั้งหมดออกเป็น (A) Training Set และ Test Set และ (B) Training Set, Validation Set และ 
    Test Set (เครดิตภาพ: Wikimedia Commons)}
    \label{fig:dataset_splitting}
\end{figure}

ภาพที่ \ref{fig:dataset_splitting} แสดงสัดส่วนแบบคร่าว ๆ ในการแบ่งชุดข้อมูลหลักออกเป็น Training Set และ Test Set 
และแสดงการแบ่งชุดข้อมูล Training Set อีกครั้งให้เป็น Actual Training Set ที่จะถูกนำไปใช้ในการฝึกสอนโมเดลจริง ๆ และ Validation 
Set ที่จะถูกนำมาทดสอบโมเดลเพื่อเป็นการหยั่งเชิงความสามารถของโมเดลก่อนที่จะนำไปใช้ทำนายค่าของเอาต์พุตของข้อมูลใน Test Set%
\footnote{โดยทั่วไปแล้วหลาย ๆ คนมักจะทำการแบ่งชุดข้อมูลโดยใช้อัตราส่วนคือ 80\% (สำหรับ Training Set) และ 20\% (สำหรับ Test
Set) ตามหลักการของ Pareto ซึ่งสามารถอ่านเพิ่มเติมได้ที่ \url{https://en.wikipedia.org/wiki/Pareto_principle}}

แล้วขั้นตอนการลด Bias นั่นมันเกิดขึ้นได้อย่างไร คำตอบก็คือในการแบ่งข้อมูลออกมาเป็น Validation Set (เช่นแบ่งออกมา 20\% จากทั้งหมด)
โดยทำการสุ่มเลือกบางส่วนของข้อมูลออกมา ซึ่งถ้าหากเราทำวนไปแบบนี้ไปเรื่อย ๆ เราจะเรียกว่าเป็นการทำ Validation แบบข้ามไปมาทั่วทั้ง 
Training Set ซึ่งเมื่อเรานำ Training Set แต่ละชุดไปฝึกสอนโมเดล เราจะได้ประสิทธิภาพของโมเดลแบบเฉลี่ย เปรียบเสมือนเป็นการเกลี่ย% 
หาความเท่ากันของข้อมูลนั่นเอง (กระจายออกไปให้เสมอกัน) ท้ายที่สุดแล้วถ้าเราแบ่งชุดข้อมูลตามที่ได้อธิบายมา เราจะมีอัตราส่วนของชุดข้อมูล%
ย่อย ๆ แต่ละประเภท ดังนี้

\begin{itemize}
    \item Training: 80\%
    \begin{itemize}
        \item Actual Training Set: 60\%
        
        \item Cross Validation: 20\%
    \end{itemize}
    
    \item Testing: 20\%
\end{itemize}

%--------------------------
\section{การสร้างชุดข้อมูล}
\label{sec:create_dataset}
\idxth{ชุดข้อมูล!การสร้างชุดข้อมูล}
%--------------------------

ขั้นตอนการสร้างชุดข้อมูลประกอบไปด้วย 3 ขั้นหลักดังนี้

\paragraph{1. รวบรวมข้อมูล (Data Collection)} สิ่งแรกที่เราจะทำในการมองหา Dataset ก็คือแหล่งข้อมูลที่เราสามารถนำข้อมูลมาใช้ได้ 
ถ้าหากแหล่งข้อมูลไม่น่าเชื่อถือเราก็มักจะได้ชุดข้อมูลที่มีคุณภาพต่ำซึ่งอาจจะมีข้อผิดพลาดในชุดข้อมูลด้วยเช่นกัน เช่น ข้อมูลที่ถูกสร้างขึ้นมา 
(ข้อมูลปลอมหรือ Fake Data) 

\paragraph{2. ประมวลผลข้อมูลก่อน (Data Preprocessing)} หลักการสำคัญข้อหนึ่งของวิทยาศาสตร์เชิงข้อมูลเลยก็คือทำความสะอาดชุดข้อมูล 
(Data Cleaning) รวมไปถึงการทราบที่มาที่ไปและการทำความเข้าใจชุดข้อมูล เราต้องถามตัวเองก่อนว่าชุดข้อมูลที่เราสนใจนั้นเคยถูกใช้มาก่อน%
หน้านี้แล้วหรือยัง ถ้าหากว่ายังและเป็นชุดข้อมูลใหม่ เราควรจะต้องตั้งข้อสังเกตหรือสมมติฐานไว้ก่อนว่าชุดข้อมูลชุดนี้อาจจะมีข้อมูลที่ผิดปกติซ่อนอยู่ได้ 
หรืออาจจะมีข้อมูลที่ไม่ครบถ้วนขาดหายไป เราสามารถทำการประมวลผลชุดข้อมูลก่อนนำไปใช้งานจริงได้โดยการดูที่คุณภาพของ Features รวมไปถึง 
Bias ภายในชุดข้อมูล บางชุดข้อมูลมี Features เยอะมากแต่ว่ามี Bias เยอะมาก ๆ กับ Features เพียงแค่ 2-3 Features นอกจากนี้แล้ว%
ปริมาณของชุดข้อมูลก็มีผลด้วยเพราะว่าถ้าหากเรามีปริมาณข้อมูลที่น้อยเกินไปก็อาจจะเกิดปัญหา Overfitting ได้ในภายหลัง

\paragraph{3. การทำคำอธิบายประกอบ (Annotatation)} หลังจากทำความสะอาดข้อมูลเสร็จเรียบร้อยแล้วสิ่งที่เราควรจะต้องในลำดับต่อไปคือ 
Annotate นั่นคือเป็นการทำให้มั่นใจว่าข้อมูลของเรานั้นมันสามารถนำไปใช้ในการสอนเครื่องจักรเข้าใจได้ อธิบายง่าย ๆ คือทำให้คอมพิวเตอร์สามารถ%
เรียนรู้จากข้อมูลได้ นั่นก็เพราะว่าเครื่องจักรไม่สามารถเข้าใจข้อมูลเหมือนอย่างที่มนุษย์เข้าใจ ดังนั้นเราควรจะต้องทำการเพิ่มคำอธิบายเชิงดิจิทัลให้%
กับข้อมูลนั่นก็คือการระบุค่าเฉพาะสำหรับข้อมูลนั้น ๆ หรือที่เรียกว่าการติดป้าย (Labeling)

%--------------------------
\section{ปริภูมิเคมี}
\label{sec:chem_space}
\idxboth{ปริภูมิเคมี}{Chemical Space}
%--------------------------

ปริภูมิเคมี (Chemical Space)\autocite{kirkpatrick2004} เป็นแนวคิดที่อธิบายว่าจำนวนและชนิดของโมเลกุลนั้นไม่ที่สิ้นสุด (Infinity) 
ซึ่งเปรียบเสมือนจำนวนดวงดาวในจักรวาลที่ก็ไม่มีที่สิ้นสุดเหมือนกัน ในขณะที่นักดาราศาสตร์พยายามสำรวจค้นหาดาวดวงใหม่นั้น นักเคมีก็สำรวจหา%
โมเลกุลชนิดใหม่ที่ซ่อนอยู่ในปริภูมิเคมี การค้นพบโมเลกุลใหม่นั้นอาจนำมาซึ่งคุณสมบัติเชิงเคมีที่น่าสนใจที่เราสามารถนำเอาองค์ความรู้ไปพัฒนาและ%
ต่อยอดในการศึกษาเคมีแขนงต่าง ๆ ได้ เช่น นำโมเลกุลที่ค้นพบใหม่นี้ไปศึกษาปฏิกิริยาใหม่ ๆ หรือรวมไปถึงการนำไปใช้ประโยชน์ในระยะยาวและ%
ใช้จริงในระดับอุตสาหกรรม เช่น การพัฒนาวัสดุหรือผลิตภัณฑ์ชนิดใหม่ 

%--------------------------
\section{ชุดข้อมูลเคมีควอนตัม}
\label{ssec:step_create_qm_dataset}
\idxth{ชุดข้อมูล!การสร้างชุดข้อมูลเคมีควอนตัม}
\idxen{Dataset!Create Dataset}
%--------------------------

%--------------------------
\section{ชุดข้อมูลมาตรฐาน}
\label{sec:std_dataset}
\idxth{ชุดข้อมูล!ชุดข้อมูลมาตรฐาน}
\idxen{Dataset!Standard Dataset}
%--------------------------

QM9 เป็นหนึ่งในชุดข้อมูลเคมีควอนตัมที่ได้รับความนิยมและถูกนำมาใช้ในงานวิจัย ML สำหรับเคมีควอนตัมเป็นอย่างมาก ซึ่งถูกใช้อย่างแพร่หลายตั้งแต่ปี 
ค.ศ. 2014 เป็นต้นมา\autocite{ruddigkeit2012,ramakrishnan2014} โดยบทความงานวิจัยแรกที่นำ QM9 มาใช้ในการทดสอบประสิทธิภาพ%
ของโมเดล ML นั้นได้รายงานค่าความผิดพลาดของโมเดล ML ที่ใช้ในการทดสอบว่ามีค่าคลาดเคลื่อนไม่เกิน 10 kcal/mol ซึ่งในเชิงการวัดนั้นถือว่า%
มีคลาดเคลื่อนที่เยอะมาก ๆ และในเวลาต่อมาก็ได้มีการพัฒนาระเบียบวิธีวิจัยรวมไปถึงโมเดล ML และ Descriptor ใหม่ ๆ จนทำให้ในปัจจุบันนั้น%
นักวิจัยสามารถที่จะทำนายหรือพยากรณ์ค่าพลังงานของโมเลกุลทางเคมีอินทรีย์ขนาดเล็กได้แม่นยำมากโดยมีค่าความคลาดเคลื่อนประมาณ 1 kcal/mol 
หรือต่ำกว่านั้น ซึ่งค่า 1 kcal/mol นี้ถือได้ว่าเป็น \textit{ค่าความถูกต้องทางเคมี (Chemical Accuracy)} ซึ่งเป็นค่ามาตรฐานที่ต่ำที่สุดที่%
เทคนิคทางการทดลองสามารถวัดได้ โดยถ้าหากค่าความคลาดเคลื่อนทางการคำนวณที่ต่ำไปกว่า 1 kcal/mol แล้วเทคนิคต่าง ๆ ในเชิงการทดลอง%
จะไม่สามารถระบุความแตกต่างของความคลาดเคลื่อนที่แม่นยำได้อีกต่อไป

QM9 ประกอบไปด้วยข้อมูลคุณสมบัติอิเล็กทรอนิกส์ของโมเลกุลมากถึง 134,000 โมเลกุล โดยทุกโมเลกุลมีธาตุพื้นฐานเป็นองค์ประกอบ ประกอบไปด้วย 
คาร์บอน (C), ไนโตรเจน, (N), ออกซิเจน (O), ไฮโดรเจน (H), และฟลูออรีน (F) โดย Feature หลักของ QM9 ก็จะมีพิกัดคาร์ทีเซียนของ%
อะตอมทุกอะตอมในโมเลกุลซึ่งได้มาจากการคำนวณการปรับโครงสร้าง (Geometry Optimization) ด้วยระเบียบวิธี B3LYP/6-31G(2df,p) 
และนอกจากนี้ยังมีค่า Label หรือค่าที่ไว้ใช้ในการเปรียบเทียบการพยากรณ์ดังแสดงในตารางที่ \ref{tab:qm9_feature}%
\footnote{โมเดล ML ที่เหมาะสมสำหรับการฝึกสอนด้วย QM9 นั้นจะต้องไม่ขึ้นกับ Translation, Rotation และ Permutation}

\begin{table}[H]
    \centering
    \caption{ข้อมูล Feature ของชุดข้อมูล QM9}
    \label{tab:qm9_feature}
    \small
    \begin{tabular}{llll}\toprule
    \textbf{ดัชนี} &\textbf{ชื่อ} &\textbf{หน่วย} &\textbf{คำอธิบาย} \\\midrule
    0 &index &- &Consecutive, 1-based integer identifier of molecule \\
    1 &A &GHz &Rotational constant A \\
    2 &B &GHz &Rotational constant B \\
    3 &C &GHz &Rotational constant C \\
    4 &mu &Debye &Dipole moment \\
    5 &alpha &Bohr$^3$ &Isotropic polarizability \\
    6 &homo &Hartree &พลังงานของ Highest occupied molecular orbital (HOMO) \\
    7 &lumo &Hartree &พลังงานของ Lowest unoccupied molecular orbital (LUMO) \\
    8 &gap &Hartree &Gap (พลังงานระหว่าง LUMO and HOMO) \\
    9 &r2 &Bohr$^2$ &Electronic spatial extent \\
    10 &zpve &Hartree &Zero point vibrational energy \\
    11 &U0 &Hartree &Internal energy at 0 K \\
    12 &U &Hartree &Internal energy at 298.15 K \\
    13 &H &Hartree &Enthalpy at 298.15 K \\
    14 &G &Hartree &Free energy at 298.15 K \\
    15 &Cv &cal/(mol K) &Heat capacity at 298.15 K \\
    \bottomrule
    \end{tabular}
\end{table}

ชุดข้อมูล QM9 สามารถดาวน์โหลดมาใช้งานได้ฟรีจากเว็บไซต์ \url{http://quantum-machine.org/datasets/} โดยจะมีข้อมูลพิกัดคาร์ทีเซียน
(Cartesian Coordinates), คุณลักษณะ (Features), และค่าพลังงานซึ่งเป็น Target ของเรา คราวนี้เราลองมาดูโค้ดสำหรับการใช้งาน 
QM9 โดยใช้ภาษา Python ตามด้านล่างนี้ได้เลย

\noindent ทำการเรียกใช้ไลบรารี่และอ่านไฟล์ของชุดข้อมูล
\begin{lstlisting}[style=MyPython]
# Import libraries
import ase.io as aio
import pandas as pd

# Read qm9.csv
qm9_data = pd.read_csv('./qm9.csv', index_col=0)
\end{lstlisting}

\noindent เราสามารถใช้คำสั่งด้านล่างในการแสดง Target ได้

\begin{lstlisting}[style=MyPython]
# Convert energy from Hartree to kcal/mol
target = qm9_data['u0'] * 627.5096080305927 
print(target)

# OUTPUT
0         -25400.917498
40       -121896.331092
80       -146555.740566
120      -122481.233425
160      -168344.348805
              ...      
119800   -242900.012196
119840   -230424.279698
119880   -266202.856340
119920   -252980.566249
119960   -288738.466448
Name: u0, Length: 3000, dtype: float64
\end{lstlisting}

\noindent อ่านพิกัดคาร์ทีเซียนของโมเลกุล

\begin{lstlisting}[style=MyPython]
# We read xyz coorinates of all the molecules with ase aio.read tool
ase_mols = [aio.read('data/qm9/qm9_xyz/' + mol + '.xyz') for mol in qm9_data.mol_id]
\end{lstlisting}

\noindent ตรวจสอบขนาดของโมเลกุลที่ใหญ่ที่สุดในชุดข้อมูล

\begin{lstlisting}[style=MyPython]
# Check the size of molecules in the dataset
size=[]
for mol in ase_mols:
    num = len(mol.get_atomic_numbers())
    size.append(num)
max(size) #maximum size of the molecule in the dataset

# OUTPUT
27
\end{lstlisting}

โดยโค้ดด้านบนที่เราใช้สำหรับการโหลดชุดข้อมูล QM9 นั้นเราจะนำไปใช้ต่อในบทที่ \ref{sec:pred_tot_ener} สำหรับการทำนายพลังงานรวม%
ของโมเลกุลของชุดข้อมูล QM9 \ref{ssec:pred_spec_ir}

นอกจาก QM9 แล้วยังมีชุดข้อมูลอื่น ๆ ที่นักวิจัยมักจะนำมาใช้ในการฝึกสอนโมเดลและทำวิจัย เช่น 

\begin{itemize}
    \item QM7\autocite{blum2009,rupp2012a}
    
    \item QM7b\autocite{blum2009,montavon2013}
    
    \item QM8\autocite{ruddigkeit2012,ramakrishnan2015}
    
    \item ISO17\autocite{schutt2017,schutt2017a,ramakrishnan2014}
\end{itemize}

\noindent ซึ่งก็จะมี Label สำหรับวัตถุประสงค์ในการฝึกสอนโมเดลในการเพิ่มความสามารถการพยากรณ์คุณสมบัติเคมีของโมเลกุลที่ต่างกันออกไป 
โดยชุดข้อมูลที่กล่าวมาทั้งหมดนั้นสามารถดาวน์โหลดได้ฟรีจากเว็บไซต์ \url{http://quantum-machine.org/datasets} เช่นกัน

%--------------------------
\section{การวิเคราะห์ชุดข้อมูล}
\label{sec:dataset_analysis}
\idxth{ชุดข้อมูล!การวิเคราะห์ชุดข้อมูล}
\idxen{Dataset!Dataset Analysis}
%--------------------------

หลังจากที่เราเลือกชุดข้อมูลที่ต้องการนำมาศึกษาแล้ว ลำดับต่อไปคือการคำนวณ Input Feature หรือจะเรียกว่า Feature Vector ก็ได้
ซึ่งจะถูกนำไปใช้ในการฝึกสอนโมเดล ML ต่อไป โดย Feature Vector ที่ผู้เขียนจะยกตัวอย่างให้ได้ศึกษานั้นก็จะเป็น Feature ที่ใช้ Descriptor 
แบบง่ายนั่นก็คือ Coulomb Matrix (CM) ผู้เขียนจะยังคงใช้ชุดข้อมูล QM9 และใช้โค้ดต่อไปนี้ในการคำนวณ CM ของโมเลกุลตัวอย่างเพียงแค่ 3,000 
โมเลกุลเท่านั้น

\begin{lstlisting}[style=MyPython]
import numpy as np
from qml.representations import * 

cm = []
size = 27 # Maximum size of molecule in the set

# Run for loop over every molecule in the database
for structure in ase_mols: 
    # ASE prints atomic numbers 
    atomic_numbers = structure.get_atomic_numbers() 
    # ASE prints coordinates
    coordinates=structure.get_positions() 
    cm1 = generate_coulomb_matrix(atomic_numbers,
    # CM representation is saved into cm1
    coordinates, size = size, sorting="row-norm") 
    # All CM representations are added into one variable
    cm.append(cm1) 

# Transform cm into numpy array
cm = np.array(cm) 
# Check size of cm
print(cm.shape)

# OUTPUT
(3000, 378)
\end{lstlisting}

หลังจากที่เราคำนวณ CM ของโมเลกุลในชุดข้อมูลเสร็จเรียบร้อยแล้ว สิ่งที่หลายคนทำในละดับต่อไปก็คือสร้างโมเดลแล้วนำ Feature Vector 
หรืออินพุตที่ได้ไปใช้ในการฝึกสอนโมเดลทันทีเลย ซึ่งการทำแบบนี้นั้นจริง ๆ แล้วไม่เหมาะสมเท่าไหร่นัก นั่นก็เพราะว่าเราควรจะต้องทำความเข้าใจ 
Feature ที่เราคำนวณออกมาก่อนโดยทำการวิเคราะห์เพื่อดูลักษณะการกระจายตัวหรือการจัดกลุ่มซึ่งสามารถบอกแนวโน้มรวมไปถึง Bias ได้

เราสามารถใช้เทคนิค Unsupervised ML แบบง่าย ๆ ที่ไม่ซับซ้อน เช่น Principal Component Analysis (PCA) ซึ่งเป็นวิธีที่ลดจำนวนมิติ 
(Dimensionality Reduction) ของข้อมูลให้อยู่ในรูปขององค์ประกอบเชิงตั้งฉาก (Orthogonal Component) ที่อธิบายปริมาณของความแปรปรวน
(Variance) ที่มากที่สุด หรือจะใช้วิธี t-distributed Stochastic Neighbor Embedding (t-SNE) ซึ่งเป็นวิธีที่สามารถแสดงข้อมูลที่มีมิติสูง ๆ 
(Highd-dimensional Data) ได้เช่นเดียวกัน\autocite{JMLR:v9:vandermaaten08a,belkina2019} โดยจะทำเปลี่ยนความเหมือนกันระหว่าง%
ข้อมูลสองชุดให้เป็นความน่าจะเป็นร่วม (Joint Probability) แล้วทำการปรับค่า Kullback-Leibler Divergence ระหว่างความน่าจะเป็นร่วม%
ของข้อมูลที่อยู่ในมิติต่ำและมิติสูงให้น้อยที่สุด (Minimization)\footnote{เทคนิค t-SNE ถูกพัฒนาต่อมาจากเทคนิค SNE ซึ่งแรกเริ่มนั้นพัฒนาโดย
Geoffrey Hinton และ Sam Roweis แห่งมหาวิทยาลัยโทรอนโต ประเทศแคนาดา\autocite{NIPS2002_6150ccc6} หลังจากนั้น Laurens 
van der Maaten ได้ทำการเพิ่ม t-distributed เข้าไป} ซึ่งเราสามารถใช้ทั้ง PCA และ t-SNE ในการวิเคราะห์เพื่อดูลักษณะหรืออธิบายง่าย ๆ 
คือดู \enquote{\textit{รูปร่างหน้าตา}} ของ CM ของทั้ง 3,000 โมเลกุลที่คำนวณออกมาได้โดยใช้โค้ดต่อไปนี้
\idxen{t-distributed Stochastic Neighbor Embedding}

\noindent สร้างโมเดล t-SNE
\begin{lstlisting}[style=MyPython]
import sklearn

tsne_cm = sklearn.manifold.TSNE(n_components=2)
tsne_cm_data = tsne_cm.fit_transform(cm)
\end{lstlisting}

\noindent สร้างโมเดล PCA
\begin{lstlisting}[style=MyPython]
pca_cm = sklearn.decomposition.PCA(n_components=2)
pca_cm_data = pca_cm.fit_transform(cm)
\end{lstlisting}

\noindent พลอตกราฟค่าที่ได้จากการ Fit ข้อมูล
\begin{lstlisting}[style=MyPython]
import matplotlib.pyplot as plt

fig, axs = plt.subplots(1,2,figsize=(15,9))
axs[0].set_title('Principal Components')
axs[1].set_title('t-SNE')

plot1 = axs[0].scatter(pca_cm_data[:, 0], pca_cm_data[:, 1], c=target, cmap='jet', s=1)
plot2 = axs[1].scatter(tsne_cm_data[:, 0], tsne_cm_data[:, 1], c=target, cmap='jet', s=1)

cbar = fig.colorbar(plot2, ax=axs);
cbar.set_label('Internal energy at 0 K [kcal/mol]')
plt.show()
\end{lstlisting}

\noindent พลอตที่ได้จะเป็นแบบนี้ โดยข้อมูลแต่ละจุดนั้นจะถูกไฮไลต์ด้วยสีที่มีสเกลแตกต่างกันไปซึ่งแสดงค่าของพลังงานภายในของโมเลกุล

\begin{figure}[H]
    \centering
    \includegraphics[width=\linewidth]{fig/cm_pca_tsne.png}
    \caption{การกระจายตัวของ Coulomb Matrix Feature ที่ถูกลดจำนวนมิติให้เหลือเพียงแค่ 2 มิติ (Principal Component = 2)}
    \label{fig:cm_pca_tsne}
\end{figure}
 % ชุดข้อมูลทางเคมี
% LaTeX source for ``การเรียนรู้ของเครื่องสำหรับเคมีควอนตัม (Machine Learning for Quantum Chemistry)''
% Copyright (c) 2022 รังสิมันต์ เกษแก้ว (Rangsiman Ketkaew).

% License: Creative Commons Attribution-NonCommercial-NoDerivatives 4.0 International (CC BY-NC-ND 4.0)
% https://creativecommons.org/licenses/by-nc-nd/4.0/

\chapter{การทำนายคุณสมบัติของโมเลกุล}
\label{ch:predict_molprop}

%--------------------------
\section{แนวทางการปฏิบัติ}
%--------------------------

แนวทางปฏิบัติ (Best Practice) หรือลำดับขั้นตอนสำหรับการนำ ML มาประยุกต์กับเคมีควอนตัมสามารถแบ่งออกเป็น 6 ขั้นตอนง่าย ๆ ได้ดังนี้
\idxen{Best Practice}

\begin{enumerate}
    \item ทำความสะอาดข้อมูลดิบ (Raw Data Cleaning)
    \item เลือก Representation/Descriptor ที่จะนำมาคำนวณ Feature 
    \item เลือกอัลกอริทึม ML ที่เหมาะสมกับโจทย์ของเรา 
    \item ฝึกสอนโมเดลและทำนายคำตอบ
    \item ศึกษาผลกระทบจากการเปลี่ยน Hyperparameter และทำ Validation
    \item ประเมินประสิทธิภาพของโมเดลและวิเคราะห์ผลการทำนาย
\end{enumerate}

ขั้นตอนแรกสุดเลยก็คือเป็นการเตรียมข้อมูลดิบนั่นก็คือข้อมูลทางเคมีเบื้องต้นที่เรามี โดยส่วนใหญ่แล้วนักเคมีเชิงคำนวณมักจะเริ่มต้นด้วยข้อมูลพิกัด%
คาร์ทีเซียน (Cartesian Coordinates) ของชุดโมเลกุลที่ต้องการศึกษา เช่น โมเลกุลอินทรีย์ ลำดับถัดมาคือเราจะต้องเลือก Representation 
ที่เราต้องการนำมาคำนวณมาคุณสมบัติต่าง ๆ ของโมเลกุล ซึ่งเรามักจะได้มาจากการคำนวณด้วยวิธีแบบดั้งเดิม โดยการคำนวณ Representation 
นั้นก็จะมีให้เลือกมากมาย ขึ้นอยู่กับความสอดคล้องของอินพุต (Feature ที่เราคำนวณ) กับเอาต์พุตที่เราต้องการจะทำนาย ซึ่งขั้นตอนนี้จะเป็นการ%
สร้างชุดข้อมูลนั่นเอง เมื่อเราได้ชุดข้อมูลแล้ว เราอาจจะมีขั้นตอนที่แทรกเข้ามาเพื่อช่วยให้เราเข้าใจชุดข้อมูลได้มากขึ้น เช่น เราอาจจะใช้สถิติเข้ามา%
ช่วยคำนวณค่าทางสถิติของชุดข้อมูลก่อนนำไปฝึกสอนโมเดล เช่น คำนวณค่ากลางหรือพารามิเตอร์ที่ให้เราเข้าใจการกระจายตัวในชุดข้อมูลรวมไปถึง%
ความสำคัญ (Importance) ของ Feature แต่ละตัวในชุดข้อมูล การทำแบบนี้จะช่วยให้เราเข้าใจว่า Feature ตัวไหนที่ \enquote{น่าจะ} มีผล%
ต่อประสิทธิภาพของโมเดลเรามากที่สุด เมื่อเราได้ชุดข้อมูลที่มีความเหมาะสมแล้ว ขั้นตอนต่อมาคือการเลือกเทคนิคหรืออัลกอริทึมของ ML ที่เราต้องการ%
จะใช้ เริ่มต้นเราอาจจะยังไม่ต้องไปใช้เทคนิคที่อลังการมากก็ได้ ซึ่งการเลือกใช้เทคนิคง่าย ๆ เช่น Ridge Regression ก็อาจจะทำให้เรามีโมเดล MLP
ที่มีประสิทธิภาพมาก ๆ แล้วก็ได้ เพราะการที่เราไปใช้เทคนิคที่ซับซ้อนตั้งแต่แรกนั้น มันอาจจะไม่ได้การันตีว่าเราจะได้โมเดลที่ดีเสมอไป และนอกจากนี้%
ยังเสียเวลาอีกด้วย ตัวอย่างเช่น ผู้เขียนมักจะเห็นหลาย ๆ คนที่เริ่มฝึกสอนโมเดลด้วย Deep Neural Network โดยการใช้เทคนิคขั้นสูงกับข้อมูลที่%
มีความเรียบง่าย (ขี่ช้างจับตั๊กแตน) ซึ่งตรงจุดนี้บางครั้งมันก็มีความไม่เหมาะสมระหว่างเทคนิคและข้อมูลที่เรามี โดยตรงจุดนี้ก็ขึ้นอยู่กับวิจารณญาน%
ของแต่คนครับ เมื่อเราเลือกวิธี ML ได้แล้ว ขั้นตอนต่อมาก็คือการสร้างโมเดลและฝึกสอนกับ Training Set โดยในขั้นตอนนี้เราอาจจะลองสร้างหลาย ๆ 
โมเดลและทำการปรับ Hyperparameter ไปด้วยก็ได้ (ควรจะเปลี่ยนค่าอย่างเป็นระบบ ไม่ควรเปลี่ยนแบบมั่ว ๆ) นอกจากนี้เราอาจจะยังทำ Validation 
ด้วยก็ได้ เมื่อเราได้โมเดลที่ถูกฝึกสอนมาแล้ว ลำดับต่อมาก็คือการทำนายหรือพยากรณ์คำตอบนั่นเอง โดยเราจะต้องมาประเมินประสิทธิภาพของโมเดล%
ในขั้นตอนนี้ด้วย เราจะต้องมาวิเคราะห์ถึงปัจจัยที่ส่งผลต่อค่าที่เราได้ออกมา พยายามหาความเชื่อมโยงระหว่าง Feature, ML Algorithm และ%
พารามิเตอร์อื่น ๆ เมื่อเราได้โมเดลที่ถูกฝึกสอนมาอย่างดีและมีประสิทธิภาพที่อยู่ในเกณฑ์ที่ยอมรับได้แล้วนั้น เราก็จะมีโมเดลที่พร้อมจะไปใช้งานจริงครับ

\begin{figure}[H]
    \centering
    \includegraphics[width=\linewidth]{fig/ml_pipeline.png}
    \caption{แนวทางและขั้นตอนการสร้างโมเดลปัญญาประดิษฐ์ (เครดิตภาพ: https://vitalflux.com)}
    \label{fig:ml_pipeline}
\end{figure}

%--------------------------
\section{การเลือกโมเดลที่เหมาะสม}
%--------------------------

\begin{figure}[H]
    \centering
    \includegraphics[width=0.9\linewidth]{fig/ml_map.png}
    \caption{แผนภาพการเลือกใช้โมเดลปัญญาประดิษฐ์ (เครดิตภาพ: https://scikit-learn.org)}
    \label{fig:ml_map}
\end{figure}

\begin{itemize}
    \item Linear kernel : $K(x_i, x_j) = x_i \cdot x_j$
    \item Polynomial : $K(x_i, x_j; a, b) = (x_i \cdot x_j + a)^b$
    \item Gaussian kernel : $K(x_i, x_j; w, \sigma) = \mathrm{exp}\left(-\frac{|x_i-x_j|^2}{2\sigma^2}\right)$
    \item Laplacian kernel : $K(x_i, x_j; w, \gamma) = \mathrm{exp}\left(-{|x_i-x_j|}\right)$
\end{itemize}

%--------------------------
\section{การทำนายพลังงานรวมของโมเลกุล}
%--------------------------

%--------------------------
\section{การทำนายพื้นผิวพลังงานศักย์}
%--------------------------

การทำนายพื้นผิวพลังงานศักย์ (Potential Energy Surface หรือ PES)

Machine Learning Potentials (MLP) แบ่งออกได้เป็นสองประเภทคือ Kernel-based Potential กับ Neural Network-based Potential

- Gaussian Approximation Potentials (GAP)\autocite{bartok2010}
- Moment Tensor Potentials (MTP)\autocite{shapeev2016}
- Spectral Neighbor Analysis Potentials (SNAP)\autocite{thompson2015}


- High-dimensional Neural Network Potentials (HDNNP)\autocite{behler2007}
- ANAKIN-ME หรือเรียกสั้น ๆ ว่า ANI (ชื่อเต็มคือ (Accurate NeurAl networK engINe for Molecular Energies)
    - ANI-1x\autocite{smith2017}
    - ANI-1ccx\autocite{smith2018}
    - ANI-2x\autocite{smith2019}

%--------------------------
\section{การจำลอง Force Field}
%--------------------------

การทำนายหรือพยากรณ์แรง (Forces) และพลังงาน (Energies) ของโมเลกุลนั้นเรียกอีกอย่างหนึ่งว่าการสร้างโมเดล ML Force-field 
เริ่มต้นสมมติว่าเรามีชุดข้อมูลที่มี Feature Vector ซึ่งเขียนแทนด้วย $\mathbf{D}$ และมีอนุพันธ์แบบเวกเตอร์ (Divergence) เป็น 
$\nabla_{\mathbf{r_i}} \mathbf{D}$ และมีข้อมูลเพิ่มเติมคือพลังงาน $E$ และแรง $\mathbf{F}$ ของระบบ (โมเลกุล) สำหรับการฝึกสอน
เราจะสร้าง Neural Network (เขียนแทนด้วย $f$) เพื่อสร้างโมเดลสำหรับการทำนายพลังงาน ($\hat{E} = f(\mathbf{D})$)
\footnote{เครื่องหมาย $\hat{}$\, (อ่านว่า \enquote{hat}) ที่อยู่ด้านบนของตัวแปรเป็นส่งที่บ่งบอกว่าค่าที่เราจะทำนายของตัวแปรนั้น ๆ}
ซึ่งเราสามารถคำนวณแรงได้โดยตรงจากค่าติดลบของ Gradient ของพลังงานเทียบกับพิกัดตำแหน่งของอะตอมนั้น ๆ ดังนั้นแรงที่ได้จะเป็นปริมาณต่ออะตอม
ยกตัวอย่างเช่นแรงของอะตอม $i$ สามารถคำนวณได้จากสมการต่อไปนี้ (โดยใช้เวกเตอร์แบบแถว)
\idxen{Force Field}

\begin{align}\label{eq:force_pred}
\hat{\mathbf{F}}_i &= - \nabla_{\mathbf{r_i}} f(\mathbf{D}) \\
&= - \nabla_{\mathbf{D}} f \cdot \nabla_{\mathbf{r_i}} \mathbf{D}\\
&= - \begin{bmatrix}
    \frac{\partial f}{\partial D_1} & \frac{\partial f}{\partial D_2} & \dots
\end{bmatrix}
\begin{bmatrix}
    \frac{\partial D_1}{\partial x_i} & \frac{\partial D_1}{\partial y_i} & \frac{\partial D_1}{\partial z_i}\\
    \frac{\partial D_2}{\partial x_i} & \frac{\partial D_2}{\partial y_i} & \frac{\partial D_2}{\partial z_i}\\
    \vdots & \vdots & \vdots \\
\end{bmatrix}
\end{align}

จากสมการ \ref{eq:force_pred} นั้นเราอธิบายได้ว่า $\nabla_{\mathbf{D}} f$ เป็นค่าอนุพันธ์ของคำตอบของโมเดล ML ซึ่งจะเทียบกับ
Descriptor $\mathbf{D}$ และ $\nabla_{\mathbf{r_i}} \mathbf{D}$ เป็น อนุพันธ์ของ Descriptor ที่เทียบกับตำแหน่งของอะตอม
ซึ่งตามที่เราได้ศึกษามาก่อนหน้านี้ว่า Neural Network นั้นจะให้คำตอบที่เป็นแบบ Analytical Solution

%--------------------------
\section{การทำนายพลังงานกระตุ้นของปฏิกิริยาเคมี}
%--------------------------

พลังงานกระตุ้น (Activation Energy) เป็นค่าพลังงานที่บ่งบอกถึงความยากง่ายในการทำให้ปฏิริยาเคมีสามารถดำเนินไปได้ ณ สภาวะหนึ่ง ๆ

%--------------------------
\section{การทำนายประจุของอะตอม}
%--------------------------

%--------------------------
\section{การทำนายไดโพลโมเมนต์}
%--------------------------

%--------------------------
\section{การทำนายค่าคู่ควบเชิงเล็กอิเล็กทรอนิกส์}
%--------------------------

ค่าคู่ควบเชิงเล็กอิเล็กทรอนิกส์ (Electronic Coupling) คือค่าความเกี่ยวเนื่องเชิงอิเล็กทรอนิกส์ระหว่าง 2 สถานะใด ๆ เช่นสถานะเริ่มต้นและ%
สถานะสิ้นสุดในกระบวนการทางควอนตัม

- Electron transfer Coupling
- Nonadiabatic Coupling

%--------------------------
\section{การทำนายสเปคตรัม}
\idxth{การทำนายสเปคตรัม}
%--------------------------

%--------------------------
\subsection{การทำนายอินฟราเรดสเปกโทรโกปี}
\idxth{การทำนายสเปคตรัม!อินฟราเรด}
%--------------------------

%--------------------------
\subsection{การทำนายรามานสเปกโทรโกปี}
\idxth{การทำนายสเปคตรัม!รามาน}
%--------------------------

%--------------------------
\section{บทความวิชาการเพิ่มเติม}
%--------------------------

นอกเหนือจากการนำ ML ไปใช้สำหรับการทำนายพารามิเตอร์ต่าง ๆ แล้ว ถ้าหากผู้อ่านสนใจการประยุกต์ใช้ ML กับงานทางด้านอื่น ๆ ของเคมีควอนตัม 
สามารถอ่านบทความวิชาการเพิ่มเติมได้จากวารสารวิชาการชั้นแนวหน้า เช่น Journal of Chemical Theory and Computation (JCTC), 
Journal of Chemical Physics (JCP), และ Journal of Physical Chemistry A (JPCA)

โดยผู้เขียนได้เลือกงานวิจัยที่มีความโดดเด่นและเหมาะสำหรับผู้เริ่มต้นศึกษา ML และเคมีควอนตัม ซึ่งน่าจะช่วยให้ผู้อ่านเห็นภาพรวมของโจทย์งานวิจัย%
ในปัจจุบันที่กำลังมาแรง บทความที่คัดเลือกมาประกอบไปด้วยบทความการทบทวนงานวิจัย (Review) ที่ใช้ ML ในการเรียนรู้ Force Field 
สำหรับงานทางด้านเคมีควอนตัมและการจําลองพลวัตเชิงโมเลกุล (QM/MD) หรือนำมาใช้ในการทำนายพื้นที่พลังงานอิสระ (Free Energy Landscape)
ไปจนถึงการพัฒนาโมเดล ML เพื่อทำนายคุณสมบัติเชิงโมเลกุล เช่น ไดโพลโมเมนต์ (Dipole Moment) และสภาพการเกิดขึ้น (Polarizability)

\begin{enumerate}
    \item \enquote{PhysNet: A Neural Network for Predicting Energies, Forces, Dipole Moments, and 
    Partial Charges}\autocite{unke2019}\\
    ตีพิมพ์เมื่อวันที่ 01 พฤษภาคม ค.ศ. 2019
    
    \item \enquote{Comparison of the Performance of Machine Learning Models in Representing High-Dimensional 
    Free Energy Surfaces and Generating Observables}\autocite{cendagorta2020}\\
    ตีพิมพ์เมื่อวันที่ 10 เมษายน ค.ศ. 2020
    
    \item \enquote{Kernel-Based Machine Learning for Efficient Simulations of Molecular Liquids}\autocite{scherer2020}\\
    ตีพิมพ์เมื่อวันที่ 13 เมษายน ค.ศ. 2020

    \item \enquote{Machine Learning Force Fields}\autocite{unke2021}\\
    ตีพิมพ์เมื่อวันที่ 11 มีนาคม ค.ศ. 2021\\

    \item \enquote{The Rise of Neural Networks for Materials and Chemical Dynamics}\autocite{kulichenko2021}\\
    ตีพิมพ์เมื่อวันที่ 1 กรกฎาคม ค.ศ. 2021\\

\end{enumerate}
 % การทำนายคุณสมบัติของโมเลกุล
% LaTeX source for ``การเรียนรู้ของเครื่องสำหรับเคมีควอนตัม (Machine Learning for Quantum Chemistry)''
% Copyright (c) 2022 รังสิมันต์ เกษแก้ว (Rangsiman Ketkaew).

% License: Creative Commons Attribution-NonCommercial-NoDerivatives 4.0 International (CC BY-NC-ND 4.0)
% https://creativecommons.org/licenses/by-nc-nd/4.0/

\chapter{โมเดลการเรียนรู้ของเครื่องสำหรับเคมีควอนตัม}
\label{ch:chem_ml}

%--------------------------
\section{SchNet และ SchNetOrb}
%--------------------------

%--------------------------
\section{GDML และ sGDML}
%--------------------------

%--------------------------
\section{$\Delta$ML}
%--------------------------

Delta-ML ($\Delta$ML) เป็นเทคนิคที่ใช้ค่าความแตกต่างระหว่างค่าอ้างอิง (Reference หรือจะเรียก Label ก็ได้) จากวิธีการคำนวณที่มีความ%
แม่นยำต่ำกับความแม่นยำสูงมาใช้ในการเทรนโมเดล (จึงเป็นที่มาว่าทำไมถึงเรียกว่า Detla) ซึ่งการทำแบบนี้จะช่วยให้โมเดลสามารถเรียนรู้การเชื่อมโยง 
(Transferability) ไปยังค่าที่ต้องการทำนายได้อย่างถูกต้องและแม่นยำมากขึ้น โดยจะมีความถูกต้องเทียบเคียงกับการใช้วิธีแบบดั้งเดิมที่มีความแม่นยำสูง 
(เช่น Post-HF) ตัวอย่างของการใช้ $\Delta$ML คือการใช้ค่าความแตกต่างของพลังงานที่ได้จากการคำนวณด้วยวิธี DFT และ CCSD(T) 
($y_{DFT} - y_{CCSD(T)}$) มาฝึกสอนโมเดล

จริง ๆ แล้ว $\Delta$ML ก็เป็นเทคนิคอันนึงที่มีแนวคิดมาจากความพยายามที่ต้องการจะทำให้โมเดลสามารถเรียนรู้ได้จากค่าความผิดพลาด (Error) 
โดยเริ่มมีการเอามาใช้กันมากขึ้นในช่วงปีที่ผ่านมา (ในช่วงแรกถูกใช้เยอะแค่ในเฉพาะกลุ่มวิจัยในโซนยุโรป สำหรับการเอามาทำนายพลังงานและ%
เกรเดียนต์ของพลังงาน (Energy Gradient) ซึ่งก็สอดคล้องกับแรงของแต่ละอะตอมในโมเลกุลโมเลกุลนั่นเอง

%--------------------------
\section{Graph Neural Network}
%--------------------------

Graph Neural Network หรือโครงข่ายประสาทแบบกราฟ เป็น NN รูปแบบหนึ่งซึ่งใช้การมองโครงสร้างข้อมูลให้อยู่ในรูปแบบของกราฟ
โดยไอเดียนี้ได้ถูกเสนอตั้งแต่ปี ค.ศ. 2008\cite{scarselli2009,zhou2020}

%--------------------------
\subsection{Message Passing Neural Network}
%--------------------------

Message Passing Neural Network หรือ MPNN ถูกนำเสนอครั้งแรกเมื่อปี ค.ศ. 2018\cite{gilmer2017}

%--------------------------
\section{Molecule Attention Transformer}
%--------------------------
 % โมเดลการเรียนรู้ของเครื่องสำหรับเคมีควอนตัม
% LaTeX source for ``การเรียนรู้ของเครื่องสำหรับเคมีควอนตัม (Machine Learning for Quantum Chemistry)''
% Copyright (c) 2022 รังสิมันต์ เกษแก้ว (Rangsiman Ketkaew).

% License: Creative Commons Attribution-NonCommercial-NoDerivatives 4.0 International (CC BY-NC-ND 4.0)
% https://creativecommons.org/licenses/by-nc-nd/4.0/

\chapter{ซอฟต์แวร์ฝึกสอนปัญญาประดิษฐ์สำหรับเคมีควอนตัม}
\label{ch:ml_lib}

%--------------------------
\section{ไลบรารี่สำหรับคำนวณ Representation}
%--------------------------

%--------------------------
\subsection{DScribe}
%--------------------------

DScribe เป็นไลบรารี่ที่สามารถคำนวณ Representation ได้เยอะมาก\cite{himanen2020}

https://singroup.github.io/dscribe/latest/index.html

%--------------------------
\section{ไลบรารี่สำหรับสร้าง Model}
%--------------------------

%--------------------------
\subsection{SchNetPack}
%--------------------------

SchNetPack\cite{schutt2019}

%--------------------------
\subsection{sGDML}
%--------------------------

เราสามารถติดตั้ง sGDML โดยใช้ Python Package Manager เช่น PIP ได้ด้วยคำสั่งต่อไปนี้

\begin{lstlisting}[style=MyBash]
pip install sgdml
\end{lstlisting}

ตรวจสอบว่า sGDML ถูกติดตั้งและพร้อมใช้งานหรือไม่

\begin{lstlisting}[style=MyBash]
(sgdml) nutt@SURFACE:~/sGDML$ which sgdml
/home/nutt/sgdml/bin/sgdml

(sgdml) nutt@SURFACE:~$ sgdml
usage: sgdml [-h] [--version] {all,create,train,validate,select,test,show,reset} ...
sgdml: error: the following arguments are required: command
\end{lstlisting}

คู่มือการใช้งานอ่านได้ที่ \url{http://quantum-machine.org/gdml/doc/}

sGDML\cite{chmiela2019}

%--------------------------
\subsection{PiNN}
%--------------------------

PiNN\cite{shao2020}

%--------------------------
\subsection{TorchANI}
%--------------------------

TorchANI\cite{gao2020}
 % ไลบรารี่การเรียนรู้ของเครื่องสำหรับเคมีควอนตัม

\appendix
\begin{appendices}
\renewcommand{\thesection}{\arabic{section}} % remove chapter number
\fancyhead[RE,LO]{\textbf{ภาคผนวก}}
% LaTeX source for ``การเรียนรู้ของเครื่องสำหรับเคมีควอนตัม (Machine Learning for Quantum Chemistry)''
% Copyright (c) 2022 รังสิมันต์ เกษแก้ว (Rangsiman Ketkaew).

% License: Creative Commons Attribution-NonCommercial-NoDerivatives 4.0 International (CC BY-NC-ND 4.0)
% https://creativecommons.org/licenses/by-nc-nd/4.0/

\chapter{คณิตศาสตร์สำหรับการเรียนรู้ของเครื่อง}
\label{ap:basic_math}

%--------------------------
\section{พีชคณิตเชิงเส้น}
%--------------------------

%--------------------------
\subsection{สเกลาร์, เวกเตอร์, และเมทริกซ์}
%--------------------------

ปริมาณทางกายภาพสามารถแบ่งออกเป็น 3 ปริมาณ ได้แก่ สเกลาร์ (Scalar), เวกเตอร์ (Vector), และเมทริกซ์ (Matrix)
โดยปริมาณแต่ละตัวที่ใช้ในคณิตศาสตร์มีความหมายง่าย ๆ ดังนี้

\begin{description}
    \item[สเกลาร์] คือปริมาณที่มีเพียงขนาดอย่างเดียว โดยสามารถบอกแต่ขนาดอย่างเดียวก็ได้ความหมายสมบูรณ์ ไม่ต้องบอกทิศทาง กล่าวคือเป็นแค่เพียงตัวเลขเดี่ยว ๆ เท่านั้น
    \item[เวกเตอร์] คือปริมาณที่ใช้ดำเนินการบนปริภูมิเวกเตอร์ (Vector Space) ซึ่งจะมีความหมายค่อนข้างกว้าง แต่จะมีนิยามคล้าย ๆ กับเวกเตอร์ในทางฟิสิกส์
    กล่าวคือเวกเตอร์จะมีทั้งขนาดและองค์ประกอบบ่งบอกทิศทาง ในส่วนของการเขียนโปรแกรมนั้นเวกเตอร์คือ Array ขนาด 1 มิติ
    \item[เมทริกซ์] คือปริมาณที่เกิดจากเวกเตอร์มากกว่า 1 เวกเตอร์มารวมกัน โดยจะมีองค์ประกอบเป็นจำนวนแถวและจำนวนหลัก 
    ถ้าหากเมทริกซ์มีเพียงแค่แถวเดียว หรือหลักเดียว เราจะกล่าวได้ว่านั่นคือเวกเตอร์นั่นเอง ในส่วนของการเขียนโปรแกรมนั้นเมทริกซ์คือ Array ขนาด 2 มิติ
\end{description}

%--------------------------
\subsection{การดำเนินการของเมทริกซ์}
%--------------------------

\paragraph{Transpose} ถ้าหากเรามีรูปแบบซึ่งจริง ๆ ก็คือเมทริกซ์ขนาด 2 มิติแล้วเราทำการคูณด้วยเมทริกซ์การหมุน (Rotation Matrix) 
สิ่งที่เราจะได้คือเราจะได้รูปภาพที่ถูกหมุนไป โดยการที่เรากระทำการหมุนเมทริกซ์นั้นเราเรียกว่า Transpose 
อธิบายง่าย ๆ คือการ Tranpose เมทริกซ์นั้นก็คือการสลับแถวกับหลักของเมทริกซ์ หรือทำการหมุนสมาชิกที่ไม่ใช่แถวทแยง (Off-diagonal) รอบ ๆ แนวทแยงนั่นเอง

\paragraph{การบวกและการลบ} เมทริกซ์สองเมทริกซ์ที่มีขนาดเท่ากัน (จำนวนแถวและจำนวนหลักเท่ากัน) สามารถบวกและลบกันได้ 
โดยให้ทำการบวกหรือลบสมาชิกที่มีดัชนีตรงกันได้โดยตรงเลย

\paragraph{การคูณด้วยสเกลาร์} การคูณเมทริกซ์ด้วยปริมาณสเกลาร์สามารถทำได้ง่าย ๆ โดยให้คูณสมาชิกทุกตัวของเมทริกซ์ด้วยตัวเลขตัวนั้น

\paragraph{การคูณเมทริกซ์ด้วยเมทริกซ์}
สมมติว่าเรามีเมทริกซ์ A กับเมทริกซ์ B การที่เมทริกซ์สองตัวนี้จะคูณกันได้นั้นจะต้องไม่ขัดกับเงื่อนไขดังต่อไปนี้
\enquote{สมาชิกของผลคูณของเมทริกซ์ในแถวที่ i หลักที่ j จะเกิดสมาชิกในแถวที่ i ของเมทริกซ์ที่อยู่หน้า คูณกับสมาชิกในหลักที่ j ของเมทริกซ์หลักเป็นคู่ ๆ แล้วนำมาบวกกัน}

%--------------------------
\subsection{ประเภทของเมทริกซ์}
%--------------------------

เมทริกซ์แบบพิเศษมีด้วยกันหลากหลายแบบด้วยกัน โดยเมทริกซ์แบบพิเศษที่มักเจอมีดังต่อไปนี้

\paragraph{Identity} เมทริกซ์ที่สมาชิกในแนวทแยงมีค่ากับเท่า 1 ทุกตัวและสมาชิกที่เหลือเป็น 0 ทั้งหมด

%--------------------------
\subsection{เทนเซอร์}
%--------------------------

บางครั้งเราจำเป็นที่จะต้องจัดการข้อมูลที่มีจำนวนของมิติที่มากกว่า 2 มิติ นั่นคือเราไม่สามารถใช้เวกเตอร์หรือเมทริกซ์ได้อีกต่อไป
โดยเราจะต้องใช้เทนเซอร์ (Tensor) แทน เพราะว่าเทนเซอร์คือ Array ที่มีจำนวนมิติ $n$ มิติ สรุปง่าย ๆ คือเวกเตอร์นั้นคือเทนเซอร์ 1 มิติ 
และเมทริกซ์คือเทนเซอร์ 2 มิติ แล้วถ้าเป็น 3 มิติล่ะ เราจะเรียกว่าเป็นคิวป์ (Cube) และเทนเซอร์ 4 มิติ เราก็จะเรียกว่าเป็นเวกเตอร์ของคิวป์ 
และ 5 มิติก็จะเป็นเมทริกซ์ของคิวป์นั่นเอง
 % พีชคณิตเชิงเส้น
% LaTeX source for ``การเรียนรู้ของเครื่องสำหรับเคมีควอนตัม (Machine Learning for Quantum Chemistry)''
% Copyright (c) 2022 รังสิมันต์ เกษแก้ว (Rangsiman Ketkaew).

% License: Creative Commons Attribution-NonCommercial-NoDerivatives 4.0 International (CC BY-NC-ND 4.0)
% https://creativecommons.org/licenses/by-nc-nd/4.0/

%--------------------------
\section{การเขียนโปรแกรมสำหรับการเรียนรู้ของเครื่อง}
%--------------------------
\label{ap:programming}

%--------------------------
\subsection{ชุดโปรแกรมสำหรับการเรียนรู้ของเครื่อง}
%--------------------------

%--------------------------
\subsection{ชุดโปรแกรมสำหรับการเรียนรู้เชิงลึก}
%--------------------------
 % ไลบรารี่การเรียนรู้ของเครื่อง
% LaTeX source for ``การเรียนรู้ของเครื่องสำหรับเคมีควอนตัม (Machine Learning for Quantum Chemistry)''
% Copyright (c) 2022 รังสิมันต์ เกษแก้ว (Rangsiman Ketkaew).

% License: Creative Commons Attribution-NonCommercial-NoDerivatives 4.0 International (CC BY-NC-ND 4.0)
% https://creativecommons.org/licenses/by-nc-nd/4.0/

%--------------------------
\section{เทคนิคการเขียนโมเดล TensorFlow}
\label{ap:coding_tf}
%--------------------------

%--------------------------
\subsection{การปรับแต่ง Loss Function}
%--------------------------

\begin{lstlisting}[style=MyPython]
import tensorflow as tf
import tensorflow.keras.backend as kb
import numpy as np

def custom_loss(y_actual, y_pred): 
    custom_loss=tf.experimental.numpy.log10(kb.sum(kb.abs(y_actual - y_pred)) / y_actual.shape[0])
    return custom_loss

x = np.random.randint(1, 4, size=(1000,))
x = np.asarray(x).T

y = x ** 2
y = np.asarray(y).T
x = x.astype(np.float32)
y = y.astype(np.float32)

keras_model = tf.keras.Sequential(
    [
        tf.keras.layers.Dense(32, activation=tf.nn.relu, input_shape=[1]),
        tf.keras.layers.Dense(32, activation=tf.nn.relu),
        tf.keras.layers.Dense(1),
    ]
)

optimizer = tf.keras.optimizers.RMSprop(0.001)
keras_model.compile(loss=custom_loss, optimizer=optimizer)
keras_model.fit(x, y, batch_size=20, epochs=50)
\end{lstlisting}
 % เทคนิคการเขียนโมเดล TensorFlow
% LaTeX source for ``การเรียนรู้ของเครื่องสำหรับเคมีควอนตัม (Machine Learning for Quantum Chemistry)''
% Copyright (c) 2022 รังสิมันต์ เกษแก้ว (Rangsiman Ketkaew).

% License: Creative Commons Attribution-NonCommercial-NoDerivatives 4.0 International (CC BY-NC-ND 4.0)
% https://creativecommons.org/licenses/by-nc-nd/4.0/

%--------------------------
\section{โปรแกรมทางด้านเคมีควอนตัม}
\label{ap:qm_software}
%--------------------------

%--------------------------
\subsection{Gaussian}
\label{ssec:software_gaussian}
%--------------------------

โปรแกรม Gaussian เป็นโปรแกรมที่เรียกได้ว่าเป็นตำนานของโปรแกรมทางด้านเคมีควอนตัม นั่นก็เพราะว่า Gaussian ได้ถูกพัฒนามาอย่างยาวนาน 
ซึ่งถือว่าเป็นโปรแกรมแรกของงานวิจัยสายนี้เลยก็ว่าได้ โดย Gaussian ได้ถูกพัฒนาขึ้นในกลุ่มวิจัยของศาสตราจารย์ John A. Pople ในช่วงปี ค.ศ.
1970 และมีการพัฒนาต่อเรื่อยมาจนถึงปัจจุบัน โดยเวอร์ชั่นล่าสุดของ Gaussian (ณ วันที่ผู้เขียนเขียนหนังสือเล่มนี้) คือเวอร์ชั่น 16\autocite{g16}

คุณสมบัติหรือ Feature ของโปรแกรม Gaussian นั้นคือสามารถคำนวณคุณสมบัติเชิงอิเล็กทรอนิกส์ของโมเลกุลขนาดเล็ก (ไม่เกิน 50 อะตอม) 
ขนาดกลาง (50 - 120 อะตอม) และขนาดใหญ่ (มากกว่า 120 อะตอม) ได้อย่างแม่นยำ\footnote{ความแม่นยำและความถูกต้องของผลการคำนวณ%
อ้างอิงตามประสบการณ์ของผู้เขียน โดยมีปัจจัยที่ส่งผลต่อค่าความถูกต้อง เช่น วิธีที่ใช้ในการคำนวณและ Basis Set} โดยจุดเด่นของ Gaussian 
ก็คือการคำนวณคุณสมบัติเชิงอิเล็กทรอนิกส์ของโมเลกุลด้วยวิธี DFT และด้วยอัลกอริทึมของตัวโปรแกรมนั้น ทำให้ Gaussian ได้รับการยอมรับว่าเป็น%
หนึ่งในโปรแกรมที่ให้ผลการคำนวณที่ถูกต้องและน่าเชื่อถือ และสามารถนำไปเปรียบเทียบกับผลการทดลองได้ สำหรับ Gaussian นั้นรองรับการคำนวณ%
แบบวิธี OpenMP นั่นคือสามารถทำการประมวลผลแบบขนาดได้โดยใช้หน่วยประมวลผล CPU หลายตัวพร้อม ๆ กันได้ และนอกจากนี้แล้วในเวอร์ชั่น 16
ตัวโปรแกรมยังรองรับกราฟฟิคการ์ด GPU สำหรับการคำนวณโดยใช้วิธี DFT อีกด้วย 

รายละเอียดเพิ่มเติมเกี่ยวกับโปรแกรม Gaussian ดูได้ที่เว็บไซต์ \url{https://gaussian.com/}

%--------------------------
\subsection{ORCA}
\label{ssec:orca}
%--------------------------

โปรแกรม ORCA เป็นอีกหนึ่งโปรแกรมทางเคมีควอนตัมที่มีประสิทธิภาพและความสามารถในการคำนวณสูง\autocite{neese2012,neese2018} 
สามารถคำนวณได้หลายวิธี โดยสามารถคำนวณ DFT และวิธี Semi-empirical ได้ รวมไปถึงวิธี Post Hartree-Fock อื่น ๆ ด้วย โดย ORCA 
ถูกใช้อย่างแพร่หลายในงานวิจัยทางด้านเคมีอินทรีย์และเคมีอนินทรีย์ โดยเฉพาะการศึกษาสารประกอบเชิงซ้อนของโลหะทรานซิชัน (Transition 
Metal Complex) ซึ่งเป็นโมเลกุลที่มีขนาดใหญ่และมีความซับซ้อนในเชิงของโครงสร้างอิเล็กทรอนิกส์มากกว่าโมเลกุลอินทรีย์ขนาดเล็ก และ ORCA 
ยังมีความโดดเด่นในด้านของความแม่นยำและความเร็วในการคำนวณเกี่ยวกับคุณสมบัติเชิงสเปกตรัมของโมเลกุล

โปรแกรม ORCA ถูกพัฒนาในกลุ่มวิจัยของศาสตราจารย์ Frank Neese โดยสามารถดาวน์โหลดตัวโปรแกรม (เฉพาะไฟล์ Binary ที่ถูกคอมไพล์แล้ว)
มาใช้ได้ฟรีสำหรับวัตถุประสงค์ด้านการศึกษาและการทำงานวิจัย 

รายละเอียดเพิ่มเติมเกี่ยวกับโปรแกรม ORCA ดูได้ที่เว็บไซต์ \url{https://orcaforum.kofo.mpg.de/app.php/portal}

%--------------------------
\subsection{NWChem}
\label{ssec:nwchem}
%--------------------------

โปรแกรม NWChem เป็นโปรแกรมการคำนวณทางเคมีควอนตัมและพลศาสตร์เชิงโมเลกุล (Molecular Dynamics)\autocite{apra2020} 
พัฒนาโดยสถาบัน Pacific Northwest National Laboratory (PNNL) ในช่วงปี ค.ศ. 1990 โดยรองรับการคำนวณด้วยวิธี DFT และ 
Post Hartree-Fock เช่น M$\o$llor-Plesset (MP), Configuration Interaction (CI), Coupled Cluster (CC) และ
Multiconfiguration SCF (MCSCF) 

NWChem นั้นถูกพัฒนาเพื่อให้สามารถประมวลผลบน Supercomputer ที่มีประสิทธิภาพสูงได้ NWChem ถูกเขียนขึ้นโดยใช้ภาษา Fortran 77
และใช้ไลบรารี่ทางด้านพีชคณิตสำหรับการประมวลผล เช่น BLAS, LAPACK, และ ScaLAPACK และสามารถประมวลผลแบบขนาดได้โดยใช้วิธี 
Message-Passing Interface (MPI) นอกจากนี้ NWChem ยังรองรับการประมวลผลด้วย GPU สำหรับการคำนวณด้วยวิธี Coupled Cluster 
ซึ่งถือว่าเป็นจุดเด่นของ NWChem เลยก็ว่าได้ โปรแกรม NWChem เป็นแบบ Open-source มีการพัฒนาอย่างต่อเนื่องเรื่อยมาจนถึงปัจจุบัน ซึ่งนักวิจัย 
นักศึกษา และคนทั่วไปสามารถร่วมพัฒนาและใช้งานโปรแกรมได้ฟรี

รายละเอียดเพิ่มเติมเกี่ยวกับโปรแกรม NWChem ดูได้ที่เว็บไซต์ \url{https://nwchemgit.github.io/}
 % โปรแกรมทางด้านเคมีควอนตัม
\end{appendices}

\backmatter
% LaTeX source for ``การเรียนรู้ของเครื่องสำหรับเคมีควอนตัม (Machine Learning for Quantum Chemistry)''
% Copyright (c) 2022 รังสิมันต์ เกษแก้ว (Rangsiman Ketkaew).

% License: Creative Commons Attribution-NonCommercial-NoDerivatives 4.0 International (CC BY-NC-ND 4.0)
% https://creativecommons.org/licenses/by-nc-nd/4.0/

{
\begin{@empty}
    \chapter*{บรรณานุกรม}
    \fancyhead[RE,LO]{\textbf{บรรณานุกรม}}
    \addcontentsline{toc}{chapter}{บรรณานุกรม}
    \printbibliography[heading=none]
\end{@empty}
}

% LaTeX source for ``การเรียนรู้ของเครื่องสำหรับเคมีควอนตัม (Machine Learning for Quantum Chemistry)''
% Copyright (c) 2022 รังสิมันต์ เกษแก้ว (Rangsiman Ketkaew).

% License: Creative Commons Attribution-NonCommercial-NoDerivatives 4.0 International (CC BY-NC-ND 4.0)
% https://creativecommons.org/licenses/by-nc-nd/4.0/

{
\printindex[th]
\addcontentsline{toc}{chapter}{ดรรชนีภาษาไทย}
\printindex[en]
\addcontentsline{toc}{chapter}{ดรรชนีภาษาอังกฤษ}
}

% LaTeX source for ``การเรียนรู้ของเครื่องสำหรับเคมีควอนตัม (Machine Learning for Quantum Chemistry)''
% Copyright (c) 2022 รังสิมันต์ เกษแก้ว (Rangsiman Ketkaew).

% License: Creative Commons Attribution-NonCommercial-NoDerivatives 4.0 International (CC BY-NC-ND 4.0)
% https://creativecommons.org/licenses/by-nc-nd/4.0/

{
\thispagestyle{empty}

\begin{center}
    ประวัติผู้แต่ง
\end{center}

รังสิมันต์ เกษแก้ว สำเร็จการศึกษาปริญญาตรี (พ.ศ. 2559) และปริญญาโท (พ.ศ. 2562) สาขาเคมี จากภาควิชาเคมี 
คณะวิทยาศาสตร์และเทคโนโลยี มหาวิทยาลัยธรรมศาสตร์ หลังจากนั้นทำงานเป็น Consultant ให้กับบริษัท New Equilibrium Biosciences 
(พ.ศ. 2562 - 2563) ปัจจุบันกำลังศึกษาปริญญาเอกสาขาเคมีทฤษฎีและเคมีคำนวณที่ภาควิชาเคมี มหาวิทยาลัยแห่งซูริค ประเทศสวิตเซอร์แลนด์

\noindent ผลงานอื่น ๆ สามารถดูได้ที่ \url{https://rangsimanketkaew.github.io}

\vfill
}

% Book back cover
\includepdf[pages=-]{cover_back.pdf}

\end{document}
