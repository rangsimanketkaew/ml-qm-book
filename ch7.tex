% LaTeX source for ``การเรียนรู้ของเครื่องสำหรับเคมีควอนตัม (Machine Learning for Quantum Chemistry)''
% Copyright (c) 2022 รังสิมันต์ เกษแก้ว (Rangsiman Ketkaew).

% License: Creative Commons Attribution-NonCommercial-NoDerivatives 4.0 International (CC BY-NC-ND 4.0)
% https://creativecommons.org/licenses/by-nc-nd/4.0/

\chapter{คุณสมบัติเชิงอิเล็กทรอนิกส์ของโมเลกุล}
\label{ch:el_prop}

เนื้อหาในบทนี้จะเกี่ยวกับคุณสมบัติเชิงอิเล็กทรอนิกส์ (Electronic Properties) ของโมเลกุล ซึ่งเป็นวัตถุประสงค์ของการใช้ ML เข้ามาศึกษาเคมีควอนตัม
โมเลกุลเป็นหน่วยพื้นฐานของสิ่งต่าง ๆ รอบตัวเรา ซึ่งโมเลกุลก็คือเป็นกลุ่มของอะตอมหลาย ๆ อะตอมมารวมกัน และในอะตอมนั้นเราสนใจอิเล็กตรอนเป็นพิเศษ
ในวิชาควอนตัมนั้นเราจะอธิบายพฤติกรรมของโมเลกุลโดยมุ่งเน้นไปที่อิเล็กตรอนซึ่งสามารถที่จะอธิบายได้ด้วยฟังก์ชันทางคณิตศาสตร์ที่เรียกว่าฟังก์ชันคลื่น (Wavefunction)
โดยหน้าตาของ Wavefunction นั้นจริง ๆ แล้วไม่มีการนิยามแบบตายตัว โดยหนึ่งในสมการที่โด่งดังที่สุดสมการหนึ่งของวงการวิทยาศาสตร์นั่นคือสมการชโรดิงเงอร์ 
(Schrödinger Equation) โดยถูกนำมาใช้ในการศึกษาระบบทางกลศาสตร์ควอนตัม ซึ่งการแก้สมการชโรดิงเงอร์ได้นั้นจะทำให้ได้มาซึ่ง Wavefunction 

สมการชโรดิงเงอร์สามารถแบ่งออกได้เป็นสองแบบ คือ แบบที่ไม่ขึ้นกับเวลาและแบบที่ขึ้นกับเวลา ดังนี้

\noindent Time-dependent Schrödinger Equation

\begin{equation}
    i \hbar \frac{d}{d t} \ket{\Psi(t)} = \hat{H} \ket{\Psi(t)}
\end{equation}

\noindent Time-independent Schrödinger Equation

\begin{equation}
    \hat{H}\ket{\Psi} = E \ket{\Psi}
\end{equation}

โดย Wavefunction ($\Psi(t)$) ที่เป็น Eigenfunction นั้นจะบรรจุข้อมูลเชิงอิเล็กทรอนิกส์ทุกอย่างเกี่ยวกับระบบของเราเอาไว้ ซึ่งในที่นี้ก็คือโมเลกุล
โดยสมการข้างต้นเป็นการคำนวณหาพลังงานของระบบโดยใช้ Hamiltonian Operator ($\hat{H}$) ซึ่งเป็น Operator ที่สอดคล้องกับพลังงาน
ซึ่งจริง ๆ แล้ว Eigenvalue ของสมการข้างต้นจะเป็นคุณสมบัติของโมเลกุลอะไรก็ได้ ตราบใดที่เราใช้ Operator ที่สอดคล้องกับคุณสมบัตินั้น ๆ 

หนึ่งในเป้าหมายสำคัญของกลศาสตร์ควอนตัมเชิงโมเลกุลก็คือการแก้สมการ Time-independent Schrödinger Equation และคำนวณหาโครงสร้างเชิงอิเล็กทรอนิกส์ 
(Electronic Structures) ของอะตอมและโมเลกุล โดยหัวข้อแรกของบทนี้ที่เราจะมาดูกันแบบละเอียดก็คือการใช้เทคนิคเชิงคำนวณและอาศัยการประมาณค่าในการแก้สมการดังกล่าว
โดยทั่วไปนั้นจะมีวิธีการหลัก ๆ 2 วิธีที่สามารถช่วยให้เราหาคำตอบของสมการชโรดิงเงอร์ ได้นั่นคือ \textbf{ab initio method} 
ซึ่งเป็นวิธีที่ความถูกแม่นยำของผลลัพธ์ที่ได้จากการแก้สมการนั้นจะขึ้นอยู่กับ Model ที่เรานำมาใช้ในการอธิบาย Wavefunction ของโมเลกุล 
และเป็นที่ทราบกันดีว่าสำหรับโมเลกุลที่มีขนาดใหญ่นั้น วิธีการ ab initio นี้จะมีความสิ้นเปลืองสูงมาก ดังนั้นจึงเป็นที่มาของวิธีการที่สอง นั่นคือ 
\textbf{Semiempirical method} ซึ่งจะใช้เทคนิคการมอง Hamiltonian ในรูปแบบที่ง่ายกว่า และอาศัยค่า Parameter ที่ได้จากการทดลองเพื่อเพิ่มความแม่นยำ 
อย่างไรก็ตาม วิธี Density Functional Theory (DFT) ก็ถูกพัฒนาขึ้นมาเพื่อแก้ปัญหาที่เราจะต้องมาแก้หรือประมาณค่า Wavefunction ตรง ๆ 
จึงทำให้ความสิ้นเปลืองของการคำนวณนั้นต่ำมากเมื่อเทียบกับสองวิธีข้างต้นทีได้กล่าวไป

%--------------------------
\section{การแก้สมการฟังก์ชันคลื่นเพื่อคำนวณพลังงาน}
%--------------------------

%--------------------------
\subsection{วิธี Self-Consistent Field}
%--------------------------

ในหัวข้อนี้เราจะมาพูดถึงการแก้สมการชโรดิงเงอร์โดยใช้วิธีที่ชื่อว่า Self-Consistent Field (SCF) ซึ่งเป็นการประมาณค่า Hamiltonian แบบวนซ้ำ
เริ่มต้นเราจะต้องมาดูกันก่อนว่าการมอง Wavefunction ของระบบหลายอิเล็กตรอนสำหรับวิธี SCF นั้นจะมีการตัดสิ่งที่ซับซ้อนออกไปนั่นก็คือ
Electron-Electron Repulsion หรืออันตรกิริยาระหว่างอิเล็กตรอน โดย Wavefunction สามารถถูกอธิบายได้ด้วยสมการต่อไปนี้

\begin{equation}
    H^{\circ} \Psi^{\circ} = E^{\circ} \Psi^{\circ}
\end{equation}

โดยกำหนดให้ $H^{\circ} = \sum^{N}_{i=1} h_{i}$ เมื่อ $h$ คือ Hamiltonian สำหรับหนึ่งอิเล็กตรอน (อิเล็กตรอนตัวที่ $i$) ในระบบที่มีอิเล็กตรอน $N$ ตัว 
นั่นคือสมการสำหรับระบบที่มีอิเล็กตรอน $N$ ตัวนั้น จะสามารถถูกแยกออกมาได้เป็นสมการของระบบหนึ่งอิเล็กตรอนได้ $N$ สมการ 
และ Wavefunction ของอิเล็กตรอนหนึ่งตัวนั้นจริง ๆ แล้วก็คือออบิทัล (Orbital) เราจึงสามารถเขียนสมการของอิเล็กตรอนหนึ่งตัวได้เป็น

\begin{equation}
    h_{i} \Psi^{\circ}(i) = E^{\circ}_{m} \Psi^{\circ}(i)
\end{equation}

โดยที่ $E^{\circ}_{m}$ คือพลังงานของอิเล็กตรอนหนึ่งตัวใน Molecular Orbital (MO) ซึ่งเขียนแทนด้วย $m$ นั่นเอง สำหรับระบบที่อิเล็กตรอนไม่ขึ้นต่อกันและกัน

ด้วยเหตุนี้ Wavefunction รวมของระบบ ($\Psi^{\circ}$) จึงสามารถเขียนให้อยู่ในรูปของ Wavefunction ของอิเล็กตรอนหนึ่งตัวได้ดังนี้

\begin{equation}
    \Psi^{\circ} = \psi^{\circ}_{a}(1) \psi^{\circ}_{b}(1) \dots \psi^{\circ}_{z}(N)
\end{equation}

ซึ่ง Wavefunction ด้านบนนี้จะขึ้นอยู่กับพิกัดของอิเล็กตรอนทุกตัวและขึ้นกับตำแหน่งของนิวเคลียสหรืออะตอมด้วย
\footnote{ตอนนี้เราจะยังไม่พิจารณาสปินของอิเล็กตรอนที่จะต้องสอดคล้องและไม่ขัดกับหลักกีดกันของเพาลีหรือ Pauli Exclusion 
ซึ่งจะมีการรวม Spinorbital สำหรับ Molecular Orbital $m$ ($\varphi_{m}$) เข้าไปด้วย}

สำหรับกระบวนการหรือขั้นตอนที่เราจะนำมาใช้ในการแก้สมการของระบบอิเล็กตรอนหลายตัวนั้น เราจะพิจารณาสมการรูทฮาน (Roothaan Equation) เป็นหลัก
ซึ่งเป็นวิธีหนึ่งในการแก้สมการ Hartree-Fock (HF) ซึ่งมีการกำหนดตัวดำเนินการใหม่ขึ้นมาใช้แทน Hamiltonian นั่นก็คือ Fock Operator 
โดยที่ Fock Operator ถูกนิยามในเทอมของ Coulomb Operator และ Exchange Operator ขึ้นมา นั่นก็คือ Fock Operator 
ซึ่งเขียนสมการสำหรับอิเล็กตรอน 1 ตัวได้เป็น

\begin{equation}
    \label{eq:fock}
    f_{1} \psi_{m}(1) = \epsilon_{n} \psi_{m}(1)
\end{equation}

%--------------------------
\subsection{สมการ Roothaan}
%--------------------------

สำหรับการแก้สมการ HF ตรง ๆ โดยใช้ SCF นั้นสามารถทำได้ตรง ๆ แต่ว่าผลเฉลยที่ที่มาจากการประมาณ (Numerical Solution) นั้นมีความซับซ้อนมาก
ดังนั้นนักฟิสิกส์และนักเคมีชาวดัตช์ที่ชื่อว่า Clemens C.J. Roothaan ได้เสนอวิธีการใหม่สำหรับการอธิบาย MO เรียกว่า Linear Combination of Atomic Orbitals (LCAO)
เรามาดูกันเลยว่าวิธีการนี้มีรายละเอียดอย่างไร

เริ่มต้นเราจะนิยามฟังก์ชันพื้นฐาน (Basis Function) สำหรับระบบที่มีอิเล็กตรอน $N$ ตัวขึ้นมาก่อน ซึ่งเขียนแทนด้วย $\chi_{o}$
ซึ่งไอเดียตอนนี้ก็คือเราจะมองว่า Basis Function แบบที่ง่ายที่สุดที่เราสามาถนำมาใช้ได้นั่นก็คือ Atomic Orbital (O) ซึ่งสามารถที่จะเขียน Spatial Wavefunction 
(ฟังก์ชันคลื่นที่ขึ้นกับตำแหน่ง ของ AO) ให้อยู่ในผลรวมเชิงเส้นของการคูณระหว่างสัมประสิทธิ์เชิงเส้นที่เรายังไม่ทราบค่า (Unknown Coefficients, $c_{om}$) 
กับ $\chi_{o}$ ดังนี้

\begin{equation}
    \label{eq:lcao}
    \psi_{m} = \sum^{N_{o}}_{o=1} c_{om} \chi_{o} 
\end{equation}

เมื่อเราแทนสมการ \ref{eq:lcao} เข้าไปในสมการ \ref{eq:fock} เราจะได้

\begin{equation}
    \label{eq:lcao_in_fock}
    f_{1} \sum^{N_{o}}_{o=1} \chi_{o}(1) = \epsilon \sum^{N_{o}}_{o=1} c_{om} \chi_{o}(1)
\end{equation}

แล้วทำการคูณสมการ \ref{eq:lcao_in_fock} ทั้งสองข้างด้วย $\chi^{*}_{o}(1)$ และทำการอินทิเกรตทั่วทั้ง Space ซึ่งจะทำให้เราได้ความสัมพันธ์ต่อไปนี้

\begin{equation}
    \label{eq:lcao_in_fock_int}
    \sum^{N_{o}}_{o=1} c_{om} \int \chi^{*}_{o}(1) f_{1} \chi_{o}(1) d\tau_{1} =
    \epsilon_{m} \sum^{N_{o}}_{o=1} c_{om} \int \chi^{*}_{o}(1) \chi_{o}(1) d\tau_{1}
\end{equation}

จากสมการข้างต้นเราจะพบว่าจะมีผลคูณของ Basis Function ทั้งสองฝั่ง โดยทางฝั่งซ้ายนั้นเราสามารถนิยาม Fock Matrix (F) ได้

\begin{equation}
    F_{o'o} = \int \chi^{*}_{o'}(1) f_{1} \chi_{o}(1) d\tau_{1}
\end{equation}

และทางฝั่งขวา เรานิยามสิ่งที่เรียกว่า Overlap Matrix (S) ซึ่งเป็น Matrix ทีอ่ธิบายถึงการซ้อนทับกันระหว่าง State 2 อัน

\begin{equation}
    S_{o'o} = \int \chi^{*}_{o'}(1) \chi_{o}(1) d\tau_{1}
\end{equation}

ถึงเราสามารถเขียนสมการ \ref{eq:lcao_in_fock_int} ให้อยู่ในรูปของสมการที่เรียกว่า Roothaan Equation ได้กระชับ ๆ ดังนี้

\begin{equation}
    \label{eq:roothaan}
    F c = \epsilon S c
\end{equation}

โดยที่ $c$ คือเมทริกซ์ขนาด $N_{o} \times N_{o}$ ซึ่งประกอบไปด้วยสมาชิกของ Coefficient $c_{om}$ และ $\epsilon$ คือเมทริกซ์ที่มีขนาด
$N_{o} \times N_{o}$ เช่นเดียวกันซึ่งเป็นเมทริกซ์แบบ Diagonal Matrix (สมาชิกที่ไม่ใช่แนวทแยงมีค่าเป็น 0 ทั้งหมด) ซึ่งก็คือพลังงานของ Orbital นั่นเอง
ซึ่งตรงจุดนี้เราต้องไม่ลืมว่า Fock Operator ($f_{1}$) นั้นถูกกำหนดให้อยู่ในรูปของ Integral บน MO และขึ้นอยู่กับค่าของ Coefficient $c_{om}$ ด้วย

สำหรับการแก้สมการ \ref{eq:roothaan} นั้นสามารถทำได้ผ่าน Determinant ดังนี้

\begin{equation}
    det|F - \epsilon S| = 0
\end{equation}

ซึ่งสมการด้านบนไม่สามารถแก้ได้แบบตรงไปตรงมาเพราะว่าสมาชิกของเมทริกซ์ $F_{o'o}$ นั้นเกี่ยวเนื่องโดยตรงกับ Integral ของ Coulomb และ Exchange Operators
ซึ่งขึ้นอยู่กับ Spatial Wavefunction นั่นจึงทำให้เป็นปัญหาแบบงูกินหาง ดังนั้นเราจึงต้องใช้กระบวนการวนซ้ำ (Iteration) ในการแก้ปัญหาจนกว่าตำตอบจะลู่เข้านั่นเอง

%--------------------------
\subsection{การแก้สมการ Roothaan ด้วย Self-Consistent Field}
%--------------------------

