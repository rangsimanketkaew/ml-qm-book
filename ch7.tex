% LaTeX source for ``การเรียนรู้ของเครื่องสำหรับเคมีควอนตัม (Machine Learning for Quantum Chemistry)''
% Copyright (c) 2022 รังสิมันต์ เกษแก้ว (Rangsiman Ketkaew).

% License: Creative Commons Attribution-NonCommercial-NoDerivatives 4.0 International (CC BY-NC-ND 4.0)
% https://creativecommons.org/licenses/by-nc-nd/4.0/

\chapter{คุณสมบัติเชิงอิเล็กทรอนิกส์ของโมเลกุล}
\label{ch:el_prop}

เนื้อหาในบทนี้จะเกี่ยวกับคุณสมบัติเชิงอิเล็กทรอนิกส์ (Electronic Properties) ของโมเลกุล ซึ่งเป็นวัตถุประสงค์ของการใช้ ML เข้ามาศึกษาเคมีควอนตัม
โมเลกุลเป็นหน่วยพื้นฐานของสิ่งต่าง ๆ รอบตัวเรา ซึ่งโมเลกุลก็คือเป็นกลุ่มของอะตอมหลาย ๆ อะตอมมารวมกัน และในอะตอมนั้นเราสนใจอิเล็กตรอนเป็นพิเศษ
ในวิชาควอนตัมนั้นเราจะอธิบายพฤติกรรมของโมเลกุลโดยมุ่งเน้นไปที่อิเล็กตรอนซึ่งสามารถที่จะอธิบายได้ด้วยฟังก์ชันทางคณิตศาสตร์ที่เรียกว่าฟังก์ชันคลื่น (Wavefunction)
โดยหน้าตาของ Wavefunction นั้นจริง ๆ แล้วไม่มีการนิยามแบบตายตัว โดยหนึ่งในสมการที่โด่งดังที่สุดสมการหนึ่งของวงการวิทยาศาสตร์นั่นคือสมการชโรดิงเงอร์ 
(Schrödinger Equation) โดยถูกนำมาใช้ในการศึกษาระบบทางกลศาสตร์ควอนตัม ซึ่งการแก้สมการชโรดิงเงอร์ได้นั้นจะทำให้ได้มาซึ่ง Wavefunction 

สมการชโรดิงเงอร์สามารถแบ่งออกได้เป็นสองแบบ คือ แบบที่ไม่ขึ้นกับเวลาและแบบที่ขึ้นกับเวลา ดังนี้

\noindent Time-dependent Schrödinger Equation

\begin{equation}
    i \hbar \frac{d}{d t} \ket{\Psi(t)} = \hat{H} \ket{\Psi(t)}
\end{equation}

\noindent Time-independent Schrödinger Equation

\begin{equation}
    \hat{H}\ket{\Psi} = E \ket{\Psi}
\end{equation}

%--------------------------
\section{ฟังก์ชันคลื่น}
%--------------------------

- Primary output

- Secondary output

%--------------------------
\section{ระเบียบวิธีทางเคมีควอนตัม}
%--------------------------

