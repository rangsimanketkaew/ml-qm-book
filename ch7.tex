% LaTeX source for ``การเรียนรู้ของเครื่องสำหรับเคมีควอนตัม (Machine Learning for Quantum Chemistry)''
% Copyright (c) 2022 รังสิมันต์ เกษแก้ว (Rangsiman Ketkaew).

% License: Creative Commons Attribution-NonCommercial-NoDerivatives 4.0 International (CC BY-NC-ND 4.0)
% https://creativecommons.org/licenses/by-nc-nd/4.0/

\chapter{คุณสมบัติเชิงอิเล็กทรอนิกส์ของโมเลกุล}
\label{ch:el_prop}

เนื้อหาในบทนี้จะเกี่ยวกับคุณสมบัติเชิงอิเล็กทรอนิกส์ (Electronic Properties) ของโมเลกุล ซึ่งเป็นวัตถุประสงค์ของการใช้ ML เข้ามาศึกษาเคมีควอนตัม
โมเลกุลเป็นหน่วยพื้นฐานของสิ่งต่าง ๆ รอบตัวเรา ซึ่งโมเลกุลก็คือเป็นกลุ่มของอะตอมหลาย ๆ อะตอมมารวมกัน และในอะตอมนั้นเราสนใจอิเล็กตรอนเป็นพิเศษ
ในวิชาควอนตัมนั้นเราจะอธิบายพฤติกรรมของโมเลกุลโดยมุ่งเน้นไปที่อิเล็กตรอนซึ่งสามารถที่จะอธิบายได้ด้วยฟังก์ชันทางคณิตศาสตร์ที่เรียกว่าฟังก์ชันคลื่น (Wavefunction)
โดยหน้าตาของ Wavefunction นั้นจริง ๆ แล้วไม่มีการนิยามแบบตายตัว โดยหนึ่งในสมการที่โด่งดังที่สุดสมการหนึ่งของวงการวิทยาศาสตร์นั่นคือสมการชโรดิงเงอร์ 
(Schrödinger Equation) โดยถูกนำมาใช้ในการศึกษาระบบทางกลศาสตร์ควอนตัม ซึ่งการแก้สมการชโรดิงเงอร์ได้นั้นจะทำให้ได้มาซึ่ง Wavefunction 

สมการชโรดิงเงอร์สามารถแบ่งออกได้เป็นสองแบบ คือ แบบที่ไม่ขึ้นกับเวลาและแบบที่ขึ้นกับเวลา ดังนี้

\noindent Time-dependent Schrödinger Equation

\begin{equation}
    i \hbar \frac{d}{d t} \ket{\Psi(t)} = \hat{H} \ket{\Psi(t)}
\end{equation}

\noindent Time-independent Schrödinger Equation

\begin{equation}
    \hat{H}\ket{\Psi} = E \ket{\Psi}
\end{equation}

โดย Wavefunction ($\Psi(t)$) ที่เป็น Eigenfunction นั้นจะบรรจุข้อมูลเชิงอิเล็กทรอนิกส์ทุกอย่างเกี่ยวกับระบบของเราเอาไว้ ซึ่งในที่นี้ก็คือโมเลกุล
โดยสมการข้างต้นเป็นการคำนวณหาพลังงานของระบบโดยใช้ Hamiltonian Operator ($\hat{H}$) ซึ่งเป็น Operator ที่สอดคล้องกับพลังงาน
ซึ่งจริง ๆ แล้ว Eigenvalue ของสมการข้างต้นจะเป็นคุณสมบัติของโมเลกุลอะไรก็ได้ ตราบใดที่เราใช้ Operator ที่สอดคล้องกับคุณสมบัตินั้น ๆ 

หนึ่งในเป้าหมายสำคัญของกลศาสตร์ควอนตัมเชิงโมเลกุลก็คือการแก้สมการ Time-independent Schrödinger Equation และคำนวณหาโครงสร้างเชิงอิเล็กทรอนิกส์ 
(Electronic Structures) ของอะตอมและโมเลกุล โดยหัวข้อแรกของบทนี้ที่เราจะมาดูกันแบบละเอียดก็คือการใช้เทคนิคเชิงคำนวณและอาศัยการประมาณค่าในการแก้สมการดังกล่าว
โดยทั่วไปนั้นจะมีวิธีการหลัก ๆ 2 วิธีที่สามารถช่วยให้เราหาคำตอบของสมการชโรดิงเงอร์ ได้นั่นคือ \textbf{ab initio method} 
ซึ่งเป็นวิธีที่ความถูกแม่นยำของผลลัพธ์ที่ได้จากการแก้สมการนั้นจะขึ้นอยู่กับ Model ที่เรานำมาใช้ในการอธิบาย Wavefunction ของโมเลกุล 
และเป็นที่ทราบกันดีว่าสำหรับโมเลกุลที่มีขนาดใหญ่นั้น วิธีการ ab initio นี้จะมีความสิ้นเปลืองสูงมาก ดังนั้นจึงเป็นที่มาของวิธีการที่สอง นั่นคือ 
\textbf{Semiempirical method} ซึ่งจะใช้เทคนิคการมอง Hamiltonian ในรูปแบบที่ง่ายกว่า และอาศัยค่า Parameter ที่ได้จากการทดลองเพื่อเพิ่มความแม่นยำ 
อย่างไรก็ตาม วิธี Density Functional Theory (DFT) ก็ถูกพัฒนาขึ้นมาเพื่อแก้ปัญหาที่เราจะต้องมาแก้หรือประมาณค่า Wavefunction ตรง ๆ 
จึงทำให้ความสิ้นเปลืองของการคำนวณนั้นต่ำมากเมื่อเทียบกับสองวิธีข้างต้นทีได้กล่าวไป

%--------------------------
\section{การแก้สมการฟังก์ชันคลื่นเพื่อคำนวณพลังงาน}
%--------------------------

%--------------------------
\subsection{วิธี Self-Consistent Field}
%--------------------------

ในหัวข้อนี้เราจะมาพูดถึงการแก้สมการชโรดิงเงอร์โดยใช้วิธีที่ชื่อว่า Self-Consistent Field (SCF) ซึ่งเป็นการประมาณค่า Hamiltonian แบบวนซ้ำ
เริ่มต้นเราจะต้องมาดูกันก่อนว่าการมอง Wavefunction ของระบบหลายอิเล็กตรอนสำหรับวิธี SCF นั้นจะมีการตัดสิ่งที่ซับซ้อนออกไปนั่นก็คือ
Electron-Electron Repulsion หรืออันตรกิริยาระหว่างอิเล็กตรอน โดย Wavefunction สามารถถูกอธิบายได้ด้วยสมการต่อไปนี้

\begin{equation}
    H^{\circ} \Psi^{\circ} = E^{\circ} \Psi^{\circ}
\end{equation}

โดยกำหนดให้ $H^{\circ} = \sum^{N}_{i=1} h_{i}$ เมื่อ $h$ คือ Hamiltonian สำหรับหนึ่งอิเล็กตรอน (อิเล็กตรอนตัวที่ $i$) ในระบบที่มีอิเล็กตรอน $N$ ตัว 
นั่นคือสมการสำหรับระบบที่มีอิเล็กตรอน $N$ ตัวนั้น จะสามารถถูกแยกออกมาได้เป็นสมการของระบบหนึ่งอิเล็กตรอนได้ $N$ สมการ 
และ Wavefunction ของอิเล็กตรอนหนึ่งตัวนั้นจริง ๆ แล้วก็คือออบิทัล (Orbital) เราจึงสามารถเขียนสมการของอิเล็กตรอนหนึ่งตัวได้เป็น

\begin{equation}
    h_{i} \Psi^{\circ}(i) = E^{\circ}_{m} \Psi^{\circ}(i)
\end{equation}

โดยที่ $E^{\circ}_{m}$ คือพลังงานของอิเล็กตรอนหนึ่งตัวใน Molecular Orbital (MO) ซึ่งเขียนแทนด้วย $m$ นั่นเอง สำหรับระบบที่อิเล็กตรอนไม่ขึ้นต่อกันและกัน

ด้วยเหตุนี้ Wavefunction รวมของระบบ ($\Psi^{\circ}$) จึงสามารถเขียนให้อยู่ในรูปของ Wavefunction ของอิเล็กตรอนหนึ่งตัวได้ดังนี้

\begin{equation}
    \Psi^{\circ} = \psi^{\circ}_{a}(1) \psi^{\circ}_{b}(1) \dots \psi^{\circ}_{z}(N)
\end{equation}

ซึ่ง Wavefunction ด้านบนนี้จะขึ้นอยู่กับพิกัดของอิเล็กตรอนทุกตัวและขึ้นกับตำแหน่งของนิวเคลียสหรืออะตอมด้วย
\footnote{ตอนนี้เราจะยังไม่พิจารณาสปินของอิเล็กตรอนที่จะต้องสอดคล้องและไม่ขัดกับหลักกีดกันของเพาลีหรือ Pauli Exclusion 
ซึ่งจะมีการรวม Spinorbital สำหรับ Molecular Orbital $m$ ($\varphi_{m}$) เข้าไปด้วย}

สำหรับกระบวนการหรือขั้นตอนที่เราจะนำมาใช้ในการแก้สมการของระบบอิเล็กตรอนหลายตัวนั้น เราจะพิจารณาสมการรูทฮาน (Roothaan Equation) เป็นหลัก
ซึ่งเป็นวิธีหนึ่งในการแก้สมการ Hartree-Fock (HF) ซึ่งมีการกำหนดตัวดำเนินการใหม่ขึ้นมาใช้แทน Hamiltonian นั่นก็คือ Fock Operator 
โดยที่ Fock Operator ถูกนิยามในเทอมของ Coulomb Operator และ Exchange Operator ขึ้นมา นั่นก็คือ Fock Operator 
ซึ่งเขียนสมการสำหรับอิเล็กตรอน 1 ตัวได้เป็น

\begin{equation}
    \label{eq:fock}
    f_{1} \psi_{m}(1) = \varepsilon_{n} \psi_{m}(1)
\end{equation}

%--------------------------
\subsection{สมการ Roothaan}
%--------------------------

สำหรับการแก้สมการ HF ตรง ๆ โดยใช้ SCF นั้นสามารถทำได้ตรง ๆ ด้วยวิธีการเชิงตัวเลข (Numerical Method) แต่ว่าผลเฉลยที่ได้มานั้นมีความซับซ้อนมาก
ดังนั้นนักฟิสิกส์และนักเคมีชาวดัตช์ที่ชื่อว่า Clemens C.J. Roothaan จึงได้เสนอวิธีการใหม่สำหรับการอธิบาย MO โดยเรียกวิธีนั้นว่า 
Linear Combination of Atomic Orbitals (LCAO) เรามาดูกันเลยว่าวิธีการนี้มีรายละเอียดอย่างไร

เริ่มต้นเราจะนิยามฟังก์ชันพื้นฐาน (Basis Function) สำหรับระบบที่มีอิเล็กตรอน $N$ ตัวขึ้นมาก่อน ซึ่งเขียนแทนด้วย $\chi_{o}$
ซึ่งไอเดียตอนนี้ก็คือเราจะมองว่า Basis Function แบบที่ง่ายที่สุดที่เราสามาถนำมาใช้ได้นั่นก็คือ Atomic Orbital (O) ซึ่งสามารถที่จะเขียน Spatial Wavefunction 
(ฟังก์ชันคลื่นที่ขึ้นกับตำแหน่ง ของ AO) ให้อยู่ในผลรวมเชิงเส้นของการคูณระหว่างสัมประสิทธิ์เชิงเส้นที่เรายังไม่ทราบค่า (Unknown Coefficients, $c_{om}$) 
กับ $\chi_{o}$ ดังนี้

\begin{equation}
    \label{eq:lcao}
    \psi_{m} = \sum^{N_{o}}_{o=1} c_{om} \chi_{o} 
\end{equation}

เมื่อเราแทนสมการ \ref{eq:lcao} เข้าไปในสมการ \ref{eq:fock} เราจะได้

\begin{equation}
    \label{eq:lcao_in_fock}
    f_{1} \sum^{N_{o}}_{o=1} \chi_{o}(1) = \varepsilon \sum^{N_{o}}_{o=1} c_{om} \chi_{o}(1)
\end{equation}

แล้วทำการคูณสมการ \ref{eq:lcao_in_fock} ทั้งสองข้างด้วย $\chi^{*}_{o}(1)$ และทำการอินทิเกรตทั่วทั้ง Space ซึ่งจะทำให้เราได้ความสัมพันธ์ต่อไปนี้

\begin{equation}
    \label{eq:lcao_in_fock_int}
    \sum^{N_{o}}_{o=1} c_{om} \int \chi^{*}_{o}(1) f_{1} \chi_{o}(1) d\tau_{1} =
    \varepsilon_{m} \sum^{N_{o}}_{o=1} c_{om} \int \chi^{*}_{o}(1) \chi_{o}(1) d\tau_{1}
\end{equation}

จากสมการข้างต้นเราจะพบว่าจะมีผลคูณของ Basis Function ทั้งสองฝั่ง โดยทางฝั่งซ้ายนั้นเราสามารถนิยาม Fock Matrix (F) ได้

\begin{equation}
    \label{eq:matrix_fock}
    F_{o'o} = \int \chi^{*}_{o'}(1) f_{1} \chi_{o}(1) d\tau_{1}
\end{equation}

และทางฝั่งขวา เรานิยามสิ่งที่เรียกว่า Overlap Matrix (S) ซึ่งเป็น Matrix ทีอ่ธิบายถึงการซ้อนทับกันระหว่าง State 2 อัน

\begin{equation}
    \label{eq:matrix_overlap}
    S_{o'o} = \int \chi^{*}_{o'}(1) \chi_{o}(1) d\tau_{1}
\end{equation}

ถึงเราสามารถเขียนสมการ \ref{eq:lcao_in_fock_int} ให้อยู่ในรูปของสมการที่เรียกว่า Roothaan Equation ได้กระชับ ๆ ดังนี้

\begin{equation}
    \label{eq:roothaan}
    F c = \varepsilon S c
\end{equation}

โดยที่ $c$ คือเมทริกซ์ขนาด $N_{o} \times N_{o}$ ซึ่งประกอบไปด้วยสมาชิกของ Coefficient $c_{om}$ และ $\varepsilon$ คือเมทริกซ์ที่มีขนาด
$N_{o} \times N_{o}$ เช่นเดียวกันซึ่งเป็นเมทริกซ์แบบ Diagonal Matrix (สมาชิกที่ไม่ใช่แนวทแยงมีค่าเป็น 0 ทั้งหมด) ซึ่งก็คือพลังงานของ Orbital นั่นเอง
ซึ่งตรงจุดนี้เราต้องไม่ลืมว่า Fock Operator ($f_{1}$) นั้นถูกกำหนดให้อยู่ในรูปของ Integral บน MO และขึ้นอยู่กับค่าของ Coefficient $c_{om}$ ด้วย

สำหรับการแก้สมการ \ref{eq:roothaan} นั้นสามารถทำได้ผ่าน Determinant ดังนี้

\begin{equation}
    \label{eq:scf_secular}
    det|F - \varepsilon S| = 0
\end{equation}

ซึ่งสมการด้านบนไม่สามารถแก้ได้แบบตรงไปตรงมาเพราะว่าสมาชิกของเมทริกซ์ $F_{o'o}$ นั้นเกี่ยวเนื่องโดยตรงกับ Integral ของ Coulomb Operator 
และ Exchange Operators ซึ่งขึ้นอยู่กับ Spatial Wavefunction นั่นจึงทำให้เป็นปัญหาแบบงูกินหาง ดังนั้นเราจึงต้องใช้กระบวนการวนซ้ำ (Iteration) 
ในการแก้ปัญหาจนกว่าคำตอบหรือผลลัพธ์ที่เราต้องการจากสมการ (พลังงาน) จะลู่เข้านั่นเอง

%--------------------------
\subsection{การแก้สมการ Roothaan ด้วย Self-Consistent Field}
%--------------------------

\begin{figure}[H]
    \centering
    \includegraphics[width=0.8\linewidth]{fig/ch7-scf.png}
    \caption{แผนภาพเปรียบเทียบภาพโปรแกรมแบบดั้งเดิมกับการเรียนรู้ของเครื่อง}
    \label{fig:scf}
\end{figure}

ภาพที่ \ref{fig:scf} แสดงแผนผงอัลกอริทึมของวิธี SCF โดยเริ่มจากการเลือก Atomic Basis Function ซึ่งถือว่าเป็นองค์ประกอบหลักของนำไปสร้าง (Formulate) 
$S$ โดยใช้สมการ \ref{eq:matrix_overlap} กับ $c_{om}$ ซึ่งเราจะใช้วิธีการสร้างค่าเริ่มต้นด้วยวิธี Guess ซึ่งมีด้วยกันหลายวิธี เช่น

\begin{enumerate}
    \item \textbf{H{\"u}ckel guess} : ใช้ H{\"u}ckel Orbital
    \item \textbf{Superposition of Atomic Densities (SAD)} : ใช้ผลรวมของ Atomic Density ในการสร้าง Density Matrix
    \item \textbf{Generalized Wolfsberg-Helmholtz (GWH)} : เป็นวิธีการที่อาศัย H{\"u}ckel Theory โดยการใช้ Overlap Matrix และ 
    Core Hamiltonian\cite{wolfsberg1952}
    \item \textbf{CORE} : ทำการทำ Core Hamiltonian ให้เกิดเมทริกซ์รูปทแยง (Diagonalization)
    \item \textbf{Harris} : ใช้ Harris Functional ซึ่งเป็น Non-self-consistent Approximation สำหรับ Kohn-Sham Orbital\cite{harris1985}
\end{enumerate}

ซึ่งโปรแกรมเคมีเชิงคำนวณต่างก็มีการเลือกใช้ Guess Method สำหรับการเดา Coefficient หรือ Wavefunction เริ่มต้นในการแก้ SCF แตกต่างกันไป
โปรแกรม Gaussian ใช้วิธี Harris สำหรับการคำนวณ HF และ DFT และใช้ H{\"u}ckel หรือ CORE สำหรับ Semiempirical Methods, 
โปรแกรม Q-Chem และ Psi ใช้วิธี SAD กับ GWH เป็นวิธีเริ่มต้นโดยอัตโนมัติ เป็นต้น

หลังจากสร้าง Coefficient Matrix ขั้นตอนต่อไปคือการสร้าง Fock Matrix $F$ โดยใช้สมการ \ref{eq:matrix_fock} 
หลังจากนั้นเราจะทำการแก้สมการลักษณะเฉพาะ (Secular Equation) สมการที่ \ref{eq:scf_secular} เพื่อหา Energy Matrix 
แล้วก็ทำการวนซ้ำขั้นตอนการสร้าง $S$ กับ $F$ ไปปรับหาค่าพลังงานไปเรื่อย ๆ จนกว่าค่าความคลาดเคลื่อนหรือ Error จะมีค่าน้อยกว่าค่าที่กำหนดไว้ (Threshold)
แล้วจึงสิ้นสุดกระบวนการ SCF เมื่อค่าพลังงานนั้นลู่เข้า

%--------------------------
\subsection{การคำนวณอนุพันธ์ของพลังงานและเมทริกซ์เฮสเซียน}
%--------------------------

หลังจากที่เราสามารถหาพลังงานเชิงอิเล็กทรอนิกส์ (Electronic Energy) ได้แล้ว ลำดับถัดไปที่เราสามารถคำนวณได้ก็คือคุณสมบัติต่าง ๆ ของโมเลกุล
สิ่งแรกที่เราทำได้และถือว่าสำคัญมาก ๆ ในงานวิจัยทางด้านเคมีควอนตัมก็คือการหาโครงสร้างที่เหมาะสมหรือเสถียรที่สุดของโมเลกุลโดยใช้หลักเกณฑ์พลังงานรวมที่ต่ำที่สุด
ซึ่งการที่เราทราบโครงสร้างที่เหมาะสมที่สุดนั้นมีประโยชน์อย่างมากเพราะเราสามารถนำผลการคำนวณไปเทียบกับผลจากการทดลองด้วยเทคนิค X-ray Crystallography,
Electron Diffractiom, หรือ Microwave Spectroscopy เป็นต้น โดยการหาโครงสร้างที่สภาวะเหมาะสมหรือสมดุล (Equilibrium Structure) 
นั้นสามารถทำได้โดยหาอนุพันธ์ของพลังงานศักย์ของโมเลกุลเทียบกับพิกัดนิวเคลียร์ ซึ่งวิธีการที่เราสามารถนำมาหาอนุพันธ์เพื่อให้ได้ผลลัพเชิงวิเคราะห์ (Analytical Method) 
เรียกว่า Gradient Method ซึ่งเร็วและให้ผลลัพธ์ที่แม่นยำกว่าระเบียบวิธีเชิงตัวเลข (Numerical Method)

สำหรับอนุพันธ์ของพลังงานนั้นเราจะเริ่มต้นพิจารณากรณีที่ง่าย ๆ นั่นก็คือโมเลกุลที่มีอะตอมสองอะตอม ซึ่งพลังงานศักย์ของโมเลกุลซึ่งเขียนแทนด้วย $E$ 
นี้จะมีเทอมที่เป็นแรงผลักระหว่างนิวเคลียสด้วยซึ่งจะขึ้นกับระยะห่างระหว่างนิวเคลียส (Internuclear Distance, $R$) สำหรับโครงสร้างที่อยู่ในสมดุลนั้น 
แรง (Force) ที่กระทำต่อนิวเคลียสโดยอิเล็กตรอนนั้นจะเท่ากับศูนย์ ซึ่งแรงดังกล่าวเป็นแรงย่อยมีนิยามคืออนุพันธ์อันดับที่หนึ่งของพลังงานศักย์เทียบกับพิกัดของนิวเคลียสที่ $i$

\begin{align}
    f_{i} &= - \pdv{E}{q_{i}} \\
    &= 0
\end{align}

โดยการคำนวณหาอนุพันธ์ข้างต้นด้วยวิธีการวิเคราะห์หรือ Analytical Method นั้นเราจะต้องทำการคำนวณหาอนุพันธ์ของอินทิกรัลของอิเล็กตรอนหนึ่งตัวและอิเล็กตรอนสองตัว
(One-electron กับ Two-electron Integrals) เทียบกับพิกัดนิวเคลียร์ นั่นคือเราจะต้องทำการหาอนุพันธ์ของ Basis Function นั่นเอง\footnote{Basis Function 
ก็คือ Basis ที่เกิดขึ้นมาจาก Atomic Orbtials ที่ถูก centered หรือมีตำแหน่งอยู่ที่จุดอ้างอิงของนิวเคลียสของอะตอมในโมเลกุล} 
ซึ่งเราสามารถทำได้ผ่านการใช้กฎลูกโซ่ (Chain Rule) โดยทำการหาอนุพันธ์ของพลังงานศักย์เทียบกับ Expansion Coefficient

ลำดับถัดมาคือการหาเมทริกซ์เฮสเซียน (Hessian Matrix) ซึ่งสามารถทำได้โดยการหาอนุพันธ์ย่อยอันดับที่สองของพลังงานศักย์เทียบกับนิวเคลียสของอะตอมตัวที่ $i$ และ $j$
($\pdv{E}{q_{i}}{q_{j}}$) ซึ่งช่วยให้เราสามารถระบุได้ว่าค่าพลังงานที่คำนวณออกมาได้นั้นสอดคล้องกับจุดต่ำสุดหรือสูงสุดบนพื้นผิวพลังงานศักย์ (Potential Energy Surface) 
โดยจะสอดคล้องกับอนุพันธ์อันดับที่สองที่ได้ค่าออกมาเป็นบวก (สำหรับ Minimum Point) และลบ (สำหรับ Maximum Point) ตามลำดับ

%--------------------------
\section{Electron Density และ Density Matrix}
%--------------------------

%--------------------------
\section{พลังงานของ Fronteir Orbitals}
%--------------------------

%--------------------------
\subsection{พลังงานของ HOMO และ LUMO}
%--------------------------

%--------------------------
\subsection{ผลต่างของพลังงานของ HOMO และ LUMO}
%--------------------------

%--------------------------
\section{Dipole Moment}
%--------------------------

%--------------------------
\section{Polarizability}
%--------------------------

%--------------------------
\section{Electron Transfer}
%--------------------------

%--------------------------
\subsection{Electron Transfer Coupling}
%--------------------------

%--------------------------
\subsection{Reorganization Energy}
%--------------------------

%--------------------------
\section{Excited State Properties}
%--------------------------

%--------------------------
\subsection{Excited State Energies}
%--------------------------

%--------------------------
\subsection{Nonadiabatic Coupling}
%--------------------------
