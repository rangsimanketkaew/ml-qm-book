% LaTeX source for ``ปัญญาประดิษฐ์สำหรับเคมีควอนตัม (Machine Learning for Quantum Chemistry)''
% Copyright (c) 2022 รังสิมันต์ เกษแก้ว (Rangsiman Ketkaew).

% License: Creative Commons Attribution-NonCommercial-NoDerivatives 4.0 International (CC BY-NC-ND 4.0)
% https://creativecommons.org/licenses/by-nc-nd/4.0/

\documentclass[a4paper,12pt,twoside,openany]{book}
\usepackage[
    width=5.5in,
    height=8.5in,
    hmarginratio=3:2,
    vmarginratio=1:1
]{geometry}

% Adjust size of the body text
\AtBeginDocument{\fontsize{14}{16.2}\selectfont}

\usepackage[T1]{fontenc}
\usepackage{textcomp}
\usepackage{url}
\usepackage{graphicx}
\usepackage{booktabs}  % For nicely typeset tabular material
\usepackage{multirow}  % For multple rows table
\usepackage{changepage,threeparttable}  % For wide tables

\usepackage{amsmath}
\usepackage{amsthm}
\usepackage{amssymb}
\usepackage{mathtools}
\usepackage{physics}
% \usepackage{exercise}
\usepackage{setspace}
\usepackage{csquotes}
\usepackage{enumitem}
\usepackage{hyphenat}
\usepackage[bookmarks]{hyperref}

% Table of Contents
\usepackage{tocloft}
\renewcommand{\contentsname}{\hspace*{\fill}\bfseries\huge สารบัญ\hspace*{\fill}}
\setcounter{tocdepth}{1} % Level of TOC
\renewcommand\cftchapafterpnum{\vskip10pt}
\renewcommand\cftsecafterpnum{\vskip10pt}
\renewcommand\cftsubsecafterpnum{\vskip10pt}

% Header
\usepackage{fancyhdr}
\fancyhf{}
\fancyhead[LE,RO]{\textbf{\thepage}}
\fancyhead[RE,LO]{\textbf{\leftmark}}

% Part
\renewcommand\thepart{\arabic{part}}

% Chapter
\renewcommand{\partname}{ส่วนที่}
\renewcommand{\chaptername}{บทที่}

% Appendix
\usepackage[toc,page]{appendix}
\renewcommand{\appendixname}{ภาคผนวก}
\renewcommand{\appendixtocname}{ภาคผนวก}
\renewcommand{\appendixpagename}{ภาคผนวก}

% Figure & Table Captions
\usepackage{caption}
\captionsetup{labelsep=space,justification=centering,singlelinecheck=off}
\captionsetup[figure]{name={ภาพ}}
\captionsetup[table]{name={ตาราง}}

% Paragraph
\usepackage{indentfirst}
\setlength{\parindent}{2em}
\setlength{\parskip}{1em}

% Indices
\usepackage{imakeidx}
\makeindex
\makeindex[name=th,title={ดรรชนีภาษาไทย}]
\makeindex[name=en,title={ดรรชนีภาษาอังกฤษ}]
\newcommand{\idxth}[1]{\index[th]{#1}}
\newcommand{\idxen}[1]{\index[en]{#1}}
\newcommand{\idxboth}[2]{\index[th]{#1}\index[en]{#2}}

%----------------------------------------------------

%%% Option 1. Font for Thai %%%

\usepackage[utf8]{inputenc}
\usepackage{xltxtra} % this will load the fontspec, metalogo, and realscripts packages
\XeTeXlinebreaklocale "th_TH"
\XeTeXlinebreakskip = 0pt plus 1pt  
\defaultfontfeatures{Scale=1.23}
\setmainfont[
  Renderer=Basic,
  BoldFont={THSarabunNew_Bold.ttf},
  ItalicFont={THSarabunNew_Italic.ttf},
  BoldItalicFont={THSarabunNew_BoldItalic.ttf},
]{THSarabunNew.ttf} 

% \setmainfont{TH Sarabun New}

%%% Option 2. Fonts for different languages %%%
% Fonts need to be installed on the system

% \usepackage{fontspec,xunicode}
% \defaultfontfeatures{Mapping=tex-text}
% \setmainfont[Renderer=Basic, Numbers=OldStyle, Scale = 1.0]{TeX Gyre Pagella}
% \setmathrm[Renderer=Basic, Scale=0.90]{TeX Gyre Pagella}
% \setsansfont[Renderer=Basic, Scale=0.90]{TeX Gyre Heros} 
% \setmonofont[Renderer=Basic]{TeX Gyre Cursor}
% \newfontfamily{\thaifont}{TH Sarabun New}[Scale=MatchLowercase] 

% \usepackage[Latin, Thai]{ucharclasses}
% \setTransitionTo{Thai}{\thaifont}
% \setTransitionFrom{Thai}{\normalfont}

% \XeTeXlinebreaklocale "th_TH"
% \XeTeXlinebreakskip = 0pt plus 1pt  

% \defaultfontfeatures{Scale=1.23}

%----------------------------------------------------

% Bibliography
\usepackage[
  backend=bibtex,
  autocite=superscript,
  sorting=none,
  style=numeric, 
  natbib=true,
  mincitenames=1,
  doi=false,
  url=false, 
  isbn=false,
  backref=true
]{biblatex}
\addbibresource{references.bib}
