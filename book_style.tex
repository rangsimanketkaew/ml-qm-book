% LaTeX source for ``ปัญญาประดิษฐ์สำหรับเคมีควอนตัม (Machine Learning for Quantum Chemistry)''
% Copyright (c) 2022 รังสิมันต์ เกษแก้ว (Rangsiman Ketkaew).

% License: Creative Commons Attribution-NonCommercial-NoDerivatives 4.0 International (CC BY-NC-ND 4.0)
% https://creativecommons.org/licenses/by-nc-nd/4.0/

\documentclass[a4paper,12pt,twoside,openany]{book}
\usepackage[
    width=5.5in,
    height=8.5in,
    hmarginratio=3:2,
    vmarginratio=1:1
]{geometry}

% Adjust size of the body text
\AtBeginDocument{\fontsize{14}{16.2}\selectfont}

\usepackage[T1]{fontenc}
\usepackage{textcomp}
\usepackage{url}
\usepackage{graphicx}
\usepackage{float}
\usepackage{subcaption}
\usepackage{booktabs}  % For nicely typeset tabular material
\usepackage{multirow}  % For multple rows table
\usepackage{changepage,threeparttable}  % For wide tables
\usepackage{wrapfig} % wrapping text around figure
\usepackage[perpage]{footmisc} % reset footnote counter each page
\usepackage{alertmessage} % alert box
\usepackage{pdfpages} % for adding PDF pages

\usepackage{amsmath}
\usepackage{amsthm}
\usepackage{amssymb}
\usepackage{mathtools}
\usepackage{physics}
\usepackage{bm}
\usepackage{algorithm}
\usepackage{algpseudocode}
% \usepackage{exercise}
\usepackage[version=3]{mhchem}
\usepackage{setspace}
\usepackage{csquotes}
\usepackage{enumitem}
\usepackage{hyphenat}
\usepackage[bookmarks]{hyperref}
% import lstlisting setting
%%%%%%% List of commands: %%%%%%%
% Code block:
% \begin{lstlisting}[style=MyBash]{}
% \begin{lstlisting}[style=MyPython]{}
% \begin{lstlisting}[style=MyC++]{}
% \begin{lstlisting}[style=MyJSON]{}
% ----------------------------------
% Code in line:
% \bashinline{}
% \pyinline{}
% \cppinline{}
% \inlinehighlight{}
%%%%%%%%%%%%%%%%%%%%%%%%%%%%%%%%%

\usepackage{listings} % for code listing
\usepackage{xcolor} % for color

% define color
\colorlet{shadecolor}{gray!10} % increase the value will produce darker color
\colorlet{punct}{red!60!black}
\colorlet{numb}{magenta!60!black}
\definecolor{mymauve}{rgb}{0.58,0,0.82}
\definecolor{deepblue}{rgb}{0,0,0.5}
\definecolor{deepred}{rgb}{0.6,0,0}
\definecolor{deepgreen}{rgb}{0,0.5,0}
\definecolor{pythoncolor}{RGB}{102,102,255}

\lstset{
  backgroundcolor=\color{shadecolor},   % choose the background color; you must add \usepackage{color} or \usepackage{xcolor}; should come as last argument
  basicstyle=\ttfamily\normalsize\linespread{0.5},        % the size of the fonts that are used for the code
  keywordstyle=\color{blue}\ttfamily,       % keyword style
  commentstyle=\color{pink}\ttfamily,    	   % comment style
  breaklines=true,                 % sets automatic line breaking
  breakatwhitespace=true,         % sets if automatic breaks should only happen at whitespace
  captionpos=b,                    % sets the caption-position to bottom
  deletekeywords={...},            % if you want to delete keywords from the given language
  escapeinside={\%*}{*)},          % if you want to add LaTeX within your code
  extendedchars=true,              % lets you use non-ASCII characters; for 8-bits encodings only, does not work with UTF-8
%   firstnumber=1000,                % start line enumeration with line 1000
  frame=tlrb,	                   % adds a frame around the code, use a combination of t l r and b
  frameshape={RYR}{Y}{Y}{RYR},     % rounded corner
  keepspaces=true,                 % keeps spaces in text, useful for keeping indentation of code (possibly needs columns=flexible)
  columns=flexible,
%   basewidth={.88em},
  numbers=left,                    % where to put the line-numbers; possible values are (none, left, right)
  numbersep=10pt,                   % how far the line-numbers are from the code
  numberstyle=\normalsize\color{gray}, % the style that is used for the line-numbers
  rulecolor=\color{lightgray},         % if not set, the frame-color may be changed on line-breaks within not-black text (e.g. comments (green here))
  showspaces=false,                % show spaces everywhere adding particular underscores; it overrides 'showstringspaces'
  showstringspaces=false,          % underline spaces within strings only
  showtabs=false,                  % show tabs within strings adding particular underscores
  stepnumber=1,                    % the step between two line-numbers. If it's 1, each line will be numbered
  stringstyle=\color{mymauve},     % string literal style
  tabsize=4,	                   % sets default tabsize to 2 spaces
  title=\lstname,                   % show the filename of files included with \lstinputlisting; also try caption instead of title
  xleftmargin = 0.8cm,             % left margin for the code
  xrightmargin = -0.5cm,           % right margin for the code
  framexleftmargin = 2em,          % left margin for the whole frame
%   aboveskip=3mm,
  belowskip=-1.5 \baselineskip,
}

\lstdefinestyle{MyBash}{
    language=Bash,
    keywordstyle=\color{blue},
    stringstyle=\color{black},
    commentstyle=\color{pink},
    morekeywords={}, % for letter 
    otherkeywords={}, % for non-letter
    deletekeywords={cd,echo,enable,export,jobs,local,source,test},
    morecomment=[l][\color{magenta}]{\#},
}

\lstdefinestyle{MyPython}{
    language=Python,
    keywordstyle=\color{deepblue},
    emph={MyClass,__init__},          % Custom highlighting
    emphstyle=\color{deepred},    % Custom highlighting style
    stringstyle=\color{deepgreen},
    commentstyle=\color{pink},
    morekeywords={self},              % Add keywords here
    morecomment=[l][\color{magenta}]{\#},
}

\lstdefinestyle{MyC++}{
    language=C++,
    keywordstyle=\color{blue},
    stringstyle=\color{red},
    commentstyle=\color{pink},
    morekeywords={},
    morecomment=[l][\color{magenta}]{\/\/},
}

\lstdefinestyle{MyFortran}{
    language=Fortran,
    keywordstyle=\color{blue},
    stringstyle=\color{red},
    commentstyle=\color{green},
    morecomment=[l][\color{magenta}]{!\ } % Comment only with space after !
}

\lstdefinelanguage{MyJSON}{
    stringstyle=\color{black},
    morestring=[b]",
    literate=
     *{0}{{{\color{numb}0}}}{1}
      {1}{{{\color{numb}1}}}{1}
      {2}{{{\color{numb}2}}}{1}
      {3}{{{\color{numb}3}}}{1}
      {4}{{{\color{numb}4}}}{1}
      {5}{{{\color{numb}5}}}{1}
      {6}{{{\color{numb}6}}}{1}
      {7}{{{\color{numb}7}}}{1}
      {8}{{{\color{numb}8}}}{1}
      {9}{{{\color{numb}9}}}{1}
      {:}{{{\color{punct}{:}}}}{1}
      {,}{{{\color{punct}{,}}}}{1}
      {\{}{{{\color{mymauve}{\{}}}}{1}
      {\}}{{{\color{mymauve}{\}}}}}{1}
      {[}{{{\color{mymauve}{[}}}}{1}
      {]}{{{\color{mymauve}{]}}}}{1},
}

\lstdefinestyle{nonumber}{
    number=none,
}

% for skipping lines with dots
% https://tex.stackexchange.com/questions/476658/how-to-skip-lines-in-lstlisting-with-dots
%------------------------
\let\origthelstnumber\thelstnumber
\makeatletter
\newcommand*\Suppressnumber{%
  \lst@AddToHook{OnNewLine}{%
    \let\thelstnumber\relax%
  }%
}

\newcommand*\Reactivatenumber{%
  \lst@AddToHook{OnNewLine}{%
   \let\thelstnumber\origthelstnumber%
  }%
}
\makeatother
%------------------------

% define code in line with highlighting
\newcommand{\bashinline}[1]{\colorbox{shadecolor}{\lstinline[style=MyBash]{#1}}}
\newcommand{\pyinline}[1]{\colorbox{shadecolor}{\lstinline[style=MyPython]{#1}}}
% inline highlight using a special color
\newcommand{\inlinehighlight}[1]{\colorbox{shadecolor}{\lstinline[
    style=MyPython,
    basicstyle=\ttfamily\small\linespread{0.5}\color{pythoncolor},
]{#1}}}
\newcommand{\cppinline}[1]{\colorbox{shadecolor}{\lstinline[style=MyC++]{#1}}}


% Table of Contents
\usepackage{tocloft}
\renewcommand{\contentsname}{\hspace*{\fill}\bfseries\huge สารบัญ\hspace*{\fill}}
\setcounter{tocdepth}{1} % Level of TOC
% https://tex.stackexchange.com/a/397486/117615
\renewcommand\cftchappresnum{บทที่ }
\renewcommand\cftchapafterpnum{\vskip10pt}
\newlength\mylen
\settowidth\mylen{\bfseries บทที่ :\hspace{3em}}
\cftsetindents{chap}{0pt}{\mylen}

\renewcommand\cftsecafterpnum{\vskip10pt}
\renewcommand\cftsubsecafterpnum{\vskip10pt}

% Header
\usepackage{fancyhdr}
\fancyhf{}
\fancyhead[LE,RO]{\textbf{\thepage}}
\fancyhead[RE,LO]{\textbf{\leftmark}}

% Part
\renewcommand\thepart{\arabic{part}}

% Chapter
\renewcommand{\partname}{ส่วนที่}
\renewcommand{\chaptername}{บทที่}

% Section & Subsection
\usepackage{float} % for [H] option

% Appendix
\usepackage[toc,page]{appendix}
\renewcommand{\appendixname}{ภาคผนวก}
\renewcommand{\appendixtocname}{ภาคผนวก}
\renewcommand{\appendixpagename}{ภาคผนวก}

% Figure & Table Captions
\usepackage{caption}
\captionsetup{labelsep=space,justification=centering,singlelinecheck=off}
\captionsetup[figure]{name={ภาพ}}
\captionsetup[table]{name={ตาราง}}

% Paragraph
\usepackage{indentfirst}
\setlength{\parindent}{2em}
\setlength{\parskip}{1em}

% Equation
\counterwithin{equation}{chapter}

% Indices
\usepackage{imakeidx}
\makeindex
\makeindex[name=th,title={ดรรชนีภาษาไทย}]
\makeindex[name=en,title={ดรรชนีภาษาอังกฤษ}]
\newcommand{\idxth}[1]{\index[th]{#1}}
\newcommand{\idxen}[1]{\index[en]{#1}}
\newcommand{\idxboth}[2]{\index[th]{#1}\index[en]{#2}}

% URL font
\urlstyle{same} % use the same font as main font

%----------------------------------------------------

%%% Option 1. Font for Thai %%%

\usepackage[utf8]{inputenc}
\usepackage{xltxtra} % this will load the fontspec, metalogo, and realscripts packages
\XeTeXlinebreaklocale "th_TH"
\XeTeXlinebreakskip = 0pt plus 1pt  
\defaultfontfeatures{Scale=1.23}
% \setmainfont{TH Sarabun New}
\setmainfont[
  ExternalLocation=font/,
  Renderer=Basic,
  BoldFont={THSarabunNew_Bold.ttf},
  ItalicFont={THSarabunNew_Italic.ttf},
  BoldItalicFont={THSarabunNew_BoldItalic.ttf},
]{THSarabunNew.ttf} 

%%% Option 2. Fonts for different languages %%%
% Fonts need to be installed on the system

% \usepackage{fontspec,xunicode}
% \defaultfontfeatures{Mapping=tex-text}
% \setmainfont[Renderer=Basic, Numbers=OldStyle, Scale = 1.0]{TeX Gyre Pagella}
% \setmathrm[Renderer=Basic, Scale=0.90]{TeX Gyre Pagella}
% \setsansfont[Renderer=Basic, Scale=0.90]{TeX Gyre Heros} 
% \setmonofont[Renderer=Basic]{TeX Gyre Cursor}
% \newfontfamily{\thaifont}{TH Sarabun New}[Scale=MatchLowercase] 

% \usepackage[Latin, Thai]{ucharclasses}
% \setTransitionTo{Thai}{\thaifont}
% \setTransitionFrom{Thai}{\normalfont}

% \XeTeXlinebreaklocale "th_TH"
% \XeTeXlinebreakskip = 0pt plus 1pt  

% \defaultfontfeatures{Scale=1.23}

%----------------------------------------------------

\usepackage{tikz}
%\usepackage{etoolbox} % for \ifthen
\usepackage{listofitems} % for \readlist to create arrays
\usetikzlibrary{arrows.meta} % for arrow size
\usepackage[outline]{contour} % glow around text
\contourlength{1.4pt}

\tikzset{>=latex} % for LaTeX arrow head
\usepackage{xcolor}
\colorlet{myred}{red!80!black}
\colorlet{myblue}{blue!80!black}
\colorlet{mygreen}{green!60!black}
\colorlet{myorange}{orange!70!red!60!black}
\colorlet{mydarkred}{red!30!black}
\colorlet{mydarkblue}{blue!40!black}
\colorlet{mydarkgreen}{green!30!black}
\tikzstyle{node}=[thick,circle,draw=myblue,minimum size=22,inner sep=0.5,outer sep=0.6]
\tikzstyle{node in}=[node,green!20!black,draw=mygreen!30!black,fill=mygreen!25]
\tikzstyle{node hidden}=[node,blue!20!black,draw=myblue!30!black,fill=myblue!20]
\tikzstyle{node convol}=[node,orange!20!black,draw=myorange!30!black,fill=myorange!20]
\tikzstyle{node out}=[node,red!20!black,draw=myred!30!black,fill=myred!20]
\tikzstyle{connect}=[thick,mydarkblue] %,line cap=round
\tikzstyle{connect arrow}=[-{Latex[length=4,width=3.5]},thick,mydarkblue,shorten <=0.5,shorten >=1]
\tikzset{ % node styles, numbered for easy mapping with \nstyle
  node 1/.style={node in},
  node 2/.style={node hidden},
  node 3/.style={node out},
}
\def\nstyle{int(\lay<\Nnodlen?min(2,\lay):3)} % map layer number onto 1, 2, or 3

%----------------------------------------------------

% Bibliography
\usepackage[
  backend=bibtex,
  autocite=superscript,
  sorting=none,
  style=numeric, 
  natbib=true,
  mincitenames=1,
  doi=false,
  url=false, 
  isbn=false,
  backref=true
]{biblatex}
\addbibresource{references.bib}

\renewcommand*{\bibfont}{\fontsize{14}{16.2}\selectfont}
