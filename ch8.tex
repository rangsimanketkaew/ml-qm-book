% LaTeX source for ``การเรียนรู้ของเครื่องสำหรับเคมีควอนตัม (Machine Learning for Quantum Chemistry)''
% Copyright (c) 2022 รังสิมันต์ เกษแก้ว (Rangsiman Ketkaew).

% License: Creative Commons Attribution-NonCommercial-NoDerivatives 4.0 International (CC BY-NC-ND 4.0)
% https://creativecommons.org/licenses/by-nc-nd/4.0/

\chapter{ลักษณะเฉพาะของอะตอมและโมเลกุล}
\label{ch:feature}

ลักษณะเฉพาะ (Feature) คือคุณลักษณะเด่นที่มีความพิเศษและบ่งบอกความเฉพาะตัวของอะตอมหรือโมเลกุลนั้น ๆ ซึ่งอาจจะเรียกว่าเป็นคุณลักษณะแบบพิเศษก็ได้ 
(Special Attributes) นอกจากนี้เรายังเรียกสิ่งที่เป็น Feature ว่ามันคือการแสดงความเป็นตัวแทนของสิ่งที่เราสนใจอีกด้วยหรือ Representation
คำถามที่ตามมาก็คือแล้ว Feature หรือ Representation มีความสำคัญอย่างไร จริง ๆ แล้วมีความสำคัญมาก ๆ อาจจะกล่าวได้ว่าสำคัญที่สุดเลยก็ว่าได้ 
เพราะมันเป็น Input ที่เรานำมาใช้สร้าง Model ดังนั้นมันจึงเป็นปัจจัยที่กำหนดประสิทธิภาพของ Model นั่นเอง 

เนื่องจากว่ามนุษย์สามารถแยกแยะโมเลกุลแต่ละตัวออกจากกันได้ แต่ว่าคอมพิวเตอร์ไม่สามารถทำได้ เพราะว่ามันเข้าใจข้อมูลที่เป็นแบบดิจิตอลในภาษาเครื่องจักรเท่านั้น (Machine Language)
ดังนั้นเราจึงต้องมีการใช้ Representation เพื่ออธิบายโมเลกุลในรูปแบบของค่า Parameter ที่คอมพิวเตอร์สามารถเข้าใจได้ เช่น แปลงโมเลกุลเป็นเวกเตอร์ของตัวเลข 
สำหรับการอธิบายโมเลกุลแบบง่าย ๆ นั้นสามารถทำได้โดยใช้ Representation เพื่อมาอธิบายข้อมูลเชิงโครงสร้าง (Structural Properties) 
ซึ่งสามารถใช้ข้อมูลทางเคมีทั่วไปได้ ยกตัวอย่างเช่น รูปร่างของโมเลกุล, จำนวนหมู่ฟังก์ชัน, ชนิดของพันธะระหว่างอะตอมคาร์บอน, และจำนวนวงเบนซีน 
ซึ่งข้อมูลเหล่านี้เราสามารถคำนวณออกมาได้ง่าย ๆ ไม่มีความซับซ้อนอะไร แต่ปัญหาคือ Representation ที่เป็น Structural-based นั้นมีข้อมูลที่น้อยเกินไป
จึงทำให้ไม่สามารถถูกนำมาใช้เป็น Input สำหรับการสร้าง Model เพื่อทำนายคุณสมบัติหรือ Parameter ทางเคมีที่ซับซ้อนหรือละเอียดกว่าได้ เช่น 
พลังงานพันธะ (Bond Energy), ค่าความถี่การสั่น (Vibration), ไดโพลโมเมนต์ (Dipole Moment), ฯลฯ 
นั่นก็เพราะว่า Input ของเรามันเป็น Representation ที่ไม่มี Correlation กับ Output ที่เราต้องการทำนายโดยตรง

ดังนั้นถ้าหากเราต้องการที่จะทำนาย Output ที่มีความละเอียดอยู่ในระดับอะตอมหรือเชิงอิเล็กทรอนิกส์ เราควรจะใช้ Representation ที่อยู่ในระดับเดียวกัน 
และ Representation Input เหล่านี้ควรจะต้องเก็บข้อมูลทางควอนตัมเคมีและฟิสิกส์ไว้ด้วย โดยการพัฒนา Representation โดยใช้องค์ความรู้ทางฟิสิกส์
(Physics-inspired Representation) ก็เป็นหนึ่งในหัวข้องานวิจัยที่กำลังมาแรงในขณะนี้ ข้อมูลทางฟิสิกส์ที่เราเพิ่มเข้าไปก็เปรียบเสมือนเป็นองค์ประกอบที่เพิ่มความถูกต้อง
(Correction) ให้กับ Representation ให้มากขึ้น เราใส่ระบุความเป็นสมมาตร (Symmetricity) ของโมเลกุลเข้าไปได้ เป็นต้น 

%--------------------------
\section{การแปลงข้อมูลเชิงโมเลกุล}
%--------------------------

โมเลกุลประกอบไปด้วยอะตอมหลายอะตอมมารวมกัน เราจึงเปรียบเทียบโมเลกุลเป็นประโยคหรือข้อความและเปนียบเทียบอะตอมเป็นคำแต่ละคำได้
ดังที่บอกไปในข้างต้นว่าการทำให้คอมพิวเตอร์เข้าใจความเชื่อมโยงระหว่างอะตอมในโมเลกุลนั้นต้องมีการกำหนดหรือเลือกใช้ Representation ที่เหมาะสม
โดยคุณสมบัติของ Representation ที่ดีนั้นไม่เพียงแต่จะต้องไม่ขึ้นกับการเคลื่อนที่เชิงการหมุน (Rotational Motion) และ การเคลื่อนที่เชิงเส้นด้วย (Transitional Motion) เท่านั้น 
แต่ควรจะต้องมีความเรียบง่ายและไม่ซับซ้อนหรือยุ่งยากเกินไปในการคำนวณเพื่อสร้าง Machine Code จากพิกัดตำแหน่งคาร์ทคีเชียน (Cartesian Coordinates)

%--------------------------
\section{ลักษณะเฉพาะเชิงโครงสร้าง}
%--------------------------

%--------------------------
\subsection{Internal Coordinates}
%--------------------------

$Z$ matrix หรือเรียกอย่างว่า Internal Coordinates (แปลตรงว่าพิกัดภายในของโมเลกุล) เป็น Representation แบบที่ง่ายมาก 
อาจจะเรียกได้ว่าง่ายที่สุดเลยก็ว่าได้ ถูกใช้เป็น Descriptor มาอย่างยาวนาน โดยถูกใช้อย่างแพร่หลายในยุคแรก ๆ ของการเรียนรู้ของเครื่องสำหรับเคมี
Internal Coordinates สามารถอธิบายได้ทั้งโมเลกุล โดยองค์ประกอบของ Representation อันนี้มีความยาวพันธะระหว่างอะตอม (Bond Distance) 
มุมพันธะ (Bond Angle) และมุมบิดเบี้ยว (Dihedral Angle หรือ Torsion) ซึ่งจำนวนของอะตอมที่ถูกเลือกมาคำนวณ Internal Coordinates 
นั้นมักจะเป็นอะตอมที่เรียงกันอยู่หรืออยู่ใกล้กัน อย่างไรก็ตาม เราสามารถคำนวณหา Internal Coordinates ของโมเลกุลได้โดยพิจารณาอะตอมทุก ๆ คู่
หรือทุกความเป็นไปได้ทั้งหมดภายในโมเลกุล

%--------------------------
\subsection{Geometric Descriptors}
%--------------------------

Geometric Descriptors คือลักษณะเฉพาะเชิงเรขาคณิต

All geometric descriptors that provide information about the spatial coordinates of atoms in 
a molecule relate to the symbolic representation of the molecule. There are several geometric descriptors, 
including the molecular $Z$ matrix, standard and effective coordination numbers. 
The descriptor of the molecular matrix represents each atom's coordinates $(x, y, z)$ in Cartesian space. 
In contrast to most descriptors, these descriptors are able to distinguish isomeric molecules 
(e.g. cis/trans stereoisomers). However, geometric-based representations can still be problematic 
since they omit electronic structure. More sophisticated representations developed in 
the last decade leverage and include electronic descriptions such as atomic force, electron configuration, 
and correlation between orbitals.\cite{musil2021} It can also be difficult to calculate 
the geometric descriptors due to their complexity.\cite{keith2021}

%--------------------------
\subsection{Inverse Distance Matrix}
%--------------------------

%--------------------------
\section{ลักษณะเฉพาะเชิงอิเล็กทรอนิกส์}
%--------------------------

%--------------------------
\subsection{Coulomb Matrix}
%--------------------------

Coulomb Matrix หรือ เมทริกซ์คูลอมบ์

The Coulomb matrix (CM) introduced by \citeauthor{rupp2012} is widely employed because of its simplicity 
and relatively less requirement of \textit{a priori} knowledge of chemical properties of a molecule.\cite{rupp2012}
It stores information about how atoms interact with each other. Each pair of atoms in a molecule carries 
the pairwise electrostatic potential energy. In addition, every atom has a set of Cartesian coordinates 
representing its location in space and has a charge attached to it. The CM between atoms $i$ and $j$ 
is simply given by

\begin{equation}
C_{ij} =
\begin{cases}
 & 0.5 Z_i^{2.4} \text{ if } i = j \\ 
 & \frac{Z_i Z_j}{R_{ij}} \text{ if } i \neq j
\end{cases}
\end{equation}

\noindent where $Z$ is the atomic number and $R_{ij}$ is the interatomic separation. The off-diagonal entries 
of the CM reflect Coulomb's repulsions between the nuclei, and the exponents in the diagonal entries 
correspond to a polynomial fit linking the atomic number to the overall energies of the unbound atoms. 

Despite being invariant to molecular translation and rotation, the CM is not geometrically invariant to 
atom permutations. To solve this puzzle, a variety of alternative representation methods, such as 
Permutation-Invariant Polynomials (PIP)\cite{braams2009}, Randomly Sorted Coulomb Matrices (RSCM)\cite{hansen2013}, 
Bag of Bonds (BoB)\cite{hansen2013}, and Permutation Invariant Vectors (PIV)\cite{gallet2013} have been 
proposed for producing a representation of the molecule that is independent of the ordering of the atoms.

%--------------------------
\subsection{Smooth Overlap of Atomic Positions}
%--------------------------

Smooth Overlap of Atomic Positions (SOAP) can encode atomic geometries in their chemical environment 
using a local expansion of a Gaussian smeared atomic density based on the radial basis ($g_{n}(r)$) and 
real spherical harmonic functions ($Y_{lm}(\theta, \phi)$).\cite{bartok2013,de2016} SOAP is suitable for predicting 
local properties such as atomic forces or chemical shifts but requires partitioning of global properties 
such as total energies. The SOAP kernel between two atomic environments ($\mathcal{X}$ and $\mathcal{X}'$) 
can be retrieved as a normalized polynomial kernel of the partial power spectra. The working equation of 
the SOAP kernel ($K^\mathrm{SOAP}$) retrieved as a normalized polynomial kernel of the two neighbor partial 
power spectra $\mathbf{p}$ and $\mathbf{p}'$ is given by

\begin{equation}
    K^\mathrm{SOAP}(\mathbf{p}, \mathbf{p'}) = \left( \frac{\mathbf{p} \cdot \mathbf{p'}}{\sqrt{\mathbf{p} 
    \cdot \mathbf{p}~\mathbf{p'} \cdot \mathbf{p'}}}\right)^{\xi},
\end{equation}

\noindent where $\xi$ is a positive integer. The elements of the $\mathbf{p}$ vector are defined as 

\begin{equation}
    p^{Z_1 Z_2}_{n n' l} = \pi \sqrt{\frac{8}{2l+1}}\sum_m {c^{Z_1}_{n l m}}^{\dagger} c^{Z_2}_{n' l m}
\end{equation}

\noindent where $n$ and $n'$ are indices for radial basis functions up to $n_{max}$, $l$ is 
the angular degree of the spherical harmonics up to $l_{max}$, $m$ is an integer such that 
$\abs{m} \leq l$, and $Z_{1}$ and $Z_{2}$ are atomic species. The coefficients $c^{Z}_{n'lm}$ and 
${c^{Z}_{nlm}}^{\dagger}$ are defined as the inner products of spherical harmonic functions with 
the Gaussian smoothed atomic density for atoms with the atomic number $Z$ ($\rho^Z$), and its complex conjugate, 
respectively.\cite{de2016}

%--------------------------
\subsection{Atom-centered Symmetry Functions}
%--------------------------

Atom-centered Symmetry Functions (ACSF) generalize the output of multiple two- and three-body functions 
to estimate the local electronic environment near atoms using a fingerprint method that can be customized 
to detect specific structural features for symmetric functions.\cite{behler2011} Radial symmetric functions 
for the central atom $i$ with neighboring atom $j$ are given as follows:

\begin{equation}
    G^{radial}_{i} = \sum^{N_{atom}}_{j \neq i} e^{\eta (\boldsymbol{r}_{ij} - \mu)^{2}} f_{c}(\boldsymbol{r}_{ij})
\end{equation}

\noindent where $\eta$ and $\mu$ are the parameters controlling the width and the position of 
the Gaussian function. $f_{c}$ is a cutoff function that selects the relevant regions close to 
the central nucleus to be encoded into the ACSF and $\boldsymbol{r}_{ij}$ is the displacement between 
the atoms $i$ and $j$. Angular symmetric functions have been defined.\cite{behler2011} 
Moreover, an input required to ACSFs is fine-tuning of internal parameters that can be properly used 
to define the Gaussian function. A similar representation to ACSFs is the Spectrum of London and 
Axilrod-Teller-Muto (SLATM), which has been recently used in the literature, but mostly for 
kernel ridge regression (KRR) models.\cite{faber2018} Polynomial functions of the inverse of 
the interatomic distances have also been suggested but are not discussed in this review.\cite{musil2021}

%--------------------------
\subsection{Gaussian-type Orbital-based Density Vectors}
%--------------------------

Gaussian-type Orbital (GTO)-based Density Vector is a descriptor function alternative to 
the ACSF and symmetric polynomial function.\cite{kwac2021} The GTO-based density vector is given by

\begin{equation}
    \rho^{i}_{L,\alpha,r_{s}} = \sum^{l_{x}+l_{y}+l_{z} = L}_{l_{x},l_{y},l_{z}} 
    \frac{L!}{l_{x}!l_{y}!l_{z}!} \left ( \sum^{n_{type}}_{t=1} c_{t} \sum^{N^{t}_{atom}}_{j=1} 
    \phi^{\alpha,r_{s}}_{l_{x}l_{y}l_{z}} (\boldsymbol{r}_{ij}) \right )
\end{equation}

\noindent where $l_{x}+l_{y}+l_{z} = L$ specifies the orbital angular momentum, $n_{type}$ is the atomic type 
in the molecule under study, $c_{t}$ is the type-dependent weight, $N^{t}_{atom}$ is the number of atoms of 
the type $t$, and $\phi^{\alpha,r_{s}}_{l_{x}l_{y}l_{z}} (r_{ij})$ is the Gaussian orbital centered at 
each atom with the parameters $\alpha$ and $r_{s}$ which determine the radial distributions of orbital functions, 
given as

\begin{equation}
    \phi^{\alpha,r_{s}}_{l_{x}l_{y}l_{z}} (\boldsymbol{r}_{ij}) = x^{l_{x}}y^{l_{y}}z^{l_{z}} e^{-\alpha 
    |r-r_{s}|^{2}}
\end{equation}

\noindent where $\boldsymbol{r}_{ij} = (x,y,z)$ indicates the vector from nuclei $i$ to nuclei $j$ and $r$ 
is the magnitude of $\boldsymbol{r}_{ij}$. GTOs with $L=0,1$ are usually considered for constructing 
the density vectors for small organic molecules containing light atoms such as C, H, O, and N.\cite{kwac2021}
