% LaTeX source for ``การเรียนรู้ของเครื่องสำหรับเคมีควอนตัม (Machine Learning for Quantum Chemistry)''
% Copyright (c) 2022 รังสิมันต์ เกษแก้ว (Rangsiman Ketkaew).

% License: Creative Commons Attribution-NonCommercial-NoDerivatives 4.0 International (CC BY-NC-ND 4.0)
% https://creativecommons.org/licenses/by-nc-nd/4.0/

\chapter{วิธีเคอร์เนล}
\label{ch:kernel}

%--------------------------
\section{เคอร์แนลคืออะไร}
%--------------------------

%--------------------------
\section{ฟังก์ชันเคอร์แนล}
%--------------------------

%--------------------------
\subsection{Linear Regression}
%--------------------------

%--------------------------
\subsection{Ridge Regression}
%--------------------------

%--------------------------
\section{Kernel Ridge Regression}
%--------------------------

Kernel Ridge Regression (KRR) เป็นการต่อยอดจาก Ridge Regression หรืออธิบายง่าย ๆ ว่า KRR คือ RR ในเวอร์ชันที่เป็น Nonlinear problem

%--------------------------
\section{Gaussian Process Regression}
%--------------------------

%--------------------------
\section{Support Vector Machine}
%--------------------------

