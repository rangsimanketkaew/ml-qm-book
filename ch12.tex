% LaTeX source for ``การเรียนรู้ของเครื่องสำหรับเคมีควอนตัม (Machine Learning for Quantum Chemistry)''
% Copyright (c) 2022 รังสิมันต์ เกษแก้ว (Rangsiman Ketkaew).

% License: Creative Commons Attribution-NonCommercial-NoDerivatives 4.0 International (CC BY-NC-ND 4.0)
% https://creativecommons.org/licenses/by-nc-nd/4.0/

\chapter{ซอฟต์แวร์ฝึกสอนปัญญาประดิษฐ์สำหรับเคมีควอนตัม}
\label{ch:ml_lib}

%--------------------------
\section{ไลบรารี่สำหรับคำนวณ Representation}
%--------------------------

%--------------------------
\subsection{DScribe}
%--------------------------

DScribe เป็นไลบรารี่ที่สามารถคำนวณ Representation ได้เยอะมาก\autocite{himanen2020}

คู่มือการใช้งานอ่านได้ที่ \url{https://singroup.github.io/dscribe/latest/index.html}

\noindent ตัวอย่างโค้ดของการใช้ DScribe ในการสร้าง Coulomb Matrix (CM)

\begin{lstlisting}[style=MyPython]
from dscribe.descriptors import CoulombMatrix

atomic_numbers = [1, 8]
rcut = 6.0
nmax = 8
lmax = 6

# Setting up the CM descriptor
cm = CoulombMatrix(
    n_atoms_max=6,
)
\end{lstlisting}

\noindent ตัวอย่างโค้ดของการใช้ DScribe ในการสร้าง SOAP kernel

\begin{lstlisting}[style=MyPython]
from ase.build import molecule

# Molecule created as an ASE.Atoms
water = molecule("H2O")

# Create SOAP output for the system
soap_water = soap.create(water, positions=[0])

print(soap_water)
print(soap_water.shape)

# Create output for multiple system
samples = [molecule("H2O"), molecule("NO2"), molecule("CO2")]
positions = [[0], [1, 2], [1, 2]]
# Serial
coulomb_matrices = soap.create(samples, positions)
# Parallel     
coulomb_matrices = soap.create(samples, positions, n_jobs=2)
\end{lstlisting}

%--------------------------
\section{ไลบรารี่สำหรับสร้างโมเดล}
%--------------------------

%--------------------------
\subsection{SchNetPack}
%--------------------------

SchNetPack เป็นชุดโปรแกรมสำหรับการฝึกสอนโมเดล Neural Network สำหรับการทำนายคุณสมบัติเชิงอะตอม\autocite{schutt2019}
ถูกเขียนโดยภาษา Python 100\% และใช้ไลบรารี PyTorch เป็น Backend สำหรับการสร้างโมเดล SchNet ซึ่งเป็นตัวโมเดลหลักของ SchNetPack
ผู้อ่านสามารถดาวน์โหลดซอร์สโค้ดของ SchNetPack และศึกษาวิธีการติดตั้งและใช้งานได้ที่เว็บไซต์ \url{https://schnetpack.readthedocs.io/}

\noindent Feature หลัก ๆ ของ SchNetPack มีดังนี้

\begin{itemize}
    \item รองรับการสร้างโมเดล SchNet ซึ่งถูกพัฒนาโดยใช้อัลกอริทึม Convolutional Neural Network สำหรับโมเลกุลโดยเฉพาะ%
    \autocite{schutt2017,schutt2017a}
    \item รองรับการสร้างโมเดล PaiNN ซึ่งเป็น Equivariant Message-Passing สำหรับโมเลกุลเช่นเดียวกัน\autocite{schutt2021}
    \item สามารถทำนายค่าเอาต์พุตได้หลากหลาย เช่น Dipole Moments, Polarizability, Stress และคุณสมบัติอื่น ๆ ของโมเลกุล
    \item มีโมดูลสำหรับ Electrostatics และ Ewald Summation
    \item รองรับการเร่งความเร็วด้วย GPU
\end{itemize}

%--------------------------
\subsection{sGDML}
%--------------------------

sGDML เป็นไลบรารี่ที่รับความนิยมในการสร้างโมเดลของโมเลกุลโดยการใช้ค่าพลังงานและแรงในการฝึกสอนโมเดล\autocite{chmiela2019}
คู่มือการใช้งานอ่านได้ที่ \url{http://quantum-machine.org/gdml/doc/}

\noindent เราสามารถติดตั้ง sGDML โดยใช้ Python Package Manager เช่น PIP ได้ด้วยคำสั่งต่อไปนี้

\begin{lstlisting}[style=MyBash]
pip install sgdml
\end{lstlisting}

\noindent ตรวจสอบว่า sGDML ถูกติดตั้งและพร้อมใช้งานหรือไม่

\begin{lstlisting}[style=MyBash]
(sgdml) nutt@SURFACE:~/sGDML$ which sgdml
/home/nutt/sgdml/bin/sgdml

(sgdml) nutt@SURFACE:~$ sgdml
usage: sgdml [-h] [--version] {all,create,train,validate,select,test,show,reset} ...
sgdml: error: the following arguments are required: command
\end{lstlisting}

%--------------------------
\subsection{PiNN}
%--------------------------

PiNN\autocite{shao2020}

%--------------------------
\subsection{TorchANI}
%--------------------------

TorchANI\autocite{gao2020}
