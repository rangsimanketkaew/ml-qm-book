% LaTeX source for ``การเรียนรู้ของเครื่องสำหรับเคมีควอนตัม (Machine Learning for Quantum Chemistry)''
% Copyright (c) 2022 รังสิมันต์ เกษแก้ว (Rangsiman Ketkaew).

% License: Creative Commons Attribution-NonCommercial-NoDerivatives 4.0 International (CC BY-NC-ND 4.0)
% https://creativecommons.org/licenses/by-nc-nd/4.0/

\chapter{การเรียนรู้ของเครื่อง}
\label{ch:ml}

%--------------------------
\section{ความสำคัญของ ML}
%--------------------------
\index{การเรียนรู้ของเครื่อง}

การเรียนรู้ของเครื่องหรือ Machine Learning (ML) เป็นวิทยาการคอมพิวเตอร์ประเภทหนึ่งที่เราทำให้เครื่องจักรสมองกลเกิด \enquote{สติปัญญา} 
ซึ่งกระบวนการดังกล่าวนั้นเรียกว่าการเรียนรู้ของเครื่องจักร ซึ่งในบริบทนี้เครื่องจักรสมองกลที่เรารู้จักกันดีก็คือโปรแกรมที่ถูกติดตั้งอยู่ในคอมพิวเตอร์นั่นเอง
โดยวิธีการเรียนรู้ก็คือเราป้อนข้อมูลและคำตอบเข้าไปให้กับโปรแกรม โปรแกรมจะทำการสร้างโมเดลที่สามารถอธิบายความสัมพันธ์ระหว่างข้อมูลที่เราป้อนเข้าไปได้ 
ซึ่งขั้นตอนที่เกิดขึ้นระหว่างการเรียนรู้ก็คือการฝึกสอนโมเดล (Model training) ซึ่งโปรแกรมสามารถแปลข้อมูลทั้งหมดเป็นโมเดลที่ปรับปรุงได้ 
นั่นหมายความว่าเทคโนโลยีการเรียนรู้ของเครื่องสามารถทำให้คอมพิวเตอร์เรียนรู้วิธีทำงานเลียนเบียบคล้ายมนุษย์ได้นั่นเอง 

%--------------------------
\section{บทบาทของ ML ในควอนตัมเคมี}
%--------------------------
\index{เคมีควอนตัม}

เคมีควอนตัม (Quantum Chemistry) เป็นแขนงหนึ่งของวิชาเคมีเชิงฟิสิกส์ (Physical Chemistry) ซึ่งเป็นการผสมผสานระหว่างกลศาสตร์ควอนตัม 
(Quantum Mechanics) กับการศึกษาอะตอมและโมเลกุลเข้าด้วยกัน กล่าวคือเรานำมาความทางด้านกลศาสตร์มาศึกษาอะตอมและโมเลกุล
โดยนักวิทยาศาสตร์ได้ศึกษาและค้นคว้างานวิจัยศาสตร์ด้านนี้มากว่าหนึ่งศควรรษ นับตั้งแต่ช่วงต้นปี ค.ศ. 1920 โดยได้มีการพัฒนาทฤษฎีต่าง ๆ มากมาย 
แต่สิ่งที่น่าสนใจก็คือจุดเปลี่ยนที่สำคัญของเคมีควอนตัมยุคใหม่ก็คือทฤษฎีฟังก์ชันความหนาแน่น หรือ Density Functional Theory (DFT) 
ซึ่งถูกคิดค้นมากว่าครึ่งศตวรรษ ถ้าหากใครที่เคยเรียนวิชาเคมีเชิงฟิสิกส์ หรือฟิสิกส์เชิงโมเลกุล หรือฟังการนำเสนอผลงานวิชาการตามงานประชุมวิชาการเคมีก็น่าจะเคยได้ยินชื่อทฤษฎีนี้กันมาบ้าง 

DFT เป็นทฤษฎีที่เรานำมาใช้ในการศึกษาคุณสมบัติของโมเลกุล ไม่ว่าจะเป็นขนาดเล็กอย่างเช่นโมเลกุลของสารประกอบอินทรีย์ และอนินทรีย์ 
หรือจะเป็นโมเลกุลขนาดใหญ่ เช่น โปรตีน, วัสดุโลหะ, และพอลิเมอร์ นั่นก็เพราะว่า DFT เป็นวิธีการคำนวณที่ให้ผลแม่นยำและไม่เปลืองพลังในการคำนวณมากนัก 
นั่นจึงทำให้ทฤษฎี DFT ได้รับการเชิดชูเกียรติด้วยรางวัลโนเบลสาขาเคมีในปี ค.ศ. 1998 และถูกนำมาใช้อย่างแพร่หลายในงานวิจัยไม่เพียงแต่ในสาขาเคมีเท่านั้น 
แต่ยังรวมไปถึงสาขาฟิสิกส์และชีววิทยาอีกด้วย แต่ในความเป็นจริงนั้น DFT ไม่ได้ให้ผลการคำนวณที่แม่นยำสูงมากนัก และยังไม่สามารถคำนวณคุณสมบัติของระบบยางระบบได้ 
จึงทำให้ในปัจจุบันนั้นได้มีการพัฒนาระเบียบวิธีใหม่ ๆ ขึ้นมาเพื่อปรับปรุงประสิทธิภาพหรือความสามารถของ DFT ให้เทียบเท่ากับวิธีที่อ้างอิงด้วยวิธีฟังก์ชันคลื่น หรือ Wavefunction Theory (WFT)

ในขณะเดียวกันนั้น ML ก็ถูกนำมาใช้ประโยชน์ในงานวิจัยเคมีมานานกว่า 30 ปีแล้ว แต่ในปัจจุบันนั้น เทคโนโลยีต่าง ๆ เช่น Supercomputing Cloud และ Graphical Processing Unit (GPU)
ได้เข้ามามีบทบาทอย่างมากในวิทยาศาสตร์เชิงคำนวณ (Computaitonal Science) โดยเฉพาะเคมีเชิงคำนวณ (Computational Chemistry) 
จึงทำให้มีจุดเปลี่ยนที่ทำให้ความสนใจของนักวิจัยในช่วง 10 ปีที่ผ่านมานี้ในหันมาทำงานวิจัยโดยใช้ ML กันมากขึ้น นั่นก็เพราะว่าในปัจจุบันนั้น ML สามารถศึกษาได้ง่ายขึ้นเมื่อเทียบกับในอดีต
ทุกวันนี้เราไม่จำเป็นต้องมานั่งเขียนโค้ดเพื่อสร้างโมเดล ML แบบเริ่มจากศูนย์กันแล้ว ตอนนี้เรามี Library แบบ Open-source ต่าง ๆ มากมายให้เลือกใช้ 
เช่น TensorFlow, PyTorch, Scikit-learn, หรือแม้แต่ Matlab ที่ก็มีฟังก์แบบสำเร็จรูปมาให้เราใช้งานได้เลย ซึ่งทำให้เราสามารถเลือกใช้โมเดล ML ต่าง ๆ ได้ตามต้องการ

ขอยกตัวอย่างงานวิจัยหนึ่งที่ตอนนี้กำลังเป็นหัวข้อที่มาแรง (อย่างน้อย ๆ ก็ ณ วันที่ผู้เขียนกำลังเขียนหนังสือเล่มนี้) นั่นคือการใช้ ML สร้างโมเดลที่ใช้ออกแบบ 
Exchange-Correlation (XC) Functional ที่มันมีความ Universal ให้กับ DFT ซึ่งถ้าหากเราทำสำเร็จหรือใกล้เคียง เราจะมีโมเดล XC ที่จะนำไปใช้ในการคำนวณอะไรก็ได้
เรียกได้ว่าเป็น XC สารพัดประโยชน์ (General-purpose) เลยก็ว่าได้ แต่ในความเป็นจริงนั้น XC นั้นก็เปรียบเสมือนเป็นกล่องดำ (Black Box) 
ซึ่งไม่มีใครที่รู้หน้าตาสมการหรือผลเฉลยทั่วไปของมันที่แน่นอน นั่นก็เพราะมันเป็นเทอมที่อธิบายอันตรกิริยาระหว่างอิเล็กตรอน ดังนั้นเราจึงทำได้เพียงหารูปแบบที่เป็นการประมาณเท่านั้น
ตรงจุดนี้เองที่ ML ก็เข้ามามีบทบาท เพราะมันก็คือเป็นการประมาณค่าแบบหนึ่งที่ใช้หลักการทางสถิติเข้ามาช่วยในการหาความสัมพันธ์ระหว่างของสองสิ่ง 
ถึงแม้ว่าตอนนี้มันจะยังอยู่ในขั้นของการพัฒนา แต่สิ่งหนึ่งที่เราเห็นได้เลยก็คือ ML มันช่วยลดระยะเวลาในคำนวณคุณสมบัติเชิงอิเล็กทรอนิกส์ (Electronic Properties) 
ของโมเลกุลอย่างเห็นได้ชัด

\begin{table}[!htp]
    \centering
    \caption{ตารางเปรียบเทียบความซับซ้อนเชิงคำนวณของวิธีทางเคมีควอนตัม\cite{rupp2015}}
    \begin{tabular}{lll}\toprule
    ตัวย่อ &วิธี &Runtime \\\midrule
    FCI &Full Configuration Interaction (CISDTQ) &$\mathcal(O)(N^{10})$ \\
    CC &Coupled Cluster (CCSD(T)) &$\mathcal(O)(N^{7})$ \\
    FCI &Full Configuration Interaction (CISD) &$\mathcal(O)(N^{6})$ \\
    MP2 &M$\o$llor-Plesset second order perturbation theory &$\mathcal(O)(N^{5})$ \\
    QMC &Quantum Monte Carlo &$\mathcal(O)(N^{3}) - \mathcal(O)(N^{4})$ \\
    HF &Hartree-Fock &$\mathcal(O)(N^{3}) - \mathcal(O)(N^{4})$ \\
    DFT &Density Functional Theory (Kohn-Sham) &$\mathcal(O)(N^{3})$ \\
    TB &Tight Binding &$\mathcal(O)(N^{3})$ \\
    MM &Molecular Mechanics &$\mathcal(O)(N^{2})$ \\
    \bottomrule
    \end{tabular}
\end{table}

%--------------------------
\section{เริ่มต้นศึกษา ML}
%--------------------------

การมีความรู้พื้นฐานก่อนเริ่มศึกษา ML อย่างจริงจังนั้นมันเป็นสิ่งสำคัญ ผู้เขียนได้สรุป 5 สิ่งสำคัญที่ควรจะต้องรู้

\begin{enumerate}
    \item \textbf{พีชคณิตเชิงเส้นและแคลคูลัสแบบหลายตัวแปร} : ทั้งสองวิชานี้ถือว่าเป็นรากฐานของ ML เลยก็ว่าได้ 
    เพราะว่าโมเดลทุกรูปแบบของ ML นั้นต่างก็ล้วนแต่เป็นคณิตศาสตร์ ถ้าหากเราต้องการที่จะพัฒนาอังกอริธึมใหม่ ๆ 
    หรือปรับปรุงอัลกอริทึมที่มีอยู่แล้ว เราจะต้องอาศัยความรู้พีชคณิตเชิงเส้น (เวกเตอร์และเมทริกซ์) และแคลคูลัส (การหาอนุพันธ์) 
    แต่ถ้าหากว่าเราเน้นไปทางสายแอพพลิเคชัน เราก็อาจจะไม่จำเป็นต้องรู้แบบลึกหรือละเอียดมากก็ได้ เพราะว่าปัจจุบันนี้มี Library สำเร็จรูปให้เราเลือกใช้มากมาย
    \item \textbf{สถิติ} : เนื่องจากว่าในขั้นตอนก่อนที่จะเริ่มสร้างและเทรนโมเดล ML นั้น เราจะต้องใช้เวลาส่วนใหญ่ 
    (อาจจะมากถึง 80\%) ไปกับการรวบรวมข้อมูล ทำความสะอาดข้อมูล การศึกษาการกระจายตัวของข้อมูล การตั้งและทดสอบสมมติฐาน 
    การทำการถดถอย (Regression) การแยกประเภท (Classification) เราจึงจำเป็นจะต้องใช้สถิติเข้ามาช่วยเพื่อให้เข้าใจถึงรายละเอียด
    ของชุดข้อมูลที่เรากำลังจะเล่นกับมัน ยิ่งเข้าใจข้อมูลมากเท่าไหร่ ยิ่งช่วยให้เราสามารถเลือกใช้โมเดล ML ได้เหมาะสมเท่านั้น 
    \item \textbf{โปรแกรมมิ่ง} : สิ่งสำคัญลำดับถัดมาคือทักษะในการเขียนโปรแกรมหรือเขียนโค้ด ถึงแม้ว่าเราจะมีความรู้ด้านทฤษฎีที่แม่นยำ 
    แต่ถ้าหากเราไม่สามารถเขียนโปรแกรมได้ แล้วก็ไม่สามารถสร้างโมเดลหรือนำ ML มาใช้งานจริงได้เลย ดังนั้นเราควรจะต้องเรียนรู้การเขียนโปรแกรม
    ให้ได้อย่างน้อยสัก 1 ภาษา ซึ่งภาษาที่ได้รับความนิยมมากที่สุดสำหรับงานทางด้านวิทยาศาสตร์ข้อมูล ณ ตอนนี้คือภาษา Python 
    นั่นก็เพราะตัวภาษาเองมี Syntax ที่ง่าย มี Library ให้เลือกใช้เยอะ มี Community ที่ใหญ่มาก ไม่ต้องกลัวเลยว่าถ้าหากมีปัญหา
    เกี่ยวกับการเขียน Python แล้วจะไม่มีคนช่วยหรือหาวิธีแก้ปัญหาไม่ได้
    \item \textbf{แนวคิดของ ML} : แนวคิดหรือ Concept ทางด้าน ML (วิทยาศาสตร์ข้อมูล) เป็นสิ่งที่สำคัญมากเช่นเดียวกัน
    เราควรจะทราบคำศัพท์เฉพาะทางและความหมาย (Terminology) ประเภทของ ML แนวทางการนำ ML มาใช้ (Best Practice)
    \item \textbf{ฝึกทำโจทย์จริง} : ตัวช่วยที่ดีที่สุดให้เราเรียนรู้ ML ได้ง่ายและเร็วนั่นก็คือการฝึกฝน ลองหาโจทย์จริง ๆ มาฝึกทำ
    หรืออาจจะลองเก็บเกี่ยวประสบการณ์โดยเข้าร่วมการแข่งขันวิทยาศาสตร์ข้อมูล ซึ่ง ณ ปัจจุบันก็มีการจัดแข่งขันบ่อยมาก ๆ เรียกได้ว่ามีสนาม
    ให้ได้ฝึกฝนวิทยายุทธเป็นร้อย ๆ พัน ๆ เลย
\end{enumerate}

%--------------------------
\section{คำศัพท์ ML}
%--------------------------

\begin{description}[style=nextline]
    \item[Algorithm] วิธีหรือขั้นตอนกระบวนการคิดคำนวณทางคณิตศาสตร์เพื่อให้ได้ผลลัพธ์ออกมา
    \item[Classification] การจำแนกข้อมูลหรือการทำนายค่าที่มีความไม่ต่อเนื่อง เช่น ประเภทของยานพาหนะ ชนิดของผลไม้
    \item[Data set หรือ Dataset] ชุดข้อมูลที่ได้เตรียมไว้ ประกอบไปด้วยข้อมูล input และ/หรือ output
    \item[Descriptor] Vector ของข้อมูล เช่น Feature vector
    \item[Features/Attribute/Representation] คุณลักษณะเด่นของข้อมูล
    \item[Model]  ชุดคำสั่งหรือโปรแกรมที่ถูกสร้างขึ้นมาโดยมีความสามารถในการคำนวณ ประมวลผลและตัดสินใจ
    \item[Target/Class/Label/Output] คำตอบหรือเป้าหมายที่ต้องการคำนวณ ประมาณค่า หรือทำนาย
    \item[Training] กระบวนการสร้างและฝึกสอน Model โดยใช้ Training set 
    \item[Prediction] กระบวนการทำนายค่าของ Model โดยจะทำนายค่า Output ของข้อมูลใหม่ที่ถูกป้อนเข้าไป
    \item[Regression]  การทำนายค่าที่มีความต่อเนื่อง เช่น ราคาสินค้า ปริมาณน้ำมัน
    \item[Reinforment learning] การเรียนรู้แบบเสริมแรง 
    \item[Supervised learning] การเรียนรู้ของ Model แบบมีผู้สอน (Output)
    \item[Test set] ชุดข้อมูลที่ใช้ทดสอบความถูกต้องและแม่นยำของ Model
    \item[Training set] ชุดข้อมูลที่นำมาใช้ในการสอนคอมพิวเตอร์เพื่อสร้าง Model
    \item[Unsupervised learning] การเรียนรู้ของ Model แบบไม่มีผู้สอน (Output-free)
    ซึ่งยังสามารถแบ่งออกได้เป็นสองประเภทคือ 1. Binary classification กับ 2. Multi-class classification
    \item[Validation set] ชุดข้อมูลสำหรับประเมินประสิทธิภาพของ Model ก่อนที่จะนำไปทดสอบกับ Test set จริง 
    โดย data set ประเภทนี้มักจะถูกนำมาใช้ในการทำ Cross-validation
\end{description}
