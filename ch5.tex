% LaTeX source for ``การเรียนรู้ของเครื่องสำหรับเคมีควอนตัม (Machine Learning for Quantum Chemistry)''
% Copyright (c) 2022 รังสิมันต์ เกษแก้ว (Rangsiman Ketkaew).

% License: Creative Commons Attribution-NonCommercial-NoDerivatives 4.0 International (CC BY-NC-ND 4.0)
% https://creativecommons.org/licenses/by-nc-nd/4.0/

\chapter{การเลือกและปรับแต่งโมเดล}
\label{ch:reg_sel_model}

การทำให้โมเดลมีความถูกต้อง (Regularization) และการเลือกโมเดล (Model Selection)

%--------------------------
\section{Cross Validation}
\idxen{Cross Validation}
%--------------------------

วิธีการตรวจสอบ Model วิธีแรกนี้เป็นวิธีที่ได้รับความนิยมเป็นอย่างมากเพราะว่าสามารถทำได้ง่ายและให้ผลลัพธ์ที่น่าเชื่อถือ นั่นก็คือ \enquote{K-Fold Cross Validation}
หรือเรียกสั้น ๆ ว่า Cross Validation วิธีนี้เริ่มด้วยการแบ่งข้อมูล $k$ ให้มีขนาดของแต่ละส่วนเท่า ๆ กัน หลังจากนั้นทำเก็บข้อมูลหนึ่งส่วนไว้ใช้สำหรับเป็นตัวทดสอบ Model 
นั่นก็คือการทำ Validation แล้วทําวนไปเช่นนี้จนครบจํานวนที่แบ่งไว้ เช่น การทดสอบด้วยวิธี 5-fold Cross Validation
ในรอบแรกเราจะทำการเทรนโมเดลด้วยชุดข้อมูลที่เกิดจากการวมส่วนที่ 2, 3, 4, และ 5 และทำการทดสอบด้วยข้อมูลส่วนที่ 1 
และในรอบที่สองเราจะเปลี่ยนมาเทรนโมเดลด้วยข้อมูลของส่วนที่ 1, 3, 4, และ 5 แล้วนำ Model มาทดสอบด้วยข้อมูลส่วนที่ 2

%--------------------------
\section{การคัดเลือก Feature}
\idxen{Feature Selection}
%--------------------------

การคัดเลือก Feature (Feature Selection) เป็นการหา Feature ที่เหมาะสมที่สุดสำหรับการใช้อธิบายข้อมูลของโมเลกุล โดยเราจะทำการเรียงลำดับความสำคัญของ Feature 
แล้วทำการเลือกเฉพาะ Feature ที่คิดว่าสอดคล้องกับ Output ที่ต้องการทำนาย และคัด Feature ที่มีความสำคัญน้อยออกไปเพื่อหลีกเลี่ยง Bias ที่อาจจะเกิดขึ้น

%--------------------------
\section{ปัญหา Bias-Variance}
\idxen{Bias-Variance}
%--------------------------

%--------------------------
\section{การเพิ่มประสิทธิภาพการเรียนรู้และแก้ปัญหา Overfitting}
%--------------------------

\begin{description} 
    \item[Overfitting] โมเดลตอบสนองต่อ Noise ที่มากเกินไป ทำให้เกิดการเรียนรู้และจดจำ Noise และไม่สามารถที่จะเรียนรู้รายละเอียดจริง ๆ ของข้อมูลได้
    ซึ่งส่งผลให้ทำนายข้อมูลไม่ได้หรือผิดพลาดมากกว่าที่คาดไว้หรือยอมรับได้ โดยกรณีนี้ Model จะมีค่าความแปรปรวนของข้อมูลสูง (High Variance)
    
    \item[Underfitting] โมเดลของเราไม่สามารถหาความสัมพันธ์ระหว่าง Input $X$ กับ Output $Y$ ได้เพราะว่ามีข้อมูลที่ใช้ในการเทรนน้อยเกินไป
    หรือดึงข้อมูลออกมาจาก Training Set ได้ไม่เพียงพอที่จะเรียนรู้ โดยในกรณีนี้ Model จะมีค่าความเอนเอียงสูง (High Bias)

    \item[Noisy] โมเดลไม่มี Overfitting และ Underfitting แต่ยังยังมีค่า Error ของการเรียนรู้ที่ยังสูงอยู่มาก ซึ่งสาเหตุก็อาจจะมาจากการที่ชุดข้อมูล Noise มากเกินไปนั่นเอง
\end{description}

เราสามารถจัดการกับปัญหาข้างต้นด้วยวิธีการต่อไปนี้

%--------------------------
\subsection{Data Augmentation}
\idxen{Data Augmentation}
%--------------------------

%--------------------------
\subsection{Dropout}
\idxen{Dropout}
%--------------------------

%--------------------------
\subsection{Early Stopping}
\idxen{Early Stopping}
%--------------------------

%--------------------------
\subsection{Ensembling}
\idxen{Ensembling}
%--------------------------

%--------------------------
\subsection{Injecting Noise}
\idxen{Injecting Noise}
%--------------------------

%--------------------------
\subsection{L1 Regularization}
\idxen{Regularization!L1}
%--------------------------

%--------------------------
\subsection{L2 Regularization}
\idxen{Regularization!L2}
%--------------------------
