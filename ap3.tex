% LaTeX source for ``การเรียนรู้ของเครื่องสำหรับเคมีควอนตัม (Machine Learning for Quantum Chemistry)''
% Copyright (c) 2022 รังสิมันต์ เกษแก้ว (Rangsiman Ketkaew).

% License: Creative Commons Attribution-NonCommercial-NoDerivatives 4.0 International (CC BY-NC-ND 4.0)
% https://creativecommons.org/licenses/by-nc-nd/4.0/

%--------------------------
\section{เทคนิคการเขียนโมเดล TensorFlow}
\label{ap:coding_tf}
%--------------------------

%--------------------------
\subsection{การปรับแต่ง Loss Function}
%--------------------------

\begin{lstlisting}[style=MyPython]
import tensorflow as tf
import tensorflow.keras.backend as kb
import numpy as np

def custom_loss(y_actual, y_pred): 
    custom_loss=tf.experimental.numpy.log10(kb.sum(kb.abs(y_actual - y_pred)) / y_actual.shape[0])
    return custom_loss

x = np.random.randint(1, 4, size=(1000,))
x = np.asarray(x).T

y = x ** 2
y = np.asarray(y).T
x = x.astype(np.float32)
y = y.astype(np.float32)

keras_model = tf.keras.Sequential(
    [
        tf.keras.layers.Dense(32, activation=tf.nn.relu, input_shape=[1]),
        tf.keras.layers.Dense(32, activation=tf.nn.relu),
        tf.keras.layers.Dense(1),
    ]
)

optimizer = tf.keras.optimizers.RMSprop(0.001)
keras_model.compile(loss=custom_loss, optimizer=optimizer)
keras_model.fit(x, y, batch_size=20, epochs=50)
\end{lstlisting}

%--------------------------
\subsection{การฝึกสอนโมเดลด้วยขนานบน GPU หลายตัว}
%--------------------------
\begin{lstlisting}[style=MyPython]
import tensorflow as tf

# Use all avialable GPUs
mirrored_strategy = tf.distribute.MirroredStrategy()
# Specify which GPU to be used
mirrored_strategy = tf.distribute.MirroredStrategy(devices=["/gpu:0", "/gpu:1"])

with mirrored_strategy.scope():
    model = tf.keras.Sequential([tf.keras.layers.Dense(1, input_shape=(1,))])

model.compile(loss='mse', optimizer='sgd')

dataset = tf.data.Dataset.from_tensors(([1.], [1.])).repeat(100).batch(10)
model.fit(dataset, epochs=2)
model.evaluate(dataset)
\end{lstlisting}
