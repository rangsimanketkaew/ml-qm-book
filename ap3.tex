% LaTeX source for ``การเรียนรู้ของเครื่องสำหรับเคมีควอนตัม (Machine Learning for Quantum Chemistry)''
% Copyright (c) 2022 รังสิมันต์ เกษแก้ว (Rangsiman Ketkaew).

% License: Creative Commons Attribution-NonCommercial-NoDerivatives 4.0 International (CC BY-NC-ND 4.0)
% https://creativecommons.org/licenses/by-nc-nd/4.0/

%--------------------------
\section{เทคนิคการเขียนโมเดล TensorFlow}
\label{ap:coding_tf}
%--------------------------

TensorFlow ถือได้ว่าเป็นไลบรารี่ Neural Network ที่ได้รับความนิยทมากที่สุดในโลกก็ว่าได้ ด้วยฟังก์ชันและโมเดลที่มีให้เลือกใช้งานได้หลากหลาย
ทำให้ TensorFlow ถูกนำมาใช้งานในการสร้างและฝึกสอนโมเดล Neural Network ในงานประเภทต่าง ๆ

%--------------------------
\subsection{การเขียน TensorFlow เบื้องต้น}
%--------------------------

ตัวอย่างด้านล่างคือการสร้างและฝึกสอนโมเดล Neural Network ด้วย Keras ซึ่งเป็น API ของ TensorFlow จะเห็นได้ว่าเราสามารถเขียนโค้ด 
Python โดยเริ่มจากการนำเข้าชุดข้อมูลตัวอย่างซึ่งเป็น Mnist มีการแปลงข้อมูล ตามด้วยการสร้าง Neural Network โดยใช้วิธี Sequential 
กำหนดจำนวนชั้น ประเภทของแต่ละชั้น และมีพารามิเตอร์ที่เราสามารถกำหนดได้ เช่น จำนวนโหนดหรือหน่วยเรียนรู้ของแต่ละชั้นและ Activation 
function หลังจากนั้นก็ทำการ Compile โมเดลซึ่งเราสามารถกำหนด Optimizer และ Loss Function ได้ด้วย เมื่อทำการสร้างโมเดลเสร็จแล้ว 
ก็จะต่อด้วยการฝึกสอนหรือเทรน (Train) โมเดลตามจำนวนรอบ (Epoch) และขั้นตอนสุดท้ายคือการทดสอบโมเดลโดยการทำนายและประเมินผล

\begin{lstlisting}[style=MyPython]
import tensorflow as tf
mnist = tf.keras.datasets.mnist

(x_train, y_train),(x_test, y_test) = mnist.load_data()
x_train, x_test = x_train / 255.0, x_test / 255.0

model = tf.keras.models.Sequential([
  tf.keras.layers.Flatten(input_shape=(28, 28)),
  tf.keras.layers.Dense(128, activation='relu'),
  tf.keras.layers.Dropout(0.2),
  tf.keras.layers.Dense(10, activation='softmax')
])

model.compile(optimizer='adam',
              loss='sparse_categorical_crossentropy',
              metrics=['accuracy'])

model.fit(x_train, y_train, epochs=5)
model.evaluate(x_test, y_test)
\end{lstlisting}

จากโค้ดด้านบนจะเห็นได้เลยว่าจริง ๆ แล้วการเขียนโค้ดการเรียนรู้ของเครื่องโดยเฉพาะ Neural Network ไม่ได้ยากเลย เพราะว่าในยุคสมัยนี้เรามี 
Framework ต่าง ๆ ที่ถูกพัฒนาขึ้นมาเพื่อช่วยอำนวยความสะดวกให้เรากับในการเขียนโค้ด เพียงแค่ไม่กี่สิบบรรทัดเราก็สามารถสร้างโมเดลได้แล้ว
เมื่อเราเข้าใจพื้นฐานในการสร้างโมเดลแล้ว ถ้าหากเราต้องการที่จะต่อยอดโดยการปรับแต่งโมเดลเพื่อให้สามารถจัดการกับงานที่ซับซ้อนขึ้นก็ไม่ใช่เรื่องยาก

%--------------------------
\subsection{การปรับแต่ง Loss Function}
%--------------------------

\begin{lstlisting}[style=MyPython]
import tensorflow as tf
import tensorflow.keras.backend as kb
import numpy as np

def custom_loss(y_actual, y_pred): 
    custom_loss=tf.experimental.numpy.log10(kb.sum(kb.abs(y_actual - y_pred)) / y_actual.shape[0])
    return custom_loss

x = np.random.randint(1, 4, size=(1000,))
x = np.asarray(x).T

y = x ** 2
y = np.asarray(y).T
x = x.astype(np.float32)
y = y.astype(np.float32)

keras_model = tf.keras.Sequential(
    [
        tf.keras.layers.Dense(32, activation=tf.nn.relu, input_shape=[1]),
        tf.keras.layers.Dense(32, activation=tf.nn.relu),
        tf.keras.layers.Dense(1),
    ]
)

optimizer = tf.keras.optimizers.RMSprop(0.001)
keras_model.compile(loss=custom_loss, optimizer=optimizer)
keras_model.fit(x, y, batch_size=20, epochs=50)
\end{lstlisting}

%--------------------------
\subsection{การฝึกสอนโมเดลด้วยขนานบน GPU หลายตัว}
%--------------------------
\begin{lstlisting}[style=MyPython]
import tensorflow as tf

# Use all avialable GPUs
mirrored_strategy = tf.distribute.MirroredStrategy()
# Specify which GPU to be used
mirrored_strategy = tf.distribute.MirroredStrategy(devices=["/gpu:0", "/gpu:1"])

with mirrored_strategy.scope():
    model = tf.keras.Sequential([tf.keras.layers.Dense(1, input_shape=(1,))])

model.compile(loss='mse', optimizer='sgd')

dataset = tf.data.Dataset.from_tensors(([1.], [1.])).repeat(100).batch(10)
model.fit(dataset, epochs=2)
model.evaluate(dataset)
\end{lstlisting}
