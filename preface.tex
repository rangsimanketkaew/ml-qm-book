% LaTeX source for ``การเรียนรู้ของเครื่องสำหรับเคมีควอนตัม (Machine Learning for Quantum Chemistry)''
% Copyright (c) 2022 รังสิมันต์ เกษแก้ว (Rangsiman Ketkaew).

% License: Creative Commons Attribution-NonCommercial-NoDerivatives 4.0 International (CC BY-NC-ND 4.0)
% https://creativecommons.org/licenses/by-nc-nd/4.0/

{
% \pagenumbering{gobble}

\chapter*{\centering คำนำ}
\addcontentsline{toc}{chapter}{คำนำ}

ในปัจจุบันได้มีการนำการเรียนรู้ของเครื่องไปใช้ประโยชน์ในหลากหลายด้าน เช่น คอมพิวเตอร์วิทัศน์, คอมพิวเตอร์กราฟฟิก, ภาษาศาสตร์, เศรษฐศาสตร์, 
อุตสาหกรรม, เกษตรกรรม รวมไปถึงวิทยาศาสตร์และวิศวกรรม โดยสาขาเคมีนั้นก็เป็นอีกหนึ่งศาสตร์ที่ได้มีการนำการเรียนรู้ของเครื่องเข้ามาประยุกต์ใช้%
มาเป็นระยะเวลานานนับตั้งแต่ช่วงปี ค.ศ. 1990 โดยนักวิจัยได้ใช้การเรียนรู้ของเครื่องในการพัฒนายาสำหรับรักษาโรค การศึกษาโครงสร้างโปรตีน 
การศึกษาคุณสมบัติของโมเลกุลและสารประกอบ การออกแบบโมเลกุลและวัสดุชนิดใหม่ ๆ รวมไปถึงการศึกษาปฏิริยาเคมีและตัวเร่งปฏิกิริยา

โดยผู้เขียนได้เล็งเห็นว่าการประยุกต์ใช้การเรียนรู้ของเครื่องกับเคมีควอนตัมนั้นเป็นสิ่งใหม่ที่หลาย ๆ คนต่างก็ให้ความสนใจ ทั้งนักศึกษา อาจารย์ 
และนักวิจัย ดังนั้นผู้เขียนจึงได้เรียบเรียงหนังสือเล่มนี้ขึ้นมาเพื่อเป็นแนวทางให้แก่ผู้ที่ต้องการศึกษาทางด้านนี้ โดยหนังสือเล่มนี้ครอบคลุม%
ทฤษฎีและเทคนิคการเรียนรู้ของเครื่องแบบต่าง ๆ รวมไปถึงทฤษฎีเคมีควอนตัมซึ่งเป็นสาขาหนึ่งของเคมีเชิงฟิสิกส์ที่ช่วยให้เราสามารถเข้าใจองค์ความรู้%
พื้นฐานของอะตอมและโมเลกุลได้เป็นอย่างดี และหัวข้อเคมีควอนตัมที่สามารถนำการเรียนรู้ของเครื่องไปประยุกต์ใช้ได้ รวมไปถึงรายละเอียดเชิงเทคนิค%
ในการพัฒนาวิธีแบบใหม่เพื่อปรับปรุงประสิทธิภาพการพยากรณ์ของแบบจำลองการเรียนรู้ของเครื่อง

ผู้เขียนหวังเป็นอย่างยิ่งว่าหนังสือเล่มนี้จะช่วยให้ผู้อ่านทุกท่านได้รับความรู้และความเข้าใจที่ครบถ้วนเกี่ยวกับการเรียนรู้ของเครื่องสำหรับเคมีควอนตัม

\medskip

\begin{flushright}
รังสิมันต์ เกษแก้ว \\
9 ตุลาคม พ.ศ. 2565
\end{flushright}
}
