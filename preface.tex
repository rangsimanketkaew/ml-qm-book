% LaTeX source for ``การเรียนรู้ของเครื่องสำหรับเคมีควอนตัม (Machine Learning for Quantum Chemistry)''
% Copyright (c) 2022 รังสิมันต์ เกษแก้ว (Rangsiman Ketkaew).

% License: Creative Commons Attribution-NonCommercial-NoDerivatives 4.0 International (CC BY-NC-ND 4.0)
% https://creativecommons.org/licenses/by-nc-nd/4.0/

\chapter*{\centering คำนำ}

ในปัจจุบันมีการนำการเรียนรู้ของเครื่องไปใช้ประโยชน์ในงานหลากหลายด้าน เช่น คอมพิวเตอร์วิทัศน์, คอมพิวเตอร์กราฟฟิก, ภาษาศาสตร์, เศรษฐศาสตร์, 
อุตสาหกรรม, เกษตรกรรม รวมไปถึงวิทยาศาสตร์และวิศวกรรม โดยสาขาเคมีนั้นก็ถือว่าเป็นอีกหนึ่งศาสตร์ที่ได้มีการนำการเรียนรู้ของเครื่อง เข้ามาประยุกต์ใช้
โดยเฉพาะงานวิจัยทางด้านเคมีนั้นมีมานานกว่า 30 ปีแล้ว ไม่ว่าจะเป็นการนำมาใช้ในการพัฒนายาสำหรับรักษาโรค การศึกษาโปรตีน 
การศึกษาคุณสมบัติของโมเลกุลและสารประกอบ การออกแบบโมเลกุลขนาดเล็ก ขนาดใหญ่และวัสดุชนิดใหม่ ๆ รวมไปถึงการศึกษาปฏิริยาเคมี

โดยผู้เขียนได้เล็งเห็นว่าการนำการเรียนรู้ของเครื่อง มาใช้สำหรับเคมีนั้นเป็นศาสตร์แขนงใหม่ที่คนไทยหลาย ๆ คนไม่ว่าจะเป็น นักเรียน นักศึกษา อาจารย์ นักวิจัย
ต่างก็ให้ความสนใจ ดังนั้นผู้เขียนจึงได้เขียนหนังสือเล่มนี้ขึ้นมาเพื่อเป็นแนวทางหรือจุดเริ่มต้นให้แก่ผู้ที่อยากจะศึกษาศาสตร์ทางด้านนี้ 
โดยหัวข้อทางเคมีที่หนังสือเล่มนี้จะลงละเอียดก็คือเคมีควอนตัม ซึ่งเป็นศาสตร์ที่ช่วยให้เราสามารถเข้าใจองค์ความรู้พื้นฐานของโมเลกุลได้เป็นอย่างดี 
ซึ่งในหนังสือเล่มนี้จะมีการอธิบายตั้งแต่ทฤษฎีและเทคนิคการเรียนรู้ของเครื่อง ต่าง ๆ ความรู้ทางควอนตัมเคมีหรือหัวข้อที่สามารถนำการเรียนรู้ของเครื่อง 
เข้าไปประยุกต์ใช้ได้ รวมไปถึงรายละเอียดเชิงเทคนิคในการพัฒนาวิธีหรือเทคนิคใหม่ ๆ เพื่อปรับปรุงประสิทธิภาพของโมเดลการเรียนรู้ของเครื่อง 

สุดท้ายนี้ ผู้เขียนหวังเป็นอย่างยิ่งว่าหนังสือเล่มนี้จะช่วยให้ผู้อ่านทุกท่านได้รับความรู้ ความเข้าใจที่ชัดเจนและครบถ้วนเกี่ยวกับการเรียนรู้ของเครื่องสำหรับเคมีควอนตัม

\medskip

\begin{flushright}
รังสิมันต์ เกษแก้ว
\end{flushright}
