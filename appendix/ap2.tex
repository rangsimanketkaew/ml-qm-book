% LaTeX source for ``การเรียนรู้ของเครื่องสำหรับเคมีควอนตัม (Machine Learning for Quantum Chemistry)''
% Copyright (c) 2022 รังสิมันต์ เกษแก้ว (Rangsiman Ketkaew).

% License: Creative Commons Attribution-NonCommercial-NoDerivatives 4.0 International (CC BY-NC-ND 4.0)
% https://creativecommons.org/licenses/by-nc-nd/4.0/

%--------------------------
\section{ไลบรารี่การเรียนรู้ของเครื่อง}
\label{ap:library_ml}
\idxboth{ไลบรารี่}{Library}
%--------------------------

%--------------------------
\subsection{ไลบรารี่สำหรับการเรียนรู้ของเครื่องแบบทั่วไป}
\idxboth{ไลบรารี่!การเรียนรู้ของเครื่องแบบทั่วไป}{Library!Machine Learning}
%--------------------------

ในปัจจุบันได้มีการพัฒนาเครื่องมือหรือชุดโปรแกรม (Framework) สำหรับงานทางด้านสถิติ วิทยาศาสตร์ข้อมูล และปัญญาประดิษฐ์เป็นจำนวนมาก
และแน่นอนว่าชุดเครื่องมือเหล่านี้ก็มีไลบรารี่สำหรับการเขียนโปรแกรมด้วยภาษาคอมพิวเตอร์ เช่น Python ซึ่งได้รับความนิยมเป็นอันดับนั่นก็เพราะตัว%
ภาษามีไวยากรณ์ (Syntax) ที่เรียนรู้ได้ง่ายไม่ซับซ้อน จึงทำให้ผู้ที่หัดเขียนโปรแกรมเบื้องต้นเลือกใช้ภาษา Python ในการฝึกฝน 

ไลบรารี่ที่ได้รับความนิยมในการเขียนโปรแกรมสำหรับการสร้างโมเดล การทำนายหรือพยากรณ์คำตอบ และการวิเคราะห์ข้อมูล รวมไปถึงการนำเสนอข้อมูล%
ในรูปแบบของกราฟและตารางนั้นมีหลายตัวด้วยกัน โดยผู้เขียนหยิบเลือกมาเฉพาะไลบรารี่ที่หลายคนเลือกใช้

\paragraph{\textbf{NumPy}}
ย่อมาจาก \textit{Numerical Python} ชุดเครื่องมือสำหรับการทำงานเกี่ยวกับ N-dimensional array (Numpy Array) ซึ่งเป็นโครงสร้างข้อมูล%
ที่มีความสำคัญและถูกใช้งานบ่อยมาก นั่นก็เพราะว่า Array นั้นคือโครงสร้างหลักที่เราสามารถนำมาใช้ในการสร้างเวกเตอร์ เมทริกซ์ รวมไปถึงเทนเซอร์
ซึ่งเราจะนำมาใช้ในการเก็บข้อมูลที่มีขนาดหลายมิติ เช่น Feature โดยสามารถเก็บให้อยู่ในรูปของเวกเตอร์แบบ 1 มิติหรือจะเป็นเมทริกซ์แบบ 2 มิติก็ได้
นอกจากนี้ NumPy ยังเป็นไลบรารี่พื้นฐานของไลบรารี่อื่น ๆ อีกมายมาก ไม่ว่าจะเป็น SciPy, Scikit-Learn และ SymPy เป็น
\idxen{Library!NumPy}

\paragraph{SciPy} 
ย่อมาจาก \textit{Scientific Python} เป็นไลบรารี่ที่ถูกพัฒนาต่อยอดมาจาก NumPy โดยใช้ Array เป็นโครงสร้างข้อมูล
โดย SciPy จะมีโมดูล (Module) ที่จะเน้นไปทางการคำนวณทางด้านคณิตศาสตร์และวิทยาศาสตร์ มีฟังก์ชันพื้นฐานให้เราเลือกใช้มากมาย เช่น 
พีชคณิตเชิงเส้น (Linear Algebra), สถิติและการวิเคราะห์ และการแก้สมการเชิงอนุพันธ์สามัญ (Ordinary Differential Equation)
\idxen{Library!SciPy}

\paragraph{Scikit-Learn}
ไลบรารี่ที่ถูกนำมาใช้ในการสร้างโมเดล ML แบบที่เป็น Regression และ Classification มากที่สุดตัวหนึ่ง โดยสามารถสร้างโมเดลได้ทั้งแบบ 
Supervised ML และ Unsupervised ML โดยมีโมเดลให้เราเลือกใช้งานหลายประเภท เช่น Linear Regression, Logistic Regression, 
Ridge Regression, Support Vector Machines, Random Forests และ Nearest Neighbors นอกจากนี้ Scikit-Learn ยังมีฟังก์ชัน%
ในการจัดการข้อมูล จัดแบ่งข้อมูลและวัดผลโมเดลด้วย อย่างไรก็ตาม Scikit-Learn ไม่ได้ถูกออกแบบมาเพื่อสร้างโมเดล Neural Network
\idxen{Library!Scikit-Learn}

\paragraph{Pandas}
ไลบรารี่ยอดฮิตที่หลายคนเลือกใช้เพื่อนำมาจัดการและวิเคราะห์ข้อมูลในรูปแบบของตาราง (คล้าย ๆ Excel) โดยมีการจัดเก็บโครงสร้างของข้อมูล%
ในรูปแบบ DataFrame ซึ่งเปรียบเสมือกเป็นตารางสำหรับข้อมูล 2 มิติ โดย Pandas มีฟังก์ชันที่อำนวยความสะอวดในการจัดการข้อมูลในตาราง (Cell) 
และยังทำงานร่วมกับ NumPy ได้อีกด้วย
\idxen{Library!Pandas}

\paragraph{Matplotlib}
ไลบรารี่สำหรับพล็อตกราฟแบบต่าง ๆ ทั้ง 2 มิติและ 3 มิติ ซึ่งช่วยให้เราวิเคราะห์ข้อมูลได้ง่ายขึ้น โดยผู้ใช้งานสามารถนำเข้าข้อมูลประเภทแบบ 
List, Array หรือ DataFrame เช่น กราฟเส้น (Line Plots), กราฟแท่ง (Bar Charts), ฮิสโตแกรม (Histograms), แผนภูมิวงกลม 
(Pie Charts) และแผนภูมิกระจาย (Scatter Plots) รวมไปถึงพื้นผิว (Surface Plot)
\idxen{Library!Matplotlib}

\paragraph{Anaconda}
ชุดซอฟต์แวร์ที่ช่วยให้เราสามารถติดตั้งและจัดการไลบรารี่สำหรับ Python ที่กล่าวมาข้างต้นได้หมดทุกตัว โดยมีข้อดีคือเราสามารถสร้าง Environment 
หลาย ๆ อันสำหรับแต่ละโปรเจกต์ได้ และใน Environment นั้น ๆ เราก็สามารถติดตั้งไลบรารี่ด้วยเวอร์ชั่นที่เราต้องการได้ ช่วยให้แก้ปัญหาการขัดแย้ง%
กันระหว่างเวอร์ชั่นที่ต่างกันของไลบรารี่ตัวเดียวกันได้
\idxen{Library!Anaconda}

%--------------------------
\subsection{ไลบรารี่สำหรับการเรียนรู้เชิงลึก}
\idxboth{ไลบรารี่!การเรียนรู้เชิงลึก}{Library!Deep Learning}
%--------------------------

ในการสร้างโมเดล Neural Network ที่มีความซับซ้อนนั้นไลบรารี่ที่กล่าวมาก่อนหน้านี้นั้นไม่สามารถทำได้ ดังนั้นเราจะต้องใช้ไลบรารี่ที่ถูกออกแบบ%
มาโดยเฉพาะสำหรับการสร้างโมเดลและฝึกสอนโมเดล ซึ่งในปัจจุบันมีไลบรารี่ 2 ตัวที่ได้รับความนิยมสำหรับ Neural Network ดังนี้

\paragraph{TensorFlow}
TensorFlow เป็นไลบรารี่ที่พัฒนาโดย Google สามารถใช้งานได้ฟรีและยังเป็นไลบรารี่แบบ Open Source เหมาะสำหรับการคำนวณเชิงตัวเลขที่รวดเร็ว 
เราสามารถใช้ TensorFlow สร้างโมเดล Neural Network และปรับแต่งโมเดลได้ตามต้องการเนื่องจากว่าตัว Framework นั้นยืดหยุ่นมาก
\idxen{Library!TensorFlow}

\paragraph{PyTorch} 
PyTorch เป็นไลบรารีที่พัฒนาขึ้นมาแข่งกับ TensorFlow ซึ่งในช่วงหลังได้รับการสนับสนุนโดย Facebook โดยจุดประสงค์หลักของ PyTorch 
ก็คือพัฒนาเพื่อใช้กับคอมพิวเตอร์วิทัศน์ (Computer Vision) และการประมวลผลภาษาธรรมชาติ (Natural Language Processing) 
จุดเด่นของ PyTorch ก็คือสามารถใช้ GPU ในการเพิ่มความเร็วในการฝึกสอนได้อย่างมีประสิทธิภาพ
\idxen{Library!PyTorch}

ไลบรารี่ทั้งคู่สามารถทำงานได้อย่างมีประสิทธิภาพพอ ๆ กัน\footnote{จริง ๆ แล้วมีบทความที่เปรียบเทียบประสิทธิภาพระหว่าง TensorFlow และ 
PyTorch แต่โดยส่วนตัวแล้วผู้เขียนคิดว่าความสามารถก็ไม่ได้ต่างกันมาก สิ่งที่มีผลจริง ๆ สำหรับการใช้ Neural Network คือการเลือกใช้โมเดลมากกว่า}
ซึ่งทั้งสองตัวก็มีทั้งข้อดีและข้อเสียต่างกัน ดังนั้นการเลือกใช้ไลบรารี่จึงขึ้นอยู่กับความถนัดและความชอบของแต่ละคน นอกจากนี้ไลบรารี่ทั้งคู่ยังรองรับ
GPU ในการฝึกสอนโมเดลได้อย่างมีประสิทธิภาพอีกด้วย

%--------------------------
\subsection{ไลบรารี่สำหรับการเรียนรู้เชิงลึก}
\idxboth{ไลบรารี่!การติดตั้ง}{Library!Installation}
%--------------------------

การติดตั้งชุดโปรแกรมหรือชุดคำสั่งที่เขียนด้วยภาษา Python นั้นสามารถทำได้ง่าย ๆ ผ่านตัวติดตั้งชุดโปรแกรมสองตัวซึ่งได้รับความนิยมมากที่สุด นั่นคือ

\begin{description}
    \item[pip] เป็นตัวจัดการชุดชุดโปรแกรมของ Python ซึ่งเป็นตัวติดตั้งแบบมาตรฐาน โดยจะทำการเชื่อมต่อไปยัง Repository ของชุดโปรแกรม%
    ที่ชื่อว่า Python Package Index (PyPI) โดยฐานข้อมูลของ PyPI นั้นเข้าไปดูได้ที่ \url{https://pypi.org/}
    
    \item[conda] เป็นตัวจัดการชุดคำสั่งและดูระบบต่าง ๆ ของเครื่อง ซึ่ง Repository ที่ conda ใช้นั้นจะแยกออกมาจาก PyPI โดย conda 
    มีความจุดเด่นในเรื่องของการจัดการ Dependencies ต่าง ๆ ของชุดโปรแกรมที่ถูกติดตั้งมาจากทั้งภายนอกและภายใน นอกจากภาษา Python 
    แล้ว conda ยังรอบรับการจัดการของภาษาอื่น ๆ ได้อีกด้วย เช่น R, Ruby, Lua, Scala, Java, JavaScript, C/\cpp, และ Fortran
    ซึ่ง conda นี้เป็นหนึ่งในผลิตภันฑ์ของ Anaconda บริษัทที่ให้คำแนะนำและบริการเทคโลยีเกี่ยวกับการจัดการชุดโปรแกรมภายในองค์กร
    โดยฐานข้อมูลของ Conda นั้นเข้าไปดูได้ที่ \url{https://anaconda.org/}
\end{description}

จากประสบการณ์ส่วนตัวของผู้เขียนนั้น pip จะได้เปรียบในเรื่องของการเวอร์ชั่นของชุดโปรแกรมที่จะได้รับการอัพเดทอยู่ตลอดเวลา นั่นก็เพราะว่า PyPI
เป็น Repository ที่นักพัฒนาชุดโปรแกรมของ Python ส่วนใหญ่นั้นจะเลือกใช้ในการจัดเก็บชุดโปรแกรมของตนเอง แต่จุดอ่อนอย่างหนึ่งของ pip 
ก็คือการตรวจสอบความขัดแย้ง (Conflict) ระหว่างเวอร์ชั่นและ Dependencies ของโปรแกรมหลาย ๆ โปรแกรมในกรณีที่ต้องใช้ร่วมกัน 
ซึ่งตรงจุดนี้เองที่ conda สามารถทำได้ดีกว่า pip กล่าวคือ ในการติดตั้งชุดโปรแกรมนั้น conda จะทำการตรวจสอบก่อนว่าโปรแกรมนั้น ๆ 
จะมีปัญหาที่ซ้อนทับกับโปรแกรมอื่น ๆ หรือไม่ เช่น อาจจะใช้โปรแกรมอีกตัวหนึ่งร่วมกันแต่ว่าใช้เวอร์ชั่นที่ต่างกัน ซึ่ง conda ก็จะหลีกเลี่ยงปัญหานี้
นอกจากนี้ยังมีการจัดการแบ่งสิ่งแวดล้อม (Environment) แยกสำหรับแต่ละโปรเจ็คเพื่อป้องกันปัญหาดังกล่าวได้อีกด้วย

ตัวอย่างด้านล่างคือการติดตั้ง TensorFlow บนระบบปฏิบัติการ Linux หรือ Windows Subsystem for Linux 2 (WSL2)
\noindent ณ วันที่ผู้เขียนกำลังเขียนบทนี้ TensorFlow เวอร์ชั่นเสถียร (Stable Version) คือ 2.10.0 ซึ่งรอบรับ Python 3.7-3.10%
\footnote{รายละเอียดของ Configuration ของ TensorFlow สามารถดูได้บนเว็บไซต์หลัก}

\subsubsection{ติดตั้ง TensorFlow ด้วย pip สำหรับกรณีที่รองรับ CPU อย่างเดียว}

\begin{lstlisting}[style=MyBash]
python3 -m pip install tensorflow
\end{lstlisting}

หลังจากนั้นสามารถตรวจสอบการติดตั้งและเรียกใช้งาน TensorFlow ได้ตามปกติโดยรันคำสั่งต่อไปนี้

\begin{lstlisting}[style=MyBash]
python3 -c "import tensorflow as tf; print(tf.reduce_sum(tf.random.normal([1000, 1000])))"
\end{lstlisting}

\subsubsection{ติดตั้ง TensorFlow ด้วย conda สำหรับกรณีที่รองรับ GPU}

สำหรับกรณีที่ต้องการติดตั้ง TensorFlow ที่รองรับการทำงานร่วมกับ GPU ด้วยนั้น ผู้เขียนแนะนำให้ติดตั้งด้วย conda เพราะว่าสามารถติดตั้ง Driver
ของการ์ดจอและ CUDA toolkit ได้โดยอัตโนมัติ ซึ่งจะแตกต่างจากกรณีที่ติดตั้งด้วย pip ซึ่งเราจะต้องทำการติดตั้ง Driver ของ GPU และ 
CUDA toolkit เอง

\begin{lstlisting}[style=MyBash]
conda update --all -y
conda install tensorflow-gpu
\end{lstlisting}

หลังจากนั้นสามารถตรวจสอบการติดตั้งและยืนยันว่า TensorFlow สามารถตรวจพบ GPU ของเครื่องได้โดยรันคำสั่งต่อไปนี้

\begin{lstlisting}[style=MyBash]
python3 -c "import tensorflow as tf; print(tf.config.list_physical_devices('GPU'))"
\end{lstlisting}

ในกรณีข้างบนที่ใช้ conda ในการตั้งติด TensorFlow นั้นตัว conda จะทำการค้นหาแพคเกจของ TensorFlow ในแชนแนลหลักที่เป็นค้าเริ่มต้นก่อน
นั่นคือแชนแนล anaconda โดยจะทำการติดตั้ง TensorFlow เวอร์ชั่นล่าสุดที่แชนแนลนี้มี ซึ่งเวอร์ชั่นล่าสุดที่แชนแนลนี้มีก็ไม่ใช่เวอร์ชั่นล่าสุดที่มีใน
PyPI ดังนั้นถ้าหากเราต้องการที่จะติดตั้ง TensorFlow เวอร์ชั่นล่าสุดที่เทียบเท่ากับการติดตั้งด้วย pip เราจะต้องเปลี่ยนแชนแนลที่จะให้ conda
นั้นไปทำการค้นหาแพคเกจ ซึ่งสามารถทำได้โดยใช้คำสั่งต่อไปนี้ซึ่งผมกำหนดให้แชนแนลเป็น \inlinehighlight{conda-forge}

\begin{lstlisting}[style=MyBash]
    conda install --channel conda-forge tensorflow-gpu
\end{lstlisting}
