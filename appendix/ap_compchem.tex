% LaTeX source for ``การเรียนรู้ของเครื่องสำหรับเคมีควอนตัม (Machine Learning for Quantum Chemistry)''
% Copyright (c) 2022 รังสิมันต์ เกษแก้ว (Rangsiman Ketkaew).

% License: Creative Commons Attribution-NonCommercial-NoDerivatives 4.0 International (CC BY-NC-ND 4.0)
% https://creativecommons.org/licenses/by-nc-nd/4.0/

%--------------------------
\chapter{โปรแกรมทางด้านเคมีควอนตัม}
\label{ch:qm_software}
%--------------------------
\idxboth{โปรแกรมเคมีควอนตัม}{Quantum Chemistry Software}

%--------------------------
\section{Gaussian}
\label{sec:software_gaussian}
\idxen{Quantum Chemistry Software!Gaussian}
%--------------------------

\begin{figure}[H]
    \centering
    \includegraphics[width=0.25\linewidth]{fig/logo_gaussian.jpg}
    \label{fig:logo_gaussian}
\end{figure}

โปรแกรม Gaussian เป็นโปรแกรมที่เรียกได้ว่าเป็นตำนานของโปรแกรมทางด้านเคมีควอนตัม นั่นก็เพราะว่า Gaussian ได้ถูกพัฒนามาอย่างยาวนาน 
ซึ่งถือว่าเป็นโปรแกรมแรกของงานวิจัยสายนี้เลยก็ว่าได้ โดย Gaussian ได้ถูกพัฒนาขึ้นในกลุ่มวิจัยของศาสตราจารย์ John A. Pople ในช่วงปี ค.ศ.
1970 และมีการพัฒนาต่อเรื่อยมาจนถึงปัจจุบัน โดยเวอร์ชั่นล่าสุดของ Gaussian (ณ วันที่ผู้เขียนเขียนหนังสือเล่มนี้) คือเวอร์ชั่น 16\autocite{g16}

คุณสมบัติหรือ Feature ของโปรแกรม Gaussian นั้นคือสามารถคำนวณคุณสมบัติเชิงอิเล็กทรอนิกส์ของโมเลกุลขนาดเล็ก (ไม่เกิน 50 อะตอม) 
ขนาดกลาง (50 - 120 อะตอม) และขนาดใหญ่ (มากกว่า 120 อะตอม) ได้อย่างแม่นยำ\footnote{ความแม่นยำและความถูกต้องของผลการคำนวณ%
อ้างอิงตามประสบการณ์ของผู้เขียน โดยมีปัจจัยที่ส่งผลต่อค่าความถูกต้อง เช่น วิธีที่ใช้ในการคำนวณและ Basis Set} โดยจุดเด่นของ Gaussian 
ก็คือการคำนวณคุณสมบัติเชิงอิเล็กทรอนิกส์ของโมเลกุลด้วยวิธี DFT และด้วยอัลกอริทึมของตัวโปรแกรมนั้น ทำให้ Gaussian ได้รับการยอมรับว่าเป็น%
หนึ่งในโปรแกรมที่ให้ผลการคำนวณที่ถูกต้องและน่าเชื่อถือ และสามารถนำไปเปรียบเทียบกับผลการทดลองได้ สำหรับ Gaussian นั้นรองรับการคำนวณ%
แบบวิธี OpenMP นั่นคือสามารถทำการประมวลผลแบบขนาดได้โดยใช้หน่วยประมวลผล CPU หลายตัวพร้อม ๆ กันได้ และนอกจากนี้แล้วในเวอร์ชั่น 16
ตัวโปรแกรมยังรองรับกราฟฟิคการ์ด GPU สำหรับการคำนวณโดยใช้วิธี DFT อีกด้วย 

ตัวอย่างไฟล์อินพุตของโปรแกรม Gaussian สำหรับการคำนวณพลังงานของโมเลกุล Formaldehyde ด้วยวิธี Hartree-Fock

\begin{lstlisting}[style=plain]
%chk=formaldehyde.chk
%mem=128MB
#P HF/6-31G(d) scf=tight

HF/6-31G(d) sp formaldehyde

0 1
C1
O2  1  r2
H3  1  r3  2  a3
H4  1  r4  2  a4  3  d4

r2=1.20
r3=1.0
r4=1.0
a3=120.
a4=120.
d4=180.
\end{lstlisting}

\vspace{1em}

รายละเอียดเพิ่มเติมเกี่ยวกับโปรแกรม Gaussian ดูได้ที่เว็บไซต์ \url{https://gaussian.com}

%--------------------------
\section{ORCA}
\label{sec:software_orca}
\idxen{Quantum Chemistry Software!ORCA}
%--------------------------

\begin{figure}[H]
    \centering
    \includegraphics[width=0.4\linewidth]{fig/logo_orca.png}
    \label{fig:logo_orca}
\end{figure}

โปรแกรม ORCA เป็นอีกหนึ่งโปรแกรมทางเคมีควอนตัมที่มีประสิทธิภาพและความสามารถในการคำนวณสูง\autocite{neese2012,neese2018} 
สามารถคำนวณได้หลายวิธี โดยสามารถคำนวณ DFT และวิธี Semi-empirical ได้ รวมไปถึงวิธี Post Hartree-Fock อื่น ๆ ด้วย โดย ORCA 
ถูกใช้อย่างแพร่หลายในงานวิจัยทางด้านเคมีอินทรีย์และเคมีอนินทรีย์ โดยเฉพาะการศึกษาสารประกอบเชิงซ้อนของโลหะทรานซิชัน (Transition 
Metal Complex) ซึ่งเป็นโมเลกุลที่มีขนาดใหญ่และมีความซับซ้อนในเชิงของโครงสร้างอิเล็กทรอนิกส์มากกว่าโมเลกุลอินทรีย์ขนาดเล็ก และ ORCA 
ยังมีความโดดเด่นในด้านของความแม่นยำและความเร็วในการคำนวณเกี่ยวกับคุณสมบัติเชิงสเปกตรัมของโมเลกุล

โปรแกรม ORCA ถูกพัฒนาในกลุ่มวิจัยของศาสตราจารย์ Frank Neese โดยสามารถดาวน์โหลดตัวโปรแกรม (เฉพาะไฟล์ Binary ที่ถูกคอมไพล์แล้ว)
มาใช้ได้ฟรีสำหรับวัตถุประสงค์ด้านการศึกษาและการทำงานวิจัย 

ตัวอย่างไฟล์อินพุตของโปรแกรม ORCA สำหรับการคำนวณพลังงานของโมเลกุลน้ำด้วยวิธี Hartree-Fock

\begin{lstlisting}[style=plain]
!HF DEF2-SVP

%SCF
   MAXITER 500
END

* xyz 0 1
O   0.0000   0.0000   0.0626
H  -0.7920   0.0000  -0.4973
H   0.7920   0.0000  -0.4973
*
\end{lstlisting}

\vspace{1em}

รายละเอียดเพิ่มเติมเกี่ยวกับโปรแกรม ORCA ดูได้ที่เว็บไซต์ \url{https://orcaforum.kofo.mpg.de/app.php/portal} หรือดูคู่มือและ%
แบบฝึกหัดสอนการใช้โปรแกรมได้ที่เว็บไซต์ \url{https://www.orcasoftware.de/tutorials_orca/index.html}

%--------------------------
\section{NWChem}
\label{sec:software_nwchem}
\idxen{Quantum Chemistry Software!NWChem}
%--------------------------

\begin{figure}[H]
    \centering
    \includegraphics[width=0.8\linewidth]{fig/logo_nwchem.png}
    \label{fig:logo_nwchem}
\end{figure}

โปรแกรม NWChem เป็นโปรแกรมการคำนวณทางเคมีควอนตัมและพลศาสตร์เชิงโมเลกุล (Molecular Dynamics)\autocite{apra2020} 
พัฒนาโดยสถาบัน Pacific Northwest National Laboratory (PNNL) ในช่วงปี ค.ศ. 1990 โดยรองรับการคำนวณด้วยวิธี DFT และ 
Post Hartree-Fock เช่น M\"{o}llor-Plesset (MP), Configuration Interaction (CI), Coupled Cluster (CC) และ
Multiconfiguration SCF (MCSCF) 

NWChem นั้นถูกพัฒนาเพื่อให้สามารถประมวลผลบน Supercomputer ที่มีประสิทธิภาพสูงได้ NWChem ถูกเขียนขึ้นโดยใช้ภาษา Fortran 77
และใช้ไลบรารี่ทางด้านพีชคณิตสำหรับการประมวลผล เช่น BLAS, LAPACK, และ ScaLAPACK และสามารถประมวลผลแบบขนาดได้โดยใช้วิธี 
Message-Passing Interface (MPI) นอกจากนี้ NWChem ยังรองรับการประมวลผลด้วย GPU สำหรับการคำนวณด้วยวิธี Coupled Cluster 
ซึ่งถือว่าเป็นจุดเด่นของ NWChem เลยก็ว่าได้ โปรแกรม NWChem เป็นแบบ Open-source มีการพัฒนาอย่างต่อเนื่องเรื่อยมาจนถึงปัจจุบัน ซึ่งนักวิจัย 
นักศึกษา และคนทั่วไปสามารถร่วมพัฒนาและใช้งานโปรแกรมได้ฟรี

สำหรับผู้อ่านที่ใช้งานระบบปฏิบัติการ Debian หรือ Ubuntu นั้นก็สามารถติดตั้งโปรแกรม NWChem ได้อย่างง่ายดายด้วยคำสั่ง

\begin{lstlisting}[style=MyBash]
sudo apt-get install nwchem
\end{lstlisting}

\vspace{1em}

ตัวอย่างไฟล์อินพุตของโปรแกรม NWChem สำหรับการคำนวณพลังงานของโมเลกุลน้ำด้วยวิธี Hartree-Fock

\begin{lstlisting}[style=plain]
start h2o 
title "Water in 6-31g basis set" 

geometry units au  
  O      0.00000000    0.00000000    0.00000000  
  H      0.00000000    1.43042809   -1.10715266  
  H      0.00000000   -1.43042809   -1.10715266 
end  

basis  
  H library 6-31g  
  O library 6-31g  
end

task scf
\end{lstlisting}

\vspace{1em}

รายละเอียดเพิ่มเติมเกี่ยวกับโปรแกรม NWChem ดูได้ที่เว็บไซต์ \url{https://nwchemgit.github.io}

%--------------------------
\section{PySCF}
\label{sec:software_pyscf}
\idxen{Quantum Chemistry Software!PySCF}
%--------------------------

\begin{figure}[H]
    \centering
    \includegraphics[width=0.5\linewidth]{fig/logo_pyscf.png}
    \label{fig:logo_pyscf}
\end{figure}

โปรแกรม PySCF เป็นโปรแกรมสำหรับการคำนวณทางเคมีควอนตัมซึ่งถูกเขียนขึ้นมาโดยใช้ภาษา Python โดยทีมนักวิจัยจาก California Institute 
of Technology\autocite{sun2018} จุดเด่นของโปรแกรม PySCF ก็คือมีขนาดที่เล็ก (Lightweight) และมีโมดูลในการคำนวณโครงสร้าง%
เชิงอิเล็กทรอนิกส์ที่หลากหลาย สามารถพัฒนาต่อได้ง่ายไม่ซับซ้อนเพราะว่าใช้ภาษา Python ในการเขียน และใช้งานได้ง่าย เหมาะกับผู้ที่ต้องการเขียน%
หรือแก้ไขโค้ดโปรแกรมทางเคมีควอนตัม นอกจากนี้ผู้ใช้งานสามารถใช้ PySCF ในการคำนวณคุณสมบัติของโมเลกุล ปรับแต่ง Hamiltonians ของ 
Wavefunction โดยใช้ทฤษฎีสนามเฉลี่ย (Mean-Field Theory)

โค้ดเกือบทั้งหมดของโปรแกรม PySCF ถูกเขียนด้วยภาษา Python แต่ว่าโค้ดของโปรแกรมบางส่วนที่เกี่ยวข้องกับการคำนวณที่ซับซ้อนนั้นถูกเขียนด้วย%
ภาษา C ซึ่งช่วยลดระยะเวลาในการคำนวณและทำให้ PySCF นั้นมีประสิทธิภาพเทียบเท่าได้กับโปรแกรมคำนวณทางเคมีควอนตัมโปรแกรมอื่น ๆ ที่ช่วย%
ด้วยภาษาระดับล่าง เช่น C หรือ Fortran นอกเหนือจากโมดูลหลักที่ใช้ในการคำนวณโครงสร้างเชิงอิเล็กทรอนิกส์

การติดตั้งโปรแกรม PySCF นั้นก็สามารถทำได้ง่ายมาก เพียงแค่ผู้ใช้งานมีโปรแกรม Python เวอร์ชั่น 3 ก็สามารถติดตั้งได้โดยใช้คำสั่งต่อไปนี้

\begin{lstlisting}[style=MyBash]
pip install pyscf[geomopt]
\end{lstlisting}

\vspace{1em}
\noindent ซึ่งจะโมดูล Geometry Optimization ของ PySCF ด้วย ถ้าหากไม่ต้องการติดตั้งโมดูลนี้ก็ให้ใช้แค่ \bashinline{pyscf}

\begin{lstlisting}[style=plain]
from pyscf import gto, scf

mol = gto.Mole()
mol.verbose = 5
#mol.output = 'out_h2o'
mol.atom = '''
O        0.000000    0.000000    0.117790
H        0.000000    0.755453   -0.471161
H        0.000000   -0.755453   -0.471161'''
mol.basis = 'ccpvdz'
mol.symmetry = 1
mol.build()

mf = scf.RHF(mol)
mf.kernel()
\end{lstlisting}

\vspace{1em}

รายละเอียดเพิ่มเติมเกี่ยวกับโปรแกรม PySCF ดูได้ที่เว็บไซต์ \url{https://pyscf.org}
