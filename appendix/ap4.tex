% LaTeX source for ``การเรียนรู้ของเครื่องสำหรับเคมีควอนตัม (Machine Learning for Quantum Chemistry)''
% Copyright (c) 2022 รังสิมันต์ เกษแก้ว (Rangsiman Ketkaew).

% License: Creative Commons Attribution-NonCommercial-NoDerivatives 4.0 International (CC BY-NC-ND 4.0)
% https://creativecommons.org/licenses/by-nc-nd/4.0/

%--------------------------
\section{โปรแกรมทางด้านเคมีควอนตัม}
\label{ap:qm_software}
%--------------------------
\idxboth{โปรแกรมเคมีควอนตัม}{Quantum Chemistry Software}

%--------------------------
\subsection{Gaussian}
\label{ssec:software_gaussian}
\idxen{Quantum Chemistry Software!Gaussian}
%--------------------------

โปรแกรม Gaussian เป็นโปรแกรมที่เรียกได้ว่าเป็นตำนานของโปรแกรมทางด้านเคมีควอนตัม นั่นก็เพราะว่า Gaussian ได้ถูกพัฒนามาอย่างยาวนาน 
ซึ่งถือว่าเป็นโปรแกรมแรกของงานวิจัยสายนี้เลยก็ว่าได้ โดย Gaussian ได้ถูกพัฒนาขึ้นในกลุ่มวิจัยของศาสตราจารย์ John A. Pople ในช่วงปี ค.ศ.
1970 และมีการพัฒนาต่อเรื่อยมาจนถึงปัจจุบัน โดยเวอร์ชั่นล่าสุดของ Gaussian (ณ วันที่ผู้เขียนเขียนหนังสือเล่มนี้) คือเวอร์ชั่น 16\autocite{g16}

คุณสมบัติหรือ Feature ของโปรแกรม Gaussian นั้นคือสามารถคำนวณคุณสมบัติเชิงอิเล็กทรอนิกส์ของโมเลกุลขนาดเล็ก (ไม่เกิน 50 อะตอม) 
ขนาดกลาง (50 - 120 อะตอม) และขนาดใหญ่ (มากกว่า 120 อะตอม) ได้อย่างแม่นยำ\footnote{ความแม่นยำและความถูกต้องของผลการคำนวณ%
อ้างอิงตามประสบการณ์ของผู้เขียน โดยมีปัจจัยที่ส่งผลต่อค่าความถูกต้อง เช่น วิธีที่ใช้ในการคำนวณและ Basis Set} โดยจุดเด่นของ Gaussian 
ก็คือการคำนวณคุณสมบัติเชิงอิเล็กทรอนิกส์ของโมเลกุลด้วยวิธี DFT และด้วยอัลกอริทึมของตัวโปรแกรมนั้น ทำให้ Gaussian ได้รับการยอมรับว่าเป็น%
หนึ่งในโปรแกรมที่ให้ผลการคำนวณที่ถูกต้องและน่าเชื่อถือ และสามารถนำไปเปรียบเทียบกับผลการทดลองได้ สำหรับ Gaussian นั้นรองรับการคำนวณ%
แบบวิธี OpenMP นั่นคือสามารถทำการประมวลผลแบบขนาดได้โดยใช้หน่วยประมวลผล CPU หลายตัวพร้อม ๆ กันได้ และนอกจากนี้แล้วในเวอร์ชั่น 16
ตัวโปรแกรมยังรองรับกราฟฟิคการ์ด GPU สำหรับการคำนวณโดยใช้วิธี DFT อีกด้วย 

รายละเอียดเพิ่มเติมเกี่ยวกับโปรแกรม Gaussian ดูได้ที่เว็บไซต์ \url{https://gaussian.com}

%--------------------------
\subsection{ORCA}
\label{ssec:software_orca}
\idxen{Quantum Chemistry Software!ORCA}
%--------------------------

โปรแกรม ORCA เป็นอีกหนึ่งโปรแกรมทางเคมีควอนตัมที่มีประสิทธิภาพและความสามารถในการคำนวณสูง\autocite{neese2012,neese2018} 
สามารถคำนวณได้หลายวิธี โดยสามารถคำนวณ DFT และวิธี Semi-empirical ได้ รวมไปถึงวิธี Post Hartree-Fock อื่น ๆ ด้วย โดย ORCA 
ถูกใช้อย่างแพร่หลายในงานวิจัยทางด้านเคมีอินทรีย์และเคมีอนินทรีย์ โดยเฉพาะการศึกษาสารประกอบเชิงซ้อนของโลหะทรานซิชัน (Transition 
Metal Complex) ซึ่งเป็นโมเลกุลที่มีขนาดใหญ่และมีความซับซ้อนในเชิงของโครงสร้างอิเล็กทรอนิกส์มากกว่าโมเลกุลอินทรีย์ขนาดเล็ก และ ORCA 
ยังมีความโดดเด่นในด้านของความแม่นยำและความเร็วในการคำนวณเกี่ยวกับคุณสมบัติเชิงสเปกตรัมของโมเลกุล

โปรแกรม ORCA ถูกพัฒนาในกลุ่มวิจัยของศาสตราจารย์ Frank Neese โดยสามารถดาวน์โหลดตัวโปรแกรม (เฉพาะไฟล์ Binary ที่ถูกคอมไพล์แล้ว)
มาใช้ได้ฟรีสำหรับวัตถุประสงค์ด้านการศึกษาและการทำงานวิจัย 

รายละเอียดเพิ่มเติมเกี่ยวกับโปรแกรม ORCA ดูได้ที่เว็บไซต์ \url{https://orcaforum.kofo.mpg.de/app.php/portal}

%--------------------------
\subsection{NWChem}
\label{ssec:software_nwchem}
\idxen{Quantum Chemistry Software!NWChem}
%--------------------------

โปรแกรม NWChem เป็นโปรแกรมการคำนวณทางเคมีควอนตัมและพลศาสตร์เชิงโมเลกุล (Molecular Dynamics)\autocite{apra2020} 
พัฒนาโดยสถาบัน Pacific Northwest National Laboratory (PNNL) ในช่วงปี ค.ศ. 1990 โดยรองรับการคำนวณด้วยวิธี DFT และ 
Post Hartree-Fock เช่น M\"{o}llor-Plesset (MP), Configuration Interaction (CI), Coupled Cluster (CC) และ
Multiconfiguration SCF (MCSCF) 

NWChem นั้นถูกพัฒนาเพื่อให้สามารถประมวลผลบน Supercomputer ที่มีประสิทธิภาพสูงได้ NWChem ถูกเขียนขึ้นโดยใช้ภาษา Fortran 77
และใช้ไลบรารี่ทางด้านพีชคณิตสำหรับการประมวลผล เช่น BLAS, LAPACK, และ ScaLAPACK และสามารถประมวลผลแบบขนาดได้โดยใช้วิธี 
Message-Passing Interface (MPI) นอกจากนี้ NWChem ยังรองรับการประมวลผลด้วย GPU สำหรับการคำนวณด้วยวิธี Coupled Cluster 
ซึ่งถือว่าเป็นจุดเด่นของ NWChem เลยก็ว่าได้ โปรแกรม NWChem เป็นแบบ Open-source มีการพัฒนาอย่างต่อเนื่องเรื่อยมาจนถึงปัจจุบัน ซึ่งนักวิจัย 
นักศึกษา และคนทั่วไปสามารถร่วมพัฒนาและใช้งานโปรแกรมได้ฟรี

รายละเอียดเพิ่มเติมเกี่ยวกับโปรแกรม NWChem ดูได้ที่เว็บไซต์ \url{https://nwchemgit.github.io}

%--------------------------
\subsection{PySCF}
\label{ssec:software_pyscf}
\idxen{Quantum Chemistry Software!PySCF}
%--------------------------
