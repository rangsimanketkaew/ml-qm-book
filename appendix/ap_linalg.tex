% LaTeX source for ``การเรียนรู้ของเครื่องสำหรับเคมีควอนตัม (Machine Learning for Quantum Chemistry)''
% Copyright (c) 2022 รังสิมันต์ เกษแก้ว (Rangsiman Ketkaew).

% License: Creative Commons Attribution-NonCommercial-NoDerivatives 4.0 International (CC BY-NC-ND 4.0)
% https://creativecommons.org/licenses/by-nc-nd/4.0/

%--------------------------
\chapter{พีชคณิตเชิงเส้น}
%--------------------------
\label{ch:basic_math}

%--------------------------
\section{สเกลาร์, เวกเตอร์, และเมทริกซ์}
\label{sec:quantity}
%--------------------------

ปริมาณทางกายภาพสามารถแบ่งออกเป็น 3 ปริมาณ ได้แก่ สเกลาร์ (Scalar), เวกเตอร์ (Vector), และเมทริกซ์ (Matrix)
โดยปริมาณแต่ละตัวที่ใช้ในคณิตศาสตร์มีความหมายง่าย ๆ ดังนี้

\begin{description}
    \item[สเกลาร์] คือปริมาณที่มีเพียงขนาดอย่างเดียว โดยสามารถบอกแต่ขนาดอย่างเดียวก็ได้ความหมายสมบูรณ์ ไม่ต้องบอกทิศทาง 
    กล่าวคือ เป็นแค่เพียงตัวเลขเดี่ยว ๆ เท่านั้น
    \idxboth{สเกลาร์}{Scalar}
    
    \item[เวกเตอร์] คือปริมาณที่ใช้ดำเนินการบนปริภูมิเวกเตอร์ (Vector Space) ซึ่งจะมีความหมายค่อนข้างกว้าง แต่จะมีนิยามคล้าย ๆ 
    กับเวกเตอร์ในทางฟิสิกส์ กล่าวคือ เวกเตอร์จะมีทั้งขนาดและองค์ประกอบบ่งบอกทิศทาง ในส่วนของการเขียนโปรแกรมนั้นเวกเตอร์คือ Array 
    ขนาด 1 มิติ
    \idxboth{เวกเตอร์}{Vector}

    เครื่องหมายที่ใช้แทนเวกเตอร์มีดังต่อไปนี้

    \begin{equation}
        v = \begin{bmatrix}
        1 \\
        2 \\
        3 \\
        \end{bmatrix}
        =
        \begin{pmatrix}
        1 \\
        2 \\
        3 \\
        \end{pmatrix}
        =
        \begin{bmatrix}
        1 & 2 & 3\\
        \end{bmatrix}
    \end{equation}

    \item[เมทริกซ์] คือปริมาณที่เกิดจากเวกเตอร์มากกว่า 1 เวกเตอร์มารวมกัน โดยจะมีองค์ประกอบเป็นจำนวนแถวและจำนวนหลัก 
    ถ้าหากเมทริกซ์มีเพียงแค่แถวเดียว หรือหลักเดียว เราจะกล่าวได้ว่านั่นคือเวกเตอร์นั่นเอง ในส่วนของการเขียนโปรแกรมนั้นเมทริกซ์คือ Array 
    ขนาด 2 มิติ
    \idxboth{เมทริกซ์}{Matrix}

    ตัวอย่างของเมทริกซ์มีดังต่อไปนี้

    \begin{equation}
        a = \begin{bmatrix}
        a^{2} & 2a & 8\\
        18 & 7a-4 & 10\\
        \end{bmatrix}
    \end{equation}

    \begin{equation}
        b = \begin{bmatrix}
        a^{2} & 2a & 8\\
        18 & 7a-4 & 10\\
        \end{bmatrix}
    \end{equation}

\end{description}

%--------------------------
\section{ประเภทของเมทริกซ์}
\label{sec:matrix_type}
%--------------------------

เมทริกซ์แบบพิเศษมีด้วยกันหลากหลายแบบด้วยกัน โดยเมทริกซ์แบบพิเศษที่มักเจอมีดังต่อไปนี้

\paragraph{Zero Matrix หรือ Null Matrix} เมทริกซ์ที่มีสมาชิกทุกตัวเป็น 0 หมด มีนิยามดังต่อไปนี้

\begin{equation}
    0_{K_{m,n}} = 
    \begin{bmatrix}
    0_K & 0_K & \cdots & 0_K \\
    0_K & 0_K & \cdots & 0_K \\
    \vdots & \vdots & \ddots  & \vdots \\
    0_K & 0_K & \cdots & 0_K 
    \end{bmatrix}_{m \times n}
\end{equation}

\noindent ตัวอย่างของ Zero Matrix มีดังนี้

\begin{align}
    0_{1,1} &= \begin{bmatrix}
    0 \end{bmatrix}
    \\
    0_{2,2} &= \begin{bmatrix}
    0 & 0 \\
    0 & 0 \end{bmatrix}
    \\
    0_{2,3} &= \begin{bmatrix}
    0 & 0 & 0 \\
    0 & 0 & 0 \end{bmatrix}
\end{align}

\paragraph{Identity Matrix หรือ Unit Matrix} ใช้สัญลักษณ์ $I$ หรือ $I_{n}$ เมทริกซ์ที่สมาชิกในแนวทแยง (Diagonal Elements) 
มีค่ากับเท่า 1 ทุกตัวและสมาชิกที่ไม่ได้อยู่ในแนวทแยง (Off-diagonal Elements) เป็น 0 ทั้งหมด ถ้าในกรณีที่สมาชิกในแนวทแยงอย่างน้อย%
หนึ่งตัวที่ไม่มีค่าเท่ากับ 1 แต่สมาชิกนอกแนวทแยงยังเป็น 0 อยู่ เราจะเรียกเมทริกซ์นั้นว่า เมทริกซ์แนวทแยง (Diagonal Matrix)

\noindent ตัวอย่างของ Identity Matrix ตามขนาด มีดังนี้

\begin{align}
    I_1 &= \begin{bmatrix} 1 \end{bmatrix}
    ,\ \\
    I_2 &= \begin{bmatrix}
    1 & 0 \\
    0 & 1 \end{bmatrix}
    ,\ \\
    I_3 &= \begin{bmatrix}
    1 & 0 & 0 \\
    0 & 1 & 0 \\
    0 & 0 & 1 \end{bmatrix}
    ,\ \\ 
    & \dots ,\ \\
    I_n &= \begin{bmatrix}
    1 & 0 & 0 & \cdots & 0 \\
    0 & 1 & 0 & \cdots & 0 \\
    0 & 0 & 1 & \cdots & 0 \\
    \vdots & \vdots & \vdots & \ddots & \vdots \\
    0 & 0 & 0 & \cdots & 1 \end{bmatrix}
\end{align}

%--------------------------
\section{การดำเนินการของเมทริกซ์}
\label{sec:matrix_operate}
%--------------------------

\paragraph{Transpose} ถ้าหากเรามีรูปแบบซึ่งจริง ๆ ก็คือเมทริกซ์ขนาด 2 มิติแล้วเราทำการคูณด้วยเมทริกซ์การหมุน (Rotation Matrix) 
สิ่งที่เราจะได้คือเราจะได้รูปภาพที่ถูกหมุนไป โดยการที่เรากระทำการหมุนเมทริกซ์นั้นเราเรียกว่า Transpose 
อธิบายง่าย ๆ คือการ Tranpose เมทริกซ์นั้นก็คือการสลับแถวกับหลักของเมทริกซ์ หรือทำการหมุนสมาชิกที่ไม่ใช่แถวทแยง (Off-diagonal) 
รอบ ๆ แนวทแยงนั่นเอง
\idxboth{เมทริกซ์!ทรานโพส}{Matrix!Transpose}

\paragraph{การบวกและการลบ} เมทริกซ์สองเมทริกซ์ที่มีขนาดเท่ากัน (จำนวนแถวและจำนวนหลักเท่ากัน) สามารถบวกและลบกันได้ 
โดยให้ทำการบวกหรือลบสมาชิกที่มีดัชนีตรงกันได้โดยตรงเลย

\paragraph{การคูณด้วยสเกลาร์} การคูณเมทริกซ์ด้วยปริมาณสเกลาร์สามารถทำได้ง่าย ๆ โดยให้คูณสมาชิกทุกตัวของเมทริกซ์ด้วยตัวเลขตัวนั้น

\paragraph{การคูณเมทริกซ์ด้วยเมทริกซ์แบบจุด (Dot Product)}
สมมติว่าเรามีเมทริกซ์ A กับเมทริกซ์ B การที่เมทริกซ์สองตัวนี้จะคูณกันได้นั้นจะต้องไม่ขัดกับเงื่อนไขดังต่อไปนี้
\enquote{สมาชิกของผลคูณของเมทริกซ์ในแถวที่ $i$ หลักที่ $j$ จะเกิดสมาชิกในแถวที่ $i$ ของเมทริกซ์ที่อยู่หน้า คูณกับสมาชิกในหลักที่ $j$ 
ของเมทริกซ์หลักเป็นคู่ ๆ แล้วนำมาบวกกัน}

\paragraph{การคูณเมทริกซ์ด้วยเมทริกซ์แบบอาดามาร์ (Hadamard Product)} เป็นการคูณสมาชิกของเมทริกซ์ที่มีขนาดมิติเท่ากันและนำสมาชิก%
ที่มีตำแหน่งตรงกันมาคูณกันโดยตรง

\begin{equation}
    \begin{bmatrix}
    a_1 & a_2 \\
    a_3 & a_4 \\
    \end{bmatrix}
    \odot
    \begin{bmatrix}
    b_1 & b_2 \\
    b_3 & b_4 \\
    \end{bmatrix}
    =
    \begin{bmatrix}
    a_1 \cdot b_1 & a_2 \cdot b_2 \\
    a_3 \cdot b_3 & a_4 \cdot b_4 \\
\end{bmatrix}
\end{equation}

%--------------------------
\section{เทนเซอร์}
\label{sec:tensor}
%--------------------------

บางครั้งเราจำเป็นที่จะต้องจัดการข้อมูลที่มีจำนวนของมิติที่มากกว่า 2 มิติ นั่นคือเราไม่สามารถใช้เวกเตอร์หรือเมทริกซ์ได้อีกต่อไป
โดยเราจะต้องใช้เทนเซอร์ (Tensor) แทน เพราะว่าเทนเซอร์คือ Array ที่มีจำนวนมิติ $n$ มิติ สรุปง่าย ๆ คือเวกเตอร์นั้นคือเทนเซอร์ 1 มิติ 
และเมทริกซ์คือเทนเซอร์ 2 มิติ แล้วถ้าเป็น 3 มิติล่ะ เราจะเรียกว่าเป็นคิวป์ (Cube) และเทนเซอร์ 4 มิติ เราก็จะเรียกว่าเป็นเวกเตอร์ของคิวป์ 
และ 5 มิติก็จะเป็นเมทริกซ์ของคิวป์นั่นเอง
\idxboth{เทนเซอร์}{Tensor}
