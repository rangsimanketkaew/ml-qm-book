% LaTeX source for ``การเรียนรู้ของเครื่องสำหรับเคมีควอนตัม (Machine Learning for Quantum Chemistry)''
% Copyright (c) 2022 รังสิมันต์ เกษแก้ว (Rangsiman Ketkaew).

% License: Creative Commons Attribution-NonCommercial-NoDerivatives 4.0 International (CC BY-NC-ND 4.0)
% https://creativecommons.org/licenses/by-nc-nd/4.0/

{
% \pagenumbering{gobble}

\chapter*{\centering คำนำ}
\addcontentsline{toc}{chapter}{คำนำ}

ในปัจจุบันได้มีการนำการเรียนรู้ของเครื่อง (Machine Learning) ไปใช้ประโยชน์ในหลากหลายด้าน เช่น คอมพิวเตอร์วิทัศน์, คอมพิวเตอร์กราฟฟิก, 
ภาษาศาสตร์, เศรษฐศาสตร์, อุตสาหกรรม, เกษตรกรรม รวมไปถึงวิทยาศาสตร์และวิศวกรรม โดยสาขาเคมีนั้นก็เป็นอีกหนึ่งศาสตร์ที่ได้มีการนำ 
Machine Learning เข้ามาประยุกต์ใช้มาเป็นระยะเวลานานนับตั้งแต่ช่วงปี ค.ศ. 1990 โดยนักวิจัยได้พัฒนาเทคนิค Machine Learning 
เพื่อใช้ในการศึกษาคุณสมบัติของโมเลกุลอินทรีย์และอนินทรีย์ ศึกษาโครงสร้างโปรตีน ออกแบบโมเลกุลยา รวมไปถึงศึกษาปฏิริยาเคมีและตัวเร่งปฏิกิริยา

ผู้เขียนเล็งเห็นว่าการประยุกต์ใช้ Machine Learning กับเคมีควอนตัมนั้นเป็นสิ่งใหม่ที่หลายคนต่างก็ให้ความสนใจ ทั้งนักศึกษา อาจารย์ และนักวิจัย 
ดังนั้นผู้เขียนจึงได้เรียบเรียงหนังสือเล่มนี้ขึ้นมาเพื่อใช้เป็นแนวทางสำหรับการศึกษาและทำงานวิจัยทางด้านนี้ โดยหนังสือเล่มนี้ครอบคลุมทฤษฎีและเทคนิค 
Machine Learning, ทฤษฎีโครงสร้างเชิงอิเล็กทรอนิกส์ซึ่งเป็นสาขาหนึ่งของเคมีควอนตัมที่ช่วยให้นักวิจัยเข้าใจองค์ความรู้พื้นฐานของอะตอมและโมเลกุล, 
หัวข้อเคมีควอนตัมที่สามารถนำการเรียนรู้ของเครื่องไปประยุกต์ใช้ได้ รวมไปถึงรายละเอียดเชิงลึกในการพัฒนาวิธีคำนวณแบบใหม่เพื่อปรับปรุงประสิทธิ%
ภาพการพยากรณ์ของแบบจำลอง Machine Learning

ผู้เขียนหวังเป็นอย่างยิ่งว่าหนังสือเล่มนี้จะช่วยให้ผู้อ่านทุกท่านได้รับความรู้และความเข้าใจที่ครบถ้วนเกี่ยวกับ Machine Learning สำหรับเคมีควอนตัม

\medskip

\begin{flushright}
รังสิมันต์ เกษแก้ว \\
9 ตุลาคม พ.ศ. 2565
\end{flushright}
}
