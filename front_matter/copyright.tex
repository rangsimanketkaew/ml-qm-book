% LaTeX source for ``การเรียนรู้ของเครื่องสำหรับเคมีควอนตัม (Machine Learning for Quantum Chemistry)''
% Copyright (c) 2022 รังสิมันต์ เกษแก้ว (Rangsiman Ketkaew).

% License: Creative Commons Attribution-NonCommercial-NoDerivatives 4.0 International (CC BY-NC-ND 4.0)
% https://creativecommons.org/licenses/by-nc-nd/4.0/

{
~\vfill
\thispagestyle{empty}
\setlength{\parindent}{0em}

รังสิมันต์ เกษแก้ว

การเรียนรู้ของเครื่องสำหรับเคมีควอนตัม\\
Machine Learning for Quantum Chemistry

ปกด้านหน้า: แขนกลและโมเลกุล\\
ปกด้านหลัง: ปริภูมิเคมีของชุดข้อมูล QM9

\bigskip

\par{ฉบับพิมพ์ครั้งที่ 1 พ.ศ. 2565}

สงวนลิขสิทธิ์ตาม พ.ร.บ. ลิขสิทธิ์ พ.ศ. 2537/2540

อนุญาตให้ผู้อื่นเผยแพร่ผลงานชิ้นนี้ได้ ตราบใดที่ให้เครดิตแก่ผู้เขียนในฐานะผู้สร้างต้นฉบับและลิงก์กลับไปที่สัญญาอนุญาตของเจ้าของผลงาน 
ไม่อนุญาตให้นำไปใช้เพื่อการค้าและดัดแปลงแก้ไขไม่ว่าด้วยวิธีใด เว้นแต่จะได้รับอนุญาตเป็นลายลักษณ์อักษรจากผู้เขียน

หนังสือเล่มนี้อยู่ภายใต้ลิขสิทธิ์สัญญาอนุญาตแบบเปิด A Creative Commons Attribution-NonCommercial-NoDerivatives 4.0 
International (CC BY-NC-ND 4.0), https://creativecommons.org/licenses/by-nc-nd/4.0/.

% Use absolute path of the file
\includegraphics[scale=1.2]{front_matter/by-nc-nd.pdf}

\bigskip

ซอร์สโค้ดของหนังสือเล่มนี้ถูกเขียนขึ้นโดยใช้ภาษา {\fontfamily{lmr}\selectfont \LaTeX} เผยแพร่ที่ 
\url{https://github.com/rangsimanketkaew/ml-qm-book} และไฟล์ PDF ถูกสร้างขึ้นโดยใช้ {\fontfamily{lmr}\selectfont 
\XeLaTeX} เผยแพร่ที่ \url{https://rangsimanketkaew.github.io/ml-qm-book}

หากต้องการติดต่อผู้เขียน กรุณาส่งอีเมลมาที่ rangsiman1993[at]gmail[dot]com
}
