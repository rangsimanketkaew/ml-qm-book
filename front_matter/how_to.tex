% LaTeX source for ``การเรียนรู้ของเครื่องสำหรับเคมีควอนตัม (Machine Learning for Quantum Chemistry)''
% Copyright (c) 2022 รังสิมันต์ เกษแก้ว (Rangsiman Ketkaew).

% License: Creative Commons Attribution-NonCommercial-NoDerivatives 4.0 International (CC BY-NC-ND 4.0)
% https://creativecommons.org/licenses/by-nc-nd/4.0/

{
% \pagenumbering{gobble}

\chapter*{\centering คำแนะนำในการอ่านหนังสือเล่มนี้}
\addcontentsline{toc}{chapter}{คำแนะนำในการอ่านหนังสือเล่มนี้}

ผู้เขียนเรียบเรียงหนังสือเล่มนี้ขึ้นมาเพื่อให้บุคคลที่สนใจการเรียนรู้ของเครื่องและเคมีควอนตัมได้ศึกษาเพื่อเป็นแนวทางในการทำงานวิจัยในสาขานี้ต่อไป 
โดยผู้เขียนได้ศึกษาจากหนังสือต่างประเทศรวมไปบทความงานวิจัยที่ตีพิมพ์ในวารสารวิชาการชั้นนำทางด้านปัญญาประดิษฐ์และเคมีทฤษฎีที่ได้รับการยอมรับ

หนังสือเล่มนี้เน้นไปที่การอธิบายทฤษฏีประกอบกับสมการทางคณิตศาสตร์ที่ใช้ในเคมีควอนตัมอย่างกระชับ โดยผู้เขียนพยายามเลือกใช้คำและสำนวน%
ที่ไม่เป็นทางการมากนักในการอธิบายเนื้อหาที่ซับซ้อนเพื่อให้ผู้อ่านสามารถเข้าใจได้ง่ายขึ้น ดังนั้นสไตล์การเขียนของผู้เขียนจึงเป็นในเชิงที่ใช้ภาษาพูด 
โดยมีการใส่ความคิดเห็นส่วนตัว แนวคิดและมุมมองของผู้เขียนที่ได้จากการอภิปรายกับเพื่อนร่วมงานและผู้เชี่ยวชาญในสาขาที่คิดว่าเหมาะสมเข้าไปด้วย

หนังสือเล่มนี้ประกอบไปด้วยเนื้อหาสามส่วน ดังนี้ 
%
\begin{enumerate}[topsep=0pt]
    \item \textbf{การเรียนรู้ของเครื่อง}: ในส่วนแรกผู้อ่านจะได้ศึกษาที่มาและความสำคัญของการเรียนรู้ของเครื่องและการเชื่อมโยงเพื่อนำไป%
    ใช้กับเคมีควอนตัมโดยเฉพาะในการทำงานวิจัย ผู้อ่านจะได้ศึกษาอัลกอริทึมการเรียนรู้ของเครื่องแบบต่าง ๆ ทั้งการเรียนรู้แบบมีผู้สอนและแบบไม่มี%
    ผู้สอนซึ่งเป็นอัลกอริทึมมาตรฐานที่นักวิจัยใช้เพื่อพัฒนาโมเดลสำหรับการทำนายเอาต์พุตที่ต้องการ นอกจากนี้ผู้อ่านจะได้เรียนรู้ปัญหาของการเรียนรู้%
    ของเครื่องที่มักจะพบเจอได้บ่อยและขั้นตอนหรือเทคนิคการแก้ปัญหาดังกล่าว และการเลือกโมเดลการเรียนรู้ของเครื่องเพื่อให้เหมาะสมกับโจทย์%
    ปัญหาที่ต้องการแก้
    
    \item \textbf{การทำนายคุณสมบัติเชิงเคมีควอนตัม}: ในส่วนที่สองจะเป็นการอธิบายทฤษฎีของโครงสร้างเชิงอิเล็กทรอนิกส์ (Electronic 
    Structure) ของอะตอมและโมเลกุล ผู้อ่านจะได้ศึกษาระบบอิเล็กตรอน วิธีการคำนวณที่อ้างอิงกับฟังก์ชันคลื่นและความหนาแน่นของอิเล็กตรอน 
    รวมถึงทฤษฎีของคุณสมบัติเชิงโมเลกุลแบบต่าง ๆ ซึ่งเป็นสิ่งที่นักวิจัยเคมีทฤษฎีให้ความสนใจและเกี่ยวข้องโดยตรงกับหัวข้อถัดไปนั่นก็คือทฤษฎี%
    ของตัวอธิบายเชิงอิเล็กทรอนิกส์ (Electronic-based Representation) แบบต่าง ๆ ที่ได้รับการพัฒนาตั้งแต่อดีตจนถึงปัจจุบันและถูกนำ%
    มาใช้สำหรับการคำนวณลักษณะเฉพาะ (Feature) ของโมเลกุล นอกจากนี้ผู้อ่านจะได้ศึกษาชุดข้อมูลทางเคมีควอนตัมที่เราจะนำมาใช้เป็นอินพุต%
    สำหรับการฝึกสอนโมเดล ผู้อ่านจะได้ศึกษาการพัฒนาโมเดลการเรียนรู้ของเครื่องรวมไปถึงโมเดลเฉพาะทางแบบต่าง ๆ สำหรับเคมีควอนตัมที่ได้%
    รับการพัฒนาในช่วงสิบปีที่ผ่านมาและการวิเคราะห์ผลการทำนายของโมเดลซึ่งจะเป็นประโยชน์ในการทำงานวิจัยและเผยแพร่ผลงานวิชาการต่อไป

    \item \textbf{ภาคผนวก}: ในส่วนที่สามเป็นภาคผนวกซึ่งจะรวบรวมหัวข้อพื้นฐานที่เป็นประโยชน์ต่อการทำความเข้าใจเนื้อหาหลักในสองส่วน%
    แรก ประกอบไปด้วยพื้นฐานพีชคณิตเชิงเส้นซึ่งเกี่ยวข้องกับเมทริกซ์ การเขียนโปรแกรมการเรียนรู้ของเครื่อง โดยผู้อ่านจะได้ฝึกการเขียนโปรแกรม%
    ของโครงข่ายประสาท (Neural Network) ด้วยไลบรารี่ TensorFlow เป็นต้น นอกจากนี้ยังมีการแนะนำโปรแกรมเคมีเชิงคำนวณที่ผู้อ่าน%
    สามารถใช้งานเพื่อคำนวณคุณสมบัติเชิงโครงสร้างและเชิงอิเล็กทรอนิกส์ของโมเลกุลได้อีกด้วย
\end{enumerate}

ในการอธิบายทฤษฎีนั้นผู้เขียนได้ใส่โค้ดเพื่อให้ผู้อ่านสามารถนำไปเขียนโปรแกรมและทดสอบได้ด้วยตนเองเข้าไปด้วย โดยสามารถดาวน์โหลดโค้ดได้ที่ 
\url{https://github.com/rangsimanketkaew/ml-qm-book-code} ซึ่งจากประสบการณ์ส่วนตัวของผู้เขียนพบว่าการเขียนโปรแกรมนั้น%
สามารถช่วยให้เข้าใจทฤษฎีต่าง ๆ ทางเคมีควอนตัมได้ง่ายขึ้น รวมไปถึงเข้าใจวิธีคิดในการคำนวณอย่างเป็นขั้นเป็นตอน ดังนั้นผู้เขียนจึงขอแนะนำ%
ให้ผู้อ่านศึกษาทฤษฎีควบคู่ไปพร้อมกับโค้ดอย่างละเอียด

ในกรณีที่ผู้อ่านมีข้อเสนอแนะหรือพบข้อผิดพลาดของหนังสือสามารถแจ้งผู้เขียนได้โดยกรอกแบบฟอร์มที่ 
\url{https://cutt.ly/ml-qm-book-feedback} สุดท้ายนี้ผู้เขียนขอให้ผู้อ่านมีความสุขกับการอ่านหนังสือเล่มนี้ครับ :)

\medskip

\begin{flushright}
รังสิมันต์ เกษแก้ว
\end{flushright}
}
