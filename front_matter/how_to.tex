% LaTeX source for ``การเรียนรู้ของเครื่องสำหรับเคมีควอนตัม (Machine Learning for Quantum Chemistry)''
% Copyright (c) 2022 รังสิมันต์ เกษแก้ว (Rangsiman Ketkaew).

% License: Creative Commons Attribution-NonCommercial-NoDerivatives 4.0 International (CC BY-NC-ND 4.0)
% https://creativecommons.org/licenses/by-nc-nd/4.0/

{
% \pagenumbering{gobble}

\chapter*{\centering คำแนะนำในการอ่านหนังสือเล่มนี้}
\addcontentsline{toc}{chapter}{คำแนะนำในการอ่านหนังสือเล่มนี้}

เนื่องจากว่าในปัจจุบันยังไม่มีหนังสือเกี่ยวกับการเรียนรู้ของเครื่องสำหรับเคมีควอนตัม ผู้เขียนจึงเรียบเรียงหนังสือเล่มนี้ขึ้นมาเพื่อให้บุคคลที่สนใจได้%
ศึกษาเพื่อเป็นแนวทางในการทำงานวิจัยในสาขานี้ต่อไป ผู้เขียนไม่ได้แปลหนังสือเล่มนี้มาจากหนังสือหรือตำราต่างประเทศ แต่เรียงเรียงจากการอ่าน%
และสรุปจากบทความงานวิจัยที่ตีพิมพ์ในวารสารวิชาการชั้นนำทางด้านปัญญาประดิษฐ์และเคมีทฤษฎีที่ได้รับการยอมรับ

หนังสือเล่มนี้จะเน้นไปที่การอธิบายทฤษฏีประกอบกับสมการทางคณิตศาสตร์อย่างกระชับ โดยผู้เขียนใช้สำนวนที่ไม่เป็นทางการมากนักและพยายามอธิบาย%
ความหมายของคำศัพท์เฉพาะทางเพื่อให้ผู้อ่านสามารถเข้าใจเนื้อหาที่ซับซ้อนได้ง่ายขึ้น ดังนั้นสไตล์การเขียนของผู้เขียนจะเป็นในเชิงที่ใช้ภาษาพูด โดยมี%
การใส่ความคิดเห็นส่วนตัว แนวคิดและมุมมองของผู้เขียนที่ได้จากการอภิปรายกับเพื่อนร่วมงานและผู้เชี่ยวชาญในสาขาที่คิดว่าเหมาะสมเข้าไปด้วย

หนังสือเล่มนี้ประกอบไปด้วยเนื้อหาสามส่วน ดังนี้ 
%
\begin{enumerate}[topsep=0pt]
    \item \textbf{การเรียนรู้ของเครื่อง}: ในส่วนแรกผู้อ่านจะได้ศึกษาที่มาและความสำคัญของการเรียนรู้ของเครื่องและการเชื่อมโยงเพื่อนำไป%
    ใช้กับเคมีควอนตัม โดยเฉพาะในการทำงานวิจัย ผู้อ่านจะได้ศึกษาอัลกอริทึมต่าง ๆ ของการเรียนรู้ของเครื่องทั้งการเรียนรู้แบบมีผู้สอนและแบบไม่มี%
    ผู้สอนซึ่งเป็นอัลกอริทึมมาตรฐานที่นักวิจัยใช้เพื่อพัฒนาโมเดลสำหรับการทำนายเอาต์พุตที่ต้องการ นอกจากนี้ผู้อ่านจะได้เรียนรู้ปัญหาของการเรียนรู้%
    ของเครื่องที่มักจะพบเจอได้บ่อยและขั้นตอนหรือเทคนิคการแก้ปัญหาดังกล่าว และการเลือกโมเดลการเรียนรู้ของเครื่องเพื่อให้เหมาะสมกับโจทย์ทาง%
    สถิติที่ต้องการแก้
    
    \item \textbf{การทำนายคุณสมบัติเชิงเคมีควอนตัม}: ในส่วนที่สองจะเป็นการอธิบายทฤษฎีของโครงสร้างเชิงอิเล็กทรอนิกส์ (Electronic 
    Structure) ของอะตอมและโมเลกุล นอกจากนี้ผู้อ่านจะได้ศึกษาชุดข้อมูลทางเคมีควอนตัม ทฤษฎีของคุณสมบัติเชิงโมเลกุลแบบต่าง ๆ ซึ่งเป็น%
    สิ่งที่นักวิจัยเคมีทฤษฎีให้ความสนใจและเกี่ยวข้องโดยตรงกับหัวข้อถัดไปนั่นก็คือทฤษฎีของตัวอธิบายเชิงอิเล็กทรอนิกส์ (Electronic-based 
    Descriptor) แบบต่าง ๆ ที่ได้รับการพัฒนาตั้งแต่อดีตจนถึงปัจจุบันและถูกนำมาใช้สำหรับการคำนวณลักษณะเฉพาะ (Feature) ของโมเลกุล 
    เมื่อศึกษา Feature ที่เราจะนำมาใช้เป็นอินพุตสำหรับการฝึกสอนโมเดลแล้ว ผู้อ่านจะได้ศึกษาการสร้างและพัฒนาโมเดลการเรียนรู้ของเครื่องรวม%
    ไปถึงโมเดลเฉพาะทางแบบต่าง ๆ ที่ได้รับการพัฒนาในช่วงสิบปีที่ผ่านมา และการวิเคราะห์ผลการทำนายของโมเดลซึ่งจะเป็นประโยชน์ในการทำงาน%
    วิจัยและการตีพิมพ์ต่อไป

    \item \textbf{ภาคผนวก}: ในส่วนที่สามเป็นภาคผนวกซึ่งจะรวบรวมหัวข้อพื้นฐานที่เป็นประโยชน์ต่อการทำความเข้าใจเนื้อหาหลักในสองส่วน%
    แรก ประกอบไปด้วยพื้นฐานพีชคณิตเชิงเส้นซึ่งเกี่ยวข้องกับเมทริกซ์ การเขียนโปรแกรมสำหรับปัญญา ประดิษฐ์โดยผู้อ่านจะได้ฝึกการเขียนโปรแกรม%
    ของโครงข่ายประสาท (Neural Network) ด้วยไลบรารี่ TensorFlow เป็นต้น นอกจากนี้ยังมีการอธิบายเกี่ยวกับโปรแกรมเคมีเชิงคำนวณที่%
    ผู้อ่านสามารถใช้งานเพื่อคำนวณคุณสมบัติเชิงโครงสร้างและอิเล็กทรอนิกส์ของโมเลกุลได้อีกด้วย
\end{enumerate}

ในการอธิบายทฤษฎีนั้นผู้เขียนได้ใส่โค้ดเพื่อให้ผู้อ่านสามารถนำไปเขียนโปรแกรมและทดสอบได้ด้วยตนเองเข้าไปด้วย ซึ่งจากประสบการณ์ส่วนตัวของ%
ผู้เขียนพบว่าการเขียนโปรแกรมนั้นสามารถช่วยให้เข้าใจทฤษฎีต่าง ๆ ทางเคมีควอนตัมได้ง่ายขึ้น รวมไปถึงเข้าใจวิธีคิดในการคำนวณอย่างเป็นขั้นตอน%
เป็นตอน ดังนั้นผู้เขียนจึงขอแนะนำให้ผู้อ่านศึกษาทฤษฎีควบคู่ไปพร้อมกับโค้ดอย่างละเอียด

\medskip

\begin{flushright}
รังสิมันต์ เกษแก้ว
\end{flushright}
}
