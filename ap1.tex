% LaTeX source for ``การเรียนรู้ของเครื่องสำหรับเคมีควอนตัม (Machine Learning for Quantum Chemistry)''
% Copyright (c) 2022 รังสิมันต์ เกษแก้ว (Rangsiman Ketkaew).

% License: Creative Commons Attribution-NonCommercial-NoDerivatives 4.0 International (CC BY-NC-ND 4.0)
% https://creativecommons.org/licenses/by-nc-nd/4.0/

%--------------------------
\section{พีชคณิตเชิงเส้น}
%--------------------------
\label{ap:basic_math}

%--------------------------
\subsection{สเกลาร์, เวกเตอร์, และเมทริกซ์}
%--------------------------

ปริมาณทางกายภาพสามารถแบ่งออกเป็น 3 ปริมาณ ได้แก่ สเกลาร์ (Scalar), เวกเตอร์ (Vector), และเมทริกซ์ (Matrix)
โดยปริมาณแต่ละตัวที่ใช้ในคณิตศาสตร์มีความหมายง่าย ๆ ดังนี้

\begin{description}
    \item[สเกลาร์] คือปริมาณที่มีเพียงขนาดอย่างเดียว โดยสามารถบอกแต่ขนาดอย่างเดียวก็ได้ความหมายสมบูรณ์ ไม่ต้องบอกทิศทาง กล่าวคือเป็นแค่เพียงตัวเลขเดี่ยว ๆ เท่านั้น
    \item[เวกเตอร์] คือปริมาณที่ใช้ดำเนินการบนปริภูมิเวกเตอร์ (Vector Space) ซึ่งจะมีความหมายค่อนข้างกว้าง แต่จะมีนิยามคล้าย ๆ กับเวกเตอร์ในทางฟิสิกส์
    กล่าวคือเวกเตอร์จะมีทั้งขนาดและองค์ประกอบบ่งบอกทิศทาง ในส่วนของการเขียนโปรแกรมนั้นเวกเตอร์คือ Array ขนาด 1 มิติ
    \item[เมทริกซ์] คือปริมาณที่เกิดจากเวกเตอร์มากกว่า 1 เวกเตอร์มารวมกัน โดยจะมีองค์ประกอบเป็นจำนวนแถวและจำนวนหลัก 
    ถ้าหากเมทริกซ์มีเพียงแค่แถวเดียว หรือหลักเดียว เราจะกล่าวได้ว่านั่นคือเวกเตอร์นั่นเอง ในส่วนของการเขียนโปรแกรมนั้นเมทริกซ์คือ Array ขนาด 2 มิติ
\end{description}

%--------------------------
\subsection{การดำเนินการของเมทริกซ์}
%--------------------------

\paragraph{Transpose} ถ้าหากเรามีรูปแบบซึ่งจริง ๆ ก็คือเมทริกซ์ขนาด 2 มิติแล้วเราทำการคูณด้วยเมทริกซ์การหมุน (Rotation Matrix) 
สิ่งที่เราจะได้คือเราจะได้รูปภาพที่ถูกหมุนไป โดยการที่เรากระทำการหมุนเมทริกซ์นั้นเราเรียกว่า Transpose 
อธิบายง่าย ๆ คือการ Tranpose เมทริกซ์นั้นก็คือการสลับแถวกับหลักของเมทริกซ์ หรือทำการหมุนสมาชิกที่ไม่ใช่แถวทแยง (Off-diagonal) รอบ ๆ แนวทแยงนั่นเอง

\paragraph{การบวกและการลบ} เมทริกซ์สองเมทริกซ์ที่มีขนาดเท่ากัน (จำนวนแถวและจำนวนหลักเท่ากัน) สามารถบวกและลบกันได้ 
โดยให้ทำการบวกหรือลบสมาชิกที่มีดัชนีตรงกันได้โดยตรงเลย

\paragraph{การคูณด้วยสเกลาร์} การคูณเมทริกซ์ด้วยปริมาณสเกลาร์สามารถทำได้ง่าย ๆ โดยให้คูณสมาชิกทุกตัวของเมทริกซ์ด้วยตัวเลขตัวนั้น

\paragraph{การคูณเมทริกซ์ด้วยเมทริกซ์}
สมมติว่าเรามีเมทริกซ์ A กับเมทริกซ์ B การที่เมทริกซ์สองตัวนี้จะคูณกันได้นั้นจะต้องไม่ขัดกับเงื่อนไขดังต่อไปนี้
\enquote{สมาชิกของผลคูณของเมทริกซ์ในแถวที่ i หลักที่ j จะเกิดสมาชิกในแถวที่ i ของเมทริกซ์ที่อยู่หน้า คูณกับสมาชิกในหลักที่ j ของเมทริกซ์หลักเป็นคู่ ๆ แล้วนำมาบวกกัน}

%--------------------------
\subsection{ประเภทของเมทริกซ์}
%--------------------------

เมทริกซ์แบบพิเศษมีด้วยกันหลากหลายแบบด้วยกัน โดยเมทริกซ์แบบพิเศษที่มักเจอมีดังต่อไปนี้

\paragraph{Identity} เมทริกซ์ที่สมาชิกในแนวทแยงมีค่ากับเท่า 1 ทุกตัวและสมาชิกที่เหลือเป็น 0 ทั้งหมด

%--------------------------
\subsection{เทนเซอร์}
%--------------------------

บางครั้งเราจำเป็นที่จะต้องจัดการข้อมูลที่มีจำนวนของมิติที่มากกว่า 2 มิติ นั่นคือเราไม่สามารถใช้เวกเตอร์หรือเมทริกซ์ได้อีกต่อไป
โดยเราจะต้องใช้เทนเซอร์ (Tensor) แทน เพราะว่าเทนเซอร์คือ Array ที่มีจำนวนมิติ $n$ มิติ สรุปง่าย ๆ คือเวกเตอร์นั้นคือเทนเซอร์ 1 มิติ 
และเมทริกซ์คือเทนเซอร์ 2 มิติ แล้วถ้าเป็น 3 มิติล่ะ เราจะเรียกว่าเป็นคิวป์ (Cube) และเทนเซอร์ 4 มิติ เราก็จะเรียกว่าเป็นเวกเตอร์ของคิวป์ 
และ 5 มิติก็จะเป็นเมทริกซ์ของคิวป์นั่นเอง
